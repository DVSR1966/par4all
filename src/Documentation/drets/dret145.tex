\documentclass[12pt]{article}

\usepackage[latin1]{inputenc}
\input{/usr/share/local/lib/tex/macroslocales/Dimensions.tex}

% pour importer des structures de donn�es Newgen
\newcommand{\domain}[2]{\paragraph{{#1}}\paragraph{}{\em #2}}

\newcommand{\titre}{PROJET PIPS \\
		STRUCTURES DE DONN�ES DE BASE ET LISTINGS ASSOCI�S\\[20mm]
		2eme edition 
}

\date{Septembre 1991}

\newcommand{\auteur}{
        	Fran�ois IRIGOIN \\
        	Pierre JOUVELOT \\
\vspace{0.5cm}
{\it Le pr�sent document a �t� �tabli en ex�cution du contrat
No.~88.017.01 pass� par la Direction des Recherches, Etudes et
Techniques (D�l�gation G�n�rale pour l'Armement)}
}
\newcommand{\docdate}{D�cembre 1990}
\newcommand{\numero}{E145}

\begin{document}
\input{/usr/share/local/lib/tex/macroslocales/PageTitre.tex}

{\it Le pr�sent document a �t� �tabli en ex�cution du contrat
No.~88.017.01 pass� par la Direction des Recherches, Etudes et
Techniques (D�l�gation G�n�rale pour l'Armement)}

\vspace{2cm}

\tableofcontents

\newpage

\section*{Introduction}

\section{Repr�sentation interne des programmes ({\em RI})}

\documentstyle[a4,psfig]{article}

% \input{/usr/share/local/lib/tex/macroslocales/Dimensions.tex}

\title{PIPS: Internal Representation of Fortran Code}
\author{Franc,ois Irigoin \\
        Pierre Jouvelot \\
    Ronan Keryell \\
        Re'mi Triolet\\
\\
        CRI, Ecole des Mines de Paris}

                                %\newcommand{\domain}[2]{\paragraph{{#1}}\paragraph{}{{#2}}}
      {
        \catcode `==\active
        \gdef\domain{\medskip\par\noindent
          \bgroup
          \catcode `_ \other
          \catcode `= \active
          \def={\em{\rm \string=}}
          \tt\em\vraidomain}
        \gdef\vraidomain#1{#1\egroup\medskip\par}
        }

% Correction de quelques erreurs. RK, 07/02/1994.

\renewcommand{\indexname}{Index}

\makeindex

\begin{document}
\maketitle
\sloppy

\section*{Introduction}

Ce document est utilise' directement et automatiquement par l'outil
de ge'nie logiciel NewGen pour ge'ne'rer les de'clarations des structures
de donne'es utilise'es dans le projet PIPS, ainsi que les routines de
base qui les manipulent. C'est pourquoi l'ordre des sections n'est pas
ne'cessairement naturel.

Apre`s une description rapide des structures de donne'es externes a` la
repre'sentation interne de Pips, nous pre'sentons successivement les
notions d'entite's, de code et d'expressions et la manie`re dont elles
sont utilise'es pour encoder un programme Fortran. Nous de'taillons
ensuite deux structures de donne'es supple'mentaires, les effets et les
{\em transformers} qui sont utilise's pour le calcul de de'pendance
interproce'dural et pour l'analyse syntaxique interproce'durale.

La manie`re dont les constructions de Fortran sont repre'sente'es est
de'crite dans une des sections du rapport EMP-CAII-E105. NewGen est
pre'sente' dans le rapport EMP-CRI-A191.

\section{External Data Structures}
\label{external}

\subsection{Vector}
\label{subsection-pvecteur}
\index{Pvecteur}

\domain{External Pvecteur}
{ Le domaine {\tt Pvecteur} est utilise' pour repre'senter les
expressions line'aires telles que {\tt 3I+2} (voir le domaine {\tt
normalized}) ou des contraintes line'aires telles que {\tt 3I + J <= 2}
ou {\tt 3I == J}. Ces contraintes sont utilise'es dans les syste`mes
line'aires (voir le domaine {\tt Psysteme}).

Un Pvecteur est une suite de mono^mes, un mono^me e'tant un couple
(coefficient,variable).  Le coefficient d'un tel couple est un entier,
positif ou ne'gatif. La variable est une entite', sauf dans le cas du
terme constant qui est repre'sente' par la variable pre'de'finie de nom
{\tt TCST}\footnote{Comme on rajoute le nom des modules devant les
noms de variables, il ne peut pas y avoir de conflict avec une
e'ventuelle variable {\tt TCST}.}.

Les expressions apparaissant dans le programme analyse' sont mises
sous cette forme quand c'est possible.

La structure de donne'es Pvecteur est importe'e de la bibliothe`que d'alge`bre
line'aire en nombres entiers du CAII.
}

\subsection{Set of Affine Constraints}
\label{subsection-psysteme}
\index{Psysteme}

\domain{External Psysteme}
{
Le domaine {\tt Psysteme} est utilise' pour repre'senter les syste`mes
d'e'quations et d'ine'quations line'aires qui apparaissent lors
de la phase d'analyse se'mantique (voir le domaine {\tt predicate}).

Les Psystemes sont aussi implicitement utilise's pour effectuer le
calcul de de'pendance.

Un Psysteme est forme' de six champs:
\begin{itemize}
  \item une liste d'e'galite's,
  \item le nombre des e'galite's,
  \item une liste d'ine'galite's,
  \item le nombre d'ine'galite's,
  \item la dimension de l'espace de re'fe'rence,
  \item une base de l'espace de re'fe'rence.
\end{itemize}

Comme le domaine Pvecteur, la structure de donne'es Psysteme est
importe'e de la bibliothe`que d'alge`bre line'aire en nombres entiers du
CAII.  }

\section{Entities: Variables, Functions, Operators, Constants, Labels...}
\label{entity}

\subsection{Entity}
\label{subsection-entity}
\index{Entity}

\domain{tabulated entity = name:string x type x storage x initial:value}
\domain{entity_int = entity->int}
{
Tout objet ayant un nom dans un programme Fortran est repre'sente' par
une \verb/entity/. Un tel objet peut e^tre un module, une variable, un
common, un ope'rateur, une constante, un label, etc. Pour chaque objet,
le sous-domaine \verb/name/ de l'entite' donne le nom de l'objet tel
qu'il apparai^t dans le texte source du programme pre'fixe' par le nom du
package dans lequel l'entite' est de'clare'e, le sous-domaine
\verb/type/ donne le type de l'entite', le sous-domaine \verb/storage/
le genre d'allocation me'moire utilise' pour l'entite', et finalement,
le sous-domaine \verb/initial/ donne la valeur initiale, si elle est
connue, de l'entite'. Le terme valeur initiale a ici un sens assez
large, puisqu'il s'agit par exemple du code pour les entite's
repre'sentant des modules.
}

\subsection{Type}
\label{subsection-type}
\index{Type}

\domain{Type = statement:unit + area + variable + functional + unknown:unit + void:unit}
{
Le domaine \verb/type/ repre'sente le type d'une entite'.  Le
sous-domaine \verb/statement/ est utilise' pour les labels
d'instruction.  Le sous-domaine \verb/area/ est utilise' pour les
commons.  Le sous-domaine \verb/variable/ est utilise' pour toutes les
variables, y compris les parame`tres formels et le re'sultat d'une
fonction.  Le sous-domaine \verb/functional/ est utilise' pour les
fonctions, pour les subroutines et pour le programme principal.  Le
sous-domaine \verb/void/ est utilise' pour le re'sultat d'une subroutine
ou d'un programme principal.
}

\subsubsection{Area Type}
\label{subsubsection-area}
\index{Area}

\domain{Area = size:int x layout:entity*}
{ Le domaine {\tt area} est utilise' pour repre'senter les aires de
stockage des variables telles que les commons et les aires statiques ou
dynamiques. Le sous-domaine {\tt size} donne la taille de l'aire
exprime'e en octets ({\em character storage unit} de la norme ANSI
X3.9-1978, \S~2.13), et le sous-domaine {\tt layout} donne la liste des
entite's stocke'es dans cette aire. L'ordre des de'clarations des
variables est respecte', ce qui permettrait de reproduire des programmes
sources fide`les a` ce qu'e'taient les programmes initiaux\footnote{Les
de'clarations sont en fait conserve'es sous forme textuelle pour
garantir une fide'lite' absolue.}.}

\subsubsection{Variable Type}
\label{subsubsection-variable}
\index{Variable}

\domain{Variable = basic x dimensions:dimension*}
{
Le domaine \verb/variable/ repre'sente le type d'une variable.  Le
sous-domaine \verb/basic/ donne le type Fortran de la variable.  Le
sous-domaine \verb/dimensions/ donne la liste des dimensions de la variable.
Un scalaire est un tableau de ze'ro dimension.

Chaque dimension est une expression, qui n'est pas ne'cessairement
constante dans le cas des tableaux formels. La constante pre'de'finie de
nom '*D*' est utilise'e pour les tableaux de taille non de'finie
(\verb/DIMENSION T(*)/).
}

\subsubsection{Basic Type}
\label{subsubsection-basic}
\index{Basic}

\domain{Basic = int:int + float:int + logical:int + overloaded:unit + complex:int + string:value}
{
Le domaine \verb/basic/ permet de repre'senter un type Fortran tel que
INTEGER ou REAL. La valeur de ce domaine donne la longueur en octets de
la zone me'moire occupe'e par une variable de ce type.
}

\subsubsection{Dimension}
\label{subsubsection-dimension}
\index{Dimension}

\domain{Dimension = lower:expression x upper:expression}
{
Le domaine \verb/dimension/ permet de repre'senter une dimension d'un
tableau, c'est-a`-dire un couple borne infe'rieure -- sous-domaine
\verb/lower/ -- borne supe'rieure -- sous-domaine \verb/upper/.
}

\subsubsection{Functional Type}
\label{subsubsection-functional}
\index{Functional}

\domain{Functional = parameters:parameter* x result:type}
{ Le domaine \verb/functional/ repre'sente le type d'un module,
c'est-a`-dire une fonction, une subroutine ou un programme principal. Le
sous-domaine \verb/parameters/ donne le type et le mode de passage de
chaque parame`tre, et le sous-domaine \verb/result/ donne le type du
re'sultat. Ce dernier type vaut \verb/void/ pour les subroutines et les
programmes principaux.

Il n'y a pas de moyens simples pour repre'senter les fonctions ou
sous-programmes a` nombre variable de parame`tres. Bien que ce soit
interdit pour les modules de'finis par le programmeur, de nombreux
intrinse`ques comme \verb+MIN0+ ou \verb+WRITE+ n'ont pas un profil
unique.  }

\subsubsection{Parameter Type and Mode}
\label{subsubsection-parameter}
\index{Parameter}

\domain{Parameter = type x mode}
{
Le domaine \verb/parameter/ repre'sente le type et le mode de passage d'un
parame`tre formel de module. 
}

\index{Mode}
\domain{Mode = value:unit + reference:unit}
{
Le domaine \verb/mode/ repre'sente le mode de passage d'un parame`tre
formel de module. Le domaine contient un objet du domaine \verb/value/
pour le mode de passage par valeur et \verb/reference/ pour le passage
par adresse.
}

\subsection{Storage}
\label{subsection-storage}
\index{Storage}

\domain{Storage = return:entity + ram + formal + rom:unit}
{
Le domaine \verb/storage/ permet de pre'ciser dans quelle zone de la
me'moire est stocke'e une entite'. Il y a plusieurs zones, qui ne
correspondent pas ne'cessairement a` la re'alite', c'est-a`-dire aux
zones de me'moire qui seraient affecte'es par un compilateur.

Le sous-domaine \verb/return/ permet de repre'senter les variables ayant
pour nom le nom d'une fonction et auxquelles on affecte la valeur que la
fonction doit retourner. L'entite' pointe'e par \verb/return/ est la
fonction concerne'e.

Le sous-domaine \verb/ram/ est reserve' aux variables ayant une adresse
en me'moire. Il permet de pre'ciser dans quelle fonction et
e'ventuellement dans quel common ces variables ont e'te' de'clare'es.

Le sous-domaine \verb/formal/ est re'serve' aux parame`tres formels des
modules.

Le sous-domaine \verb/rom/ est utilise' pour toutes les entite's dont la
valeur n'est pas modifiable, telles que les fonctions, les labels, les
ope'rateurs, etc.
}

\subsubsection{RAM Storage}
\label{subsubsection-ram}
\index{RAM}

\domain{Ram = function:entity x section:entity x offset:int x shared:entity*}
{
Le domaine \verb/ram/ permet de pre'ciser la de'claration d'une
variable. Le sous-domaine \verb/function/ indique dans quel module une
entite' est de'clare'e. Le sous-domaine \verb/section/ indique dans
quelle aire une entite' est stocke'e; il y a une aire par common
de'clare' et deux aires spe'ciales nomme'es \verb/STATIC/ et
\verb/DYNAMIC/ pour les entite's locales. Le sous-domaine \verb/offset/
donne l'adresse dans l'aire de la variable. Enfin, le sous-domaine {\tt
shared} donne la liste des entite's qui partagent statiquement un
morceau d'espace me'moire avec la variable concerne'. En Fortran, le
partage de me'moire vient des equivalences entre variables.  }

\subsubsection{Formal Parameter Storage}
\label{subsubsection-formal}
\index{Formal}

\domain{Formal = function:entity x offset:int}
{
Le domaine \verb/formal/ indique le module dans lequel un parame`tre formel
est de'clare' gra^ce au sous-domaine \verb/function/, et le rang de ce
parame`tre dans la liste des parame`tres gra^ce au sous-domaine
\verb/offset/.
Le premier parame`tre a un rang de 1 et non de 0.
}

\subsection{Entity Value}
\label{subsection-value}
\index{Value}

\domain{Value = code + symbolic + constant + intrinsic:unit + unknown:unit}
{
Le domaine \verb/value/ permet de repre'senter les
valeurs initiales des entite's. Le sous-domaine \verb/code/ est utilise'
pour les entite's modules. Le sous-domaine \verb/symbolic/ est utilise'
pour les entite's constantes symboliques. Le sous-domaine
\verb/constant/ est utilise' pour les entite's constantes. Le
sous-domaine \verb/intrinsic/ est utilise' pour toutes les entite's qui
ne de'pendent que du langage, telles que les intrinsics Fortran, les
ope'rateurs, les instructions, etc. Enfin le sous-domaine
\verb/unknown/ est utilise' pour les valeurs initiales inconnues.

Additional value kinds would be necessary to encode the initial value of
an area, if the overloading of the \verb+unknown+ kind becomes a
problem. Pierre Jouvelot suggested to give \verb+COMMON+ themselves as
initial value since a common represented an address.
}

\subsubsection{Symbolic Value}
\label{subsubsection-symbolic}
\index{Symbolic}

\domain{Symbolic = expression x constant}
{
Le domaine \verb/symbolic/ est utilise' pour repre'senter la valeur
initiale d'une entite' constante symbolique, c'est-a`-dire les PARAMETER
de Fortran ou les CONST de Pascal. Le sous-domaine \verb/expression/
permet de stocker l'expression qui a permis d'e'valuer la valeur
initiale contenue dans le sous-domaine \verb/constant/. Le sous-domaine
\verb/expression/ n'est utile qui si on cherche a` reproduire un texte
source fide`le.
}

\subsubsection{Constant Value}
\label{subsubsection-constant}
\index{Constant}

\domain{Constant = int + litteral:unit}
{
Le domaine \verb/constant/ est utilise' pour repre'senter la valeur
initiale des entite's constantes. Seules les entite's de type entier
nous inte'ressent, ce qui explique qu'une constante puisse e^tre soit un
\verb/int/ soit un \verb/litteral/ dont on ne garde pas la valeur (type unit).
}

\section{Code, Statements and Instructions}
\label{code}

\subsection{Module Code}
\label{subsection-code}
\index{Code}\index{Declarations}\index{Decls text}

\domain{Code = declarations:entity* x decls\_text:string}
{ 
Le domaine \verb/code/ est utilise'
pour stocker le corps des modules. Le sous-domaine \verb/declarations/
contient une liste d'entite's qui sont les variables locales,
parame`ters formels et commons de'clare's dans la fonction.

Le sous-domaine {\tt decls\_text} contient le texte exact de toutes les
de'clarations du module; ce texte est utilise' par de'faut par le
prettyprinter tant qu'il existe. Quand le code a e'te' fortement
transforme', le prettyprinter re'ge'ne`re des de'clarations
synthe'tiques.
}

\subsection{Callees}
\label{subsection-callees}
\index{Callees}

% Should be put somewhere else!

\domain{Callees = callees:string*}
{ Le domaine {\tt callees} sert a` porter des informations
interproce'durales, et sera enrichi dans le futur.  Le sous-domaine {\tt
callees} contient la liste des noms des sous-programmes et fonctions
directement appele's dans le code. Il contient une partie du callgraph.
}

\subsection{Statement}
\label{subsection-statement}
\index{Statement}

\domain{Statement = label:entity x number:int x ordering:int x comments:string x instruction}
{ 
Le
domaine \verb/statement/ permet de repe'rer les instructions d'un
module.  Le sous-domaine \verb/label/ contient une entite' qui de'finit
le label\footnote{Un statement dont l'instruction est un bloc n'a
jamais de label.}.

Le sous-domaine \verb/number/ contient un nume'ro permettant
de repe'rer le statement pour le debugging ou l'information de
l'utilisateur (valeur par defaut:
\verb+STATEMENT\_NUMBER\_UNDEFINED+). Ce nume'ro est de'fini par
l'utilisateur et n'est (en princpe) jamais modifie'. Apre`s de'roulage
de boucle, plusieurs \verb/statement/s peuvent avoir le me^me nume'ro.

Le sous-domaine \verb/ordering/ contient un entier caracte'ristique du
statement; il est forme' de la concate'nation du nume'ro de composante
dans le graphe de contro^le et du nume'ro de statement dans dans cette
composante; cet entier sert a` comparer l'ordre lexical des statements
et sa valeur par defaut est \verb+STATEMENT\_ORDERING\_UNDEFINED+. Il
est automatiquement recalcule' apre`s chaque transformation de
programme. Il est aussi utilise' comme nom absolu d'un \verb/statement/
quand des structures de donne'es comme les tables de hash-code sont
e'crites sur disque ou relues.

Le sous-domaine {\tt comments} contient le texte du commentaire associe'
a ce statement dans le programme initial; ce texte est utilise' par le
prettyprinter. Ce sont les commentaires qui pre'ce`dent le statement
qui s'y trouvent associe's. En l'absence de commentaires, ce champ
prend la valeur \verb/string\_undefined/.

Le sous-domaine \verb/instruction/ contient
l'instruction proprement dite.
}

\domain{persistant_statement_to_statement = persistant statement -> persistant statement}
{
  Used for example in use_def_elimination() to store the eventual
  statement father of a statement. The persistance is needed to avoid
  freeing the statements when the mapping is freed.
}

\subsection{Instruction}
\label{subsection-instruction}
\index{Instruction}

\domain{Instruction = block:statement* + test + loop + goto:statement + call + unstructured}
{ 
Le domaine \verb/instruction/ permet de repre'senter les instructions
d'un module. Une instruction peut e^tre un sous-domaine \verb/block/,
c'est-a`-dire une liste de \verb/statement/, un sous-domaine \verb/test/
pour les instructions de test, un sous-domaine \verb/loop/ pour les
boucles se'quentielles, un sous-domaine \verb/goto/ pour les goto qui
contient le
\verb/statement/ vers lequel le goto se branche, un sous-domaine
\verb/call/ pour toutes les autres instructions (affectation, appel de
subroutine, entre'es-sorties, return, stop, etc) ou un sous-domaine de
\verb/unstructured/ dans le cas ou` l'on traite d'un graphe de contro^le
structure'. Toutes ces instructions 
sont repre'sente'es par des appels a` des fonctions pre'de'finies dont
nous e'tudierons la nature plus loin.
}

\subsubsection{Control Flow Graph (a.k.a. Unstructured)}
\label{subsubsection-unstructured}
\index{Unstructured}

\domain{Unstructured = control x exit:control}
{

Domain \verb/unstructured/ is used to represent unstructured parts of
the code in a structured manner which as a unique statement. The entry
node of the underlying CFG is in field \verb/control/, and the
unique exit node is in field \verb/exit/. The exit node should not be
modified by users of the unstructured\footnote{FI: I do not understand
why...}.  See Figure~\ref{figure-unstructured}.

The hierarchical structure is induced by the recursive nature of
statements.  Each control node points towards a statement which can also
contain an unstructured area of the code as well as structured
part. Unstructured parts of the code can thus be contained as much as
possible as well as be recursively decomposed.

For instance, the two DO loops in:
\begin{verbatim}
      DO 200 I = 1, N
100      CONTINUE
         DO 300 J = 1, M
            T(J) = T(J) + X
300      CONTINUE
         IF(X.GT.T(I)) GO TO 100
200   CONTINUE
\end{verbatim}
are preserved as DO loops in spite of the GO~TO statement (see
Figure~\ref{figure-hierarchical-control-flow-graph}).

\begin{figure}

\begin{center}

\unitlength 3pt

\begin{picture}(90,105)(0,0)
\put(40,70){\circle*{3}}
\put(50,70){DO 200}

\put(30,35){\circle*{3}}
\put(20,40){IF}
\put(40,45){\circle*{3}}
\put(50,45){100 CONTINUE}
\put(50,35){\circle*{3}}
\put(60,35){DO 300}

\put(50,10){\circle*{3}}
\put(60,10){T(J) = T(J) + X}

% link between structured and unstructured parts
\put(50,30){\line(0,-1){20}}
\put(40,60){\line(0,-1){15}}
\put(30,37){\line(1,3){8}}
\put(50,37){\line(-1,3){8}}

% control edges
\thicklines
\put(30,35){\vector(-1,0){15}}
\put(31,36){\vector(1,1){8}}
\put(41,44){\vector(1,-1){8}}
\put(50,35){\vector(-1,0){18}}
\thinlines

% Draw the planes
\multiput(0,0)(0,30){3}{\line(1,1){25}}
\multiput(0,0)(0,30){3}{\line(1,0){105}}
\end{picture}
\end{center}
\caption{Hierarchical Control Flow Graph}
\label{figure-hierarchical-control-flow-graph}
\end{figure}

}

\subsubsection{Conditional (a.k.a. Test)}
\label{subsubsection-test}
\index{Test}

\domain{Test = condition:expression x true:statement x false:statement}
{
Le domaine \verb/test/ permet de repre'senter toutes les instructions a` base
de contro^le. Le sous-domaine \verb/condition/ contient l'expression a`
tester, et les deux sous-domaines \verb/true/ et \verb/false/ contiennent les
instructions a` exe'cuter selon la valeur du test. 

Il faut noter que chaque instruction de contro^le de Fortran,
a` l'exception de l'instruction \verb/DO/, est
transforme'e en une combinaison se'mantiquement e'quivalente de \verb/test/s
et de \verb/goto/s.
}

\subsubsection{DO Loop, Sequential or Parallel}
\label{subsubsection-loop}
\index{Loop}

\domain{Loop = index:entity x range x body:statement x label:entity x execution x locals:entity*}
{
Le domaine \verb/loop/ permet de repre'senter les boucles du type DO Fortran
ou FOR Pascal. Le sous-domaine \verb/index/ contient l'entite' indice de
boucle, le sous-domaine \verb/range/ contient les bornes de la boucle, le
sous-domaine \verb/body/ contient le corps de la boucle, c'est-a`-dire un
\verb/statement/, le sous-domaine \verb/label/ contient le label de fin de boucle,
c'est-a`-dire une entite'. Le sous-domaine \verb/execution/ de'finit le
comportement dynamique d'une boucle. Les entite's pre'sentes dans
\verb/locals/ sont propres au corps de boucle (les effets sur elles sont
masque's quand on sort de la boucle).
}

\index{Execution}
\domain{Execution = sequential:unit + parallel:unit}
{
Le domain \verb/execution/ de'finit la se'mantique d'une boucle:
\verb/sequential/ correspond a` une boucle DO classique, \verb/parallel/
de'finit un boucle dont les instances d'ite'ration peuvent e^tre
exe'cute'es en paralle`le.
}

\label{range}
\index{Range}
\domain{Range = lower:expression x upper:expression x increment:expression}
{
Le domaine \verb/range/ permet de repre'senter les bornes des boucles DO
Fortran. Il y a trois sous-domaines \verb/lower/, \verb/upper/ et \verb/increment/ de
type \verb/expression/ qui sont respectivement la borne infe'rieure, la borne
supe'rieure et l'incre'ment.
}

\subsubsection{Function Call}
\label{subsubsection-call}
\index{Call}

\domain{Call = function:entity x arguments:expression*}
{

Le domaine \verb/call/ permet de repre'senter les commandes et les
appels de fonctions Fortran sous une forme unique pseudo-fonctionelle.
Ces pseudo-fonctions jouent un ro^le important dans notre
repre'sentation interme'diaire puisque les constantes, les ope'rateurs
comme + et *, les intrinse`ques comme {\tt MOD} ou {\tt SIN} et surtout
les commandes (i.e. instructions) Fortran, a` commencer par
l'assignation et en continuant avec {\tt READ, WRITE, RETURN, CALL}
etc..., sont repre'sente'es comme les appels de fonctions de'finies par
l'utilisateur. Le nombre d'arguments de chaque pseudo-fonction varie: 0
pour les constantes, 1 ou 2 pour les ope'rateurs, etc. Les commandes
Fortran, les ope'rateurs et les intrinse`ques sont caracte'rise's par
des pseudo-fonctions pre'de'finies. Cette convention permet de diminuer
conside'rablement la taille de la de'finition de la repre'sentation
ainsi que le volume de code ne'cessaire a` de nombreux algorithmes.

Le sous-domaine \verb/function/ est une entite' qui de'finit la fonction
appele'e. Le sous-domaine \verb/arguments/ est une liste de sous-domaines
\verb/expression/ qui repre'sente les arguments d'appel de la fonction.
}

\subsubsection{Control Flow Graph (cont.)}
\label{subsubsection-control}
\index{Control}

\domain{Control = statement x predecessors:control* x successors:control*}
\domain{Controlmap = persistant statement->control}
{ 

Domain \verb/control/ is the type of nodes used to implement the CFG
implied by an unstructured instruction (see Domain
\verb/unstructured/). Each node points towards a statement which can
represent an arbitrary large piece of structured code. GOTO statements
are eliminated and represented by arcs. Nodes are doubly linked. Each
node points towards its successors (at most 2) and towards its
predecessors. The hierarchical nature of
domain \verb/statement/ is used to hide local branches from higher and
lower level pieces of code. The whole unstructured area of the code is
seen as a unique atomic statement from above, and is entirely ignored
from under. This explains the mutual recursion between \verb/control/
and \verb/statement/ (via \verb/instruction/).

\begin{figure}
\begin{center}
\mbox{\psfig{file=unstructured.idraw,width=\hsize}}
\end{center}
\caption{Control Flow Graph}
\label{figure-unstructured}
\end{figure}

All statements but \verb/test/have only one successor. The first
successor of \verb/test/ is the successor when the test condition is
evaluated to true.  And the other way round for the second one. The exit
node (see domain \verb/unstructured/) has no successor. The entry node
as well as all other nodes may have an unlimited number of precedessors.

All nodes of a CFG can be visited in a meaningless order using the
\verb/CONTROL_MAP/ macros. Look for an example in library \verb/control/
because auxiliary data structures must be decraled and freed.

}

\domain{persistant_statement_to_control = persistant statement -> persistant control}
{
  Used for example in use_def_elimination() to store the eventual control
  father of a statement in order to travel on the control graph
  associated to a statement. The persistance is needed to avoid
  freeing the control graph when the mapping is freed.
}

% \section{Repre'sentation des expressions}
\section{Expressions}
\label{expression}
\index{Expression}

\domain{Expression = syntax x normalized}
{ 
Le domaine \verb/expression/ permet de stocker les expressions.  Le
sous-domaine {\tt syntax} contient la description de l'expression telle
qu'elle apparai^t dans le texte source du programme. Le sous-domaine {\tt
normalized} contient une forme compile'e des expressions line'aires,
sous forme de Pvecteur.

Si le sous-domaine {\tt normalized} contient la valeur
{\tt normalized\_undefined}, cela signifie que la fonction de line'arisation
n'a pas e'te' appele'e pour cette expression; cela {\bf ne} signifie
{\bf pas}
que l'expression n'est pas line'aire.
}

\subsection{Abstract Tree of an Expression: Syntax}
\label{subsection-syntax}
\index{Syntax}

\domain{Syntax = reference + range + call}
{
Le domaine \verb/syntax/ permet de repre'senter les expressions telles
qu'elles apparaissent dans le texte source du programme. Un
\verb/syntax/ est soit une \verb/reference/ a` un e'le'ment de tableau
(on rappelle que les scalaires sont des tableaux a` 0 dimension) , soit
un \verb/call/ a` une fonction (les ope'rateurs sont repre'sente's par
des fonctions pre'-de'finies, y compris l'assignation; c'est pourquoi le
domaine {\tt call} est de'fini dans la section {\em Instructions}), soit
un \verb/range/, dans le cas des expressions bornes de boucles (le domaine
{\tt range} est pre'sente' au niveau des boucles, aussi dans la section
{\em Instructions}).
}

\subsubsection{Reference}
\label{subsubsection-reference}
\index{reference}

\domain{Reference = variable:entity x indices:expression*}
{
Le domaine \verb/reference/ est utilise' pour repre'senter une
re'fe'rence a` un e'le'ment de tableau\footnote{Les variables scalaires
e'tant repre'sente'es par des tableaux de dimension 0, les re'fe'rences
a des scalaires sont aussi prises en compte. Elles contiennent une liste
vide d'expressions d'indices.}.  Le sous-domaine \verb/variable/
contient une entite' de'finissant la variable re'fe'rence'e. Le
sous-domaine \verb/indices/ contient une liste {\tt expression}s qui sont les
indices de la re'fe'rence.
}

\subsubsection{Range}

See Section~\ref{range}.

\subsubsection{Function Call}

All operators, including assignment, are repreented as function calls
with side effect. See Section~\ref{subsubsection-call}.

\subsection{Affine Representation of an Expression}
\label{subsection-normalized}
\index{Normalized}

\domain{Normalized = linear:Pvecteur + complex:unit}
{ Le domaine {\tt normalized} permet de savoir si une expression est une
expression line'aire construite sur les variables simples entie`res
(sous-domaine {\tt linear}) ou non (sous-domaine {\tt complex}).

Le sous-domaine {\tt complex} est utilise' si l'expression n'est pas
line'aire (ex: {\tt I*J+4}) ou si elle est line'aire mais contient autre
chose que des re'fe'rences a` des scalaires entiers (ex: {\tt T(I-1) +
T(I) + T(I+1)}).

La forme normalise'e n'existe pas si l'expression n'a pas encore e'te'
examine'e. }

% \section{Effets des instructions}
\section{Memory Effects of Statements}
\label{effects}

\subsection{Effects}
\label{subsection-effects}
\index{Effects}

\domain{Effects =  effects:effect*}
{}

\subsection{Effect}
\label{subsection-effect}
\index{Effect}\index{Region}

\domain{Effect =  persistant reference x action x approximation x context:transformer}
{
Le domaine \verb/effect/ est utilise' pour repre'senter les effets d'un
statement sur les variables du module. Les effets des instructions sont
le point de de'part du calcul des Def-Use Chains, des de'pendances, des
Summary Data Flow Information, et des re'gions.

Le sous-domaine {\tt reference} indique sur quelle variable, scalaire ou
tableau, a lieu un effet, le sous-domaine {\tt action} pre'cise si la
re'fe'rence est lue ou e'crite, et le sous-domaine {\tt approximation}
permet de savoir si la re'fe'rence (pour les effets) ou l'ensemble des
e'le'ments de tableaux de'fini par le contexte (pour les re'gions) est lu ou
e'crit a` coup su^r ou non. 

Lors de la traduction interproce'durale des effets ou des re'gions portant
sur une variable globale non de'clare'e dans le module appelant ou sur une
variable statique apparaissant dans un {\tt SAVE}, il est important de
re'cupe'rer une re'fe'rence qui sera la me^me pour tous les sites d'appels.
Pour cela on choisit un module de re'fe'rence (dans lequel le common ou le
save est de'clare'), dont le nom pre'fixera le nom de la variable. Le choix
de ce module n'est pas trivial. A` l'heure actuel. il s'agit du nom du
module de la premie`re variable apparaissant dans la liste des variables du
common ({\tt ram_section(storage_ram(entity_storage(<my_common>)))}). Selon
l'ordre dans lequel les modules dont analyse's, le nom de module trouve'
sera diffe'rent. Il faudrait en re'alite' prendre le premier nom de module
dans l'ordre lexicographique des modules appele's (callees).

Le sous-domaine {\tt reference} permet de pre'ciser qu'un effet n'a lieu que
sur un sous-tableau en utilisant un {\tt range} comme expression d'indice.
Ceci est utilise' lors de la traduction des effets cumule's d'un proce'dure
en les effets propres d'un call site. Pour les re'gions, ce sous-domaine
pre'cise l'entite' concerne'e ainsi que la liste des variables $\phi$ qui
de'crivent ses dimensions.

Le sous-domaine {\tt context} n'est utilise' que pour repre'senter les
effets des instructions par des re'gions (telles que de'finies par Re'mi
Triolet).

}

\subsection{Nature of an Effect}
\label{subsection-action}
\index{Action}

\domain{Action = read:unit + write:unit}
{
Deux types d'effets sont utilise's dans les conditions de Bernstein
et dans les conditions propres a` chaque transformation de programme:
la lecture d'une variable et son e'criture.
}

\subsection{Approximation of an Effect}
\label{subsection-approximation}
\index{Approximation}

\domain{Approximation = may:unit + must:unit}
{
La pre'sence de tests et boucles ne permet pas de de'terminer en ge'ne'ral
si une variable est effectivement lue ou e'crite lors de l'exe'cution
d'un {\tt statement}. Il se peut me^me que certaines exe'cutions
y acce`dent et que d'autres n'y fassent pas re'fe'rence. Les effets
calcule's sont alors de type {\tt may}.

Dans quelques cas particuliers, comme une affectation simple {\tt I = 2},
l'effet est certain ({\em must}). Il peut alors e^tre utilise' dans
le calcul des {\em use-def chains} pour effectuer un {\em kill} sur les
variables scalaires. Un effet {\em must} sur un tableau ne signifie pas
que tout le tableau est lu ou e'crit mais qu'au moins un de ses
e'le'ments l'est.
}

\section{Analyse se'mantique}
\label{semantics}

\subsection{Transformer}
\label{subsection-transformer}
\index{Transformer}

\domain{Transformer = arguments:entity* x relation:predicate}
{
Le domaine {\tt transformer} de'finit une relation entre deux e'tats
me'moire. Cette relation
porte sur les valeurs des variables scalaires entie`res d'un module ou
des variables globales au programme.

Les variables qui apparaissent dans la liste des arguments sont celles
qui ont e'te' modifie'es entre les deux e'tats. Deux valeurs
sont donc associe'es a` chacune d'entre elles: la pre- et la
post-valeur.  Les post-valeurs sont porte'es par les entite's
elles-me^mes. Les pre'-valeurs sont porte'es par des entite's
spe'ciales. Les variables scalaires entie`res qui ne sont pas modifie'es
et qui n'apparaissent donc pas dans la liste des arguments n'ont qu'une
seule valeur, porte'e par l'entite' correspondant a` la variable.

La relation est de'finie par des e'galite's et des ine'galite's
line'aires entre valeurs.

Deux types de transformers sont utilise's. Le premier est propre a` un
{\tt statement} et donne une abstraction de son effet sur les variables
entie`res. Les variables qui apparaissent dans la liste des arguments
sont celles qui sont affecte'es lors de son exe'cution.  Le second,
aussi associe' a` un {\tt statement}, donne une relation entre l'e'tat
initial d'un module et l'e'tat pre'ce'dent l'exe'cution de ce {\tt statement}.

Les transformers ne sont de'finis qu'apre`s une phase d'analyse se'mantique.
}

\subsection{Predicate}
\label{subsection-predicate}
\index{Predicate}\index{Precondition}\index{Transformer}

\domain{Predicate = system:Psysteme}
{
Le domaine {\tt predicate} de'finit une relation entre valeurs de
variables scalaires entie`res. Son interpre'tation est fonction de
son utilisation. Il peut s'agir soit d'un pre'dicat valable en
un point du programme (i.e. un invariant), soit d'un pre'dicat
valable entre deux points du programme. Il s'agit alors d'une
abstraction d'une commande, c'est-a`-dire d'un {\tt transformer}.
}

\newpage

\section*{Annexe: re'capitulatif ri.newgen}

\begin{verbatim}
--         --------------------------------------------------------
--         --------------------------------------------------------
--
--                                  WARNING
--
--                THIS FILE HAS BEEN AUTOMATICALLY GENERATED
--
--                             DO NOT MODIFY IT
--
--         --------------------------------------------------------
--         --------------------------------------------------------

-- Imported domains
-- ----------------

-- External domains
-- ----------------
external Psysteme ;
external Pvecteur ;

-- Domains
-- -------
action = read:unit + write:unit ;
approximation = may:unit + must:unit ;
area = size:int x layout:entity* ;
basic = int:int + float:int + logical:int + overloaded:unit + complex:int + string:value ;
call = function:entity x arguments:expression* ;
callees = callees:string* ;
code = declarations:entity* x decls_text:string ;
constant = int + litteral:unit ;
control = statement x predecessors:control* x successors:control* ;
dimension = lower:expression x upper:expression ;
effect =  persistant reference x action x approximation x context:transformer ;
effects =  effects:effect* ;
execution = sequential:unit + parallel:unit ;
expression = syntax x normalized ;
formal = function:entity x offset:int ;
functional = parameters:parameter* x result:type ;
instruction = block:statement* + test + loop + goto:statement + call + unstructured ;
loop = index:entity x range x body:statement x label:entity x execution x locals:entity* ;
mode = value:unit + reference:unit ;
normalized = linear:Pvecteur + complex:unit ;
parameter = type x mode ;
predicate = system:Psysteme ;
ram = function:entity x section:entity x offset:int x shared:entity* ;
range = lower:expression x upper:expression x increment:expression ;
reference = variable:entity x indices:expression* ;
statement = label:entity x number:int x ordering:int x comments:string x instruction ;
storage = return:entity + ram + formal + rom:unit ;
symbolic = expression x constant ;
syntax = reference + range + call ;
tabulated entity = name:string x type x storage x initial:value ;
test = condition:expression x true:statement x false:statement ;
transformer = arguments:entity* x relation:predicate ;
type = statement:unit + area + variable + functional + unknown:unit + void:unit ;
unstructured = control x exit:control ;
value = code + symbolic + constant + intrinsic:unit + unknown:unit ;
variable = basic x dimensions:dimension* ;
\end{verbatim}

\newpage

% Cross-references for points and keywords

\documentstyle[a4,psfig]{article}

% \input{/usr/share/local/lib/tex/macroslocales/Dimensions.tex}

\title{PIPS: Internal Representation of Fortran Code}
\author{Franc,ois Irigoin \\
        Pierre Jouvelot \\
    Ronan Keryell \\
        Re'mi Triolet\\
\\
        CRI, Ecole des Mines de Paris}

                                %\newcommand{\domain}[2]{\paragraph{{#1}}\paragraph{}{{#2}}}
      {
        \catcode `==\active
        \gdef\domain{\medskip\par\noindent
          \bgroup
          \catcode `_ \other
          \catcode `= \active
          \def={\em{\rm \string=}}
          \tt\em\vraidomain}
        \gdef\vraidomain#1{#1\egroup\medskip\par}
        }

% Correction de quelques erreurs. RK, 07/02/1994.

\renewcommand{\indexname}{Index}

\makeindex

\begin{document}
\maketitle
\sloppy

\section*{Introduction}

Ce document est utilise' directement et automatiquement par l'outil
de ge'nie logiciel NewGen pour ge'ne'rer les de'clarations des structures
de donne'es utilise'es dans le projet PIPS, ainsi que les routines de
base qui les manipulent. C'est pourquoi l'ordre des sections n'est pas
ne'cessairement naturel.

Apre`s une description rapide des structures de donne'es externes a` la
repre'sentation interne de Pips, nous pre'sentons successivement les
notions d'entite's, de code et d'expressions et la manie`re dont elles
sont utilise'es pour encoder un programme Fortran. Nous de'taillons
ensuite deux structures de donne'es supple'mentaires, les effets et les
{\em transformers} qui sont utilise's pour le calcul de de'pendance
interproce'dural et pour l'analyse syntaxique interproce'durale.

La manie`re dont les constructions de Fortran sont repre'sente'es est
de'crite dans une des sections du rapport EMP-CAII-E105. NewGen est
pre'sente' dans le rapport EMP-CRI-A191.

\section{External Data Structures}
\label{external}

\subsection{Vector}
\label{subsection-pvecteur}
\index{Pvecteur}

\domain{External Pvecteur}
{ Le domaine {\tt Pvecteur} est utilise' pour repre'senter les
expressions line'aires telles que {\tt 3I+2} (voir le domaine {\tt
normalized}) ou des contraintes line'aires telles que {\tt 3I + J <= 2}
ou {\tt 3I == J}. Ces contraintes sont utilise'es dans les syste`mes
line'aires (voir le domaine {\tt Psysteme}).

Un Pvecteur est une suite de mono^mes, un mono^me e'tant un couple
(coefficient,variable).  Le coefficient d'un tel couple est un entier,
positif ou ne'gatif. La variable est une entite', sauf dans le cas du
terme constant qui est repre'sente' par la variable pre'de'finie de nom
{\tt TCST}\footnote{Comme on rajoute le nom des modules devant les
noms de variables, il ne peut pas y avoir de conflict avec une
e'ventuelle variable {\tt TCST}.}.

Les expressions apparaissant dans le programme analyse' sont mises
sous cette forme quand c'est possible.

La structure de donne'es Pvecteur est importe'e de la bibliothe`que d'alge`bre
line'aire en nombres entiers du CAII.
}

\subsection{Set of Affine Constraints}
\label{subsection-psysteme}
\index{Psysteme}

\domain{External Psysteme}
{
Le domaine {\tt Psysteme} est utilise' pour repre'senter les syste`mes
d'e'quations et d'ine'quations line'aires qui apparaissent lors
de la phase d'analyse se'mantique (voir le domaine {\tt predicate}).

Les Psystemes sont aussi implicitement utilise's pour effectuer le
calcul de de'pendance.

Un Psysteme est forme' de six champs:
\begin{itemize}
  \item une liste d'e'galite's,
  \item le nombre des e'galite's,
  \item une liste d'ine'galite's,
  \item le nombre d'ine'galite's,
  \item la dimension de l'espace de re'fe'rence,
  \item une base de l'espace de re'fe'rence.
\end{itemize}

Comme le domaine Pvecteur, la structure de donne'es Psysteme est
importe'e de la bibliothe`que d'alge`bre line'aire en nombres entiers du
CAII.  }

\section{Entities: Variables, Functions, Operators, Constants, Labels...}
\label{entity}

\subsection{Entity}
\label{subsection-entity}
\index{Entity}

\domain{tabulated entity = name:string x type x storage x initial:value}
\domain{entity_int = entity->int}
{
Tout objet ayant un nom dans un programme Fortran est repre'sente' par
une \verb/entity/. Un tel objet peut e^tre un module, une variable, un
common, un ope'rateur, une constante, un label, etc. Pour chaque objet,
le sous-domaine \verb/name/ de l'entite' donne le nom de l'objet tel
qu'il apparai^t dans le texte source du programme pre'fixe' par le nom du
package dans lequel l'entite' est de'clare'e, le sous-domaine
\verb/type/ donne le type de l'entite', le sous-domaine \verb/storage/
le genre d'allocation me'moire utilise' pour l'entite', et finalement,
le sous-domaine \verb/initial/ donne la valeur initiale, si elle est
connue, de l'entite'. Le terme valeur initiale a ici un sens assez
large, puisqu'il s'agit par exemple du code pour les entite's
repre'sentant des modules.
}

\subsection{Type}
\label{subsection-type}
\index{Type}

\domain{Type = statement:unit + area + variable + functional + unknown:unit + void:unit}
{
Le domaine \verb/type/ repre'sente le type d'une entite'.  Le
sous-domaine \verb/statement/ est utilise' pour les labels
d'instruction.  Le sous-domaine \verb/area/ est utilise' pour les
commons.  Le sous-domaine \verb/variable/ est utilise' pour toutes les
variables, y compris les parame`tres formels et le re'sultat d'une
fonction.  Le sous-domaine \verb/functional/ est utilise' pour les
fonctions, pour les subroutines et pour le programme principal.  Le
sous-domaine \verb/void/ est utilise' pour le re'sultat d'une subroutine
ou d'un programme principal.
}

\subsubsection{Area Type}
\label{subsubsection-area}
\index{Area}

\domain{Area = size:int x layout:entity*}
{ Le domaine {\tt area} est utilise' pour repre'senter les aires de
stockage des variables telles que les commons et les aires statiques ou
dynamiques. Le sous-domaine {\tt size} donne la taille de l'aire
exprime'e en octets ({\em character storage unit} de la norme ANSI
X3.9-1978, \S~2.13), et le sous-domaine {\tt layout} donne la liste des
entite's stocke'es dans cette aire. L'ordre des de'clarations des
variables est respecte', ce qui permettrait de reproduire des programmes
sources fide`les a` ce qu'e'taient les programmes initiaux\footnote{Les
de'clarations sont en fait conserve'es sous forme textuelle pour
garantir une fide'lite' absolue.}.}

\subsubsection{Variable Type}
\label{subsubsection-variable}
\index{Variable}

\domain{Variable = basic x dimensions:dimension*}
{
Le domaine \verb/variable/ repre'sente le type d'une variable.  Le
sous-domaine \verb/basic/ donne le type Fortran de la variable.  Le
sous-domaine \verb/dimensions/ donne la liste des dimensions de la variable.
Un scalaire est un tableau de ze'ro dimension.

Chaque dimension est une expression, qui n'est pas ne'cessairement
constante dans le cas des tableaux formels. La constante pre'de'finie de
nom '*D*' est utilise'e pour les tableaux de taille non de'finie
(\verb/DIMENSION T(*)/).
}

\subsubsection{Basic Type}
\label{subsubsection-basic}
\index{Basic}

\domain{Basic = int:int + float:int + logical:int + overloaded:unit + complex:int + string:value}
{
Le domaine \verb/basic/ permet de repre'senter un type Fortran tel que
INTEGER ou REAL. La valeur de ce domaine donne la longueur en octets de
la zone me'moire occupe'e par une variable de ce type.
}

\subsubsection{Dimension}
\label{subsubsection-dimension}
\index{Dimension}

\domain{Dimension = lower:expression x upper:expression}
{
Le domaine \verb/dimension/ permet de repre'senter une dimension d'un
tableau, c'est-a`-dire un couple borne infe'rieure -- sous-domaine
\verb/lower/ -- borne supe'rieure -- sous-domaine \verb/upper/.
}

\subsubsection{Functional Type}
\label{subsubsection-functional}
\index{Functional}

\domain{Functional = parameters:parameter* x result:type}
{ Le domaine \verb/functional/ repre'sente le type d'un module,
c'est-a`-dire une fonction, une subroutine ou un programme principal. Le
sous-domaine \verb/parameters/ donne le type et le mode de passage de
chaque parame`tre, et le sous-domaine \verb/result/ donne le type du
re'sultat. Ce dernier type vaut \verb/void/ pour les subroutines et les
programmes principaux.

Il n'y a pas de moyens simples pour repre'senter les fonctions ou
sous-programmes a` nombre variable de parame`tres. Bien que ce soit
interdit pour les modules de'finis par le programmeur, de nombreux
intrinse`ques comme \verb+MIN0+ ou \verb+WRITE+ n'ont pas un profil
unique.  }

\subsubsection{Parameter Type and Mode}
\label{subsubsection-parameter}
\index{Parameter}

\domain{Parameter = type x mode}
{
Le domaine \verb/parameter/ repre'sente le type et le mode de passage d'un
parame`tre formel de module. 
}

\index{Mode}
\domain{Mode = value:unit + reference:unit}
{
Le domaine \verb/mode/ repre'sente le mode de passage d'un parame`tre
formel de module. Le domaine contient un objet du domaine \verb/value/
pour le mode de passage par valeur et \verb/reference/ pour le passage
par adresse.
}

\subsection{Storage}
\label{subsection-storage}
\index{Storage}

\domain{Storage = return:entity + ram + formal + rom:unit}
{
Le domaine \verb/storage/ permet de pre'ciser dans quelle zone de la
me'moire est stocke'e une entite'. Il y a plusieurs zones, qui ne
correspondent pas ne'cessairement a` la re'alite', c'est-a`-dire aux
zones de me'moire qui seraient affecte'es par un compilateur.

Le sous-domaine \verb/return/ permet de repre'senter les variables ayant
pour nom le nom d'une fonction et auxquelles on affecte la valeur que la
fonction doit retourner. L'entite' pointe'e par \verb/return/ est la
fonction concerne'e.

Le sous-domaine \verb/ram/ est reserve' aux variables ayant une adresse
en me'moire. Il permet de pre'ciser dans quelle fonction et
e'ventuellement dans quel common ces variables ont e'te' de'clare'es.

Le sous-domaine \verb/formal/ est re'serve' aux parame`tres formels des
modules.

Le sous-domaine \verb/rom/ est utilise' pour toutes les entite's dont la
valeur n'est pas modifiable, telles que les fonctions, les labels, les
ope'rateurs, etc.
}

\subsubsection{RAM Storage}
\label{subsubsection-ram}
\index{RAM}

\domain{Ram = function:entity x section:entity x offset:int x shared:entity*}
{
Le domaine \verb/ram/ permet de pre'ciser la de'claration d'une
variable. Le sous-domaine \verb/function/ indique dans quel module une
entite' est de'clare'e. Le sous-domaine \verb/section/ indique dans
quelle aire une entite' est stocke'e; il y a une aire par common
de'clare' et deux aires spe'ciales nomme'es \verb/STATIC/ et
\verb/DYNAMIC/ pour les entite's locales. Le sous-domaine \verb/offset/
donne l'adresse dans l'aire de la variable. Enfin, le sous-domaine {\tt
shared} donne la liste des entite's qui partagent statiquement un
morceau d'espace me'moire avec la variable concerne'. En Fortran, le
partage de me'moire vient des equivalences entre variables.  }

\subsubsection{Formal Parameter Storage}
\label{subsubsection-formal}
\index{Formal}

\domain{Formal = function:entity x offset:int}
{
Le domaine \verb/formal/ indique le module dans lequel un parame`tre formel
est de'clare' gra^ce au sous-domaine \verb/function/, et le rang de ce
parame`tre dans la liste des parame`tres gra^ce au sous-domaine
\verb/offset/.
Le premier parame`tre a un rang de 1 et non de 0.
}

\subsection{Entity Value}
\label{subsection-value}
\index{Value}

\domain{Value = code + symbolic + constant + intrinsic:unit + unknown:unit}
{
Le domaine \verb/value/ permet de repre'senter les
valeurs initiales des entite's. Le sous-domaine \verb/code/ est utilise'
pour les entite's modules. Le sous-domaine \verb/symbolic/ est utilise'
pour les entite's constantes symboliques. Le sous-domaine
\verb/constant/ est utilise' pour les entite's constantes. Le
sous-domaine \verb/intrinsic/ est utilise' pour toutes les entite's qui
ne de'pendent que du langage, telles que les intrinsics Fortran, les
ope'rateurs, les instructions, etc. Enfin le sous-domaine
\verb/unknown/ est utilise' pour les valeurs initiales inconnues.

Additional value kinds would be necessary to encode the initial value of
an area, if the overloading of the \verb+unknown+ kind becomes a
problem. Pierre Jouvelot suggested to give \verb+COMMON+ themselves as
initial value since a common represented an address.
}

\subsubsection{Symbolic Value}
\label{subsubsection-symbolic}
\index{Symbolic}

\domain{Symbolic = expression x constant}
{
Le domaine \verb/symbolic/ est utilise' pour repre'senter la valeur
initiale d'une entite' constante symbolique, c'est-a`-dire les PARAMETER
de Fortran ou les CONST de Pascal. Le sous-domaine \verb/expression/
permet de stocker l'expression qui a permis d'e'valuer la valeur
initiale contenue dans le sous-domaine \verb/constant/. Le sous-domaine
\verb/expression/ n'est utile qui si on cherche a` reproduire un texte
source fide`le.
}

\subsubsection{Constant Value}
\label{subsubsection-constant}
\index{Constant}

\domain{Constant = int + litteral:unit}
{
Le domaine \verb/constant/ est utilise' pour repre'senter la valeur
initiale des entite's constantes. Seules les entite's de type entier
nous inte'ressent, ce qui explique qu'une constante puisse e^tre soit un
\verb/int/ soit un \verb/litteral/ dont on ne garde pas la valeur (type unit).
}

\section{Code, Statements and Instructions}
\label{code}

\subsection{Module Code}
\label{subsection-code}
\index{Code}\index{Declarations}\index{Decls text}

\domain{Code = declarations:entity* x decls\_text:string}
{ 
Le domaine \verb/code/ est utilise'
pour stocker le corps des modules. Le sous-domaine \verb/declarations/
contient une liste d'entite's qui sont les variables locales,
parame`ters formels et commons de'clare's dans la fonction.

Le sous-domaine {\tt decls\_text} contient le texte exact de toutes les
de'clarations du module; ce texte est utilise' par de'faut par le
prettyprinter tant qu'il existe. Quand le code a e'te' fortement
transforme', le prettyprinter re'ge'ne`re des de'clarations
synthe'tiques.
}

\subsection{Callees}
\label{subsection-callees}
\index{Callees}

% Should be put somewhere else!

\domain{Callees = callees:string*}
{ Le domaine {\tt callees} sert a` porter des informations
interproce'durales, et sera enrichi dans le futur.  Le sous-domaine {\tt
callees} contient la liste des noms des sous-programmes et fonctions
directement appele's dans le code. Il contient une partie du callgraph.
}

\subsection{Statement}
\label{subsection-statement}
\index{Statement}

\domain{Statement = label:entity x number:int x ordering:int x comments:string x instruction}
{ 
Le
domaine \verb/statement/ permet de repe'rer les instructions d'un
module.  Le sous-domaine \verb/label/ contient une entite' qui de'finit
le label\footnote{Un statement dont l'instruction est un bloc n'a
jamais de label.}.

Le sous-domaine \verb/number/ contient un nume'ro permettant
de repe'rer le statement pour le debugging ou l'information de
l'utilisateur (valeur par defaut:
\verb+STATEMENT\_NUMBER\_UNDEFINED+). Ce nume'ro est de'fini par
l'utilisateur et n'est (en princpe) jamais modifie'. Apre`s de'roulage
de boucle, plusieurs \verb/statement/s peuvent avoir le me^me nume'ro.

Le sous-domaine \verb/ordering/ contient un entier caracte'ristique du
statement; il est forme' de la concate'nation du nume'ro de composante
dans le graphe de contro^le et du nume'ro de statement dans dans cette
composante; cet entier sert a` comparer l'ordre lexical des statements
et sa valeur par defaut est \verb+STATEMENT\_ORDERING\_UNDEFINED+. Il
est automatiquement recalcule' apre`s chaque transformation de
programme. Il est aussi utilise' comme nom absolu d'un \verb/statement/
quand des structures de donne'es comme les tables de hash-code sont
e'crites sur disque ou relues.

Le sous-domaine {\tt comments} contient le texte du commentaire associe'
a ce statement dans le programme initial; ce texte est utilise' par le
prettyprinter. Ce sont les commentaires qui pre'ce`dent le statement
qui s'y trouvent associe's. En l'absence de commentaires, ce champ
prend la valeur \verb/string\_undefined/.

Le sous-domaine \verb/instruction/ contient
l'instruction proprement dite.
}

\domain{persistant_statement_to_statement = persistant statement -> persistant statement}
{
  Used for example in use_def_elimination() to store the eventual
  statement father of a statement. The persistance is needed to avoid
  freeing the statements when the mapping is freed.
}

\subsection{Instruction}
\label{subsection-instruction}
\index{Instruction}

\domain{Instruction = block:statement* + test + loop + goto:statement + call + unstructured}
{ 
Le domaine \verb/instruction/ permet de repre'senter les instructions
d'un module. Une instruction peut e^tre un sous-domaine \verb/block/,
c'est-a`-dire une liste de \verb/statement/, un sous-domaine \verb/test/
pour les instructions de test, un sous-domaine \verb/loop/ pour les
boucles se'quentielles, un sous-domaine \verb/goto/ pour les goto qui
contient le
\verb/statement/ vers lequel le goto se branche, un sous-domaine
\verb/call/ pour toutes les autres instructions (affectation, appel de
subroutine, entre'es-sorties, return, stop, etc) ou un sous-domaine de
\verb/unstructured/ dans le cas ou` l'on traite d'un graphe de contro^le
structure'. Toutes ces instructions 
sont repre'sente'es par des appels a` des fonctions pre'de'finies dont
nous e'tudierons la nature plus loin.
}

\subsubsection{Control Flow Graph (a.k.a. Unstructured)}
\label{subsubsection-unstructured}
\index{Unstructured}

\domain{Unstructured = control x exit:control}
{

Domain \verb/unstructured/ is used to represent unstructured parts of
the code in a structured manner which as a unique statement. The entry
node of the underlying CFG is in field \verb/control/, and the
unique exit node is in field \verb/exit/. The exit node should not be
modified by users of the unstructured\footnote{FI: I do not understand
why...}.  See Figure~\ref{figure-unstructured}.

The hierarchical structure is induced by the recursive nature of
statements.  Each control node points towards a statement which can also
contain an unstructured area of the code as well as structured
part. Unstructured parts of the code can thus be contained as much as
possible as well as be recursively decomposed.

For instance, the two DO loops in:
\begin{verbatim}
      DO 200 I = 1, N
100      CONTINUE
         DO 300 J = 1, M
            T(J) = T(J) + X
300      CONTINUE
         IF(X.GT.T(I)) GO TO 100
200   CONTINUE
\end{verbatim}
are preserved as DO loops in spite of the GO~TO statement (see
Figure~\ref{figure-hierarchical-control-flow-graph}).

\begin{figure}

\begin{center}

\unitlength 3pt

\begin{picture}(90,105)(0,0)
\put(40,70){\circle*{3}}
\put(50,70){DO 200}

\put(30,35){\circle*{3}}
\put(20,40){IF}
\put(40,45){\circle*{3}}
\put(50,45){100 CONTINUE}
\put(50,35){\circle*{3}}
\put(60,35){DO 300}

\put(50,10){\circle*{3}}
\put(60,10){T(J) = T(J) + X}

% link between structured and unstructured parts
\put(50,30){\line(0,-1){20}}
\put(40,60){\line(0,-1){15}}
\put(30,37){\line(1,3){8}}
\put(50,37){\line(-1,3){8}}

% control edges
\thicklines
\put(30,35){\vector(-1,0){15}}
\put(31,36){\vector(1,1){8}}
\put(41,44){\vector(1,-1){8}}
\put(50,35){\vector(-1,0){18}}
\thinlines

% Draw the planes
\multiput(0,0)(0,30){3}{\line(1,1){25}}
\multiput(0,0)(0,30){3}{\line(1,0){105}}
\end{picture}
\end{center}
\caption{Hierarchical Control Flow Graph}
\label{figure-hierarchical-control-flow-graph}
\end{figure}

}

\subsubsection{Conditional (a.k.a. Test)}
\label{subsubsection-test}
\index{Test}

\domain{Test = condition:expression x true:statement x false:statement}
{
Le domaine \verb/test/ permet de repre'senter toutes les instructions a` base
de contro^le. Le sous-domaine \verb/condition/ contient l'expression a`
tester, et les deux sous-domaines \verb/true/ et \verb/false/ contiennent les
instructions a` exe'cuter selon la valeur du test. 

Il faut noter que chaque instruction de contro^le de Fortran,
a` l'exception de l'instruction \verb/DO/, est
transforme'e en une combinaison se'mantiquement e'quivalente de \verb/test/s
et de \verb/goto/s.
}

\subsubsection{DO Loop, Sequential or Parallel}
\label{subsubsection-loop}
\index{Loop}

\domain{Loop = index:entity x range x body:statement x label:entity x execution x locals:entity*}
{
Le domaine \verb/loop/ permet de repre'senter les boucles du type DO Fortran
ou FOR Pascal. Le sous-domaine \verb/index/ contient l'entite' indice de
boucle, le sous-domaine \verb/range/ contient les bornes de la boucle, le
sous-domaine \verb/body/ contient le corps de la boucle, c'est-a`-dire un
\verb/statement/, le sous-domaine \verb/label/ contient le label de fin de boucle,
c'est-a`-dire une entite'. Le sous-domaine \verb/execution/ de'finit le
comportement dynamique d'une boucle. Les entite's pre'sentes dans
\verb/locals/ sont propres au corps de boucle (les effets sur elles sont
masque's quand on sort de la boucle).
}

\index{Execution}
\domain{Execution = sequential:unit + parallel:unit}
{
Le domain \verb/execution/ de'finit la se'mantique d'une boucle:
\verb/sequential/ correspond a` une boucle DO classique, \verb/parallel/
de'finit un boucle dont les instances d'ite'ration peuvent e^tre
exe'cute'es en paralle`le.
}

\label{range}
\index{Range}
\domain{Range = lower:expression x upper:expression x increment:expression}
{
Le domaine \verb/range/ permet de repre'senter les bornes des boucles DO
Fortran. Il y a trois sous-domaines \verb/lower/, \verb/upper/ et \verb/increment/ de
type \verb/expression/ qui sont respectivement la borne infe'rieure, la borne
supe'rieure et l'incre'ment.
}

\subsubsection{Function Call}
\label{subsubsection-call}
\index{Call}

\domain{Call = function:entity x arguments:expression*}
{

Le domaine \verb/call/ permet de repre'senter les commandes et les
appels de fonctions Fortran sous une forme unique pseudo-fonctionelle.
Ces pseudo-fonctions jouent un ro^le important dans notre
repre'sentation interme'diaire puisque les constantes, les ope'rateurs
comme + et *, les intrinse`ques comme {\tt MOD} ou {\tt SIN} et surtout
les commandes (i.e. instructions) Fortran, a` commencer par
l'assignation et en continuant avec {\tt READ, WRITE, RETURN, CALL}
etc..., sont repre'sente'es comme les appels de fonctions de'finies par
l'utilisateur. Le nombre d'arguments de chaque pseudo-fonction varie: 0
pour les constantes, 1 ou 2 pour les ope'rateurs, etc. Les commandes
Fortran, les ope'rateurs et les intrinse`ques sont caracte'rise's par
des pseudo-fonctions pre'de'finies. Cette convention permet de diminuer
conside'rablement la taille de la de'finition de la repre'sentation
ainsi que le volume de code ne'cessaire a` de nombreux algorithmes.

Le sous-domaine \verb/function/ est une entite' qui de'finit la fonction
appele'e. Le sous-domaine \verb/arguments/ est une liste de sous-domaines
\verb/expression/ qui repre'sente les arguments d'appel de la fonction.
}

\subsubsection{Control Flow Graph (cont.)}
\label{subsubsection-control}
\index{Control}

\domain{Control = statement x predecessors:control* x successors:control*}
\domain{Controlmap = persistant statement->control}
{ 

Domain \verb/control/ is the type of nodes used to implement the CFG
implied by an unstructured instruction (see Domain
\verb/unstructured/). Each node points towards a statement which can
represent an arbitrary large piece of structured code. GOTO statements
are eliminated and represented by arcs. Nodes are doubly linked. Each
node points towards its successors (at most 2) and towards its
predecessors. The hierarchical nature of
domain \verb/statement/ is used to hide local branches from higher and
lower level pieces of code. The whole unstructured area of the code is
seen as a unique atomic statement from above, and is entirely ignored
from under. This explains the mutual recursion between \verb/control/
and \verb/statement/ (via \verb/instruction/).

\begin{figure}
\begin{center}
\mbox{\psfig{file=unstructured.idraw,width=\hsize}}
\end{center}
\caption{Control Flow Graph}
\label{figure-unstructured}
\end{figure}

All statements but \verb/test/have only one successor. The first
successor of \verb/test/ is the successor when the test condition is
evaluated to true.  And the other way round for the second one. The exit
node (see domain \verb/unstructured/) has no successor. The entry node
as well as all other nodes may have an unlimited number of precedessors.

All nodes of a CFG can be visited in a meaningless order using the
\verb/CONTROL_MAP/ macros. Look for an example in library \verb/control/
because auxiliary data structures must be decraled and freed.

}

\domain{persistant_statement_to_control = persistant statement -> persistant control}
{
  Used for example in use_def_elimination() to store the eventual control
  father of a statement in order to travel on the control graph
  associated to a statement. The persistance is needed to avoid
  freeing the control graph when the mapping is freed.
}

% \section{Repre'sentation des expressions}
\section{Expressions}
\label{expression}
\index{Expression}

\domain{Expression = syntax x normalized}
{ 
Le domaine \verb/expression/ permet de stocker les expressions.  Le
sous-domaine {\tt syntax} contient la description de l'expression telle
qu'elle apparai^t dans le texte source du programme. Le sous-domaine {\tt
normalized} contient une forme compile'e des expressions line'aires,
sous forme de Pvecteur.

Si le sous-domaine {\tt normalized} contient la valeur
{\tt normalized\_undefined}, cela signifie que la fonction de line'arisation
n'a pas e'te' appele'e pour cette expression; cela {\bf ne} signifie
{\bf pas}
que l'expression n'est pas line'aire.
}

\subsection{Abstract Tree of an Expression: Syntax}
\label{subsection-syntax}
\index{Syntax}

\domain{Syntax = reference + range + call}
{
Le domaine \verb/syntax/ permet de repre'senter les expressions telles
qu'elles apparaissent dans le texte source du programme. Un
\verb/syntax/ est soit une \verb/reference/ a` un e'le'ment de tableau
(on rappelle que les scalaires sont des tableaux a` 0 dimension) , soit
un \verb/call/ a` une fonction (les ope'rateurs sont repre'sente's par
des fonctions pre'-de'finies, y compris l'assignation; c'est pourquoi le
domaine {\tt call} est de'fini dans la section {\em Instructions}), soit
un \verb/range/, dans le cas des expressions bornes de boucles (le domaine
{\tt range} est pre'sente' au niveau des boucles, aussi dans la section
{\em Instructions}).
}

\subsubsection{Reference}
\label{subsubsection-reference}
\index{reference}

\domain{Reference = variable:entity x indices:expression*}
{
Le domaine \verb/reference/ est utilise' pour repre'senter une
re'fe'rence a` un e'le'ment de tableau\footnote{Les variables scalaires
e'tant repre'sente'es par des tableaux de dimension 0, les re'fe'rences
a des scalaires sont aussi prises en compte. Elles contiennent une liste
vide d'expressions d'indices.}.  Le sous-domaine \verb/variable/
contient une entite' de'finissant la variable re'fe'rence'e. Le
sous-domaine \verb/indices/ contient une liste {\tt expression}s qui sont les
indices de la re'fe'rence.
}

\subsubsection{Range}

See Section~\ref{range}.

\subsubsection{Function Call}

All operators, including assignment, are repreented as function calls
with side effect. See Section~\ref{subsubsection-call}.

\subsection{Affine Representation of an Expression}
\label{subsection-normalized}
\index{Normalized}

\domain{Normalized = linear:Pvecteur + complex:unit}
{ Le domaine {\tt normalized} permet de savoir si une expression est une
expression line'aire construite sur les variables simples entie`res
(sous-domaine {\tt linear}) ou non (sous-domaine {\tt complex}).

Le sous-domaine {\tt complex} est utilise' si l'expression n'est pas
line'aire (ex: {\tt I*J+4}) ou si elle est line'aire mais contient autre
chose que des re'fe'rences a` des scalaires entiers (ex: {\tt T(I-1) +
T(I) + T(I+1)}).

La forme normalise'e n'existe pas si l'expression n'a pas encore e'te'
examine'e. }

% \section{Effets des instructions}
\section{Memory Effects of Statements}
\label{effects}

\subsection{Effects}
\label{subsection-effects}
\index{Effects}

\domain{Effects =  effects:effect*}
{}

\subsection{Effect}
\label{subsection-effect}
\index{Effect}\index{Region}

\domain{Effect =  persistant reference x action x approximation x context:transformer}
{
Le domaine \verb/effect/ est utilise' pour repre'senter les effets d'un
statement sur les variables du module. Les effets des instructions sont
le point de de'part du calcul des Def-Use Chains, des de'pendances, des
Summary Data Flow Information, et des re'gions.

Le sous-domaine {\tt reference} indique sur quelle variable, scalaire ou
tableau, a lieu un effet, le sous-domaine {\tt action} pre'cise si la
re'fe'rence est lue ou e'crite, et le sous-domaine {\tt approximation}
permet de savoir si la re'fe'rence (pour les effets) ou l'ensemble des
e'le'ments de tableaux de'fini par le contexte (pour les re'gions) est lu ou
e'crit a` coup su^r ou non. 

Lors de la traduction interproce'durale des effets ou des re'gions portant
sur une variable globale non de'clare'e dans le module appelant ou sur une
variable statique apparaissant dans un {\tt SAVE}, il est important de
re'cupe'rer une re'fe'rence qui sera la me^me pour tous les sites d'appels.
Pour cela on choisit un module de re'fe'rence (dans lequel le common ou le
save est de'clare'), dont le nom pre'fixera le nom de la variable. Le choix
de ce module n'est pas trivial. A` l'heure actuel. il s'agit du nom du
module de la premie`re variable apparaissant dans la liste des variables du
common ({\tt ram_section(storage_ram(entity_storage(<my_common>)))}). Selon
l'ordre dans lequel les modules dont analyse's, le nom de module trouve'
sera diffe'rent. Il faudrait en re'alite' prendre le premier nom de module
dans l'ordre lexicographique des modules appele's (callees).

Le sous-domaine {\tt reference} permet de pre'ciser qu'un effet n'a lieu que
sur un sous-tableau en utilisant un {\tt range} comme expression d'indice.
Ceci est utilise' lors de la traduction des effets cumule's d'un proce'dure
en les effets propres d'un call site. Pour les re'gions, ce sous-domaine
pre'cise l'entite' concerne'e ainsi que la liste des variables $\phi$ qui
de'crivent ses dimensions.

Le sous-domaine {\tt context} n'est utilise' que pour repre'senter les
effets des instructions par des re'gions (telles que de'finies par Re'mi
Triolet).

}

\subsection{Nature of an Effect}
\label{subsection-action}
\index{Action}

\domain{Action = read:unit + write:unit}
{
Deux types d'effets sont utilise's dans les conditions de Bernstein
et dans les conditions propres a` chaque transformation de programme:
la lecture d'une variable et son e'criture.
}

\subsection{Approximation of an Effect}
\label{subsection-approximation}
\index{Approximation}

\domain{Approximation = may:unit + must:unit}
{
La pre'sence de tests et boucles ne permet pas de de'terminer en ge'ne'ral
si une variable est effectivement lue ou e'crite lors de l'exe'cution
d'un {\tt statement}. Il se peut me^me que certaines exe'cutions
y acce`dent et que d'autres n'y fassent pas re'fe'rence. Les effets
calcule's sont alors de type {\tt may}.

Dans quelques cas particuliers, comme une affectation simple {\tt I = 2},
l'effet est certain ({\em must}). Il peut alors e^tre utilise' dans
le calcul des {\em use-def chains} pour effectuer un {\em kill} sur les
variables scalaires. Un effet {\em must} sur un tableau ne signifie pas
que tout le tableau est lu ou e'crit mais qu'au moins un de ses
e'le'ments l'est.
}

\section{Analyse se'mantique}
\label{semantics}

\subsection{Transformer}
\label{subsection-transformer}
\index{Transformer}

\domain{Transformer = arguments:entity* x relation:predicate}
{
Le domaine {\tt transformer} de'finit une relation entre deux e'tats
me'moire. Cette relation
porte sur les valeurs des variables scalaires entie`res d'un module ou
des variables globales au programme.

Les variables qui apparaissent dans la liste des arguments sont celles
qui ont e'te' modifie'es entre les deux e'tats. Deux valeurs
sont donc associe'es a` chacune d'entre elles: la pre- et la
post-valeur.  Les post-valeurs sont porte'es par les entite's
elles-me^mes. Les pre'-valeurs sont porte'es par des entite's
spe'ciales. Les variables scalaires entie`res qui ne sont pas modifie'es
et qui n'apparaissent donc pas dans la liste des arguments n'ont qu'une
seule valeur, porte'e par l'entite' correspondant a` la variable.

La relation est de'finie par des e'galite's et des ine'galite's
line'aires entre valeurs.

Deux types de transformers sont utilise's. Le premier est propre a` un
{\tt statement} et donne une abstraction de son effet sur les variables
entie`res. Les variables qui apparaissent dans la liste des arguments
sont celles qui sont affecte'es lors de son exe'cution.  Le second,
aussi associe' a` un {\tt statement}, donne une relation entre l'e'tat
initial d'un module et l'e'tat pre'ce'dent l'exe'cution de ce {\tt statement}.

Les transformers ne sont de'finis qu'apre`s une phase d'analyse se'mantique.
}

\subsection{Predicate}
\label{subsection-predicate}
\index{Predicate}\index{Precondition}\index{Transformer}

\domain{Predicate = system:Psysteme}
{
Le domaine {\tt predicate} de'finit une relation entre valeurs de
variables scalaires entie`res. Son interpre'tation est fonction de
son utilisation. Il peut s'agir soit d'un pre'dicat valable en
un point du programme (i.e. un invariant), soit d'un pre'dicat
valable entre deux points du programme. Il s'agit alors d'une
abstraction d'une commande, c'est-a`-dire d'un {\tt transformer}.
}

\newpage

\section*{Annexe: re'capitulatif ri.newgen}

\begin{verbatim}
--         --------------------------------------------------------
--         --------------------------------------------------------
--
--                                  WARNING
--
--                THIS FILE HAS BEEN AUTOMATICALLY GENERATED
--
--                             DO NOT MODIFY IT
--
--         --------------------------------------------------------
--         --------------------------------------------------------

-- Imported domains
-- ----------------

-- External domains
-- ----------------
external Psysteme ;
external Pvecteur ;

-- Domains
-- -------
action = read:unit + write:unit ;
approximation = may:unit + must:unit ;
area = size:int x layout:entity* ;
basic = int:int + float:int + logical:int + overloaded:unit + complex:int + string:value ;
call = function:entity x arguments:expression* ;
callees = callees:string* ;
code = declarations:entity* x decls_text:string ;
constant = int + litteral:unit ;
control = statement x predecessors:control* x successors:control* ;
dimension = lower:expression x upper:expression ;
effect =  persistant reference x action x approximation x context:transformer ;
effects =  effects:effect* ;
execution = sequential:unit + parallel:unit ;
expression = syntax x normalized ;
formal = function:entity x offset:int ;
functional = parameters:parameter* x result:type ;
instruction = block:statement* + test + loop + goto:statement + call + unstructured ;
loop = index:entity x range x body:statement x label:entity x execution x locals:entity* ;
mode = value:unit + reference:unit ;
normalized = linear:Pvecteur + complex:unit ;
parameter = type x mode ;
predicate = system:Psysteme ;
ram = function:entity x section:entity x offset:int x shared:entity* ;
range = lower:expression x upper:expression x increment:expression ;
reference = variable:entity x indices:expression* ;
statement = label:entity x number:int x ordering:int x comments:string x instruction ;
storage = return:entity + ram + formal + rom:unit ;
symbolic = expression x constant ;
syntax = reference + range + call ;
tabulated entity = name:string x type x storage x initial:value ;
test = condition:expression x true:statement x false:statement ;
transformer = arguments:entity* x relation:predicate ;
type = statement:unit + area + variable + functional + unknown:unit + void:unit ;
unstructured = control x exit:control ;
value = code + symbolic + constant + intrinsic:unit + unknown:unit ;
variable = basic x dimensions:dimension* ;
\end{verbatim}

\newpage

% Cross-references for points and keywords

\documentstyle[a4,psfig]{article}

% \input{/usr/share/local/lib/tex/macroslocales/Dimensions.tex}

\title{PIPS: Internal Representation of Fortran Code}
\author{Franc,ois Irigoin \\
        Pierre Jouvelot \\
    Ronan Keryell \\
        Re'mi Triolet\\
\\
        CRI, Ecole des Mines de Paris}

                                %\newcommand{\domain}[2]{\paragraph{{#1}}\paragraph{}{{#2}}}
      {
        \catcode `==\active
        \gdef\domain{\medskip\par\noindent
          \bgroup
          \catcode `_ \other
          \catcode `= \active
          \def={\em{\rm \string=}}
          \tt\em\vraidomain}
        \gdef\vraidomain#1{#1\egroup\medskip\par}
        }

% Correction de quelques erreurs. RK, 07/02/1994.

\renewcommand{\indexname}{Index}

\makeindex

\begin{document}
\maketitle
\sloppy

\section*{Introduction}

Ce document est utilise' directement et automatiquement par l'outil
de ge'nie logiciel NewGen pour ge'ne'rer les de'clarations des structures
de donne'es utilise'es dans le projet PIPS, ainsi que les routines de
base qui les manipulent. C'est pourquoi l'ordre des sections n'est pas
ne'cessairement naturel.

Apre`s une description rapide des structures de donne'es externes a` la
repre'sentation interne de Pips, nous pre'sentons successivement les
notions d'entite's, de code et d'expressions et la manie`re dont elles
sont utilise'es pour encoder un programme Fortran. Nous de'taillons
ensuite deux structures de donne'es supple'mentaires, les effets et les
{\em transformers} qui sont utilise's pour le calcul de de'pendance
interproce'dural et pour l'analyse syntaxique interproce'durale.

La manie`re dont les constructions de Fortran sont repre'sente'es est
de'crite dans une des sections du rapport EMP-CAII-E105. NewGen est
pre'sente' dans le rapport EMP-CRI-A191.

\section{External Data Structures}
\label{external}

\subsection{Vector}
\label{subsection-pvecteur}
\index{Pvecteur}

\domain{External Pvecteur}
{ Le domaine {\tt Pvecteur} est utilise' pour repre'senter les
expressions line'aires telles que {\tt 3I+2} (voir le domaine {\tt
normalized}) ou des contraintes line'aires telles que {\tt 3I + J <= 2}
ou {\tt 3I == J}. Ces contraintes sont utilise'es dans les syste`mes
line'aires (voir le domaine {\tt Psysteme}).

Un Pvecteur est une suite de mono^mes, un mono^me e'tant un couple
(coefficient,variable).  Le coefficient d'un tel couple est un entier,
positif ou ne'gatif. La variable est une entite', sauf dans le cas du
terme constant qui est repre'sente' par la variable pre'de'finie de nom
{\tt TCST}\footnote{Comme on rajoute le nom des modules devant les
noms de variables, il ne peut pas y avoir de conflict avec une
e'ventuelle variable {\tt TCST}.}.

Les expressions apparaissant dans le programme analyse' sont mises
sous cette forme quand c'est possible.

La structure de donne'es Pvecteur est importe'e de la bibliothe`que d'alge`bre
line'aire en nombres entiers du CAII.
}

\subsection{Set of Affine Constraints}
\label{subsection-psysteme}
\index{Psysteme}

\domain{External Psysteme}
{
Le domaine {\tt Psysteme} est utilise' pour repre'senter les syste`mes
d'e'quations et d'ine'quations line'aires qui apparaissent lors
de la phase d'analyse se'mantique (voir le domaine {\tt predicate}).

Les Psystemes sont aussi implicitement utilise's pour effectuer le
calcul de de'pendance.

Un Psysteme est forme' de six champs:
\begin{itemize}
  \item une liste d'e'galite's,
  \item le nombre des e'galite's,
  \item une liste d'ine'galite's,
  \item le nombre d'ine'galite's,
  \item la dimension de l'espace de re'fe'rence,
  \item une base de l'espace de re'fe'rence.
\end{itemize}

Comme le domaine Pvecteur, la structure de donne'es Psysteme est
importe'e de la bibliothe`que d'alge`bre line'aire en nombres entiers du
CAII.  }

\section{Entities: Variables, Functions, Operators, Constants, Labels...}
\label{entity}

\subsection{Entity}
\label{subsection-entity}
\index{Entity}

\domain{tabulated entity = name:string x type x storage x initial:value}
\domain{entity_int = entity->int}
{
Tout objet ayant un nom dans un programme Fortran est repre'sente' par
une \verb/entity/. Un tel objet peut e^tre un module, une variable, un
common, un ope'rateur, une constante, un label, etc. Pour chaque objet,
le sous-domaine \verb/name/ de l'entite' donne le nom de l'objet tel
qu'il apparai^t dans le texte source du programme pre'fixe' par le nom du
package dans lequel l'entite' est de'clare'e, le sous-domaine
\verb/type/ donne le type de l'entite', le sous-domaine \verb/storage/
le genre d'allocation me'moire utilise' pour l'entite', et finalement,
le sous-domaine \verb/initial/ donne la valeur initiale, si elle est
connue, de l'entite'. Le terme valeur initiale a ici un sens assez
large, puisqu'il s'agit par exemple du code pour les entite's
repre'sentant des modules.
}

\subsection{Type}
\label{subsection-type}
\index{Type}

\domain{Type = statement:unit + area + variable + functional + unknown:unit + void:unit}
{
Le domaine \verb/type/ repre'sente le type d'une entite'.  Le
sous-domaine \verb/statement/ est utilise' pour les labels
d'instruction.  Le sous-domaine \verb/area/ est utilise' pour les
commons.  Le sous-domaine \verb/variable/ est utilise' pour toutes les
variables, y compris les parame`tres formels et le re'sultat d'une
fonction.  Le sous-domaine \verb/functional/ est utilise' pour les
fonctions, pour les subroutines et pour le programme principal.  Le
sous-domaine \verb/void/ est utilise' pour le re'sultat d'une subroutine
ou d'un programme principal.
}

\subsubsection{Area Type}
\label{subsubsection-area}
\index{Area}

\domain{Area = size:int x layout:entity*}
{ Le domaine {\tt area} est utilise' pour repre'senter les aires de
stockage des variables telles que les commons et les aires statiques ou
dynamiques. Le sous-domaine {\tt size} donne la taille de l'aire
exprime'e en octets ({\em character storage unit} de la norme ANSI
X3.9-1978, \S~2.13), et le sous-domaine {\tt layout} donne la liste des
entite's stocke'es dans cette aire. L'ordre des de'clarations des
variables est respecte', ce qui permettrait de reproduire des programmes
sources fide`les a` ce qu'e'taient les programmes initiaux\footnote{Les
de'clarations sont en fait conserve'es sous forme textuelle pour
garantir une fide'lite' absolue.}.}

\subsubsection{Variable Type}
\label{subsubsection-variable}
\index{Variable}

\domain{Variable = basic x dimensions:dimension*}
{
Le domaine \verb/variable/ repre'sente le type d'une variable.  Le
sous-domaine \verb/basic/ donne le type Fortran de la variable.  Le
sous-domaine \verb/dimensions/ donne la liste des dimensions de la variable.
Un scalaire est un tableau de ze'ro dimension.

Chaque dimension est une expression, qui n'est pas ne'cessairement
constante dans le cas des tableaux formels. La constante pre'de'finie de
nom '*D*' est utilise'e pour les tableaux de taille non de'finie
(\verb/DIMENSION T(*)/).
}

\subsubsection{Basic Type}
\label{subsubsection-basic}
\index{Basic}

\domain{Basic = int:int + float:int + logical:int + overloaded:unit + complex:int + string:value}
{
Le domaine \verb/basic/ permet de repre'senter un type Fortran tel que
INTEGER ou REAL. La valeur de ce domaine donne la longueur en octets de
la zone me'moire occupe'e par une variable de ce type.
}

\subsubsection{Dimension}
\label{subsubsection-dimension}
\index{Dimension}

\domain{Dimension = lower:expression x upper:expression}
{
Le domaine \verb/dimension/ permet de repre'senter une dimension d'un
tableau, c'est-a`-dire un couple borne infe'rieure -- sous-domaine
\verb/lower/ -- borne supe'rieure -- sous-domaine \verb/upper/.
}

\subsubsection{Functional Type}
\label{subsubsection-functional}
\index{Functional}

\domain{Functional = parameters:parameter* x result:type}
{ Le domaine \verb/functional/ repre'sente le type d'un module,
c'est-a`-dire une fonction, une subroutine ou un programme principal. Le
sous-domaine \verb/parameters/ donne le type et le mode de passage de
chaque parame`tre, et le sous-domaine \verb/result/ donne le type du
re'sultat. Ce dernier type vaut \verb/void/ pour les subroutines et les
programmes principaux.

Il n'y a pas de moyens simples pour repre'senter les fonctions ou
sous-programmes a` nombre variable de parame`tres. Bien que ce soit
interdit pour les modules de'finis par le programmeur, de nombreux
intrinse`ques comme \verb+MIN0+ ou \verb+WRITE+ n'ont pas un profil
unique.  }

\subsubsection{Parameter Type and Mode}
\label{subsubsection-parameter}
\index{Parameter}

\domain{Parameter = type x mode}
{
Le domaine \verb/parameter/ repre'sente le type et le mode de passage d'un
parame`tre formel de module. 
}

\index{Mode}
\domain{Mode = value:unit + reference:unit}
{
Le domaine \verb/mode/ repre'sente le mode de passage d'un parame`tre
formel de module. Le domaine contient un objet du domaine \verb/value/
pour le mode de passage par valeur et \verb/reference/ pour le passage
par adresse.
}

\subsection{Storage}
\label{subsection-storage}
\index{Storage}

\domain{Storage = return:entity + ram + formal + rom:unit}
{
Le domaine \verb/storage/ permet de pre'ciser dans quelle zone de la
me'moire est stocke'e une entite'. Il y a plusieurs zones, qui ne
correspondent pas ne'cessairement a` la re'alite', c'est-a`-dire aux
zones de me'moire qui seraient affecte'es par un compilateur.

Le sous-domaine \verb/return/ permet de repre'senter les variables ayant
pour nom le nom d'une fonction et auxquelles on affecte la valeur que la
fonction doit retourner. L'entite' pointe'e par \verb/return/ est la
fonction concerne'e.

Le sous-domaine \verb/ram/ est reserve' aux variables ayant une adresse
en me'moire. Il permet de pre'ciser dans quelle fonction et
e'ventuellement dans quel common ces variables ont e'te' de'clare'es.

Le sous-domaine \verb/formal/ est re'serve' aux parame`tres formels des
modules.

Le sous-domaine \verb/rom/ est utilise' pour toutes les entite's dont la
valeur n'est pas modifiable, telles que les fonctions, les labels, les
ope'rateurs, etc.
}

\subsubsection{RAM Storage}
\label{subsubsection-ram}
\index{RAM}

\domain{Ram = function:entity x section:entity x offset:int x shared:entity*}
{
Le domaine \verb/ram/ permet de pre'ciser la de'claration d'une
variable. Le sous-domaine \verb/function/ indique dans quel module une
entite' est de'clare'e. Le sous-domaine \verb/section/ indique dans
quelle aire une entite' est stocke'e; il y a une aire par common
de'clare' et deux aires spe'ciales nomme'es \verb/STATIC/ et
\verb/DYNAMIC/ pour les entite's locales. Le sous-domaine \verb/offset/
donne l'adresse dans l'aire de la variable. Enfin, le sous-domaine {\tt
shared} donne la liste des entite's qui partagent statiquement un
morceau d'espace me'moire avec la variable concerne'. En Fortran, le
partage de me'moire vient des equivalences entre variables.  }

\subsubsection{Formal Parameter Storage}
\label{subsubsection-formal}
\index{Formal}

\domain{Formal = function:entity x offset:int}
{
Le domaine \verb/formal/ indique le module dans lequel un parame`tre formel
est de'clare' gra^ce au sous-domaine \verb/function/, et le rang de ce
parame`tre dans la liste des parame`tres gra^ce au sous-domaine
\verb/offset/.
Le premier parame`tre a un rang de 1 et non de 0.
}

\subsection{Entity Value}
\label{subsection-value}
\index{Value}

\domain{Value = code + symbolic + constant + intrinsic:unit + unknown:unit}
{
Le domaine \verb/value/ permet de repre'senter les
valeurs initiales des entite's. Le sous-domaine \verb/code/ est utilise'
pour les entite's modules. Le sous-domaine \verb/symbolic/ est utilise'
pour les entite's constantes symboliques. Le sous-domaine
\verb/constant/ est utilise' pour les entite's constantes. Le
sous-domaine \verb/intrinsic/ est utilise' pour toutes les entite's qui
ne de'pendent que du langage, telles que les intrinsics Fortran, les
ope'rateurs, les instructions, etc. Enfin le sous-domaine
\verb/unknown/ est utilise' pour les valeurs initiales inconnues.

Additional value kinds would be necessary to encode the initial value of
an area, if the overloading of the \verb+unknown+ kind becomes a
problem. Pierre Jouvelot suggested to give \verb+COMMON+ themselves as
initial value since a common represented an address.
}

\subsubsection{Symbolic Value}
\label{subsubsection-symbolic}
\index{Symbolic}

\domain{Symbolic = expression x constant}
{
Le domaine \verb/symbolic/ est utilise' pour repre'senter la valeur
initiale d'une entite' constante symbolique, c'est-a`-dire les PARAMETER
de Fortran ou les CONST de Pascal. Le sous-domaine \verb/expression/
permet de stocker l'expression qui a permis d'e'valuer la valeur
initiale contenue dans le sous-domaine \verb/constant/. Le sous-domaine
\verb/expression/ n'est utile qui si on cherche a` reproduire un texte
source fide`le.
}

\subsubsection{Constant Value}
\label{subsubsection-constant}
\index{Constant}

\domain{Constant = int + litteral:unit}
{
Le domaine \verb/constant/ est utilise' pour repre'senter la valeur
initiale des entite's constantes. Seules les entite's de type entier
nous inte'ressent, ce qui explique qu'une constante puisse e^tre soit un
\verb/int/ soit un \verb/litteral/ dont on ne garde pas la valeur (type unit).
}

\section{Code, Statements and Instructions}
\label{code}

\subsection{Module Code}
\label{subsection-code}
\index{Code}\index{Declarations}\index{Decls text}

\domain{Code = declarations:entity* x decls\_text:string}
{ 
Le domaine \verb/code/ est utilise'
pour stocker le corps des modules. Le sous-domaine \verb/declarations/
contient une liste d'entite's qui sont les variables locales,
parame`ters formels et commons de'clare's dans la fonction.

Le sous-domaine {\tt decls\_text} contient le texte exact de toutes les
de'clarations du module; ce texte est utilise' par de'faut par le
prettyprinter tant qu'il existe. Quand le code a e'te' fortement
transforme', le prettyprinter re'ge'ne`re des de'clarations
synthe'tiques.
}

\subsection{Callees}
\label{subsection-callees}
\index{Callees}

% Should be put somewhere else!

\domain{Callees = callees:string*}
{ Le domaine {\tt callees} sert a` porter des informations
interproce'durales, et sera enrichi dans le futur.  Le sous-domaine {\tt
callees} contient la liste des noms des sous-programmes et fonctions
directement appele's dans le code. Il contient une partie du callgraph.
}

\subsection{Statement}
\label{subsection-statement}
\index{Statement}

\domain{Statement = label:entity x number:int x ordering:int x comments:string x instruction}
{ 
Le
domaine \verb/statement/ permet de repe'rer les instructions d'un
module.  Le sous-domaine \verb/label/ contient une entite' qui de'finit
le label\footnote{Un statement dont l'instruction est un bloc n'a
jamais de label.}.

Le sous-domaine \verb/number/ contient un nume'ro permettant
de repe'rer le statement pour le debugging ou l'information de
l'utilisateur (valeur par defaut:
\verb+STATEMENT\_NUMBER\_UNDEFINED+). Ce nume'ro est de'fini par
l'utilisateur et n'est (en princpe) jamais modifie'. Apre`s de'roulage
de boucle, plusieurs \verb/statement/s peuvent avoir le me^me nume'ro.

Le sous-domaine \verb/ordering/ contient un entier caracte'ristique du
statement; il est forme' de la concate'nation du nume'ro de composante
dans le graphe de contro^le et du nume'ro de statement dans dans cette
composante; cet entier sert a` comparer l'ordre lexical des statements
et sa valeur par defaut est \verb+STATEMENT\_ORDERING\_UNDEFINED+. Il
est automatiquement recalcule' apre`s chaque transformation de
programme. Il est aussi utilise' comme nom absolu d'un \verb/statement/
quand des structures de donne'es comme les tables de hash-code sont
e'crites sur disque ou relues.

Le sous-domaine {\tt comments} contient le texte du commentaire associe'
a ce statement dans le programme initial; ce texte est utilise' par le
prettyprinter. Ce sont les commentaires qui pre'ce`dent le statement
qui s'y trouvent associe's. En l'absence de commentaires, ce champ
prend la valeur \verb/string\_undefined/.

Le sous-domaine \verb/instruction/ contient
l'instruction proprement dite.
}

\domain{persistant_statement_to_statement = persistant statement -> persistant statement}
{
  Used for example in use_def_elimination() to store the eventual
  statement father of a statement. The persistance is needed to avoid
  freeing the statements when the mapping is freed.
}

\subsection{Instruction}
\label{subsection-instruction}
\index{Instruction}

\domain{Instruction = block:statement* + test + loop + goto:statement + call + unstructured}
{ 
Le domaine \verb/instruction/ permet de repre'senter les instructions
d'un module. Une instruction peut e^tre un sous-domaine \verb/block/,
c'est-a`-dire une liste de \verb/statement/, un sous-domaine \verb/test/
pour les instructions de test, un sous-domaine \verb/loop/ pour les
boucles se'quentielles, un sous-domaine \verb/goto/ pour les goto qui
contient le
\verb/statement/ vers lequel le goto se branche, un sous-domaine
\verb/call/ pour toutes les autres instructions (affectation, appel de
subroutine, entre'es-sorties, return, stop, etc) ou un sous-domaine de
\verb/unstructured/ dans le cas ou` l'on traite d'un graphe de contro^le
structure'. Toutes ces instructions 
sont repre'sente'es par des appels a` des fonctions pre'de'finies dont
nous e'tudierons la nature plus loin.
}

\subsubsection{Control Flow Graph (a.k.a. Unstructured)}
\label{subsubsection-unstructured}
\index{Unstructured}

\domain{Unstructured = control x exit:control}
{

Domain \verb/unstructured/ is used to represent unstructured parts of
the code in a structured manner which as a unique statement. The entry
node of the underlying CFG is in field \verb/control/, and the
unique exit node is in field \verb/exit/. The exit node should not be
modified by users of the unstructured\footnote{FI: I do not understand
why...}.  See Figure~\ref{figure-unstructured}.

The hierarchical structure is induced by the recursive nature of
statements.  Each control node points towards a statement which can also
contain an unstructured area of the code as well as structured
part. Unstructured parts of the code can thus be contained as much as
possible as well as be recursively decomposed.

For instance, the two DO loops in:
\begin{verbatim}
      DO 200 I = 1, N
100      CONTINUE
         DO 300 J = 1, M
            T(J) = T(J) + X
300      CONTINUE
         IF(X.GT.T(I)) GO TO 100
200   CONTINUE
\end{verbatim}
are preserved as DO loops in spite of the GO~TO statement (see
Figure~\ref{figure-hierarchical-control-flow-graph}).

\begin{figure}

\begin{center}

\unitlength 3pt

\begin{picture}(90,105)(0,0)
\put(40,70){\circle*{3}}
\put(50,70){DO 200}

\put(30,35){\circle*{3}}
\put(20,40){IF}
\put(40,45){\circle*{3}}
\put(50,45){100 CONTINUE}
\put(50,35){\circle*{3}}
\put(60,35){DO 300}

\put(50,10){\circle*{3}}
\put(60,10){T(J) = T(J) + X}

% link between structured and unstructured parts
\put(50,30){\line(0,-1){20}}
\put(40,60){\line(0,-1){15}}
\put(30,37){\line(1,3){8}}
\put(50,37){\line(-1,3){8}}

% control edges
\thicklines
\put(30,35){\vector(-1,0){15}}
\put(31,36){\vector(1,1){8}}
\put(41,44){\vector(1,-1){8}}
\put(50,35){\vector(-1,0){18}}
\thinlines

% Draw the planes
\multiput(0,0)(0,30){3}{\line(1,1){25}}
\multiput(0,0)(0,30){3}{\line(1,0){105}}
\end{picture}
\end{center}
\caption{Hierarchical Control Flow Graph}
\label{figure-hierarchical-control-flow-graph}
\end{figure}

}

\subsubsection{Conditional (a.k.a. Test)}
\label{subsubsection-test}
\index{Test}

\domain{Test = condition:expression x true:statement x false:statement}
{
Le domaine \verb/test/ permet de repre'senter toutes les instructions a` base
de contro^le. Le sous-domaine \verb/condition/ contient l'expression a`
tester, et les deux sous-domaines \verb/true/ et \verb/false/ contiennent les
instructions a` exe'cuter selon la valeur du test. 

Il faut noter que chaque instruction de contro^le de Fortran,
a` l'exception de l'instruction \verb/DO/, est
transforme'e en une combinaison se'mantiquement e'quivalente de \verb/test/s
et de \verb/goto/s.
}

\subsubsection{DO Loop, Sequential or Parallel}
\label{subsubsection-loop}
\index{Loop}

\domain{Loop = index:entity x range x body:statement x label:entity x execution x locals:entity*}
{
Le domaine \verb/loop/ permet de repre'senter les boucles du type DO Fortran
ou FOR Pascal. Le sous-domaine \verb/index/ contient l'entite' indice de
boucle, le sous-domaine \verb/range/ contient les bornes de la boucle, le
sous-domaine \verb/body/ contient le corps de la boucle, c'est-a`-dire un
\verb/statement/, le sous-domaine \verb/label/ contient le label de fin de boucle,
c'est-a`-dire une entite'. Le sous-domaine \verb/execution/ de'finit le
comportement dynamique d'une boucle. Les entite's pre'sentes dans
\verb/locals/ sont propres au corps de boucle (les effets sur elles sont
masque's quand on sort de la boucle).
}

\index{Execution}
\domain{Execution = sequential:unit + parallel:unit}
{
Le domain \verb/execution/ de'finit la se'mantique d'une boucle:
\verb/sequential/ correspond a` une boucle DO classique, \verb/parallel/
de'finit un boucle dont les instances d'ite'ration peuvent e^tre
exe'cute'es en paralle`le.
}

\label{range}
\index{Range}
\domain{Range = lower:expression x upper:expression x increment:expression}
{
Le domaine \verb/range/ permet de repre'senter les bornes des boucles DO
Fortran. Il y a trois sous-domaines \verb/lower/, \verb/upper/ et \verb/increment/ de
type \verb/expression/ qui sont respectivement la borne infe'rieure, la borne
supe'rieure et l'incre'ment.
}

\subsubsection{Function Call}
\label{subsubsection-call}
\index{Call}

\domain{Call = function:entity x arguments:expression*}
{

Le domaine \verb/call/ permet de repre'senter les commandes et les
appels de fonctions Fortran sous une forme unique pseudo-fonctionelle.
Ces pseudo-fonctions jouent un ro^le important dans notre
repre'sentation interme'diaire puisque les constantes, les ope'rateurs
comme + et *, les intrinse`ques comme {\tt MOD} ou {\tt SIN} et surtout
les commandes (i.e. instructions) Fortran, a` commencer par
l'assignation et en continuant avec {\tt READ, WRITE, RETURN, CALL}
etc..., sont repre'sente'es comme les appels de fonctions de'finies par
l'utilisateur. Le nombre d'arguments de chaque pseudo-fonction varie: 0
pour les constantes, 1 ou 2 pour les ope'rateurs, etc. Les commandes
Fortran, les ope'rateurs et les intrinse`ques sont caracte'rise's par
des pseudo-fonctions pre'de'finies. Cette convention permet de diminuer
conside'rablement la taille de la de'finition de la repre'sentation
ainsi que le volume de code ne'cessaire a` de nombreux algorithmes.

Le sous-domaine \verb/function/ est une entite' qui de'finit la fonction
appele'e. Le sous-domaine \verb/arguments/ est une liste de sous-domaines
\verb/expression/ qui repre'sente les arguments d'appel de la fonction.
}

\subsubsection{Control Flow Graph (cont.)}
\label{subsubsection-control}
\index{Control}

\domain{Control = statement x predecessors:control* x successors:control*}
\domain{Controlmap = persistant statement->control}
{ 

Domain \verb/control/ is the type of nodes used to implement the CFG
implied by an unstructured instruction (see Domain
\verb/unstructured/). Each node points towards a statement which can
represent an arbitrary large piece of structured code. GOTO statements
are eliminated and represented by arcs. Nodes are doubly linked. Each
node points towards its successors (at most 2) and towards its
predecessors. The hierarchical nature of
domain \verb/statement/ is used to hide local branches from higher and
lower level pieces of code. The whole unstructured area of the code is
seen as a unique atomic statement from above, and is entirely ignored
from under. This explains the mutual recursion between \verb/control/
and \verb/statement/ (via \verb/instruction/).

\begin{figure}
\begin{center}
\mbox{\psfig{file=unstructured.idraw,width=\hsize}}
\end{center}
\caption{Control Flow Graph}
\label{figure-unstructured}
\end{figure}

All statements but \verb/test/have only one successor. The first
successor of \verb/test/ is the successor when the test condition is
evaluated to true.  And the other way round for the second one. The exit
node (see domain \verb/unstructured/) has no successor. The entry node
as well as all other nodes may have an unlimited number of precedessors.

All nodes of a CFG can be visited in a meaningless order using the
\verb/CONTROL_MAP/ macros. Look for an example in library \verb/control/
because auxiliary data structures must be decraled and freed.

}

\domain{persistant_statement_to_control = persistant statement -> persistant control}
{
  Used for example in use_def_elimination() to store the eventual control
  father of a statement in order to travel on the control graph
  associated to a statement. The persistance is needed to avoid
  freeing the control graph when the mapping is freed.
}

% \section{Repre'sentation des expressions}
\section{Expressions}
\label{expression}
\index{Expression}

\domain{Expression = syntax x normalized}
{ 
Le domaine \verb/expression/ permet de stocker les expressions.  Le
sous-domaine {\tt syntax} contient la description de l'expression telle
qu'elle apparai^t dans le texte source du programme. Le sous-domaine {\tt
normalized} contient une forme compile'e des expressions line'aires,
sous forme de Pvecteur.

Si le sous-domaine {\tt normalized} contient la valeur
{\tt normalized\_undefined}, cela signifie que la fonction de line'arisation
n'a pas e'te' appele'e pour cette expression; cela {\bf ne} signifie
{\bf pas}
que l'expression n'est pas line'aire.
}

\subsection{Abstract Tree of an Expression: Syntax}
\label{subsection-syntax}
\index{Syntax}

\domain{Syntax = reference + range + call}
{
Le domaine \verb/syntax/ permet de repre'senter les expressions telles
qu'elles apparaissent dans le texte source du programme. Un
\verb/syntax/ est soit une \verb/reference/ a` un e'le'ment de tableau
(on rappelle que les scalaires sont des tableaux a` 0 dimension) , soit
un \verb/call/ a` une fonction (les ope'rateurs sont repre'sente's par
des fonctions pre'-de'finies, y compris l'assignation; c'est pourquoi le
domaine {\tt call} est de'fini dans la section {\em Instructions}), soit
un \verb/range/, dans le cas des expressions bornes de boucles (le domaine
{\tt range} est pre'sente' au niveau des boucles, aussi dans la section
{\em Instructions}).
}

\subsubsection{Reference}
\label{subsubsection-reference}
\index{reference}

\domain{Reference = variable:entity x indices:expression*}
{
Le domaine \verb/reference/ est utilise' pour repre'senter une
re'fe'rence a` un e'le'ment de tableau\footnote{Les variables scalaires
e'tant repre'sente'es par des tableaux de dimension 0, les re'fe'rences
a des scalaires sont aussi prises en compte. Elles contiennent une liste
vide d'expressions d'indices.}.  Le sous-domaine \verb/variable/
contient une entite' de'finissant la variable re'fe'rence'e. Le
sous-domaine \verb/indices/ contient une liste {\tt expression}s qui sont les
indices de la re'fe'rence.
}

\subsubsection{Range}

See Section~\ref{range}.

\subsubsection{Function Call}

All operators, including assignment, are repreented as function calls
with side effect. See Section~\ref{subsubsection-call}.

\subsection{Affine Representation of an Expression}
\label{subsection-normalized}
\index{Normalized}

\domain{Normalized = linear:Pvecteur + complex:unit}
{ Le domaine {\tt normalized} permet de savoir si une expression est une
expression line'aire construite sur les variables simples entie`res
(sous-domaine {\tt linear}) ou non (sous-domaine {\tt complex}).

Le sous-domaine {\tt complex} est utilise' si l'expression n'est pas
line'aire (ex: {\tt I*J+4}) ou si elle est line'aire mais contient autre
chose que des re'fe'rences a` des scalaires entiers (ex: {\tt T(I-1) +
T(I) + T(I+1)}).

La forme normalise'e n'existe pas si l'expression n'a pas encore e'te'
examine'e. }

% \section{Effets des instructions}
\section{Memory Effects of Statements}
\label{effects}

\subsection{Effects}
\label{subsection-effects}
\index{Effects}

\domain{Effects =  effects:effect*}
{}

\subsection{Effect}
\label{subsection-effect}
\index{Effect}\index{Region}

\domain{Effect =  persistant reference x action x approximation x context:transformer}
{
Le domaine \verb/effect/ est utilise' pour repre'senter les effets d'un
statement sur les variables du module. Les effets des instructions sont
le point de de'part du calcul des Def-Use Chains, des de'pendances, des
Summary Data Flow Information, et des re'gions.

Le sous-domaine {\tt reference} indique sur quelle variable, scalaire ou
tableau, a lieu un effet, le sous-domaine {\tt action} pre'cise si la
re'fe'rence est lue ou e'crite, et le sous-domaine {\tt approximation}
permet de savoir si la re'fe'rence (pour les effets) ou l'ensemble des
e'le'ments de tableaux de'fini par le contexte (pour les re'gions) est lu ou
e'crit a` coup su^r ou non. 

Lors de la traduction interproce'durale des effets ou des re'gions portant
sur une variable globale non de'clare'e dans le module appelant ou sur une
variable statique apparaissant dans un {\tt SAVE}, il est important de
re'cupe'rer une re'fe'rence qui sera la me^me pour tous les sites d'appels.
Pour cela on choisit un module de re'fe'rence (dans lequel le common ou le
save est de'clare'), dont le nom pre'fixera le nom de la variable. Le choix
de ce module n'est pas trivial. A` l'heure actuel. il s'agit du nom du
module de la premie`re variable apparaissant dans la liste des variables du
common ({\tt ram_section(storage_ram(entity_storage(<my_common>)))}). Selon
l'ordre dans lequel les modules dont analyse's, le nom de module trouve'
sera diffe'rent. Il faudrait en re'alite' prendre le premier nom de module
dans l'ordre lexicographique des modules appele's (callees).

Le sous-domaine {\tt reference} permet de pre'ciser qu'un effet n'a lieu que
sur un sous-tableau en utilisant un {\tt range} comme expression d'indice.
Ceci est utilise' lors de la traduction des effets cumule's d'un proce'dure
en les effets propres d'un call site. Pour les re'gions, ce sous-domaine
pre'cise l'entite' concerne'e ainsi que la liste des variables $\phi$ qui
de'crivent ses dimensions.

Le sous-domaine {\tt context} n'est utilise' que pour repre'senter les
effets des instructions par des re'gions (telles que de'finies par Re'mi
Triolet).

}

\subsection{Nature of an Effect}
\label{subsection-action}
\index{Action}

\domain{Action = read:unit + write:unit}
{
Deux types d'effets sont utilise's dans les conditions de Bernstein
et dans les conditions propres a` chaque transformation de programme:
la lecture d'une variable et son e'criture.
}

\subsection{Approximation of an Effect}
\label{subsection-approximation}
\index{Approximation}

\domain{Approximation = may:unit + must:unit}
{
La pre'sence de tests et boucles ne permet pas de de'terminer en ge'ne'ral
si une variable est effectivement lue ou e'crite lors de l'exe'cution
d'un {\tt statement}. Il se peut me^me que certaines exe'cutions
y acce`dent et que d'autres n'y fassent pas re'fe'rence. Les effets
calcule's sont alors de type {\tt may}.

Dans quelques cas particuliers, comme une affectation simple {\tt I = 2},
l'effet est certain ({\em must}). Il peut alors e^tre utilise' dans
le calcul des {\em use-def chains} pour effectuer un {\em kill} sur les
variables scalaires. Un effet {\em must} sur un tableau ne signifie pas
que tout le tableau est lu ou e'crit mais qu'au moins un de ses
e'le'ments l'est.
}

\section{Analyse se'mantique}
\label{semantics}

\subsection{Transformer}
\label{subsection-transformer}
\index{Transformer}

\domain{Transformer = arguments:entity* x relation:predicate}
{
Le domaine {\tt transformer} de'finit une relation entre deux e'tats
me'moire. Cette relation
porte sur les valeurs des variables scalaires entie`res d'un module ou
des variables globales au programme.

Les variables qui apparaissent dans la liste des arguments sont celles
qui ont e'te' modifie'es entre les deux e'tats. Deux valeurs
sont donc associe'es a` chacune d'entre elles: la pre- et la
post-valeur.  Les post-valeurs sont porte'es par les entite's
elles-me^mes. Les pre'-valeurs sont porte'es par des entite's
spe'ciales. Les variables scalaires entie`res qui ne sont pas modifie'es
et qui n'apparaissent donc pas dans la liste des arguments n'ont qu'une
seule valeur, porte'e par l'entite' correspondant a` la variable.

La relation est de'finie par des e'galite's et des ine'galite's
line'aires entre valeurs.

Deux types de transformers sont utilise's. Le premier est propre a` un
{\tt statement} et donne une abstraction de son effet sur les variables
entie`res. Les variables qui apparaissent dans la liste des arguments
sont celles qui sont affecte'es lors de son exe'cution.  Le second,
aussi associe' a` un {\tt statement}, donne une relation entre l'e'tat
initial d'un module et l'e'tat pre'ce'dent l'exe'cution de ce {\tt statement}.

Les transformers ne sont de'finis qu'apre`s une phase d'analyse se'mantique.
}

\subsection{Predicate}
\label{subsection-predicate}
\index{Predicate}\index{Precondition}\index{Transformer}

\domain{Predicate = system:Psysteme}
{
Le domaine {\tt predicate} de'finit une relation entre valeurs de
variables scalaires entie`res. Son interpre'tation est fonction de
son utilisation. Il peut s'agir soit d'un pre'dicat valable en
un point du programme (i.e. un invariant), soit d'un pre'dicat
valable entre deux points du programme. Il s'agit alors d'une
abstraction d'une commande, c'est-a`-dire d'un {\tt transformer}.
}

\newpage

\section*{Annexe: re'capitulatif ri.newgen}

\begin{verbatim}
--         --------------------------------------------------------
--         --------------------------------------------------------
--
--                                  WARNING
--
--                THIS FILE HAS BEEN AUTOMATICALLY GENERATED
--
--                             DO NOT MODIFY IT
--
--         --------------------------------------------------------
--         --------------------------------------------------------

-- Imported domains
-- ----------------

-- External domains
-- ----------------
external Psysteme ;
external Pvecteur ;

-- Domains
-- -------
action = read:unit + write:unit ;
approximation = may:unit + must:unit ;
area = size:int x layout:entity* ;
basic = int:int + float:int + logical:int + overloaded:unit + complex:int + string:value ;
call = function:entity x arguments:expression* ;
callees = callees:string* ;
code = declarations:entity* x decls_text:string ;
constant = int + litteral:unit ;
control = statement x predecessors:control* x successors:control* ;
dimension = lower:expression x upper:expression ;
effect =  persistant reference x action x approximation x context:transformer ;
effects =  effects:effect* ;
execution = sequential:unit + parallel:unit ;
expression = syntax x normalized ;
formal = function:entity x offset:int ;
functional = parameters:parameter* x result:type ;
instruction = block:statement* + test + loop + goto:statement + call + unstructured ;
loop = index:entity x range x body:statement x label:entity x execution x locals:entity* ;
mode = value:unit + reference:unit ;
normalized = linear:Pvecteur + complex:unit ;
parameter = type x mode ;
predicate = system:Psysteme ;
ram = function:entity x section:entity x offset:int x shared:entity* ;
range = lower:expression x upper:expression x increment:expression ;
reference = variable:entity x indices:expression* ;
statement = label:entity x number:int x ordering:int x comments:string x instruction ;
storage = return:entity + ram + formal + rom:unit ;
symbolic = expression x constant ;
syntax = reference + range + call ;
tabulated entity = name:string x type x storage x initial:value ;
test = condition:expression x true:statement x false:statement ;
transformer = arguments:entity* x relation:predicate ;
type = statement:unit + area + variable + functional + unknown:unit + void:unit ;
unstructured = control x exit:control ;
value = code + symbolic + constant + intrinsic:unit + unknown:unit ;
variable = basic x dimensions:dimension* ;
\end{verbatim}

\newpage

% Cross-references for points and keywords

\input{ri.ind}

\end{document}
\end


\end{document}
\end


\end{document}
\end


\section{Biblioth�que de manipulation de la {\em RI} ({\em ri-util})}

\input{ri-util.listing}

\section{Initialisation des intrins�ques ({\em Bootstrap})}

\input{bootstrap.listing}

\section{Structuration du graphe de contr�le ({\em Control})}

\input{control.listing}

\section{Maintien de la coh�rence ({\em Makefile})}

%%
%% $Id$
%%
%% Copyright 1989-2010 MINES ParisTech
%%
%% This file is part of PIPS.
%%
%% PIPS is free software: you can redistribute it and/or modify it
%% under the terms of the GNU General Public License as published by
%% the Free Software Foundation, either version 3 of the License, or
%% any later version.
%%
%% PIPS is distributed in the hope that it will be useful, but WITHOUT ANY
%% WARRANTY; without even the implied warranty of MERCHANTABILITY or
%% FITNESS FOR A PARTICULAR PURPOSE.
%%
%% See the GNU General Public License for more details.
%%
%% You should have received a copy of the GNU General Public License
%% along with PIPS.  If not, see <http://www.gnu.org/licenses/>.
%%
\documentclass{article}

\usepackage[latin1]{inputenc}
\usepackage{newgen_domain}

%%\input{/usr/share/local/lib/tex/macroslocales/Dimensions.tex}

\title{PIPS: M�canisme de gestion de la coh�rence et d'encha�nement
automatique des phases (pipsmake)}
\author{Fran�ois Irigoin \\
    Pierre Jouvelot \\
    R�mi Triolet\\
\\
    CAI, �cole des mines de Paris}

\begin{document}
\maketitle
%% \sloppy

\section*{Introduction}

L'interproc�duralit� et l'interactivit� de PIPS rendent n�cessaire
la gestion d'informations partielles, li�es � un module particulier
et � une phase particuli�re, qui peuvent �tre r�utilis�es dans
diff�rents calculs et lors de plusieurs sessions successives.

L'encha�nement des calculs de ces informations et les tests de coh�rence
auraient pu �tre r�parties dans chacune des phases et analyses de PIPS.
Il a sembl� pr�f�rable de centraliser dans une biblioth�que la gestion
des d�pendances entre phases et le maintien de la coh�rence entre
informations.

La biblioth�que {\tt pipsmake} offre deux points d'entr'e principaux:
{\tt make} et {\tt apply}. Le premier permet d'obtenir une information
particuli�re sans avoir � se pr�occuper de calculer toutes les
informations qui sont n�cessaires � son calcul. Le second permet
l'�valuation d'une r�gle particuli�re.

Quand plusieurs r�gles permettent de calculer un objet particulier,
un m�canisme d'activation permet de d�finir celle qui doit �tre
utilis�e. Ceci est indispensable pour pouvoir traiter les appels
r�cursifs qu'entra�nent les encha�nements de r�gles sans avoir �
demander trop d'informations de param�trage � l'utilisateur.

Un jeu de r�gles et de ressources particuli�res peuvent �tre d�finis
statiquement dans un fichier de nom {\tt pipsmake.rc}. La biblioth�que
{\tt pipsmake} contient des modules permettant de lire un tel fichier
pour initialiser un ensemble de r�gles en m�moire et d'�crire sur
disque un ensemble de r�gles dans un format compatible avec leur
relecture.

Par d�fault, la premi�re r`egle produisant une ressource quelconque
est activ�e. Lorsque plusieurs r`egles sont disponibles pour produire
une m�me ressource, il faut que cette ressource soit l'unique produite
par cette r�gle. Ainsi est-il coh�rent de changer dynamiquement de
r�gles actives.

Ceci est automatiquement rendu possible par certains interfaces
interactifs  de pips, � condition que les r�gles disponibles comme
alternatives ainsi que leur ressource produite aient un nom d'alias.

Nous pr�sentons successivement les structures de donn�es utilis�es
pour stocker en m'emoire un ensemble de r�gles
de d�rivation, des r�gles de d�rivation, des ressources virtuelles
et des ressources r�elles.

\section{Ensemble des r�gles de d�rivation}

\domain{Makefile = rules:rule* x active\_phases:string*}
{Le domaine {\tt makefile} est utilis� par le driver de haut niveau
pour d�crire les d�pendances entre les diff�rentes phases de Pips. Un
{\tt Makefile} est une liste de r�gles ({\tt rule}), chaque r�gle
d�crivant une des phases de Pips. En outre, le {\tt Makefile} donne la
liste des phases qui sont actives � l'instant pr�sent {\tt
active\_phases}. Rappelons que chaque type de ressources peut
�ventuellement �tre produit par diff�rentes phases, mais qu'une seule
phase est utilisable � un instant donn�.

Attention, les nouvelles fonctionalit�s de production de resources
multiples impliquent une ambiguit� sur la notion de r�gle active;
les r�gles actives pouvant �tre actives pour un sous-ensemble des
r�gles qu'elles produisent (en particulier dans le cas de r�gles
partiellement cycliques). }

\section{D�finition d'une r�gle particuli�re}

\domain{Rule = phase:string x required:virtual\_resource* x produced:virtual\_resource* x preserved:virtual\_resource* x modified:virtual\_resource* x pre\_transformation:virtual\_resource*}
{Le domaine {\tt rule} permet de d�crire les actions des phases de
  Pips sur les ressources g�r�es par pips-db. Chaque phase
  n�cessite que certaines ressources soient disponibles ({\tt
    required}), elle commence par effectuer d'�ventuelles
  transformations ({\tt pre\_transformation}), puis produit une ou
  plusieurs ressources ({\tt produced}), et en modifie d'autres ({\tt
    modified}). La diff�rence entre les ressources produites et
  celles modifi�es permet au driver d'enchainer les phases dans le
  bon ordre.

Les phases de transformation agissent sur le code des modules ce qui
implique g�n�ralement que les informations qui d�corent ce module
sont perdues. Pourtant, certaines d'entre-elles font des transformations
si mineures que certaines d�corations sont pr�serv�es ({\tt
preserved}). C'est notamment le cas de la privatisation qui pr�servent
toutes les d�corations. Voici la liste des phases de Pips.
\begin{description}
\item[parser] analyse syntaxique et calcul du graphe de contr�le,
\item[linker] �dition des liens,
\item[proper-effects] calculs des effets propres des instructions,
\item[cumulated-effects] calculs des effets cumul�s des instructions,
\item[usedef] calcul des used-def chains et des def-use chains,
\item[privatizer] privatisation des variables,
\item[dgkennedy] calcul du graphe de d�pendances avec les {\em niveaux de
Kennedy}, 
\item[dgwolfe] calcul du graphe de d�pendances avec les {\em vecteurs
de direction de Wolfe},
\end{description}
}

\section{D�finition d'une ressource virtuelle}

Les ressources virtuelles sont des variables pouvant �tre instanti�es
en une ressource r�elle ou en une liste de ressource r�elle.

\domain{Virtual\_resource = name:string x owner }
{Le domaine {\tt virtual\_resource} permet de d�signer une ressource
lue ou modifi�e par une phase en pr�cisant en plus de la nature de la
ressource ({\tt datum}) si la ressource acc�d�e est celle attach�e au
module, au programme, aux modules appel�s par le module auquel la phase
est appliqu�e ou � celui qui l'appelle ({\tt owner}). Voici la liste
de toutes les ressources calculables par Pips.
\begin{description}
\item[source] fichier source Fortran d'un module; r�sultat de
l`initialisation;
\item[code] code d'un module avec graphe de contr�le structur�;
r�sultat du controlizer et du parser;
\item[entities] entites du programme; r�sultat de l`initialisation, du
parser et du linker;
\item[callees] modules appel�s directement par un module; r�sultat du
linker; 
\item[proper-effects] effets propres des instructions pour un module; le
terme {\em propre} signifie que les effets des blocs des instructions
compos�es (boucles, tests, ...) ne sont pas comptabilis�s; r�sultat
de proper-effects; 
\item[cumulated-effects] effets cumul�s des instructions pour un
module; le terme {\em cumul�} signifie que les effets des blocs des
instructions compos�es (boucles, tests, ...) sont comptabilis�s;
r�sultat de cumulated-effects; 
\item[sdfi] {\em summary data flow information} d'un module; c'est un
r�sum� des effets cumul�s de l'instruction bloc du module; r�sumer
les effets consiste � �liminer les effets sur les variables locales du
module et, dans le cas des tableaux, � globaliser chaque effet en y
supprimant les expressions d'indices; r�sultat de cumulated-effects;
\item[chains] {\em use-def} et {\em def-use chains} d'un module;
r�sultat de usedef;
\item[dgkennedy] graphe de d�pendances avec les {\em niveaux de
Kennedy}; 
\item[dgwolfe] graphe de d�pendances avec les {\em vecteurs de direction de
Wolfe};
\end{description}
}

\domain{Owner = \{ program , module , main , callees , callers , all ,  select , compilation_unit \}}
{Le domaine {\tt owner} permet de pr�ciser dans une r�gle de
d�pendances a quels modules sont rattach�es les ressources lues,
�crites, produites ou pr�serv�es. Ce peut �tre le module lui-m�me
({\tt module}), les modules appel�s par le module auquel la phase est
appliqu�e ({\tt callees}) ou � ceux qui l'appelle ({\tt caller}), ou
bien tous les modules du programme consid�r� ({\tt all}). Le programme
({\tt program}) lui-m�me caract�rise en fait un espace de travail
particulier et donc indirectement l'ensemble des modules sur lesquels on
souhaite travailler. Le nom d'un programme n'est g�n�ralement pas
d�riv� automatiquement du code source parce qu'on peut tr�s bien
souhaiter d�river plusieurs versions d'un m�me code s�quentiel
original et donner � chaque version un nom diff�rent.

Cet attribut suppl�mentaire des d�pendances
permet au top-level driver de g�rer les appels multiples rendus
n�cessaires par l'interproc�duralit� de Pips et d'�liminer
l'auto-r�cursion du gestionnaire de la base de donn�es.

\texttt{select} is a fake owner, to be used to select (or activate) rules
from other pipsmake rules. Should only be used with the bang rules?
}

\section{Ressources r�elles}

Les ressources r�elles correspondent � un ensemble de donn�es particulier
qui a �t� produit pour un module ou un programme particulier par une
phase particuli�re. Les ressources virtuelles prennent leurs valeurs
parmi ces ressources r�elles, mais les r�gles de d�rivation de
pipsmake sont toujours g�n�riques et donc toujours d�finies en terme
de ressources virtuelles.

\domain{Real\_resource = resource\_name:string x owner\_name:string}
{Le domaine {\tt real\_resource} est un domaine priv� pour pipsmake qui
sert � concr�tiser un ensemble de ressources virtuelles pour un
programme et un module donn�s.}

\end{document}
\end


\section{Exploitation des r�gles ({\em Pipsmake})}

\input{pipsmake.listing}

\section{Base de donn�es objet ({\em Database})}

%%
%% $Id$
%%
%% Copyright 1989-2010 MINES ParisTech
%%
%% This file is part of PIPS.
%%
%% PIPS is free software: you can redistribute it and/or modify it
%% under the terms of the GNU General Public License as published by
%% the Free Software Foundation, either version 3 of the License, or
%% any later version.
%%
%% PIPS is distributed in the hope that it will be useful, but WITHOUT ANY
%% WARRANTY; without even the implied warranty of MERCHANTABILITY or
%% FITNESS FOR A PARTICULAR PURPOSE.
%%
%% See the GNU General Public License for more details.
%%
%% You should have received a copy of the GNU General Public License
%% along with PIPS.  If not, see <http://www.gnu.org/licenses/>.
%%

%% newgen domain for pips detabase management.

\documentclass[a4paper]{article}

\usepackage[latin1]{inputenc}
\usepackage{newgen_domain}
\usepackage[backref,pagebackref]{hyperref}

\title{Pips Database Manager}

\begin{document}

\maketitle

\section{Old database structures}

\domain{Database = name:string x directory:string x resources:resource*}
{ Le domaine {\tt database} est utilis� par pips-db pour d�crire
l'�tat d'un programme utilisateur. Ce domaine contient son nom ({\tt
name}), le r�pertoire dans lequel il a �t� cr��, et les
informations qui ont �t� calcul�es pour les diff�rents modules ({\tt
resources}). En fait, le nom de la base et le nom de la directory
utilis�e sont directement li�, en pratique. Le nom de la directory est
le nom de la base suffis� par \textsc{database}.

Un �l�ment de type {\tt resource} est ajout� � la liste {\tt resources}
pour chaque objet calcul� pour ce programme par les phases d'analyse ou
de transformation de Pips. 

La biblioth�que qui exploite cette structure de donn�es est
\texttt{pipsdbm}.  Elle devrait �tre nomm�e \texttt{database-util} si une
r�gle de nommage �tait observ�e.

}

\domain{Resource = name:string x owner\_name:string x status x time:int x file\_time:int}
{ Le domaine {\tt resource} est utilis� par pips-db pour d�crire l'une
des informations suceptibles d'�tre calcul�es par Pips pour un module
ou un programme. Pour chaque information, il faut conna�tre son nom
({\tt name}), savoir qui elle d�core ({\tt owner\_name}), savoir si elle
est pr�sente en m�moire ou rang�e dans un fichier ({\tt status}),
connaitre sa date de cr�ation logique ({\tt time}), et eventuellement
la date (Unix) de cr�ation du fichier correspondant ({\tt file\_time}).

Le {\tt name} de la ressource est en fait un type et aurait pu �tre
d�fini comme un type �num�r�. C'est par souci de simplicit� et
g�n�ricit� de pipsdbm que nous avons choisi de le d�finir comme par
une chaine de caract�res. C'est ce {\tt name} qui permet notamment �
pipsdbm de choisir la fonction pour lire, �crire ou lib�rer une
ressource.

A un moment donn�, chaque ressource est identifi�e demani�re unique
par son nom, \texttt{name}, et le nom de son {\em owner}, \texttt{owner\_name}.

Nous avons longement discute de l'utilit� d'avoir un owner\_type pour
preciser ce que d�core la ressource: un programme (cas des entit�s),
un module (cas de toutes les autres ressources), une boucle, une
instruction, etc. Nous y avons renonc� car � l'heure actuelle, nous
n'avons que des ressources attachables � des programmes ou � des
modules, et que cette information peut �tre d�duite du {\tt name} de
la ressource. En fait, nous n'avons qu'une ressource attachable � un
programme, ce sont les entit�s (et les \texttt{user\_file}, mais ils sont
largement maltrait�s pour le moment).}

\domain{Status = memory:string + file:string}
{ Le domaine {\tt status} est utilis� par pips-db pour savoir si la
ressource concern�e peut r�sider en m�moire, auxquel cas elle peut se
trouver soit en m�moire, si elle a �t� r�clam�e par un processus,
soit sur disque, ou bien si elle est constitu�e d'un fichier, qui n'est
a priori jamais charg�e en m�moire. Dans
ce dernier cas, le sous-domaine {\tt file} donne le nom de ce fichier.
Le nom du fichier doit �tre relatif au workspace, s'il se trouve dans
le workspace, et doit �tre absolu, s'il se trouve hors du workspace, de
sorte qu'on puisse faire des \texttt{mv} et des \texttt{cp -pr} sur le workspace.
Si la ressource est en m�moire, le sous-domaine {\tt memory} contient
un pointeur vers cette ressource.

How resources in a file a and file resources differ? It cannot be seen
from this description! The list of the laters is explicitely managed
somewhere in pipsdbm! 
}

\end{document}



\section{Gestion de la base de donn�es ({\em Pipsdbm})}

\input{pipsdbm.listing}

\section{Param�tres d'ex�cution ({\em Property})}

\documentclass[a4]{article}

\usepackage[latin1]{inputenc}

\newcommand{\domain}[2]{\paragraph{{#1}}\paragraph{}{\em #2}}

\begin{document}
\sloppy
D�finition d'une propri�t�.

Une liste de propri�t�s est lue dans un fichier au d�marrage de
Pips. Cette liste permet de modifier le comportement par
d�faut de Pips. Chaque propri�t� est significative pour au moins une
passe.

Ces propri�t�s ne doivent pas �tre directement modifi�es par un
utilisateur. Elles sont normalement positionn�es par le biais des
s�lections de r�gles (cf. {\em makefile} et {\em pipsmake}). Un
utilisateur averti peut s'y risquer mais le maintien de la coh�rence
entre objets n'est plus garanti par {\em pipsmake}.

Une propri�t� peut �tre un entier, un bool�en ou une cha�ne de
caract�res. Le fichier de r�f�rence se trouve dans Production/Lib
sous le nom de {\em properties.rc}. Ce fichier est lu en premier mais les
valeurs des propri�t�s peuvent �tre modifi�es en fonction d'un
deuxi�me fichier {\em properties.rc} se trouvant dans le r�pertoire
courant.

\domain{property = int + bool + string}
{}
\end{document}
\end


\section{Gestion des param�tres ({\em Properties})}

\input{properties.listing}

\section{Parall�liseur batch: {\em Pips}}

\input{pips.listing}

\section{Biblioth�que de base ({\em Misc})}

\input{misc.listing}

\end{document}
