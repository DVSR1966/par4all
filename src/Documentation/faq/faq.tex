%% 
%% $Id$
%%
%% PIPS faq
%%
%% $Log: faq.tex,v $
%% Revision 1.2  1998/07/27 13:47:21  irigoin
%% Latex bugs fixed
%%
%% Revision 1.1  1998/07/27 11:21:48  irigoin
%% Initial revision
%%
%

%% titlepage,
\documentclass[a4paper]{article}

\usepackage{psfig,amstext,alltt,html}

\title{\bf {\Huge PIPS} \\ Frequently Asked Questions}

\author{
\begin{tabular}{rl}
  Fran�ois & IRIGOIN
\end{tabular}
}

\date{September 1997}

\renewcommand{\indexname}{Index}
\makeindex

\newcommand{\PIPSMAKERC}{\htmladdnormallink{\texttt{pipsmake-rc}}{http://www.cri.ensmp.fr/pips/pipsmake-rc.html}}

\newcommand{\PIPS}{\htmladdnormallink{{\em PIPS}}{http://www.cri.ensmp.fr/pips}}
\newcommand{\WPIPS}{\htmladdnormallink{{\em WPIPS}}{http://www.cri.ensmp.fr/pips/wpips-epips-user-manual/wpips-epips-user-manual.html}}
\newcommand{\TPIPS}{\htmladdnormallink{\texttt{tpips}}{http://www.cri.ensmp.fr/pips/line-interface.html}}
\newcommand{\PROPERTIES}{\htmladdnormallink{\emph{properties}}{http://www.cri.ensmp.fr/pips/properties-rc}}

\begin{document}
\maketitle

\begin{latexonly}
  \clearpage
  \tableofcontents  
\end{latexonly}

\newpage
\section*{Welcome}

Welcome to the not so frequently asked questions about PIPS. Not so
frequently, because the PIPS user population, however growing, still is
small. Please send more questions to \verb/pips@cri.ensmp.fr/.

\section{Some of my source code is missing. What can I do?}
\index{Source File}

PIPS is designed as a whole application compiler and it complains about
missing source code for leaf modules in the call graph. In fact PIPS may
even generate a user error and stop if the missing piece of code is needed
to answer the user request. For instance, PIPS cannot compute the side
effects of procedure \verb/P/ without analyzing the code procedure
\verb/Q/ called by \verb/P/.

Also,

\section{Why is PIPS parser so limited?}
\index{Parser}

Parsing is divided in multiple stages.

\section{My include files are in a different directory}
\index{Include}

FC: Use the Shell environment variable \verb/PIPS_CPPFLAGS/ and the usual
\verb/-I/ option to specify directories to lookup for include files.
For instance, assuming that csh is used:

\begin{quote}
\verb|setenv PIPS_CPP_FLAGS "-I ../includes -I ../../generic/includes"|
\end{quote}

The current directory is searched first. For more information, see the
preprocessing sections in \PIPSMAKERC{}.

\section{How can I execute the transformed code?}

The transformed code is stored in the current workspace by an explicit
close or by exiting \PIPS{}. The best way to do it is to open the
workspace with \TPIPS{}.

\section{How can I replay a session?}
\index{logfile\_to\_tpips}

Each PIPS workspace contains a log. This log can be processed using the
{\tt logfile\_to\_tpips} Shell script. The output is a {\tt tpips} script
which can be replayed with the {\tt tpips} interface.

Some commands, such as {\tt cd}, {\tt open}, {\tt close}, {\tt delete},
may destroy the log file or change the current workspace. It is not
possible to recreate a global log from different workspaces.

\section{What should I read to use \PIPS{}?}

You might decide to get started using the \WPIPS{} X-Window interface and a
small Fortran program. On-line help is available.

\section{How can I process several Fortran files?}

The answer depends on the PIPS interface used. With the basic Shell
interface, the argument of the {\tt -f} option can be quoted to use Shell
filename expansion. For instance, the command

\begin{quote}
{\tt Init -d -f "*.f Src/*.f" foo}
\end{quote}

deletes a possibly existing {\tt foo} workspace and creates a new one with
all Fortran files in the current directory and in the {\tt Src}
subdirectory.

The window interface, {\tt wpips}, let you select a number of Fortran
files when you create a workspace but these files must all belong to the
current directory.

The script interface, {\tt tpips}, let you select any number of Fortran
files using the standard Shell syntax. For instance, the above {\tt Init}
command would be:

\begin{quote}
{\tt delete foo}\\
{\tt create foo *.f}
\end{quote}

The standard convention is used for include files.

\section{How can I get the dependence graphs of several modules in one command?}
\index{Dependence Graph}
\index{DG}

The script interface, {\tt tpips}, let you require several data structures
in one command. For instance, you can compute the dependence graphs for
all modules in the current workspace with:

\begin{quote}
{\tt display DG\_FILE[\%ALL]}
\end{quote}

It also is possible to specify a list of relevant modules:

\begin{quote}
{\tt display DG\_FILE[A B C]}
\end{quote}

It is not possible to submit such requests with the other PIPS user interfaces.

\section{PIPS tells me to set a property? What is it? How can I do it}
\index{Property}

Properties are used to encapsulate global variables and to fine tune the
behavior of PIPS. This fine tuning was not intended for PIPS users but for
PIPS developpers. However the property mechanism has been used to extend
PIPS parser and users should be able to define {\em some} properties (see
the
\htmladdnormallink{\emph{property}}{http://www.cri.ensmp.fr/pips/properties-rc}
documentation.

Default values for properties are defined in file
\verb+$PIPS_ROOT/Share/properties.rc+. when \PIPS{} is installed. However,
they can be modified in every directory by creating a file called {\tt
  properties.rc}. Properties can also be modified dynamically with the
most recent \PIPS user interfaces, \TPIPS{} and jpips.

\TPIPS users can also modifies the property in their home directory using
a file called \verb+.tpipsrc+.

\newpage


{\small
\bibliographystyle{plain}
\bibliography{faq}
}

%% 
%% $Id$
%%
%% PIPS faq
%%
%% $Log: faq.tex,v $
%% Revision 1.2  1998/07/27 13:47:21  irigoin
%% Latex bugs fixed
%%
%% Revision 1.1  1998/07/27 11:21:48  irigoin
%% Initial revision
%%
%

%% titlepage,
\documentclass[a4paper]{article}

\usepackage{psfig,amstext,alltt,html}

\title{\bf {\Huge PIPS} \\ Frequently Asked Questions}

\author{
\begin{tabular}{rl}
  Fran�ois & IRIGOIN
\end{tabular}
}

\date{September 1997}

\renewcommand{\indexname}{Index}
\makeindex

\newcommand{\PIPSMAKERC}{\htmladdnormallink{\texttt{pipsmake-rc}}{http://www.cri.ensmp.fr/pips/pipsmake-rc.html}}

\newcommand{\PIPS}{\htmladdnormallink{{\em PIPS}}{http://www.cri.ensmp.fr/pips}}
\newcommand{\WPIPS}{\htmladdnormallink{{\em WPIPS}}{http://www.cri.ensmp.fr/pips/wpips-epips-user-manual/wpips-epips-user-manual.html}}
\newcommand{\TPIPS}{\htmladdnormallink{\texttt{tpips}}{http://www.cri.ensmp.fr/pips/line-interface.html}}
\newcommand{\PROPERTIES}{\htmladdnormallink{\emph{properties}}{http://www.cri.ensmp.fr/pips/properties-rc}}

\begin{document}
\maketitle

\begin{latexonly}
  \clearpage
  \tableofcontents  
\end{latexonly}

\newpage
\section*{Welcome}

Welcome to the not so frequently asked questions about PIPS. Not so
frequently, because the PIPS user population, however growing, still is
small. Please send more questions to \verb/pips@cri.ensmp.fr/.

\section{Some of my source code is missing. What can I do?}
\index{Source File}

PIPS is designed as a whole application compiler and it complains about
missing source code for leaf modules in the call graph. In fact PIPS may
even generate a user error and stop if the missing piece of code is needed
to answer the user request. For instance, PIPS cannot compute the side
effects of procedure \verb/P/ without analyzing the code procedure
\verb/Q/ called by \verb/P/.

Also,

\section{Why is PIPS parser so limited?}
\index{Parser}

Parsing is divided in multiple stages.

\section{My include files are in a different directory}
\index{Include}

FC: Use the Shell environment variable \verb/PIPS_CPPFLAGS/ and the usual
\verb/-I/ option to specify directories to lookup for include files.
For instance, assuming that csh is used:

\begin{quote}
\verb|setenv PIPS_CPP_FLAGS "-I ../includes -I ../../generic/includes"|
\end{quote}

The current directory is searched first. For more information, see the
preprocessing sections in \PIPSMAKERC{}.

\section{How can I execute the transformed code?}

The transformed code is stored in the current workspace by an explicit
close or by exiting \PIPS{}. The best way to do it is to open the
workspace with \TPIPS{}.

\section{How can I replay a session?}
\index{logfile\_to\_tpips}

Each PIPS workspace contains a log. This log can be processed using the
{\tt logfile\_to\_tpips} Shell script. The output is a {\tt tpips} script
which can be replayed with the {\tt tpips} interface.

Some commands, such as {\tt cd}, {\tt open}, {\tt close}, {\tt delete},
may destroy the log file or change the current workspace. It is not
possible to recreate a global log from different workspaces.

\section{What should I read to use \PIPS{}?}

You might decide to get started using the \WPIPS{} X-Window interface and a
small Fortran program. On-line help is available.

\section{How can I process several Fortran files?}

The answer depends on the PIPS interface used. With the basic Shell
interface, the argument of the {\tt -f} option can be quoted to use Shell
filename expansion. For instance, the command

\begin{quote}
{\tt Init -d -f "*.f Src/*.f" foo}
\end{quote}

deletes a possibly existing {\tt foo} workspace and creates a new one with
all Fortran files in the current directory and in the {\tt Src}
subdirectory.

The window interface, {\tt wpips}, let you select a number of Fortran
files when you create a workspace but these files must all belong to the
current directory.

The script interface, {\tt tpips}, let you select any number of Fortran
files using the standard Shell syntax. For instance, the above {\tt Init}
command would be:

\begin{quote}
{\tt delete foo}\\
{\tt create foo *.f}
\end{quote}

The standard convention is used for include files.

\section{How can I get the dependence graphs of several modules in one command?}
\index{Dependence Graph}
\index{DG}

The script interface, {\tt tpips}, let you require several data structures
in one command. For instance, you can compute the dependence graphs for
all modules in the current workspace with:

\begin{quote}
{\tt display DG\_FILE[\%ALL]}
\end{quote}

It also is possible to specify a list of relevant modules:

\begin{quote}
{\tt display DG\_FILE[A B C]}
\end{quote}

It is not possible to submit such requests with the other PIPS user interfaces.

\section{PIPS tells me to set a property? What is it? How can I do it}
\index{Property}

Properties are used to encapsulate global variables and to fine tune the
behavior of PIPS. This fine tuning was not intended for PIPS users but for
PIPS developpers. However the property mechanism has been used to extend
PIPS parser and users should be able to define {\em some} properties (see
the
\htmladdnormallink{\emph{property}}{http://www.cri.ensmp.fr/pips/properties-rc}
documentation.

Default values for properties are defined in file
\verb+$PIPS_ROOT/Share/properties.rc+. when \PIPS{} is installed. However,
they can be modified in every directory by creating a file called {\tt
  properties.rc}. Properties can also be modified dynamically with the
most recent \PIPS user interfaces, \TPIPS{} and jpips.

\TPIPS users can also modifies the property in their home directory using
a file called \verb+.tpipsrc+.

\newpage


{\small
\bibliographystyle{plain}
\bibliography{faq}
}

%% 
%% $Id$
%%
%% PIPS faq
%%
%% $Log: faq.tex,v $
%% Revision 1.2  1998/07/27 13:47:21  irigoin
%% Latex bugs fixed
%%
%% Revision 1.1  1998/07/27 11:21:48  irigoin
%% Initial revision
%%
%

%% titlepage,
\documentclass[a4paper]{article}

\usepackage{psfig,amstext,alltt,html}

\title{\bf {\Huge PIPS} \\ Frequently Asked Questions}

\author{
\begin{tabular}{rl}
  Fran�ois & IRIGOIN
\end{tabular}
}

\date{September 1997}

\renewcommand{\indexname}{Index}
\makeindex

\newcommand{\PIPSMAKERC}{\htmladdnormallink{\texttt{pipsmake-rc}}{http://www.cri.ensmp.fr/pips/pipsmake-rc.html}}

\newcommand{\PIPS}{\htmladdnormallink{{\em PIPS}}{http://www.cri.ensmp.fr/pips}}
\newcommand{\WPIPS}{\htmladdnormallink{{\em WPIPS}}{http://www.cri.ensmp.fr/pips/wpips-epips-user-manual/wpips-epips-user-manual.html}}
\newcommand{\TPIPS}{\htmladdnormallink{\texttt{tpips}}{http://www.cri.ensmp.fr/pips/line-interface.html}}
\newcommand{\PROPERTIES}{\htmladdnormallink{\emph{properties}}{http://www.cri.ensmp.fr/pips/properties-rc}}

\begin{document}
\maketitle

\begin{latexonly}
  \clearpage
  \tableofcontents  
\end{latexonly}

\newpage
\section*{Welcome}

Welcome to the not so frequently asked questions about PIPS. Not so
frequently, because the PIPS user population, however growing, still is
small. Please send more questions to \verb/pips@cri.ensmp.fr/.

\section{Some of my source code is missing. What can I do?}
\index{Source File}

PIPS is designed as a whole application compiler and it complains about
missing source code for leaf modules in the call graph. In fact PIPS may
even generate a user error and stop if the missing piece of code is needed
to answer the user request. For instance, PIPS cannot compute the side
effects of procedure \verb/P/ without analyzing the code procedure
\verb/Q/ called by \verb/P/.

Also,

\section{Why is PIPS parser so limited?}
\index{Parser}

Parsing is divided in multiple stages.

\section{My include files are in a different directory}
\index{Include}

FC: Use the Shell environment variable \verb/PIPS_CPPFLAGS/ and the usual
\verb/-I/ option to specify directories to lookup for include files.
For instance, assuming that csh is used:

\begin{quote}
\verb|setenv PIPS_CPP_FLAGS "-I ../includes -I ../../generic/includes"|
\end{quote}

The current directory is searched first. For more information, see the
preprocessing sections in \PIPSMAKERC{}.

\section{How can I execute the transformed code?}

The transformed code is stored in the current workspace by an explicit
close or by exiting \PIPS{}. The best way to do it is to open the
workspace with \TPIPS{}.

\section{How can I replay a session?}
\index{logfile\_to\_tpips}

Each PIPS workspace contains a log. This log can be processed using the
{\tt logfile\_to\_tpips} Shell script. The output is a {\tt tpips} script
which can be replayed with the {\tt tpips} interface.

Some commands, such as {\tt cd}, {\tt open}, {\tt close}, {\tt delete},
may destroy the log file or change the current workspace. It is not
possible to recreate a global log from different workspaces.

\section{What should I read to use \PIPS{}?}

You might decide to get started using the \WPIPS{} X-Window interface and a
small Fortran program. On-line help is available.

\section{How can I process several Fortran files?}

The answer depends on the PIPS interface used. With the basic Shell
interface, the argument of the {\tt -f} option can be quoted to use Shell
filename expansion. For instance, the command

\begin{quote}
{\tt Init -d -f "*.f Src/*.f" foo}
\end{quote}

deletes a possibly existing {\tt foo} workspace and creates a new one with
all Fortran files in the current directory and in the {\tt Src}
subdirectory.

The window interface, {\tt wpips}, let you select a number of Fortran
files when you create a workspace but these files must all belong to the
current directory.

The script interface, {\tt tpips}, let you select any number of Fortran
files using the standard Shell syntax. For instance, the above {\tt Init}
command would be:

\begin{quote}
{\tt delete foo}\\
{\tt create foo *.f}
\end{quote}

The standard convention is used for include files.

\section{How can I get the dependence graphs of several modules in one command?}
\index{Dependence Graph}
\index{DG}

The script interface, {\tt tpips}, let you require several data structures
in one command. For instance, you can compute the dependence graphs for
all modules in the current workspace with:

\begin{quote}
{\tt display DG\_FILE[\%ALL]}
\end{quote}

It also is possible to specify a list of relevant modules:

\begin{quote}
{\tt display DG\_FILE[A B C]}
\end{quote}

It is not possible to submit such requests with the other PIPS user interfaces.

\section{PIPS tells me to set a property? What is it? How can I do it}
\index{Property}

Properties are used to encapsulate global variables and to fine tune the
behavior of PIPS. This fine tuning was not intended for PIPS users but for
PIPS developpers. However the property mechanism has been used to extend
PIPS parser and users should be able to define {\em some} properties (see
the
\htmladdnormallink{\emph{property}}{http://www.cri.ensmp.fr/pips/properties-rc}
documentation.

Default values for properties are defined in file
\verb+$PIPS_ROOT/Share/properties.rc+. when \PIPS{} is installed. However,
they can be modified in every directory by creating a file called {\tt
  properties.rc}. Properties can also be modified dynamically with the
most recent \PIPS user interfaces, \TPIPS{} and jpips.

\TPIPS users can also modifies the property in their home directory using
a file called \verb+.tpipsrc+.

\newpage


{\small
\bibliographystyle{plain}
\bibliography{faq}
}

%% 
%% $Id$
%%
%% PIPS faq
%%
%% $Log: faq.tex,v $
%% Revision 1.2  1998/07/27 13:47:21  irigoin
%% Latex bugs fixed
%%
%% Revision 1.1  1998/07/27 11:21:48  irigoin
%% Initial revision
%%
%

%% titlepage,
\documentclass[a4paper]{article}

\usepackage{psfig,amstext,alltt,html}

\title{\bf {\Huge PIPS} \\ Frequently Asked Questions}

\author{
\begin{tabular}{rl}
  Fran�ois & IRIGOIN
\end{tabular}
}

\date{September 1997}

\renewcommand{\indexname}{Index}
\makeindex

\newcommand{\PIPSMAKERC}{\htmladdnormallink{\texttt{pipsmake-rc}}{http://www.cri.ensmp.fr/pips/pipsmake-rc.html}}

\newcommand{\PIPS}{\htmladdnormallink{{\em PIPS}}{http://www.cri.ensmp.fr/pips}}
\newcommand{\WPIPS}{\htmladdnormallink{{\em WPIPS}}{http://www.cri.ensmp.fr/pips/wpips-epips-user-manual/wpips-epips-user-manual.html}}
\newcommand{\TPIPS}{\htmladdnormallink{\texttt{tpips}}{http://www.cri.ensmp.fr/pips/line-interface.html}}
\newcommand{\PROPERTIES}{\htmladdnormallink{\emph{properties}}{http://www.cri.ensmp.fr/pips/properties-rc}}

\begin{document}
\maketitle

\begin{latexonly}
  \clearpage
  \tableofcontents  
\end{latexonly}

\newpage
\section*{Welcome}

Welcome to the not so frequently asked questions about PIPS. Not so
frequently, because the PIPS user population, however growing, still is
small. Please send more questions to \verb/pips@cri.ensmp.fr/.

\section{Some of my source code is missing. What can I do?}
\index{Source File}

PIPS is designed as a whole application compiler and it complains about
missing source code for leaf modules in the call graph. In fact PIPS may
even generate a user error and stop if the missing piece of code is needed
to answer the user request. For instance, PIPS cannot compute the side
effects of procedure \verb/P/ without analyzing the code procedure
\verb/Q/ called by \verb/P/.

Also,

\section{Why is PIPS parser so limited?}
\index{Parser}

Parsing is divided in multiple stages.

\section{My include files are in a different directory}
\index{Include}

FC: Use the Shell environment variable \verb/PIPS_CPPFLAGS/ and the usual
\verb/-I/ option to specify directories to lookup for include files.
For instance, assuming that csh is used:

\begin{quote}
\verb|setenv PIPS_CPP_FLAGS "-I ../includes -I ../../generic/includes"|
\end{quote}

The current directory is searched first. For more information, see the
preprocessing sections in \PIPSMAKERC{}.

\section{How can I execute the transformed code?}

The transformed code is stored in the current workspace by an explicit
close or by exiting \PIPS{}. The best way to do it is to open the
workspace with \TPIPS{}.

\section{How can I replay a session?}
\index{logfile\_to\_tpips}

Each PIPS workspace contains a log. This log can be processed using the
{\tt logfile\_to\_tpips} Shell script. The output is a {\tt tpips} script
which can be replayed with the {\tt tpips} interface.

Some commands, such as {\tt cd}, {\tt open}, {\tt close}, {\tt delete},
may destroy the log file or change the current workspace. It is not
possible to recreate a global log from different workspaces.

\section{What should I read to use \PIPS{}?}

You might decide to get started using the \WPIPS{} X-Window interface and a
small Fortran program. On-line help is available.

\section{How can I process several Fortran files?}

The answer depends on the PIPS interface used. With the basic Shell
interface, the argument of the {\tt -f} option can be quoted to use Shell
filename expansion. For instance, the command

\begin{quote}
{\tt Init -d -f "*.f Src/*.f" foo}
\end{quote}

deletes a possibly existing {\tt foo} workspace and creates a new one with
all Fortran files in the current directory and in the {\tt Src}
subdirectory.

The window interface, {\tt wpips}, let you select a number of Fortran
files when you create a workspace but these files must all belong to the
current directory.

The script interface, {\tt tpips}, let you select any number of Fortran
files using the standard Shell syntax. For instance, the above {\tt Init}
command would be:

\begin{quote}
{\tt delete foo}\\
{\tt create foo *.f}
\end{quote}

The standard convention is used for include files.

\section{How can I get the dependence graphs of several modules in one command?}
\index{Dependence Graph}
\index{DG}

The script interface, {\tt tpips}, let you require several data structures
in one command. For instance, you can compute the dependence graphs for
all modules in the current workspace with:

\begin{quote}
{\tt display DG\_FILE[\%ALL]}
\end{quote}

It also is possible to specify a list of relevant modules:

\begin{quote}
{\tt display DG\_FILE[A B C]}
\end{quote}

It is not possible to submit such requests with the other PIPS user interfaces.

\section{PIPS tells me to set a property? What is it? How can I do it}
\index{Property}

Properties are used to encapsulate global variables and to fine tune the
behavior of PIPS. This fine tuning was not intended for PIPS users but for
PIPS developpers. However the property mechanism has been used to extend
PIPS parser and users should be able to define {\em some} properties (see
the
\htmladdnormallink{\emph{property}}{http://www.cri.ensmp.fr/pips/properties-rc}
documentation.

Default values for properties are defined in file
\verb+$PIPS_ROOT/Share/properties.rc+. when \PIPS{} is installed. However,
they can be modified in every directory by creating a file called {\tt
  properties.rc}. Properties can also be modified dynamically with the
most recent \PIPS user interfaces, \TPIPS{} and jpips.

\TPIPS users can also modifies the property in their home directory using
a file called \verb+.tpipsrc+.

\newpage


{\small
\bibliographystyle{plain}
\bibliography{faq}
}

\input{faq.ind}

\end{document}
\end


\end{document}
\end


\end{document}
\end


\end{document}
\end
