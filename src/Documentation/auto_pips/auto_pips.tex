\documentclass[a4paper]{article}

% lang
\usepackage[utf8]{inputenc}
\usepackage[english]{babel}

% others
\usepackage{hyperref}
\usepackage{color}
\definecolor{freeblue}{rgb}{0.25,0.41,0.88}
\definecolor{thingray}{rgb}{0.9,0.9,0.95}

\usepackage{listings}

\lstset{basicstyle=\ttfamily\small,keywordstyle=\bfseries,stringstyle=\it\color{blue},commentstyle=\color{blue},breaklines=true,language=c,moredelim=[is][\bfseries\color{blue}]{!}{!},
framerule=2mm,framexleftmargin=5mm, frame=leftline, rulecolor=\color{freeblue}, breaklines=true, backgroundcolor=\color{thingray}
}

% For the smileys :-)
\usepackage{wasysym}

\makeatletter

% Test for a TeX4ht command:
\@ifundefined{HCode}{%
  % We are generating PDF, so allow to break down URL which is nicer
  % instead of ugly line overflows:
  \usepackage{breakurl}}{%
  % else we are generating HTML with TeX4ht, use plain url:
  \usepackage{url}}

\usepackage{acronym,xspace}

\usepackage{tikz}

%%%% begin tikz macros for nice icons
\newcommand{\fileicon}{
  \ifx\du\undefined
	\newlength{\du}
  \fi
	\setlength{\du}{0.25em}
\begin{tikzpicture}
\pgftransformxscale{1.000000}
\pgftransformyscale{-1.000000}
\definecolor{dialinecolor}{rgb}{0.000000, 0.000000, 1.000000}
\pgfsetstrokecolor{dialinecolor}
\definecolor{dialinecolor}{rgb}{0.000000, 0.000000, 1.000000}
\pgfsetfillcolor{dialinecolor}
\pgfsetlinewidth{0.100000\du}
\pgfsetdash{}{0pt}
\pgfsetdash{}{0pt}
\pgfsetbuttcap
\pgfsetmiterjoin
\pgfsetlinewidth{0.100000\du}
\pgfsetbuttcap
\pgfsetmiterjoin
\pgfsetdash{}{0pt}
\definecolor{dialinecolor}{rgb}{1.000000, 1.000000, 1.000000}
\pgfsetfillcolor{dialinecolor}
\fill (13.950000\du,3.650000\du)--(15.150000\du,3.650000\du)--(15.550000\du,4.050000\du)--(15.550000\du,5.650000\du)--(13.950000\du,5.650000\du)--cycle;
\definecolor{dialinecolor}{rgb}{0.000000, 0.000000, 1.000000}
\pgfsetstrokecolor{dialinecolor}
\draw (13.950000\du,3.650000\du)--(15.150000\du,3.650000\du)--(15.550000\du,4.050000\du)--(15.550000\du,5.650000\du)--(13.950000\du,5.650000\du)--cycle;
\pgfsetbuttcap
\pgfsetmiterjoin
\pgfsetdash{}{0pt}
\definecolor{dialinecolor}{rgb}{0.000000, 0.000000, 1.000000}
\pgfsetstrokecolor{dialinecolor}
\draw (15.150000\du,3.650000\du)--(15.150000\du,4.050000\du)--(15.550000\du,4.050000\du)--(15.550000\du,4.050000\du);
\end{tikzpicture}
}
\newcommand{\diricon}{
\ifx\du\undefined
  \newlength{\du}
\fi
\setlength{\du}{0.25em}
\begin{tikzpicture}
\pgftransformxscale{1.000000}
\pgftransformyscale{-1.000000}
\definecolor{dialinecolor}{rgb}{0.000000, 0.000000, 1.000000}
\pgfsetstrokecolor{dialinecolor}
\definecolor{dialinecolor}{rgb}{1.000000, 1.000000, 1.000000}
\pgfsetfillcolor{dialinecolor}
\pgfsetlinewidth{0.100000\du}
\pgfsetdash{}{0pt}
\pgfsetdash{}{0pt}
\pgfsetbuttcap
\pgfsetmiterjoin
\pgfsetlinewidth{0.100000\du}
\pgfsetbuttcap
\pgfsetmiterjoin
\pgfsetdash{}{0pt}
\definecolor{dialinecolor}{rgb}{0.901961, 0.901961, 0.901961}
\pgfsetfillcolor{dialinecolor}
\pgfpathmoveto{\pgfpoint{12.850000\du}{5.751777\du}}
\pgfpathcurveto{\pgfpoint{12.850000\du}{5.350000\du}}{\pgfpoint{12.850000\du}{5.350000\du}}{\pgfpoint{13.100000\du}{5.350000\du}}
\pgfpathcurveto{\pgfpoint{13.350000\du}{5.350000\du}}{\pgfpoint{13.350000\du}{5.350000\du}}{\pgfpoint{13.600000\du}{5.350000\du}}
\pgfpathcurveto{\pgfpoint{13.850000\du}{5.350000\du}}{\pgfpoint{13.850000\du}{5.350000\du}}{\pgfpoint{13.850000\du}{5.751777\du}}
\pgfpathcurveto{\pgfpoint{13.850000\du}{5.751777\du}}{\pgfpoint{14.600000\du}{5.751777\du}}{\pgfpoint{14.600000\du}{5.751777\du}}
\pgfpathcurveto{\pgfpoint{14.600000\du}{5.751777\du}}{\pgfpoint{14.600000\du}{7.358884\du}}{\pgfpoint{14.600000\du}{7.358884\du}}
\pgfpathcurveto{\pgfpoint{14.600000\du}{7.358884\du}}{\pgfpoint{12.850000\du}{7.358884\du}}{\pgfpoint{12.850000\du}{7.358884\du}}
\pgfpathcurveto{\pgfpoint{12.850000\du}{7.358884\du}}{\pgfpoint{12.850000\du}{6.153554\du}}{\pgfpoint{12.850000\du}{5.751777\du}}
\pgfusepath{fill}
\definecolor{dialinecolor}{rgb}{0.000000, 0.000000, 1.000000}
\pgfsetstrokecolor{dialinecolor}
\pgfpathmoveto{\pgfpoint{12.850000\du}{5.751777\du}}
\pgfpathcurveto{\pgfpoint{12.850000\du}{5.350000\du}}{\pgfpoint{12.850000\du}{5.350000\du}}{\pgfpoint{13.100000\du}{5.350000\du}}
\pgfpathcurveto{\pgfpoint{13.350000\du}{5.350000\du}}{\pgfpoint{13.350000\du}{5.350000\du}}{\pgfpoint{13.600000\du}{5.350000\du}}
\pgfpathcurveto{\pgfpoint{13.850000\du}{5.350000\du}}{\pgfpoint{13.850000\du}{5.350000\du}}{\pgfpoint{13.850000\du}{5.751777\du}}
\pgfpathcurveto{\pgfpoint{13.850000\du}{5.751777\du}}{\pgfpoint{14.600000\du}{5.751777\du}}{\pgfpoint{14.600000\du}{5.751777\du}}
\pgfpathcurveto{\pgfpoint{14.600000\du}{5.751777\du}}{\pgfpoint{14.600000\du}{7.358884\du}}{\pgfpoint{14.600000\du}{7.358884\du}}
\pgfpathcurveto{\pgfpoint{14.600000\du}{7.358884\du}}{\pgfpoint{12.850000\du}{7.358884\du}}{\pgfpoint{12.850000\du}{7.358884\du}}
\pgfpathcurveto{\pgfpoint{12.850000\du}{7.358884\du}}{\pgfpoint{12.850000\du}{6.153554\du}}{\pgfpoint{12.850000\du}{5.751777\du}}
\pgfusepath{stroke}
\pgfsetlinewidth{0.010000\du}
\pgfsetbuttcap
\pgfsetmiterjoin
\pgfsetdash{}{0pt}
\definecolor{dialinecolor}{rgb}{0.000000, 0.000000, 1.000000}
\pgfsetstrokecolor{dialinecolor}
\pgfpathmoveto{\pgfpoint{12.850000\du}{5.751777\du}}
\pgfpathcurveto{\pgfpoint{12.850000\du}{5.350000\du}}{\pgfpoint{12.850000\du}{5.350000\du}}{\pgfpoint{13.100000\du}{5.350000\du}}
\pgfpathcurveto{\pgfpoint{13.350000\du}{5.350000\du}}{\pgfpoint{13.350000\du}{5.350000\du}}{\pgfpoint{13.600000\du}{5.350000\du}}
\pgfpathcurveto{\pgfpoint{13.850000\du}{5.350000\du}}{\pgfpoint{13.850000\du}{5.350000\du}}{\pgfpoint{13.850000\du}{5.751777\du}}
\pgfpathcurveto{\pgfpoint{13.850000\du}{5.751777\du}}{\pgfpoint{14.600000\du}{5.751777\du}}{\pgfpoint{14.600000\du}{5.751777\du}}
\pgfpathcurveto{\pgfpoint{14.600000\du}{5.751777\du}}{\pgfpoint{14.600000\du}{7.358884\du}}{\pgfpoint{14.600000\du}{7.358884\du}}
\pgfpathcurveto{\pgfpoint{14.600000\du}{7.358884\du}}{\pgfpoint{12.850000\du}{7.358884\du}}{\pgfpoint{12.850000\du}{7.358884\du}}
\pgfpathcurveto{\pgfpoint{12.850000\du}{7.358884\du}}{\pgfpoint{12.850000\du}{6.153554\du}}{\pgfpoint{12.850000\du}{5.751777\du}}
\pgfusepath{stroke}
\pgfsetlinewidth{0.100000\du}
\pgfsetbuttcap
\pgfsetmiterjoin
\pgfsetdash{}{0pt}
\definecolor{dialinecolor}{rgb}{1.000000, 1.000000, 1.000000}
\pgfsetfillcolor{dialinecolor}
\fill (12.850000\du,7.358884\du)--(12.600000\du,5.952665\du)--(14.350000\du,5.952665\du)--(14.600000\du,7.358884\du)--cycle;
\definecolor{dialinecolor}{rgb}{0.000000, 0.000000, 1.000000}
\pgfsetstrokecolor{dialinecolor}
\draw (12.850000\du,7.358884\du)--(12.600000\du,5.952665\du)--(14.350000\du,5.952665\du)--(14.600000\du,7.358884\du)--cycle;
\pgfsetlinewidth{0.010000\du}
\pgfsetbuttcap
\pgfsetmiterjoin
\pgfsetdash{}{0pt}
\definecolor{dialinecolor}{rgb}{0.000000, 0.000000, 1.000000}
\pgfsetstrokecolor{dialinecolor}
\draw (12.850000\du,7.358884\du)--(12.600000\du,5.952665\du)--(14.350000\du,5.952665\du)--(14.600000\du,7.358884\du)--cycle;
% setfont left to latex
\definecolor{dialinecolor}{rgb}{0.000000, 0.000000, 1.000000}
\pgfsetstrokecolor{dialinecolor}
\node[anchor=west] at (13.600000\du,6.454886\du){};
\end{tikzpicture}
}



\newcommand{\PIPS}{\textbf{\texttt{PIPS}}}
\newcommand{\autoconf}{\texttt{autoconf}}
\newcommand{\automake}{\texttt{automake}}
\newcommand{\gnulib}{\texttt{gnulib}}
\newcommand{\file}[1]{\fileicon{}\texttt{#1}}
\newcommand{\cmd}[1]{\textbf{`}\texttt{#1}\textbf{`}}
\newcommand{\dir}[1]{\diricon\texttt{#1}}

\newcommand{\PipsDevGuidePDF}{\url{http://www.cri.ensmp.fr/pips/developer_guide.htdoc/developer_guide.pdf}\xspace}
\newcommand{\PipsDevGuideHTDOC}{\url{http://www.cri.ensmp.fr/pips/developer_guide.htdoc}\xspace}
% The PDF is indeed into the dev guide directory...
\newcommand{\PipsAutotoolsGuidePDF}{\url{http://www.cri.ensmp.fr/pips/developer_guide.htdoc/auto_pips.pdf}\xspace}
\newcommand{\PipsAutotoolsGuideHTDOC}{\url{http://www.cri.ensmp.fr/pips/auto_pips.htdoc}\xspace}

\title{Autotools for \PIPS{}}

\author{Serge \textsc{Guelton}}

% document
\begin{document}

\maketitle

\begin{abstract}
This document describes the new build infrastructure of \PIPS{},
 based on the famous \texttt{autotools} suite and completes the \PIPS{}
 developer guide.

It describes
\begin{itemize}
\item the meaning of each involved file~\ref{sec:structure};
\item maintenance processes~\ref{sec:maintain} for \PIPS{} developers;
\item installation process~\ref{sec:install} for \PIPS{} users and developers.
\end{itemize}

This new infrastructure allows better portability and quite faster
(re)compiling taking full advantage from some caching and multicore
processors.
\end{abstract}

You can get a printable version of this document on\\
\PipsAutotoolsGuidePDF{} and a HTML version on \PipsAutotoolsGuideHTDOC

%%%
\section{Introduction}

Building a large software like \PIPS{} is quite complicated:
\begin{enumerate}
\item several source languages;
\item many tools involved;
\item unusual automatic header file generations.
\end{enumerate}

If you want to ensure a good level of portability,
you have to rely on portable tools.

If you want to ensure a good level of maintainability,
you have to rely on external, asserted tools.

As \PIPS{} targets \texttt{*nix} based systems and is written mainly in \texttt{C},
\texttt{autotools} appear as a \emph{de facto} standard.
It is especially known for enforcing good portability between \texttt{MacOs}, \texttt{Linux} and \texttt{BSD}.
Through \autoconf{}, it separate configuration step from build step.
Through \gnulib{}, it ensures portability of non-standard \texttt{C} functions.

As of now, \PIPS{} compiles on  \texttt{Linux}, \texttt{BSD} and \texttt{MacOs} Operating Systems.
It can be compiled using either \texttt{gcc} or \texttt{icc}.

You should notice that indeed \PIPS{} has currently 2 different build
process that lives in parallel\footnote{Of course it is logical for a
  parallelizer to have 2 build infrastructures that can be used
  indifferently in parallel! \smiley{} It is also useful for
  fault-tolerance.}, the \texttt{GNU-make}-based traditional infrastructure
and this new one based on \texttt{autotools}. In this way we can develop
and improve the new build infrastructure without hurting traditional
users.

Besides better portability support, the \texttt{autotools}-based build
infrastructure is quite faster that the old one and can exploit available
multiprocessors for example on \texttt{GNU-make} when using the
\texttt{--jobs=...} option to specify the number of \texttt{make} process
to use.

The nasty side effect of having 2 build methods is that, when adding new
stuff in \PIPS{}, you should declare them in both build infrastructures to
avoid having different contents in \PIPS{} according to the build
infrastructure used. So refer to the companion of this guide too, the
\PIPS{} Developer Guide \PipsDevGuidePDF or \PipsDevGuideHTDOC.

%%%
\section{Prerequisites}

In this section, we shortly list all packages needed to use auto-pips.
Note that those packages are only needed for developers, not for users:
\begin{itemize}
\item \texttt{autoconf}
\item \texttt{automake}
\item \texttt{libtool}
\item \texttt{pkg-config}
\end{itemize}

%%%
\section{Infrastructure Organization}
\label{sec:structure}

In this section, we describe the configuration files used by the several tools involved in \PIPS{} build process.

\subsection{\autoconf{}}

\autoconf{} manages the configuration of the build process.
Involved files are
\begin{description}
\item[\file{configure.ac}:] central place for configuration. Running \cmd{autoreconf -vi} will produce a \file{configure} script from it.
\item[\dir{makes/m4}:] auxiliary directory where \texttt{m4} configuration macros are stored. It is read by \cmd{autoreconf}.
\item[\file{Makefile.am}:] top-level \file{Makefile.am} contains a macro variable definition \texttt{ACLOCAL\_AMFLAGS} where \cmd{autoreconf} will find its additional \texttt{m4} sources.
Running \cmd{autoreconf -vi} will produce a \file{Makefile.in} for each \file{Makefile.am}
\item[\file{configure}:] is the portable configuration script generated by \cmd{autoreconf}. Running it will turn each \file{Makefile.in} into a regular \file{Makefile}.
\item[\file{config.status}:] is a script generated by configuration step,
  that memorizes configuration parameters.
\end{description}

For in-depth documentation of \autoconf{}, feel free to read \url{http://www.gnu.org/software/autoconf/manual}.

\subsection{\automake{}}

\automake{} manages the set of makefiles involved in the build process.

Each \file{Makefile.am} in source repository describes the build process for this repository.
It follows the \cmd{make} syntax, without needing \texttt{GNU-make} extensions for the sake of portability.

For in-depth documentation of \automake{}, feel free to read \url{http://www.gnu.org/software/automake/manual}.

\subsection{\gnulib{}}

\gnulib{} manages portability of \texttt{C} functions across \texttt{*nix} flavors.
Its whole configuration is stored in \dir{src/Libs/gnulib} and \dir{src/Libs/gnulib/m4}.
A few lines have been added in \file{configure.ac} to manage \gnulib{} configuration.

For in-depth documentation of \gnulib{}, feel free to read \url{http://www.gnu.org/software/gnulib/manual}.

%%%
\section{Maintenance Processes}
\label{sec:maintain}

This section describes the process to follow when changing build infrastructure.

\subsection{Adding a \texttt{C} source file in an existing \PIPS{} library}

Let us assume you want to add the file \file{pips.c} into library \dir{src/Libs/ri-util}.
First make sure your source file includes pips configuration header, by adding 

\begin{lstlisting}
#ifdef HAVE_CONFIG_H
    #include "pips_config.h"
#endif
\end{lstlisting}

at the top of your source file, before any other include.


The only thing you have to do then is to add your file in the macro variable suffixed \texttt{\_SOURCES}
in \file{src/Libs/ri-util/Makefile.am}
That is
\begin{lstlisting}
libri_util_la_SOURCES=eval.c  ... size.c
\end{lstlisting}
Becomes
\begin{lstlisting}
libri_util_la_SOURCES=eval.c  ... size.c !pips.c!
\end{lstlisting}

\subsection{Adding a \texttt{C} header file in an existing \PIPS{} library}

Let us assume you want to add the file \file{pips.h} into library \dir{src/Libs/ri-util}.
You will have to modify \file{src/Libs/ri-util/Makefile.am}
Ask yourself the question: Do I want to install the header file with the distribution ?
\begin{itemize}
\item If the answer is yes, add your file to the \texttt{include\_HEADERS} macro variable in, or create it if it does not exist.
\item If the answer is no, add your file to the \texttt{dist\_noinst\_HEADERS}  macro variable, or create it if it does not exist.
\end{itemize}
That is write something like this
\begin{lstlisting}
include_HEADERS=!pips.h!
\end{lstlisting}

\automake{} provides a fine grain control over what gets installed and distributed.


\subsection{Adding a \TeX{} file in an existing \PIPS{} directory}
Let us assume you want to add the file \file{pips.tex} into library \dir{src/Libs/ri-util}.
You will have to modify \file{src/Libs/ri-util/Makefile.am}

First, beware that documentation is not built by default.
It is only built when user activates configure flags \texttt{--enable-doc}.

So everything you do in a makefile that is relevant to documentation must be guarded by \texttt{WITH\_DOC}
The \automake{} variable for documentation is \texttt{dist\_noinst\_DATA} for sources and \texttt{doc\_DATA} for output.
That is
\begin{lstlisting}
if !WITH_DOC!
dist_noinst_DATA=!pips.tex!
doc_DATA=!pips.pdf!
endif
\end{lstlisting}

In addition to this, you have to supply \automake{} rules to build
\textsc{pdf} from \TeX{} files, using the directive
\begin{lstlisting}
include $(top_srcdir)/makes/latex.mk
\end{lstlisting}
if it is not already there.


\subsection{Adding a library}

This one is a bit more difficult.
In the following, we assume you want to add \dir{mylib} into \dir{src/Libs}.

There are many steps involved, follow them carefully:
\begin{enumerate}
\item create a directory \dir{mylib} into \dir{src/Libs};
\item add \texttt{mylib} to the \texttt{PIPS\_SUBDIRS} macro variable in \file{src/Libs/Makefile.am};
\item add \texttt{mylib/libmylib.la} to the \texttt{libpipslibs\_la\_LIBADD} macro variable  in \file{src/Libs/Makefile.am};
\item add following template in \file{src/Libs/mylib/Makefile.am}
\begin{lstlisting}
TARGET  = !mylib!
include_HEADERS = $(TARGET).h
BUILT_SOURCES=$(TARGET).h
include $(top_srcdir)/makes/cproto.mk
noinst_LTLIBRARIES=lib!mylib!.la
lib_!mylib!_la_SOURCES= !src0.c src1.c ... srcn.c!

include $(srcdir)/../pipslibs_includes.mk
\end{lstlisting}
Where
\begin{description}
\item[\texttt{TARGET}] is used to avoid redundancy and to communicate with \file{cproto.mk}.
\item[\texttt{include\_HEADERS}] specifies that you want to distribute the header generated by \cmd{cproto}.
\item[\texttt{BUILT\_SOURCES}] specifies that  \cmd{cproto} generated header must be built before anything else.
\item[\texttt{include \$(top\_srcdir)/makes/cproto.mk}] specifies how to use \cmd{cproto}.
\item[\texttt{noinst\_LTLIBRARIES}] specifies the name of local libraries.
\item[\texttt{lib\_mylib\_la\_SOURCES}] specifies the sources of your library.
\item[\texttt{include \$(srcdir)/../pipslibs\_includes.mk}] sets preprocessor include path correctly.
\end{description}

\item add \texttt{src/Libs/mylib/Makefile} to the \texttt{AC\_CONFIG\_FILES(...)} macro function parameters in \file{configure.ac};
\end{enumerate}

\subsection{Adding a program check}

For uncommon build, one may need to depend on an extra program.
Then comes the distribution issue: how can we assert the program is installed on user/developers machines ?
First you have to ask yourself: Is the new feature that depends on this program critical or not ?
If not, you will add an optional dependency.
Otherwise it is a mandatory dependency.

You will basically add your check by filling a call to macro function \texttt{AX\_CHECK\_PROG(\textit{prog\_name})}
in \file{configure.ac}.
This will perform the check for the program, trying to find it in current \texttt{\$PATH} or
in env variable \texttt{\$\{\textit{PROG\_NAME}\}}.

The macro variable \texttt{\$(\textit{PROG\_NAME})} will be available in your \file{Makefile.am}.

The last step is to attach the result of the check to a dependency.
That way, the configure will fail or not depending on the result of the check.
To do so, you will use the macro function \texttt{AX\_DEPENDS(\textit{feature},\textit{list-of-dependencies})}.
If you have a mandatory dependency, add the name of the program to the 
\lstinline|AX_DEPENDS([minimum],[...])| line.
Otherwise, add it to the optional \texttt{AX\_DEPENDS(...)} of your choice.
To fully understand usage of \texttt{AX\_DEPENDS(...)}, please read section on passes\ref{sec:passes}.

\subsection{Adding a pass}
\label{sec:passes}
2bedone


%%%
\section{Installation Processes}
\label{sec:install}

For short for \PIPS{}
\begin{lstlisting}
wget http://ridee.enstb.org/pips/get-pips4u.sh
chmod u+x get-pips4u.sh
./get-pips4u.sh --help
\end{lstlisting}

In some cases, \PIPS{} may be included in another distribution and you may
build \PIPS{} differently. For example in Par4All, where only this
\texttt{autotools} build method is used, this is done by the Par4All
installation process and you do not need to care about the previous
installation script.

%%%
\section{Additional Checks}
\label{sec:check}

\automake{} generates a \texttt{check} rule for \cmd{make}, but this rule is not used (yet).
Instead you can try one of the following, at the top of your build directory:
\begin{itemize}
\item \cmd{make check-includes}: checks if a source file does not include useless pips headers.
It is based on the \cmd{pipslibsdeps.py} script which has some extra features, try \cmd{pipslibsdeps.py --help} !
\item \cmd{make check-properties}: check if a property is defined in \file{pipsmake-rc.tex} but never referenced;
\item \cmd{make inspect-symbols}: checks exported but unused symbols for each pips library.
\end{itemize}

%%%
\begin{acronym}
\end{acronym}

\end{document}

%%% Local Variables:
%%% mode: latex
%%% ispell-local-dictionary: "american"
%%% End:

