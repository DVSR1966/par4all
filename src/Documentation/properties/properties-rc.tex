% PIPS Project
%
% Description des enchainements possibles des passes et analyses de PIPS
% pour properties
%
% Derivation rules and aliases must be included in verbatim environments. 
% Nothing else should appear in a verbatim environment.
% 
% Modifications

\batchmode
\documentstyle[11pt]{article}
\title{Properties \\
    Low Level Tuning of PIPS}
\author{Lei Zhou\thanks{E-mail: {\tt zhou@ensmp.fr}} \hspace{2cm} 
        Fran\c{c}ois Irigoin\thanks{E-mail: {\tt irigoin@ensmp.fr}} \vspace{1cm}\\
        Centre de Recherche en Informatique \\
        Ecole des Mines de Paris \\
        77305 Fontainebleau Cedex \\
        France \\}
\date{October 1991}

\addtolength{\textwidth}{72pt}
\addtolength{\oddsidemargin}{-48pt}
\addtolength{\evensidemargin}{-48pt}
\addtolength{\textheight}{172pt}
\addtolength{\topmargin}{-60pt}

\begin{document}
\thispagestyle{empty}

\maketitle

% \begin{abstract}
% \end{abstract}

\section*{Introduction}

This paper describes...

The information here is machine and/or site independent.

Casual users should not use them. They do modify them only through pipsmake
requests.

Experimented users can modify them by inserting a file called properties.rc
in their local directory. Of course, they cannot declare new properties,
which would not be recognized by the PIPS system.

Properties are presented on a library basis, but for the general options.

The file Production/Lib/properties.rc is derived...

\section{Global Option}

Are DO loops bodies at-least executed once (F-66 style).
Useful for use/def and semantics analysis.

\begin{verbatim}
ONE_TRIP_DO FALSE
\end{verbatim}

\section{Chains}

Update dependency graph with RR dependencies.
Useful for estimation of cache memory traffic. use/def related.

\begin{verbatim}
KEEP_READ_READ_DEPENDENCE FALSE
\end{verbatim}

Do we want to mask effects in loop bodies (dangerous with current
version of Allen \& Kennedy which assumes that all the edges are
present, the ones on private variables being eventually discarded
but with a current distribution)

\begin{verbatim}
CHAINS_MASK_EFFECTS FALSE
\end{verbatim}

Do we only keep dataflow (Def -- Use) dependences in the chain graph.

\begin{verbatim}
CHAINS_DATAFLOW_DEPENDENCE_ONLY FALSE
\end{verbatim}

\section{Prettyprinter Options}

Added for homogeneity reason; could be (easily) reduced to one?  In
fact, with many target machines in mind, we should have an integer flag
``architecture", with two basic architectures, sequential and parallel.
Useless values: re-initialized in the C code.

\begin{verbatim}
PRETTYPRINT_PARALLEL FALSE

PRETTYPRINT_SEQUENTIAL TRUE
\end{verbatim}

Print parallel DO loops using FORTRAN 90 syntax. Useless value:
re-initialized in the C code.

\begin{verbatim}
PRETTYPRINT_FORTRAN90 FALSE
\end{verbatim}

Adds Cray (FMP + CFT77) compatible directives for parallelization.
Useless value: re-initialized in the C code.

\begin{verbatim}
PRETTYPRINT_CRAY FALSE
\end{verbatim}

Add statement effects as comments in output; not implemented (that way) yet.

\begin{verbatim}
PRETTYPRINT_EFFECTS FALSE
\end{verbatim}

Transform DOALL loops into sequential loops with an opposed increment
not implemented

\begin{verbatim}
PRETTYPRINT_REVERSE_DOALL FALSE
\end{verbatim}

Print statement transformers as comments in code.

\begin{verbatim}
PRETTYPRINT_TRANSFORMER FALSE
\end{verbatim}

Print statement preconditions as comments in code.

\begin{verbatim}
PRETTYPRINT_EXECUTION_CONTEXT FALSE
\end{verbatim}

Print statement regions as comments in code.

\begin{verbatim}
PRETTYPRINT_REGION FALSE
\end{verbatim}

Print regions of scalars.

\begin{verbatim}
PRETTYPRINT_SCALAR_REGIONS FALSE
\end{verbatim}

Print statement blocks

\begin{verbatim}
PRETTYPRINT_BLOCKS FALSE
\end{verbatim}

Print unstructured blocks

\begin{verbatim}
PRETTYPRINT_UNSTRUCTURED FALSE
\end{verbatim}

Print all effects for all statements regardless of PRETTYPRINT\_BLOCKS
and PRETTYPRINT\_UNSTRUCTURED

\begin{verbatim}
PRETTYPRINT_ALL_EFFECTS FALSE
\end{verbatim}

Print empty statement blocks

\begin{verbatim}
PRETTYPRINT_EMPTY_BLOCKS FALSE
\end{verbatim}

Print statement ordering information

\begin{verbatim}
PRETTYPRINT_STATEMENT_ORDERING FALSE
\end{verbatim}

Print code with DO label and CONTINUE instead of DO-ENDDO. If FALSE, all
useless CONTINUE statements are NOT prettyprinted (ie. all those in
structured parts of the code). Warning: case TRUE, generated code may be
wrong after some code transformations like distribution...

\begin{verbatim}
PRETTYPRINT_ALL_LABELS FALSE
\end{verbatim}

Print code with DO label as comment.

\begin{verbatim}
PRETTYPRINT_DO_LABEL_AS_COMMENT FALSE
\end{verbatim}

Print private variables without regard for their use

\begin{verbatim}
PRETTYPRINT_ALL_PRIVATE_VARIABLES FALSE
\end{verbatim}


Print all the declarations even if it's not declared in the original program.
Added by LZ 22/10/91

\begin{verbatim}
PRETTYPRINT_ALL_DECLARATIONS FALSE
\end{verbatim}

Manage internal RETURNs correctly if set to TRUE.  This results in a
slightly ugly (but correct) prettyprint, compared to a nicer (but
possibly incorrect) default one.

\begin{verbatim}
PRETTYPRINT_INTERNAL_RETURN TRUE
\end{verbatim}

Print transformers, preconditions and regions in a format acceptd by Foresys and
Partita. 

\begin{verbatim}
PRETTYPRINT_FOR_FORESYS FALSE
\end{verbatim}

\section{Semantic Analysis}

Perform ``meet" operations for semantics analysis.

\begin{verbatim}
SEMANTICS_FLOW_SENSITIVE FALSE
\end{verbatim}

To be refined later; basically, use callee\'s transformers instead of
callee\'s effects when computing transformers bottom-up in the call graph;
when going top-down with preconditions, should we care about unique
call site and/or perform meet operation on call site preconditions ?

\begin{verbatim}
SEMANTICS_INTERPROCEDURAL FALSE
\end{verbatim}

Go all the Halbwachs way, compute inequalities instead of sticking
to equalities; implies SEMANTICS\_FIX\_POINT and SEMANTICS\_FLOW\_SENSITIVE.

\begin{verbatim}
SEMANTICS_INEQUALITY_INVARIANT FALSE
\end{verbatim}

CPU time and memory space are cheap: compute loop fixpoint for
transformers and preconditions; this implies SEMANTICS\_FLOW\_SENSITIVE.

\begin{verbatim}
SEMANTICS_FIX_POINT FALSE
\end{verbatim}

Output semantics results on stdout

\begin{verbatim}
SEMANTICS_STDOUT FALSE
\end{verbatim}

Debug level for semantics

\begin{verbatim}
SEMANTICS_DEBUG_LEVEL 0
\end{verbatim}

\section{Dependence Test (Ricedg)}

% Module code and list of called module names.

\paragraph{Dependence test choice}

This property seems to be now obsolete. The dependence test choice is
now controlled directly and only by rules in pipsmake.rc.

\begin{verbatim}
DEPENDENCE_TEST "full"
\end{verbatim}

\paragraph{Statistics}

Provide the following counts during the dependence test. There are three
parts: numbers of dependencies and independences (fields 1-10),
dimensions of referenced arrays and dependence natures (fields 11-25)
and the same information for constant dependencies (fields 26-40),
decomposition of the dependence test in elementary steps (fields 41-49),
use and complexity of Fourier-Motzkin's pair-wise elimination (fields
50, 51 and 52-68).

\begin{itemize}

  \item[1] array reference pairs, i.e. number of tests effected
    (used to be the number of use-def, def-use and def-def pairs on arrays);

  \item[2] number of independences found (on array reference pairs);

    {\bf Note:} field 1 minus field 2 is the number of array
    dependencies. 

  \item[3] numbers of loop independent dependences between references
    in the  same statement (not useful for program transformation
    and parallelization if statements are preserved); it should
    be subtracted from field 2 to compare results with other
    parallelizers;

  \item[4] numbers of constant dependences; 

  \item[5] numbers of exact dependences;

    {\bf Note:} field 5 must be greater or equal to field 4.

  \item[6] numbers of inexact dependences involved only by the 
           elimination of equation;
  \item[7] numbers of inexact dependences involved only by the F-M
           elimination;
  \item[8] numbers of inexact dependences involved by both elimination of
           equation and F-M elimination; 

    {\bf Note:} the sum of fields 5 to 8 and field 2 equals field 1

  \item[9] number of dependences among scalar variables;
  \item[10] numbers of dependences among loop index variables;
  \item[11-40] dependence types detail table with the dimensions [5][3]
               and constant dependence detail table with the
               dimensions [5][3]; the first index is the array dimension
    (from 0 to 4 - no larger arrays has ever been found); the
    second index is the dependence nature (1: d-u, 2: u-d, 3: d-d);
    both arrays are flatten according to C rule  as 5 sequences of
    3 natures;

    {\bf Note:} the sum of fields 11 to 25 should be equal to
    the sum of field 9 and 2 minus field 1.

    {\bf Note:} the fields 26 to 40 must be less than or equal to
    the corresponding fields 11 to 25

  \item[41] numbers of independences found by the test of constant;
  \item[42] numbers of independences found by the GCD test;
  \item[43] numbers of independences found by the normalize test;
  \item[44] numbers of independences found by the lexico-positive test
             for constant Di variables;
  \item[45] numbers of independences found during the projetion on Di
            variables by the elimination of equation;
  \item[46] numbers of independences found during the projetion on Di
            variables by the Fourier-Motzkin's elimination;
  \item[47] numbers of independences found during the test of
            faisability of  Di sub-system by the elimination of equation;
  \item[48] numbers of independences found during the test of
            faisability of Di sous-system by the Fourier-Motzkin's
            elimination; 
  \item[49] numbers of independences found by the test of lexico-positive
            for Di sub-system; 

    {\bf Note:} the sum of fields 41 to 49 equals field 2

  \item[50] total number of Fourier-Motzkin's pair-wise eliminations
     used; 
  \item[51] number of Fourier-Motzkin's pair-wise elimination 
    in which the system size doesn't augment after the elimination;
  \item[52-68] complexity counter table of dimension [17]. The
               complexity of one projection by F-M is the product of the
               number of positive inequalities and the number of negatives
               inequalities that contain the eliminated variable. This
    is an histogram of the products. Products which are less than 
    or equal to 4
    imply that the total number of inequalities does not increase.
    So if no larger product exists, field 50 and 51 must be equal.
\end{itemize}

The results are stored in the currentworkspace in MODULE.resulttestfast,
MODULE.resultesttestfull, or MODULE.resulttestseman according to the
test selected.

\begin{verbatim}
RICEDG_PROVIDE_STATISTICS FALSE
\end{verbatim}

Provide the statistics above and counte all array reference pairs
including these involved in call statement.

\begin{verbatim}
RICEDG_STATISTICS_ALL_ARRAYS FALSE
\end{verbatim}

\paragraph{Algorithmic dependences}

Only take into account true flow dependences (Def -- Use) during the computation of
SCC?  Note that this is different from the
CHAINS_DATAFLOW_DEPENDENCE_ONLY option which doesn't compute the whole
graph.  Warning: this option potentially yields incorrect parallel code.

\begin{verbatim}
RICE_DATAFLOW_DEPENDENCE_ONLY FALSE
\end{verbatim}

\paragraph{Printout}

To print the dependence graph in a file called {\em module\_name}.{\tt dg}

\begin{verbatim}
PRINT_DEPENDENCE_GRAPH FALSE
\end{verbatim}

To print the dependence graph without the dependences on privatized
variables 

\begin{verbatim}
PRINT_DEPENDENCE_GRAPH_WITHOUT_PRIVATIZED_DEPS FALSE
\end{verbatim}

To print the dependence graph without the no loop carried dependences
 
\begin{verbatim}
PRINT_DEPENDENCE_GRAPH_WITHOUT_NOLOOPCARRIED_DEPS FALSE
\end{verbatim}

\paragraph{Optimization}

The default option is to compute the dependence graph only for loops
which can be parallelized using Allen \& Kennedy algorithm.
However it is possible to
compute the dependences in any case even for loop containing test, goto,
etc...
by setting this option to TRUE.

\begin{verbatim}
COMPUTE_ALL_DEPENDENCES FALSE
\end{verbatim}

\section{Effects}

print SDFI just after computation

\begin{verbatim}
EFFECTS_PRINT_SDFI TRUE
\end{verbatim}

\section{Regions}

if {\tt MUST_REGIONS} is true, then it computes regions using the algorithm
described in report E/181/CRI, called {\em $T^{-1}$ algorithm}. It provides more
accurate regions, and preserve MUST approximations more often. But it is less
efficient. Its default value is FALSE. EXACT_REGIONS is true for the moment for
backward compatibility only.

\begin{verbatim}
EXACT_REGIONS TRUE
\end{verbatim}

\begin{verbatim}
MUST_REGIONS FALSE
\end{verbatim}

The default option is to compute regions without taking into account array
bounds. Both options have their advantages and drawbacks. 

\begin{verbatim}
REGIONS_WITH_ARRAY_BOUNDS FALSE
\end{verbatim}

I intend to compute disjunctions of regions. As I have already prepared basic
operators for that purpose, I created two properties to switch between regions
and disjuctions of regions. For the moment, they are always false.

\begin{verbatim}
DISJUNCT_REGIONS FALSE
\end{verbatim}

\begin{verbatim}
DISJUNCT_IN_OUT_REGIONS FALSE
\end{verbatim}

\section{Static Complexity Evaluation}

print each ri-walk function call

\begin{verbatim}
COMPLEXITY_TRACE_CALLS FALSE
\end{verbatim}


print intermediate complexities 

\begin{verbatim}
COMPLEXITY_INTERMEDIATES FALSE
\end{verbatim}

Print the complete cost table at the beginning of the execution.

\begin{verbatim}
COMPLEXITY_PRINT_COST_TABLE FALSE
\end{verbatim}

list of variables that must appear litterally in the complexity formula.
They have relatively high priority.

\begin{verbatim}
COMPLEXITY_PARAMETERS "IMAX LOOP"
\end{verbatim}

Controls the printing of statistics:    \\
 ... = 0 : don't prettyprint statistics with complexities  \\
 ... = 1 : prettyprint statistics only for loop/block/test/unstr.
statements \\
 ... = 2 : prettyprint statistics for all statements

\begin{verbatim}
COMPLEXITY_PRINT_STATISTICS 0
\end{verbatim}

In order to change cost table easily, I added this property.
It means that unit cost can be found under this directory. We have all_1
that means all operation cost is 1 , fp_1 that means only floating
point operation is taken into account and has cost unit 1.
In the future, we will add ``sparc-2'' and ``cray-ymp'',etc.
The cost filenames are stored in the complexity-local.h, they are operation,
memory, index, transcend and trigo.

\begin{verbatim}
COMPLEXITY_COST_TABLE "all_1"
\end{verbatim}

For the moment, we have two versions of variable evaluation. First,
evaluate variable as early as possible. This mothod is easier to reach
but at least it has two drawbacks: (1) If there are several unknown
variables which are identical, when we use early variable evaluation, we
can get several UU's . (2) If user want to see the relation of
complexity. Because the variable is too early replaced by a constant, we
can not see great thing.

\begin{verbatim}
COMPLEXITY_EARLY_EVALUATION TRUE
\end{verbatim}

\section{Top level control}

Print a log of the session

\begin{verbatim}
USER_LOG_P      TRUE
\end{verbatim}

What to do on user errors: go ahead (default) or core dump (debug)

\begin{verbatim}
ABORT_ON_USER_ERROR     FALSE
\end{verbatim}

\section{Call Graph}
This section is used to see the calling relations of callees.
In this section,
we use the datebase to get the graph. It is different with icfg.
ex: if A calls B twice, in callgraph, we only have ONCE A calls B;
while in icfg, we have twice B called by A.
So there is no option on the so-called callgraph except the debugging.

debugging level (should be CALLGRAPH_DEBUG_LEVEL and numeric)

\begin{verbatim}
CALLGRAPH_DEBUG FALSE
\end{verbatim}

\section{Interprocedural Control Flow Graph}
This section is NOT identical to that of Callgraph. 
We use the different schemes to get the results.
We get the calling relations by the declaration, it should be correct.
But there are bugs somewhere.

print controlling IF's

\begin{verbatim}
ICFG_IFs FALSE
\end{verbatim}

print enclosing DO loops

\begin{verbatim}
ICFG_DOs FALSE
\end{verbatim}

to be destroyed

\begin{verbatim}
ICFG_CALLEES_TOPO_SORT FALSE
\end{verbatim}

\begin{verbatim}
ICFG_DRAW TRUE
\end{verbatim}

debugging level (should be ICFG_DEBUG_LEVEL and numeric)

\begin{verbatim}
ICFG_DEBUG FALSE
\end{verbatim}

\section{Rice (parallelization)}

TRUE to show all parallel loops  \
FALSE to generate real code

\begin{verbatim}
GENERATE_NESTED_PARALLEL_LOOPS TRUE
\end{verbatim}

\section{wp65 (PUMA project)}

\begin{verbatim}
WP65_DEBUG_LEVEL        0
\end{verbatim}

\section{HPFC {\em High Performance Fortran Compiler}}

debugging levels considered by HPFC: {\tt HPFC\_DEBUG\_LEVEL}, 
{\tt HPFC\_PARSER\_DEBUG\_LEVEL}, {\tt HPFC\_IO\_DEBUG\_LEVEL}.

  These booleans decide whether some computations are directly
generated in the output code, or computed thru calls to dedicated
runtime functions. The default is the direct expansion.

\begin{verbatim}
HPFC_EXPAND_COMPUTE_LOCAL_INDEX TRUE
\end{verbatim}

\begin{verbatim}
HPFC_EXPAND_COMPUTE_COMPUTER TRUE
\end{verbatim}

\begin{verbatim}
HPFC_EXPAND_COMPUTE_OWNER TRUE
\end{verbatim}

\begin{verbatim}
HPFC_EXPAND_CMPLID TRUE
\end{verbatim}

  These booleans control the I/O generation.

\begin{verbatim}
HPFC_NEW_IO_COMPILATION TRUE
\end{verbatim}

\begin{verbatim}
HPFC_SYNCHRONIZE_IO FALSE
\end{verbatim}

\begin{verbatim}
HPFC_IGNORE_MAY_IN_IO FALSE
\end{verbatim}


% \begin{thebibliography}{99}

% \end{thebibliography}

\end{document}
