%%
%% $Id$
%%
%% newgen domain for pips detabase management.
%%

\documentclass[a4paper]{article}

\usepackage{newgen_domain}
\usepackage[backref,pagebackref]{hyperref}

\title{Pips Database Manager \\ private data structures}
\author{Fabien Coelho}

\begin{document}

\maketitle

\section{New database structures}
\label{sec:new}

New database structures for manipulation by \texttt{pipsdbm}.
The idea is to improve \texttt{pipsdbm} performances by providing an
underlying fast mapping-based implementation instead of the old list,
which does not scale well with the number of modules.

Conceptually, the database manages resources. However, resources are both
owned by a module and have a type (from the \texttt{pipsmake} point of view),
and the altogether forms the resource identification. After discussing it
a while, it was decided that \texttt{pipsdbm} should know about this
subclassification of resources, and this is finally reflected by the
data structures.

\domain{external db_void}

\subsection{Named objects}
\label{sec:named}

The key for the mapping management is a string, however they are not
managed by functions, which need a full newgen domain. Hence this small
tabulated domain which can associate a unique object to a string.
It can be used for both owners (modules) and type strings.

\domain{tabulated db_symbol = name:string}


\subsection{A resource for pips}
\label{sec:rs}

A resource is just some pointer to the resource in memory.
It is associated a logical time and maybe a file time, to check for
possibly modified files. It may be loaded (the pointer actually points to
the resource) or unloaded (the pointer may store the name of the
resource?), or being required by pipsmake but not yet produced.

\domain{db_status = loaded:unit + stored:unit + required:unit + loaded_and_stored:unit}

\domain{db_resource = pointer:db_void x db_status x time:int x file_time:int}


\subsection{Resource mappings}
\label{sec:map}

Now the pips resources are stored internally with a two level function. 
The first one relies on the owner name, the second on the resource name.

\domain{db_owned_resources = db_symbol -> db_resource}

\domain{db_resources = db_symbol -> db_owned_resources}

\end{document}
