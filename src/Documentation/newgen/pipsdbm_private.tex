%%
%% $Id$
%%
%% Copyright 1989-2014 MINES ParisTech
%%
%% This file is part of PIPS.
%%
%% PIPS is free software: you can redistribute it and/or modify it
%% under the terms of the GNU General Public License as published by
%% the Free Software Foundation, either version 3 of the License, or
%% any later version.
%%
%% PIPS is distributed in the hope that it will be useful, but WITHOUT ANY
%% WARRANTY; without even the implied warranty of MERCHANTABILITY or
%% FITNESS FOR A PARTICULAR PURPOSE.
%%
%% See the GNU General Public License for more details.
%%
%% You should have received a copy of the GNU General Public License
%% along with PIPS.  If not, see <http://www.gnu.org/licenses/>.
%%

%% newgen domain for pips detabase management.

\documentclass[a4paper]{article}

\usepackage{newgen_domain}
\usepackage[backref,pagebackref]{hyperref}

\title{Pips Database Manager \\ Private Data Structures}
\author{Fabien Coelho}

\begin{document}

\maketitle

\section{Internal Pipsdbm Database Structures}
\label{sec:new}

This file contains the description of new database structures used
internally by \texttt{pipsdbm}. The idea is to improve
\texttt{pipsdbm} performances by providing an underlying fast
hash-table-based implementation instead of the list used externally by
\verb/pipsdbm/ API, since lists do not scale up well with the number of
modules.

Conceptually, the \verb/pipsdbm/ database manages resources. However,
resources are owned by a module and have a kind (from the
\texttt{pipsmake} point of view), and they together form the unique
resource identification. After discussing it for a while among PIPS
designers, it was decided that \texttt{pipsdbm} should know about this
subclassification of resources, and this is taken into account by
these private \verb/pipsdbm/ data structures.

All domain private to \verb/pipsdbm/ have name prefixed by \verb/db_/.

\subsection{User Resources}

The exact nature of a resource as defined by a user of the library is
not known by \verb/pipsdbm/ API. So all pointers to user resources,
e.g. \verb/CODE/ or \verb/CALLERS/, are typed as \verb/void */,
i.e. \verb/db_void/ in Newgen declarations.

\domain{external db_void}

\subsection{Named Objects}
\label{sec:named}

The key for the resource descriptor management is a string. However
they are not managed by functions, which need a full newgen
domain. 
% FI: I do not understand
Hence this small tabulated domain associates a unique object
to a string. It can be used for both owners, i.e. module names, and
resource type, i.e. strings. For instance, all occurences of the
string \verb/"foo"/ are reduced to one.

\domain{tabulated db_symbol = name:string}

% FI: since there is only one field, it looks more like a set than a
% mapping or tabulated domain...

% FI: I do not understand the reason why strings are not used as
% access keys? Why do they have to be turned to unique pointers first?
% What is gained by hashing then pointers instead of strings?

\subsection{An Internal Resource Descriptor for PIPS}
\label{sec:rs}

A resource descriptor, known here as \verb/db_resource/, contains
several fields, similar to the fields of data structure
\verb/resource/ defined in \verb/database/. First a pointer called
\verb/pointer /to the resource in memory.  This pointer is associated
to a logical time, \verb/time/, and maybe a file time,
\verb/file_time/, to check for possibly externally modified files when
PIPS is used interactively or is coupled to other tools using the PIPS
workspace files. Each resource has a status, called
\verb/db_status/. It may be \verb/loaded/ and the pointer field
\verb/db_pointer/ actually points to the resource data structure in
memory, or unloaded, \verb/stored/ (the pointer may point to the name
of the resource?), or being \verb/required/ by pipsmake but not yet
produced. For optimization, a resource can also be loaded and stored
at the same time. Note that a key information, whether the resource is
up-to-date or not, is not stored here as it depends on the rules
managed by \verb/pipsmake/ at a higher level.

\domain{db_status = loaded:unit + stored:unit + required:unit + loaded_and_stored:unit}

\domain{db_resource = pointer:db_void x db_status x time:int x file_time:int}

Note the differences with the domain \verb/resource/ and
\verb/status/. The fields \verb/name/ and \verb/owner_name/ are gone
because they are used as access keys. The field \verb/db_status/ is
more complex than the field \verb/status/ and it does not contain a
pointer to the user data structure. This pointer is moved up into
\verb/db_resource/.

Note that objects of type \verb/db_resource/ are internal to
\verb/pipsdbm/. However, they are used under the name \verb/res_id/ to
manage the \verb/make_cache/ of \verb/pipsdbm/.

\subsection{Resource Mappings}
\label{sec:map}

The PIPS resource descriptors are stored and retrieved internally with
a two-level mapping scheme.  The first one uses the owner name,
a.k.a. the module name, and the second one the resource kind
name. Note that module and resource names are both managed as
\verb/db_symbol/.

The owner name is used to reach all resources associated to a given module:

\domain{db_resources = db_symbol -> db_owned_resources}

Then the different resources can be accessed thanks to their resource
kind names: 

\domain{db_owned_resources = db_symbol -> db_resource}

\end{document}
