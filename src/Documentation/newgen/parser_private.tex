%%
%% $Id$
%%
%% Copyright 1989-2009 MINES ParisTech
%%
%% This file is part of PIPS.
%%
%% PIPS is free software: you can redistribute it and/or modify it
%% under the terms of the GNU General Public License as published by
%% the Free Software Foundation, either version 3 of the License, or
%% any later version.
%%
%% PIPS is distributed in the hope that it will be useful, but WITHOUT ANY
%% WARRANTY; without even the implied warranty of MERCHANTABILITY or
%% FITNESS FOR A PARTICULAR PURPOSE.
%%
%% See the GNU General Public License for more details.
%%
%% You should have received a copy of the GNU General Public License
%% along with PIPS.  If not, see <http://www.gnu.org/licenses/>.
%%
\documentclass{article}

\usepackage[latin1]{inputenc}
\usepackage{newgen_domain}
\usepackage[backref,pagebackref]{hyperref}

\begin{document}
\sloppy

\domain{Import entity from "ri.newgen"}
{}
\domain{Import constant from "ri.newgen"}
{}

\domain{Equivalences = chains:chain*}
{Le domaine {\tt equivalences} permet de calculer les adresses des
variables d'un programmes Fortran. C'est une s�quence de cha�nes
d'�quivalences.}

\domain{Chain = atoms:atom*}
{Le domaine {\tt chain} est utilis� pour stocker une chaine
d'�quivalence. C'est une s�quence d'atomes.}

\domain{Atom = equivar:entity x equioff:int}
{Le domaine {\tt atom} permet de conna�tre l'offset d'une variable par
rapport au d�but de la chaine d'�quivalence � laquelle cette variable
appartient. L'offset peut �tre n�gatif. Une variable peut appara�tre
dans plus d'une cha�ne.}

\domain{Data = datavars:datavar* x datavals:dataval*}
{Le domaine {\tt data} permet de calculer les valeurs initiales des
variables d'un programme Fortran. Il se compose d'une liste de positions
m�moire � initialiser et d'une liste de valeurs. }

\domain{Datavar = variable:entity x nbelements:int}
{Le domaine {\tt datavar} permet de repr�senter une s�quence de
positions de la m�moire. cette s�quence est rep�r�e par une variable
et une longueur.}

\domain{Dataval = constant x nboccurrences:int}
{Le domaine {\tt dataval} permet de repr�senter une liste de valeurs.
Toutes les valeurs sont identiques, et sont donn�es par le sous-domaine
{\tt constant}.}

\end{document}
\end
