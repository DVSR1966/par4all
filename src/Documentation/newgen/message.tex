%%
%% $Id$
%%
%% Copyright 1989-2010 MINES ParisTech
%%
%% This file is part of PIPS.
%%
%% PIPS is free software: you can redistribute it and/or modify it
%% under the terms of the GNU General Public License as published by
%% the Free Software Foundation, either version 3 of the License, or
%% any later version.
%%
%% PIPS is distributed in the hope that it will be useful, but WITHOUT ANY
%% WARRANTY; without even the implied warranty of MERCHANTABILITY or
%% FITNESS FOR A PARTICULAR PURPOSE.
%%
%% See the GNU General Public License for more details.
%%
%% You should have received a copy of the GNU General Public License
%% along with PIPS.  If not, see <http://www.gnu.org/licenses/>.
%%

\documentclass{article}

\usepackage[latin1]{inputenc}
\usepackage{newgen_domain}
\usepackage[backref,pagebackref]{hyperref}

\title{HPFC \\ (High Performance Fortran Compiler) \\ datastructure}
\author{Fabien Coelho\\
                \\
        CRI, �cole des mines de Paris}

\begin{document}
\maketitle

\section*{Introduction}
  Ce document pr�sente tr�s br�vement les domaines Newgen utilis�s
par le prototype de compilateur {\tt hpfc} pour d�crire les motifs de
messages qui doivent �tre �chang�s.

\domain{import entity from "ri.newgen"}
{}
\domain{import range from "ri.newgen"}
{}

\domain{external Pvecteur}
{}

\domain{message = array:entity x content:range* x neighbour:Pvecteur x dom:range*}
{Un message concerne un des tableaux du programme ({\tt array:entity})
et pour ce tableau, un morceau des d�clarations locales, g�n�ralement
un bord ({\tt content:range*}). Le destinataire est d�crit relativement
sous la forme de la d�signation d'un voisin dans le tableau de {\em
processors} sur lequel est distribu� le tableau ({\tt
neighbour:Pvecteur}), et concerne le domaine de {\em template} ({\tt
domain:range*}). 
}

\end{document}
