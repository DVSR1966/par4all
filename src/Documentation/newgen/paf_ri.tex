
\documentclass{article}

\usepackage[latin1]{inputenc}
\usepackage{newgen_domain}
\usepackage[backref,pagebackref]{hyperref}

\title{Structures Newgen pour l'int�gration de PAF dans PIPS}

\author{Fran�oise Lamour \\
        Arnauld Leservot \\
        Alexis Platonoff \\
        \\
        DMA/PCP, CEA CEL-V}

\begin{document}

\maketitle

\section*{Introduction}

Les structures de donn\'ees suivantes sont utilis\'ees par les phases qui 
reproduisent les calculs mis en \oe uvre dans le parall\'eliseur PAF~: le {\em 
graphe du fl\^ot de donn\'ees (DFG, ``Data Flow Graph'')}, la {\em base de 
temps (BDT)}, la {\em fonction de placement (PLC)} et les appels au logiciel de
r\'esolution de syst\`emes param\'etriques en nombres entiers ({\em PIP}).

Apr\`es une description rapide des structures de donn\'ees import\'ees et 
externes, nous pr\'esentons successivement les notions de DFG, de BDT, de PLC, 
de Quast, de Lisp\_expression et de Static\_control.

\section{Structures de donn\'ees import\'ees}

\domain{Import reference from "ri.newgen"}
{Le domaine {\tt reference} est utilis\'e pour repr\'esenter une r\'ef\'erence 
\`a un \'el\'ement de tableau. Les variables scalaires \'etant repr\'esent\'ees
par des tableaux de dimension 0, les r\'ef\'erences \`a des scalaires sont 
aussi prises en compte. Elles contiennent une liste vide d'expressions 
d'indices.}

\domain{Import predicate from "ri.newgen"}
{Le domaine {\tt predicate} d\'efinit une relation entre valeurs de variables 
scalaires enti\`eres.}

\domain{Import expression from "ri.newgen"}
{Le domaine {\tt expression} permet de stocker les expressions sous deux 
formes. La premi\`ere est une description de l'expression telle qu'elle 
appara\^{\i}t dans le texte source du programme et la deuxi\`eme est une 
description compil\'ee des expressions lin\'eaires sous forme de Pvecteur.}

\domain{Import loop from "ri.newgen"}
{Le domaine {\tt loop} permet de repr\'esenter les boucles du type DO Fortran ou
FOR Pascal.}

\domain{Import statement from "ri.newgen"}
{Le domaine {\tt statement} permet de rep\'erer les instructions d'un module.}

\domain{Import entity from "ri.newgen"}
{Tout objet ayant un nom dans un programme Fortran est repr\'esent\'e par une 
{\tt entity}. Un tel objet peut \^etre un module, une variable, un common, un 
op\'erateur, une constante, un label, etc.}

\domain{Import sccflags from "dg.newgen"}
{Le domaine {\tt sccflags} est utilis\'e par l'algorithme de calcul des 
composantes fortement connexes d'un graphe.}

\section{Structures de donn\'ees externes}

\domain{External Pvecteur}
{Le domaine {\tt Pvecteur} est utilis\'e pour repr\'esenter les expressions 
lin\'eaires telles que {\tt 3I+2} ou des contraintes lin\'eaires telles que 
{\tt 3I + J <= 2} ou {\tt 3I == J}.

Un Pvecteur est une suite de mon\^omes, un mon\^ome \'etant un couple 
(coefficient,variable). Le coefficient d'un tel couple est un entier, positif 
ou n\'egatif. La variable est une entit\'e, sauf dans le cas du terme constant 
qui est repr\'esent\'e par la variable pr\'ed\'efinie de nom {\tt TCST}.

La structure de donn\'ees Pvecteur est import\'ee de la biblioth\`eque (C3) 
d'alg\'ebre lin\'eaire en nombres entiers du CRI.}

\section{Graphe du fl\^ot de donn\'ees}

\domain{Dfg\_vertex\_label = statement:int x exec\_domain:predicate x sccflags}
{Le domaine {\tt dfg\_vertex\_label} d\'efinit la structure des informations 
qui seront attach\'ees \`a un n\oe ud du DFG. Le sous-domaine {\tt statement} 
sp\'ecifie l'instruction sur laquelle porte la d\'e\-pen\-dance~; cet entier 
correspond au champ {\tt ordering} du domaine {\tt statement}. Le sous-domaine 
{\tt exec\_domain} est le domaine d'ex\'ecution de cette instruction~; c'est un
pr\'edicats, qui correspond aux bornes
des boucles englobantes ainsi qu'aux tests englobants. Le sous-domaine 
{\tt sccflags} contient diverses informations utiles pour le calcul des 
composantes fortement connexes. La structure de graphe elle-m\^eme (domaine 
{\tt graph}) est d\'efinit dans le fichier {\tt graph.f.tex}.}

\domain{Dfg\_arc\_label = dataflows:dataflow*}
{Le domaine {\tt dfg\_arc\_label} d\'efinit la structure des informations qui 
seront attach\'ees \`a un arc du DFG, {\em i.e.} le fl\^ot de donn\'ees 
passant par cet arc. La structure de graphe elle-m\^eme (domaine {\tt graph}) 
est d\'efinit dans le fichier {\tt graph.f.tex}.}

\domain{Dataflow = reference x transformation:expression* x governing\_pred:predicate x communication}
{Le domaine {\tt dataflow} permet de d\'ecrire les informations n\'ecessaires 
\`a la description du fl\^ot d'une donn\'ee. Le sous-domaine {\tt reference} 
donne la r\'ef\'erence sur laquelle porte la d\'ependance. Le sous-domain {\tt 
transformation} donne dans l'ordre la liste d'expressions qui donnent la valeur
des indices des boucles englobantes de l'instruction source en fonction des 
indices des boucles englobantes de l'instruction destination. Le sous-domaine 
{\tt governing\_pred} donne le pr\'edicat de contr\^ole de la d\'ependance. 
Enfin, le sous-domaine {\tt communication} sp\'ecifie le type de communication 
que ce fl\^ot engendre.}

\domain{Communication = broadcast:predicate x reduction:predicate x shift:predicate}
{Le domaine {\tt communication} permet de sp\'ecifier le type de communication 
qui peut \^etre engendr\'ee par le fl\^ot auquelle elle est associ\'ee. Le 
sous-domaine {\tt broadcast} correspond \`a une diffusion. Le sous-domaine {\tt
reduction} correspond \`a une r\'ecurrence. Le sous-domaine {\tt shift} 
correspond \`a une translation.

La direction d'une communication est caract\'eris\'ee par un Pvecteur qui est 
exprim\'ee en fonction des indices des boucles englobantes. Chaque type peut 
avoir plusieurs directions, ainsi, nous utilisons la structure {\tt predicate} 
pour les sp\'ecifier. Dans ce {\tt predicate}, le Psysteme correspondant ne 
sera constitu\'e que d'\'equations, chacune repr\'esentant la direction de 
communication.}

\section{Base de temps}

\domain{Bdt = schedules:schedule*}
{Le domaine {\tt bdt} d\'efinit la base de temps pour l'ensemble des 
instructions du programme. Elle se compose tout simplement d'une liste de 
fonctions de temps, chacune associ\'ee \`a une instruction et un pr\'edicat.}

\domain{Schedule = statement:int x predicate x dims:expression*}
{Le domaine {\tt schedule} est utilis\'e pour d\'ecrire la fonction de temps 
associ\'ee \`a une instruction et un pr\'edicat. C'est en fait une liste de 
fonctions, car la base de temps peut \^etre multidimensionnelle. Le 
sous-domaine {\tt statement} correspond au champ {\tt ordering} du {\tt 
statement} auquel correspond cette base de temps. Le sous-domaine {\tt 
predicate} donne le pr\'edicat d'existence de cette fonction. Enfin, le 
sous-domaine {\tt dims} donne les diff\'erentes expressions de la fonction 
multi-dimensionnelle.}

\section{Fonction de placement}

\domain{Plc = placements:placement*}
{Le domaine {\tt plc} d\'efinit la fonction de placement pour l'ensemble des 
instructions du programme. Elle se compose tout simplement d'une liste de de 
placements, chacun associ\'e \`a une instruction.}

\domain{Placement = statement:int x dims:expression*}
{Le domaine {\tt placement} est utilis\'e pour d\'ecrire le placement 
associ\'ee \`a une instruction. C'est en fait une liste d'expressions, car la 
fonction de placement peut \^etre multidimensionnelle. Le sous-domaine {\tt 
statement} correspond au champ {\tt ordering} du {\tt statement} auquel 
correspond ce placement. Enfin, le sous-domaine {\tt dims} donne les 
diff\'erentes expressions de la fonction multi-dimensionnelle.}

\section{Repr\'esentation des quasts}

\domain{Quast = quast\_value x newparms:var\_val*}
{Le domaine {\tt quast} est utilis\'e pour contenir les informations 
renvoy\'ees par PIP. Ces informations sont l'expression (d\'ependant de 
certaines conditions) de la valeur de chaque variable du syst\`eme en fonction 
des param\`etres de structure. Le sous-domaine {\tt quast\_value} donne la 
valeur du quast.

Le sous-domaine {\tt newparms} donne la liste des nouveaux param\`etres de 
structure introduit par le quast. Pour chaque nouveau param\`etre, 
l'information est compos\'ee d'une {\tt variable} et d'une {\tt value}. La {\tt
variable} est une entit\'e repr\'esentant le nouveau param\`etre de structure 
et la {\tt value} est l'expression de sa valeur en fonction des autres 
param\`etres de structure. La cr\'eation de nouveau param\`etre de structure 
est n\'ecessaire lorsqu'apparaissent dans les expressions des divisions par des
constantes, comme par exemple~: \( n/2\).}

\domain{Quast\_value = quast\_leaf + conditional}
{Le domaine {\tt quast\_value} contient la valeur du quast dans une structure 
directement inspir\'ee de la structure lisp ``quast'' utilis\'ee dans Paf. 
Cette valeur est soit une feuille {\tt quast\_leaf},
soit un {\tt conditional}.}

\domain{Conditional = predicate x true\_quast:quast x false\_quast:quast}
{Le domaine {\tt conditional} est utilis\'e pour contenir les informations d'un
{\tt quast} qui d\'ependent d'un pr\'edicat. Si le {\tt predicate} est vrai, 
alors l'information utilis\'ee est celle contenue dans {\tt true\_quast}, sinon
l'information utilis\'ee est celle contenue dans {\tt false\_quast}.}

\domain{Quast\_leaf = solution:expression* x leaf\_label}
{Le domaine {\tt quast\_leaf} contient une {\tt solution} rendu par PIP et 
le champ {\tt leaf\_label} utilis\'e lors du calcul d'une source dans
le data flow graph.
{\tt Solution} est une liste d'expressions, dont chacune donne la valeur de la 
solution trouv\'ee par PIP pour la variable de m\^eme rang dans la liste des 
variables du syst\`eme (l'ordre de ces deux listes est donc {\em tr\`es}
important).}

\domain{Leaf\_label = statement:int x depth:int}
{}

Le domaine {\tt leaf\_label} est utilis\'e lors du calcul d'une source :
il informe sur l'origine de cette source ({\tt statement})  
ainsi que sur la profondeur de la dependance ({\tt depth}).
Le quast rendu par PIP n'utilise pas ce champ ; il est dans ce cas
mis \`a {\tt leaf\_label\_undefined}.

\domain{Var\_val = variable:entity x value:expression}
{}

Le domaine {\tt var\_val} est utilis\'e pour sp\'ecifier l'association d'une 
variable et de sa valeur exprim\'ee en fonction d'autres variables.


\section{Repr\'esentation des expressions Lisp}

\domain{Lisp\_expression = operation:string x args:expression*}
{}

Le domaine {\tt lisp\_expression} permet de d\'ecrire les informations 
contenues dans une expression num\'erique en notation Lisp. Une telle 
expression est compos\'ee d'un op\'erateur et d'une liste d'arguments (chaque 
argument \'etant une {\tt expression}). Cette structure est utilis\'ee pour les
codes r\'ealisants le transfert des structures de PAF depuis les fichiers de 
sortie \'ecrits en notation Le\_Lisp vers les structures newgen d\'efinies plus
haut.

\section{Repr\'esentation du contr\^ole statique}

\domain{Static\_control = yes:bool x params:entity* x loops:loop* x tests:expression*}
{}

Le domaine {\tt static\_control} d\'efinit la structure de l'information qui 
est associ\'ee \`a une instruction ({\em i.e.} un {\tt statement}). Le 
sous-domaine {\tt yes} indique si ce statement se trouve dans une zone \`a 
contr\^ole statique~; si ce n'est pas le cas, les trois sous-domaines suivant 
sont vides. Le sous-domaine {\tt params} donne la liste des param\`etres de 
structure de la zone \`a contr\^ole statique dans laquelle se trouve ce 
statement. Le sous-domaine {\tt loops} donne la liste des boucles englobantes 
de ce statement. Le sous-domaine {\tt tests} donne la liste des conditions 
trouv\'ees dans les tests englobants de ce statement.

\end{document}



