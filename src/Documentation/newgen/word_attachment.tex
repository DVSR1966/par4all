\documentstyle{article}

\input{/usr/share/local/lib/tex/macroslocales/Dimensions.tex}

\title{Attachment of some information to text words\\Data structures}
\author{Ronan Keryell\\
                \\
        CRI, E'cole des mines de Paris}

\newcommand{\domain}[2]{\paragraph{{#1}}\paragraph{}{{#2}}}

\begin{document}

\section*{Introduction}

This report present the data structures used to attach
informations to some word of text, mainly to use in the hypertext Emacs
interface.

\domain{import entity from "ri.newgen"}
\domain{import loop from "ri.newgen"}
\domain{import reference from "ri.newgen"}

\section{Attachments}

Are only a list of {\tt attachment}s:

\domain{attachments = attachment*}


\section{Attachment}

Here is what can be attached to a word:

\domain{attachment = attachee x begin:int x end:int}

{\tt begin} is the position of the attachment begin in the output
file and {\tt end} the position of its end.

The various objects that can be attached:
\domain{attachee = statement_line_number:int + persistent reference + persistent declaration:entity + type:string + persistent loop + persistent module_head:entity + complementary_sections:unit + complexities:unit + continuation_conditions:unit + cumulated_effects:unit + out_regions:unit + preconditions:unit + privatized_regions:unit + proper_effects:unit + proper_regions:unit + regions:unit + static_control:unit + transformers:unit + decoration:unit + comment:unit}

Persistence is need because we do not want the RI to be broken when
the attachments are freed.



\section{The mapping used to attach various internal informations}

%Each attachment point to a word of the code:
%\domain{attachment_to_word = attachment->string}

Each word can have a list of attachments:
\domain{word_to_attachments = word_pointer:int->attachments}

\verb|word_pointer| is an {\tt int} instead of a {\tt string} since
the address must be used instead of the string content. So it needs a
cast...


\end{document}
