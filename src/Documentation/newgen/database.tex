%%
%% $Id$
%%
%% Copyright 1989-2010 MINES ParisTech
%%
%% This file is part of PIPS.
%%
%% PIPS is free software: you can redistribute it and/or modify it
%% under the terms of the GNU General Public License as published by
%% the Free Software Foundation, either version 3 of the License, or
%% any later version.
%%
%% PIPS is distributed in the hope that it will be useful, but WITHOUT ANY
%% WARRANTY; without even the implied warranty of MERCHANTABILITY or
%% FITNESS FOR A PARTICULAR PURPOSE.
%%
%% See the GNU General Public License for more details.
%%
%% You should have received a copy of the GNU General Public License
%% along with PIPS.  If not, see <http://www.gnu.org/licenses/>.
%%
 
%% newgen domain for pips database management.
 
\documentclass[a4paper]{article}
 
\usepackage[latin1]{inputenc}
\usepackage{newgen_domain}
\usepackage[backref,pagebackref]{hyperref}
 
\title{Pips Database Manager}
 
\begin{document}

\maketitle

To analyze and transform code in a modular way, each PIPS pass uses
and generates many objects related to different modules. These objects
are called resources and different passes can, or at least should,
only communicate with them. They are known by the name of the module
they are related to, e.g. \verb/foo/, and by the name of their kind,
e.g. \verb/CODE/ or \verb/PRECONDITIONS/.

The passes, a.k.a. phases, are declared together with the resources in
\verb/pipsmake-rc.tex/. Library \verb/pipsmake/ is responsible for the
global consistency and for the pass isolation: passes do not have to
know about the context because \verb/pipsmake/ ensures that they are
called with all they need.

The PIPS database manager provides time stamps to support the
\verb/pipsmake/ consistency scheme and disk transfers to provide
long-term storage. Thanks to this, the compilation process can be
broken into several steps, which is very useful for large applications
and for interactions with other tools or with users.

The PIPS database manager, \verb/pipsdbm/, was first designed with
resource lists. When PIPS was used to analyze interprocedurally
hundreds of modules, the list-based data structure did not scale up
and was replaced by a two-level hash-table scheme providing a much
faster access to any resource using its owner name and its name.

The new data structure is described in
\verb/pipsdbm_private.pdf/. However, the old data structures and API
are still used by the client library, \verb/pipsmake/.

\section{Old Database Structures}
 
\domain{Database = name:string x directory:string x resources:resource*}
{ The domain {\tt database} is managed by the pipsdbm and preprocessor libraries
  to describe the current state of a PIPS execution. This Newgen domain contains its
  name ({\tt name}), the directory in which it was created which is
  also know as {\em workspace}, and the
  information which has been computed for the different modules, {\tt
    resources}. Each
  piece of information is called a {\tt
    resource}. In reality, the name of the database and the name of the
  directory used are directly linked. The name of the
  directory is the name of the base with the string \textsc{.database}
  attached as a suffix.
 
An element of type {\tt resource} is added to the list {\tt resources}
of each objet calculed for this program by the phases, analyses or
transformations of Pips.
 
The library which makes use of this data structure is
\texttt{pipsdbm}. It should be named \texttt{database-util} if a
consistent file naming rule were used across PIPS development.
 
}

\domain{Resource = name:string x owner\_name:string x status x time:int x file\_time:int}
{ The domain {\tt resource} is used by
  pipsdbm to describe any piece of information which might be
  calculated by PIPS for a module or a program. For each bit of
  information, the following fields must be known: its resource kind
  name, ({\tt name}), the function to whom this resource is related to
  ({\tt owner\_name}), whether it is present in memory or stored in
  file ({\tt status}), its logical creation date ({\tt time}), and,
  potentially, the creation date (Unix) of the corresponding file
  ({\tt file\_time}).
 
The kind {\tt name} of the resource is in fact a type and could have been
defined as an enumerated type. It is out of concern for the simple and
generic nature of pipsdbm that we have chosen to define it as a
character string. It is this {\tt name} which permits a pipsdbm to
select the proper C function for reading, writing or freeing a resource.

At any given moment, each resource is identified in a unique manner by
its resource kind name, \texttt{name}, and the name of its so-called
{\em owner}, \texttt{owner\_name}.
 
We discussed for some time the utility of having an owner\_type to
specify what the resource refers to: a program (e.g. entities, the
symbol table), a module (most other resources), a
loop, an instruction, etc. We decided against it since up to this date,
we only have those resources which may be attached to programs or
modules, and this information may be deduced from the kind {\tt name} of
the resource. In fact, we only have a few resources which may be
attached to a program; these are the entities (and the
\texttt{user\_file}, but they are largely poorly treated for the
time being). We also have summary information related to the whole
program, e.g. the program precondition.}
 
\domain{Status = memory:string + file:string}
{ The domain {\tt
    status} is used by pipsdbm to know if the resource concerned may
  either reside in memory, in which case it may be found in
  memory, if it is needed for a processus, or on disc, if it is
  composed of a file which is normally never loaded in memory. In the
  latter case, the sub-domain {\tt file} gives the file
  name.  The file name must be relative to the workspace, if it is
  found in this workspace, and must be absolute, if it is found
  outside of the workspace, in order to allow the operations
  \texttt{mv} and \texttt{cp -pr} in the workspace.  If the
  resource is in memory, the sub-domaine {\tt memory} contains a
  pointer towards this resource.
 
How do resources in a file and file resources differ? It cannot be seen
from this description. The list of the latter group is explicitely managed
somewherelsee in pipsdbm. File resources cannot be loaded into
memory. They are accessed through other files named after the resource
kind. For instance, the file PRINTED\_FILE contains the name of the
file containing the print-out of a function.

The pool of resources called database is often called the PIPS
workspace in PIPS litterature. It has been called a database initially
because a database can be memory resident.  The storage of a database,
i.e. a workspace, on disk is called a PIPS database or workspace. The
file structure of a workspace is described elsewhere (Fabien?). See
for instance the workspace slide in the PPOPP 2010 PIPS tutorial.  }
 
\end{document}
