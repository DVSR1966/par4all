\documentclass[a4]{article}

\usepackage[latin1]{inputenc}
\usepackage{newgen_domain}


\begin{document}
\sloppy

Ces structures de donn�es impl�mentent la d�tection des r�ductions
g�n�ralis�es. Voir: Jouvelot, P., et Dehbonei, Babak, ``A Unified
Semantic Approach for the Vectorization and Parallelization of
Generalized Reductions'', ACM-ICS'89, Crete.

\section {Evaluation Symbolique}

\domain{Import expression from "ri.newgen"}
{}

\domain{Import entity from "ri.newgen"}
{}

D�finition des expression symboliques utilis�es pour la d�tection des
r�ductions. 

\domain{gexpression = guard:expression x expression}
{Une expression guard�e est form�e d'une garde et d'une expression
symbolique. La garde indique quand l'expression symbolique est valide }

\domain{sexpression = gexpressions:gexpression*}
{Une expression symbolique est form�e d'une liste d'expressions
gard�es. Toutes les gardes sont mutuellement exclusives.}

\section{Pattern-matching}

Definition des patterns qui d�crivent les r�ductions g�n�ralis�es.

\domain{pattern = variable:entity x condition:expression x parameter:expression x operator:entity x indices:entity*}
{Un {\tt pattern} d�crit une r�duction. La {\tt variable} est
l'entit� sur laquelle la r�duction est effectu�e. La {\tt condition}
d�crit la garde associ�e (dans le cas de conditionnelle, e.g., max).
Le {\tt parameter} d�finit la partie ind�pendante de la r�duction.
L'{\tt operator} d�finit l'op�ration entre la variable et le
param�tre. Les {\tt indices} est la liste des variables d'unification
qui doivent �tre �gales � l'index de la boucle la plus englobante.}

\end{document}
\end
