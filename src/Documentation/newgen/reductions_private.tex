%% $RCSfile: reductions_private.tex,v $ (version $Revision$)
%% $Date: 1996/08/21 10:24:13 $, 
%%

\documentclass{article}

\title{Simple reductions}
\author{Fabien Coelho}

\newcommand{\domain}[1]{\par{#1}}

\begin{document}
\maketitle

\section{Imported domains}
\label{sec:import}

\domain{import reference from "ri.newgen"}
\domain{import entity from "ri.newgen"}
\domain{import statement from "ri.newgen"}

\section{Operators}
\label{sec:operators}

Current operators are stored directly.
We could add user defined operators if required.
None is used to mark dead reductions. 

\domain{reduction\_operator = \{ none , min , max , sum , prod , and , or \}}


\section{Reduction}
\label{sec:reduction}

A reduction is a reference describing the object on which the reduction is
performed. This object may be a scalar, a full array, part of an array,
and so on. The operator is also specified. The dependences is the list of
entities the reduction reference depends on, and is used in the
construction algorithm. The trusted references are those the effects of
contribute to the reduction. They are only used and significant in the
proper reductions construction algorithm.

\domain{preference = persistent ref:reference}

\domain{reduction = reference x op:reduction\_operator x dependences:entity* x trusted:preference*}

List of reductions are separated.

\domain{reductions = list:reduction*}


\section{Functions}
\label{sec:functions}

Function from statement to reductions to be used for proper and cumulated
reductions. 

\domain{pstatement\_reductions = persistent statement->reductions}



\end{document}
