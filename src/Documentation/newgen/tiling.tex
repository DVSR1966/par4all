%%
%% $Id$
%%
%% Copyright 1989-2009 MINES ParisTech
%%
%% This file is part of PIPS.
%%
%% PIPS is free software: you can redistribute it and/or modify it
%% under the terms of the GNU General Public License as published by
%% the Free Software Foundation, either version 3 of the License, or
%% any later version.
%%
%% PIPS is distributed in the hope that it will be useful, but WITHOUT ANY
%% WARRANTY; without even the implied warranty of MERCHANTABILITY or
%% FITNESS FOR A PARTICULAR PURPOSE.
%%
%% See the GNU General Public License for more details.
%%
%% You should have received a copy of the GNU General Public License
%% along with PIPS.  If not, see <http://www.gnu.org/licenses/>.
%%
\documentclass[a4paper]{article}

\usepackage[latin1]{inputenc}
\usepackage{newgen_domain}
\usepackage[backref,pagebackref]{hyperref}

\title{PUMA: Repr�sentation du Partitionnement}
\author{Fran�ois Irigoin \\
	Corinne Ancourt\\
\\
	CRI, Ecole des Mines de Paris}


\begin{document}
\maketitle
\sloppy

\section{Structures de donn�es externes}
\label{external}

\domain{External Pvecteur}
{ Le domaine {\tt Pvecteur} est utilis� pour repr�senter l'origine
du partitionnement, c'est-�-dire les coordonn�es de l'origine de
la {\em tile} 0 dans le syst�me de coordonn�es initial.

Un Pvecteur est une suite de mon�mes, un mon�me �tant un couple
(coefficient,variable).  Le coefficient d'un tel couple est un entier,
positif ou n�gatif. La variable est une entit�, sauf dans le cas du
terme constant qui est repr�sent� par la variable pr�d�finie de nom
{\tt TCST}. Les entit�s utilis�es dans ce cas sont les indices initiaux.

La structure de donn�es Pvecteur est import�e de la biblioth�que d'alg�bre
lin�aire en nombres entiers du CRI.
}
\domain{External matrice}
{
Le domaine {\tt matrice} est utilis� pour repr�senter la matrice $P$
de partitionnement, qui d�finit le changement de base du syst�me
de coordonn�es des {\em tiles} au syst�me de coordonn�es initial.

Les matrices sont � coefficients rationnels, repr�sent�s par
des num�rateurs entiers et un unique d�nominateur. Les num�rateurs sont
stock�s sous forme pleine. Les dimensions de la matrice sont implicites.

Comme le domaine Pvecteur, la structure de donn�es matrice est
import�e de la biblioth�que d'alg�bre lin�aire en nombres entiers du
CRI.  }

\domain{tiling = tile:matrice x origin:Pvecteur}

\end{document}
\end
