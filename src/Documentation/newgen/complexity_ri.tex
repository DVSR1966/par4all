%%
%% $Id$
%%
%% Copyright 1989-2010 MINES ParisTech
%%
%% This file is part of PIPS.
%%
%% PIPS is free software: you can redistribute it and/or modify it
%% under the terms of the GNU General Public License as published by
%% the Free Software Foundation, either version 3 of the License, or
%% any later version.
%%
%% PIPS is distributed in the hope that it will be useful, but WITHOUT ANY
%% WARRANTY; without even the implied warranty of MERCHANTABILITY or
%% FITNESS FOR A PARTICULAR PURPOSE.
%%
%% See the GNU General Public License for more details.
%%
%% You should have received a copy of the GNU General Public License
%% along with PIPS.  If not, see <http://www.gnu.org/licenses/>.
%%
\documentclass[a4]{article}
 
\usepackage[latin1]{inputenc}
\usepackage{newgen_domain}
\usepackage[backref,pagebackref]{hyperref}
 
\begin{document}
\section*{Data Structure for Complexity Models }
\today
 
\domain{Import statement from "ri.newgen"}
{
\paragraph{}
(Avoids conflicts between number types and types defined by the RI)
}
\domain{External Ppolynome}
{
\paragraph{}
The domain {\bf Ppolynome} is used to represent the complexities of
the {\bf statement} as function of parameters, variables and indices
of loops with lobal coverage.  A Ppolynome is a list of monomials,
each monomial being constited of a floating coefficient and a Pvecteur
representing the product of variables with their expopents.  }
 
\domain{Complexity = eval:Ppolynome x varcount x rangecount x ifcount}
{
\paragraph{}
In the course of the evaluation, a field {\bf complexity} is
associated with each {\bf statement}. The sub-domain {\bf eval}
contains a polynome Approching the complexity of the statement. The
sub-domains {\bf varcount}, {\bf rangecount} and {\bf ifcount} give
the information statistics concerning the mnner in which the
evaluation of the complexity took place, therefore also of the
complexity itself. They count and in so doing classify the different
types of variables/loops/tests encountered in the course of the
complexity's evaluation.  }
 
\domain{Varcount = symbolic:int x guessed:int x bounded:int x unknown:int}
{
\begin{itemize}

\item {\bf symbolic} = the number of variables which we would like to
  see appear in symbolic form in the expression of the complexity
  ("parameters" Fortran, loop indices, formal parameters of a
  subroutine). This gives false resultats of thois type of variable is
  modified in the course of execution.

\item {\bf guessed} = the number of variables the values of which
  which we have guessed thanks to semantic analysis.
    
\item {\bf bounded} = the number of variables for whih we have
  determined a maximum or minimum thanks to semantic analysis.
 
\item {\bf unknown} = the number of for which e have found no
  information, and which the user has not asked to se displayed in the
  complexity. These variables are all replaced by the pseudo-variable
  UNKNOWN\_VARIABLE, which takes at the end of the evaluation a
  default value.
\end{itemize}
}
 
\domain{Rangecount = profiled:int x guessed:int x bounded:int x unknown:int}
{
\begin{itemize}

\item {\bf profiled} = the number of loops for which the limits have
  been measured by means of profiling.
 
\item {\bf guessed} = the number of loops for which the limits were at
  worst variable polynomes of type {\bf symbolic} or {\bf guessed}
 
\item {\bf bounded} = the number of loops for which the min./max
  limits were deterlined thanks to semantic analysis.
 
\item {\bf unknown} = the number of loops for which the number of
  loops could not be discovered. In consequence 0 after profiling
  pass. All these numbers are replaced by the pseudo-variable:
  UNKNOWN\_RANGE, which will be given a default value at the end.
\end{itemize}
}
 
\domain{Ifcount = profiled:int x computed:int x halfhalf:int}
{
\begin{itemize}

\item {\bf profiled} = the number of tests for which the probability
  was messured by mens of profiling .
 
\item {\bf computed} = the number of tests for which the probability
  the probability was calculated. In particular in this case where the
  boolean expression depends only on the indices of those loops whose
  coverage global and whose {\bf range}s are of type {\bf guessed}.
 
\item {\bf halfhalf} = the number of tests the probability of which
  was determined to be equal to a $\frac{1}{2}$ by defaut.
\end{itemize}
 
}
 
\end{document}
 

