\documentclass[a4paper]{article}
\usepackage{verbatim,newgen_domain}

\title{PIPS: Extension of the Internal Representation for C}
\author{
  Fabien Coelho \\
  Fran\c{c}ois Irigoin \\
  Pierre Jouvelot \\
  Thi Viet Nga Nguyen \\
  CRI, Ecole des Mines de Paris}

\begin{document}
\noindent
\maketitle
\tableofcontents
\newpage
\section*{Introduction}
This document contains the high-level description of data structures
used in PIPS as internal representations of C programs.  These data
structures are declared using the Newgen Data Definition Language. 
Here are the goals of our work:
\begin{itemize}
\item Enough information
must be preserved to prettyprint source code from the internal
representation: this is the ultimate goal we must meet. 
\item Whole program must be stored in the internal representation. Modules written in different languages such as
Fortran and C must be stored in memory as part of one application.
\item Make the internal representation pass with the SPEC2000 CFP, SPEC1995 INT and SPEC2002 HPC
\end{itemize}
The reader is assumed knowledgeable in Newgen \cite{Jouv90} and the internal
representation used for Fortran \cite{Coel01}. Another important reference
is CIL, a C Intermediate Language developed at University of California
\cite{Necu02}, which is used to compare to our work. 
\newpage
\section{Naming}
In C, the scope of a local variable is the
block where it is declared, the scope of an external static variable is
the source file where it is declared, not the module as in Fortran. So when it is
necessary, we have to add all information
such as the current source file, module and block to the entity name in order to locate an
entity in the symbol table. Here are several objectives for naming
different kinds of entities:
\begin{itemize}
\item As short as possible
\item As significant as possible
\item As uniform as possible
\item As efficient as possible
\item To be able to recompile after modifying a file (???)
\item Faithful wrt the original declarations 
\item Compatible with the existing internal representation (Fortran)  
\end{itemize}
Before entering the details related to this naming issue, here are some
recalls of the C standard \cite{ISOC}.
\subsection{ISO C concepts}
An identifier can denote
\begin{itemize}
\item an object
\item a function
\item a tag of a structure, union or enumeration
\item a member of a structure, union or enumeration
\item a typedef name
\item a label name
\end{itemize}
\subsubsection{Scopes of identifiers (6.2.1)}
There are 4 kinds of scopes: function, file, block and function prototype.
\begin{itemize}
\item A label name has {\it function scope}.
\item If the declarator or type specifier that declares the identifier
  appears outside of any block or list of parameters, the identifier has
  {\it file scope}.
\item If the declarator or type specifier that declares the identifier
  appears inside a block or within the list of parameters, the identifier has
  {\it block scope}.
\item If the declarator or type specifier that declares the identifier
  appears within the list of parameter declarations in a function prototype, the identifier has
 {\it function prototype scope}.
\end{itemize}
\subsubsection{Linkages of identifiers (6.2.2)}
An identifier declared in different scopes or in the same scope more than
once can be made to refer to the same object or function by the {\it
  linkage} (external, internal and none). 

So the concept of global entity (\verb/TOP_LEVEL/) does not exist for C, but we
can use it to refer to external linkage entities, which include functions
and objects but not tags for structures, unions, enumerations and typedef name. 
\subsubsection{Name spaces of identifiers (6.2.3)}
There are separate name spaces : 
\begin{itemize}
\item label names
\item tags of structures, unions and enumerations 
\item members of structures or unions
\item ordinary identifiers 
\end{itemize}

\subsection{File name scope}
There may be functions or variables that have the same name but in physically
  different files, such as :
\begin{verbatim}
  in Directory1/foo.c   
     static bar()

  in Directory2/foo.c   
     static bar()
\end{verbatim}
In order to distinguish these two static functions, the two different absolute file names must be taken into account. The same situation can arrive with \verb/typedef, struct, union/
 and  \verb/enum/ types, i.e two structures with the same name in
 different files are different. 
There are several possibilities to uniquely name a source file
\begin{enumerate}
\item Use the absolute paths (problem with name length)
\item Use the relative paths (problem when copying/moving the
  database system)
\item Use specific naming in PIPS DBM, with a database that stores the
  correspondences between the actual file name (absolute or relative, FC
  ?) and the specific name. This is the current solution for Fortran.
\item By default: relative and an option for absolute 
\end{enumerate}
There may be problems related to special characters used
in a C file name which can be in conflict with characters used as
separators in the entity name. 

Another possible whole program compilation problem: different compilation units use different names for structurally equivalent
 types. CIL \cite{Necu02} resolves this problem by some merging phases : merge types and
 tags and rewrite variable declarations and function bodies.

\subsection{Block scope}
There are different solutions to handle the blocks:
\begin{enumerate}
\item Flattening like CIL \cite{Necu02} and then ignoring completely these block
  scopes
\item Flattening the blocks and conserving the scope information 
  somewhere to regenerate code. There must have special separators and tables
  to interchange between the internal names and external names. Problem for
  debugging ?
\item Dewey indexing for blocks (like trees). It can be associated to
  the above solution ?
\item Multiple symbol tables. Note: the actual solution, a unique symbol table for general
"entity" may create problem when dealing with large-scale programs. 
\end{enumerate}
We do not use flattening as in CIL because the source
programs are transformed too much, which is not appropriated for a source-to-source
compiler, although the separation of declaration and code makes it more
easier to analyze the program. For example, in CIL, local variables in inner scopes are pulled to
function scope with variable renaming like this: 
\begin{verbatim}
int main() {
  int x = 6;
  {
    int x = 7;
  }
  return x;
}

int main() {
  int x__0;
  int x__1; 
  {
    x__0 = 6;
    x__1 = 7
  }
  return x__0;
}
\end{verbatim}
Impact of block scope on analyses of PIPS ? When propagating transformers or
preconditions, we must take into account the block scope of variables, and pay
attention to the different variables that have the same name.  

\subsection{Current naming mechanism}
An entity name contains the name of the object, concatenated to a prefix
string and a special separator \verb/MODULE_SEP_STRING/ (":"). The prefix
string is the name of the package defining the scope of the object which
can be \verb/TOP-LEVEL/ or a module name. 
\begin{verbatim}
TOP-LEVEL:ABS
FOO:INIT
\end{verbatim}  
There are also special prefix to represent a main program or a label:
\verb/MAIN_PREFIX/ ("\%"), \verb/LABEL_PREFIX/ ("@").
\begin{verbatim}
TOP-LEVEL:%MAIN
TOP-LEVEL:@LAB
\end{verbatim}  
\subsection{Proposed solution}
The name of an entity can be \verb/TOP-LEVEL:name/ if the entity is a
global object or  
\verb/[file][module][block]name/ if its scope depends on the current source file,
the current module or the current block. 

The file name is a specific name stored in the PIPS DBM. 

The block name is a Dewey indexing number. 

Here are the characters corresponding to special separators (attention, \$
can be used in the identifier name): 
\begin{verbatim}
MODULE_SEP_STRING :
FILE_SEP_STRING %
BLOCK_SEP_STRING ~
\end{verbatim}  

So the entity name can be 
\begin{itemize}
\item 
\verb/TOP-LEVEL:name/
\item or 
\verb/[file%][module:][block~]name/ which can be one of the following: 
\begin{verbatim}
FILE%name
MODULE:name
FILE%MODULE:name
MODULE:BLOCK~name
FILE%MODULE:BLOCK~name
\end{verbatim}  
\end{itemize} 
The name of a label is MODULE:@label because it has function scope. 

In addition, to distinguish a structure, an union or an enumerator that has 
the same name as other program variables, we have to add special constant 
characters such as \verb/STRUCT_PREFIX, UNION_PREFIX, ENUM_PREFIX/ to 
the name prefix. We also have to keep the name of the structure and the union 
in the global name of its members in order to distinguish these members with 
other program variables, and so a \verb/MEMBER_SEP_STRING/ is needed. 
It is not necessary for the enum member, because the name of a variable must 
be different to the name of an enumerator member. 
A prefix for typedef \verb/TYPEDEF_PREFIX/ is necessary to distinguish a
defined name and to regenerate code. 
\begin{verbatim}
STRUCT_PREFIX #
UNION_PREFIX *
ENUM_PREFIX ?
TYPEDEF_PREFIX $

MEMBER_SEP_STRING ^
\end{verbatim}  

Examples: 
\begin{verbatim}
struct node {};            [file%][module:][block~]#node
struct key {int node};     [file%][module:][block~]key^node 
typedef int node;          [file%][module:][block~]$node
int key;                   [file%][module:][block~]key
union key2 {int node};     [file%][module:][block~]*key2 
                           [file%][module:][block~]key2^node 
enum hue {toto,tata};      [file%][module:][block~]?hue
                           [file%][module:][block~]toto 
int hue;                   [file%][module:][block~]hue
\end{verbatim}   

Note: C distinguishes between uppercases and lowercases. PIPS : all upper
? 
 
The scope of struct, union, enum and typedef is the current compilation
unit (current source file).
 
\newpage
\section{Storage}
The storage class determines the location and lifetime of the storage
associated with a variable. 
\subsection{External variables}
\label{external}
External variables are defined outside any function, and are thus
potentially available to many functions. Any function may access an
external variable by referring to it by name, if the name has been
declared somehow. If an external variable is to be referred to before it
is defined, or if it is defined in a different source file, then an
\verb/extern/ declaration is mandatory. 
\begin{verbatim}
Name prefix = TOP-LEVEL:
Storage = ram 
Ram_function = TOP-LEVEL-ENTITY 
Ram_section = TOP-LEVEL area 
\end{verbatim}
In Fortran, the scope of a variable is the module, or in another words, 
a global variable is always associated to the list of entities
of a module, which facilitates code regeneration. 

In the old versions of C, we can declare (extern is not a real declaration, 
since it doesn't allocate memory per se) the same variable in multiple places 
(files) with no problem at link time. This is probably what we had in mind 
when we designed this part of the RI. 
%%
This is not working anymore with more recent versions of C. As it produces 
linking error, multiple declarations are though allowed by some explicit 
arguments to the compilers (like \verb /-z muldefs/ in gcc) but it is not
cleaned either. Hence there should be some way to know the difference between
the declarations of external variable and global variables and how to 
regenerate the declaration related to external variables, such as:

\begin{verbatim}
in file1.c
   int m
   void func1 { }

in file2.c 
   int i
   extern int m 
   void func2 { }

in file3.c
   void func3() 
   { 
      extern int m
   } 
\end{verbatim}
A source file can be considered as a module, and in the first case, the
entity m can be added to the list of entities associated to this source
file. They are stored as \verb /TOP-LEVEL:m/.For the second and third case 
there should not be any memory allocation for entity m. The real problem lies 
to differentiate between entity i and entity m for second case and hence 
allocate memory to ibut not m. We need to have some information at internal 
representation for the same.

\subsection{Static variables}
\subsubsection{Internal static variables}
\begin{verbatim}
int foo() {
  ...
  {
     static int i;
     ...
  } 
}

Name = FOO:11~I
Storage = ram 
Ram_function = current module
Ram_section = *STATIC* area of current module
\end{verbatim}
The outermost block can be omitted, since in fact it is considered as 
the current module. It is not necessary to generate an entity for each block, 
we only need to number the blocks. The offset of a variable is computed from 
the declarations of variables in the same block. So variables with the same
offset but in different blocks are different. 
\subsubsection{External static variables}
An external static variable is defined outside of any function, and is
known within the remainder of the source file in which it is declared, but
not in any other file. 
\begin{verbatim}
in file foo.c
static int i = 0; 
int f() {
}
\end{verbatim}
We have a corresponding entity to this external static variable:
\begin{verbatim}
Name = FOO%I
Storage = ram 
Ram_function = source file  
Ram_section = *STATIC* area of source file 
\end{verbatim}
In addition, another entities are generated for the source file and the 
\texttt{*STATIC*} area of this source file.
\begin{verbatim}
Source file
   name = TOP-LEVEL:source_file_name
   type = functional (parameters = NIL, result = void)
   storage = rom

Area
   name = source_file_name:*STATIC*
   type = area
   storage = rom
\end{verbatim}
 The prettyprint of the external static variable is based on the list of
entities associated to the source file entity. Problem: the position of the 
variable declaration in the source file? 
\subsection{Automatic variables}
Like dynamic local variables in Fortran \cite{Coel01}. 
The outermost block can be omitted. 
\begin{verbatim}
Name prefix =  [file%]module:[block~]
Storage = ram 
Ram_function = current module 
Ram_section = *DYNAMIC* area of current module
\end{verbatim}
The keyword \verb/auto/ is kept in variable qualifiers, presented in
\ref{type_qualifiers}. 

\subsection{Formal variables}
Like formal variables in Fortran \cite{Coel01}. 
\begin{verbatim}
Name prefix = [file%]module:
Storage = formal 
\end{verbatim}
\subsection{Register variables}
A declaration of an identifier for an object with storage-class specifier
\verb/register/ suggests that access to the object be as fast as possible and 
the address of any part of an object declared with register cannot be
computed.  A \verb/register/ declaration can only be applied to automatic 
variables and to the formal parameters of a function. 
 
\begin{verbatim}
f(c,n)
register int c,n;
{
  register int i;
}

storage = return:entity + ram + formal + rom:unit
\end{verbatim}

Since a formal variable can be declared with register, creating another type of storage such as \verb/register/ to store this
information will not work. Furthermore, for a source-to-source compiler, this
information is not important, it is only used to regenerate the source
code. So we only need to add this information some where in the type
structure. A solution is presented in \ref{type_qualifiers}, together with
\verb/const, volatile/ and \verb/restrict/.

\newpage
\section{Type}
The type system of C is much more extended than Fortran 77's
which makes the backward compatibility difficult to ensure.  
Beside some usual data types such as int, float and char, C also has 
enumerated type and derived types such as array, structure, union, function
and pointer. 
Usually, these kinds of type are added to \verb/type/:
 
\verb/type += basic + array + pointer + struct + union + enum/

\verb/array = type x dimensions/

\verb/basic = int + float + .../

However, since PIPS Newgen internal representation has been designed for
Fortran where the main data structure is array, the Newgen structure
\verb/variable/ (\verb/variable = basic x dimensions/), used to represent Fortran scalar and
array variables, must be handled compatibly. 
Here are different possible solutions to deal with this:
\begin{enumerate}
\item Try to replace all the functions related to \verb/variable/ by
  macros, then \verb/variable/ is no more used in the new version.

\verb/type += basic + array + pointer + struct + union + enum/

\verb/array = type x dimensions/

\verb/basic = int + float + .../
\item Keep \verb/variable/ and \verb/array/ in parallel 

\verb/type += basic + array + variable + pointer + struct + union + enum/

\verb/array = type x dimensions/

\verb/variable = basic x dimensions/

\verb/basic = int + float + .../

\item To avoid the modifications related to \verb/variable/, which is expensive, the new types
  can be added to \verb/basic/. This method is called {\it
  array-oriented}. The
Newgen data structure \verb/basic/ is modified to enable the recursion
among array, pointer, structure, .... 
  
\verb/type = statement + area + variable + functional + void + .../

\verb/variable = basic x dimensions /

\verb/basic = int + float + logical + string + pointer + .../

However, in this method, the traversal is not always direct, i.e to access a
pointer, we have to go through \verb/variable/, ... which may create 
bugs with malloc, free and it is not easy for debugging. 
\end{enumerate}
The third solution is chosen because it requires less modifications in
the actual internal representation. 
\subsection{Basic types}
\subsubsection{Integer type}
\label{int}
C has different kinds of integer type : int, signed int, unsigned int,
short, long, long long, ... and char, signed char, unsigned char. Since 
\verb/int x/ and  \verb/signed int x/ declarations are
implementation-defined (which is found in the file
stdio.h but not in SPEC 2000), they should be distinguished. 
There are different solutions to represent all this information:
\begin{enumerate}
\item A compact representation that only uses the basic \verb/int/ and gives different values to
each kind of type:
\begin{verbatim}
char                   = 1
short_int              = 2
int                    = 4
long_int               = 6
long_long_int          = 8

unsigned_char          = 11
unsigned_short_int     = 12
unsigned_int           = 14
unsigned_long_int      = 16
unsigned_long_long_int = 18

signed_char            = 21
signed_short_int       = 22
signed_int             = 24
signed_long_int        = 26
signed_long_long_int   = 28
\end{verbatim}
We can use \verb/mod(int,10)/ to know the basic size  and
\verb/div(int,10)/ to know if the variable is unsigned, signed
or not. However, this  not compatible with the initial version
\cite{Coel01} where the value
of int is the number of bytes required to store one scalar object of this
type. 
\item A less compact but simpler solution to represent three different
  cases: unsigned, signed or not. 

\verb/basic += int:int + signed:int + unsigned:int/

In each case, different values are associated to short int,
int, long int or long long int. 
\begin{verbatim}
short_int              = 1
int                    = 2
long_int               = 4
long_long_int          = 8
\end{verbatim}
\end{enumerate}

\subsubsection{Character type} 
\begin{enumerate}
\item A character variable is in fact an integral variable, so it can be
associated to the basic \verb/int/, as in the first solution of \ref{int}.
\begin{verbatim}
unsigned_char          = 10
char                   = 20
signed_char            = 30
\end{verbatim}
\item 
It may also be better to treat character independently. 

\verb/basic += char:int/

where the value of \verb/int/ is 
\begin{verbatim}
unsigned_char          = 11
char                   = 21
signed_char            = 31
\end{verbatim}
Ambiguity between string and array of characters ?
\end{enumerate}
The first solution is choosen.
\subsubsection{Bit type}
We have to add a basic type \verb/bit/ to represent the integral bit fields
occurred in a structure declaration.   

\verb/basic += bit:int/
\begin{verbatim}
int a:1 
unsigned b:2 
signed c:3
\end{verbatim}
\subsection{Pointers}
Pointer can point to any variable: to a scalar variable, to an array variable, to a function, ...
To represent an array of pointers and to keep the initial internal representation, a new type \verb/pointer/ is
added to \verb/basic/.   

\verb/basic +=  pointer:type/

\subsubsection{Pointer to integers}
\begin{verbatim}
int *p1
name = p1
type = variable
   dimension = NIL
   basic = pointer
         pointer of type variable
             basic = int
             dimension = NIL.
\end{verbatim}
with this dynamic allocation:
\begin{verbatim}
p = malloc(5*sizeof(int));
\end{verbatim}
to store the size of the corresponding memory zone, the dimension of the
pointer can be changed to : 
\begin{verbatim}
int *p1
name = p1
type = variable
   dimension = NIL
   basic = pointer
         pointer of type variable
             basic = int
             dimension =5.
\end{verbatim}
But how about the following case ?

\subsubsection{Pointer to pointer}
\begin{verbatim}
char **p;
p = malloc(n*sizeof(char *));
for (i=0;i<n;i++)
   p[i] = malloc((i+1)*sizeof(char));
\end{verbatim}
Initially, with the declaration, we have :
\begin{verbatim}
name = p
type = variable
   dimension = NIL
   basic = pointer
         pointer of type variable
             basic = pointer of type variable (basic = int/char, dimension
             = NIL)
             dimension = NIL.
\end{verbatim}
After the first allocation, p points to an array of n pointers (attention
to the scope of n):
\begin{verbatim}
             basic = pointer of type variable (basic = int/char, dimension
             = NIL)
             dimension = n.
\end{verbatim}
How to know the memory size pointed by this each pointer ? And the size is
i-related ...? p[i] represents an entity or p only ? The zise of p[i] is
lost  There is a particular case where (i+1) is replaced by m (arrays of
same size), we can keep this information in dimension of the 2nd pointer. 

\subsubsection{Pointer to an array}
\verb/int (*p2)[13]/ is a pointer to an array of 13 integers
\begin{verbatim}
name = p2
type = variable
   dimension = NIL
   basic = pointer
         pointer of type variable
             basic = int
             dimension = [13].
\end{verbatim}
\subsubsection{Array of pointers}
\verb/int *a[13]/ is an array of 13 pointers to integers
\begin{verbatim}
name = a
type = variable
    dimension = [13]
    basic = pointer
         pointer of type variable
             basic = int
             dimension = NIL.
\end{verbatim}
\subsubsection{Function returning a pointer}
\verb/char *f(n)/, f is a function that returns a pointer to a character
string
\begin{verbatim}
name = f
type = functional 
      result = variable
          dimension = NIL
          basic = pointer,
              pointer of type variable
                 basic = char
                 dimension = NIL. 
\end{verbatim}
\subsubsection{Pointer to a function}
\verb/int (*p)()/
\begin{verbatim}
name = p
type = variable
      dimension = NIL
      basic = pointer
           pointer of type functional
                parameters = undefined
                result = int.
\end{verbatim}
\subsection{Structure, Union and Enumerated Types} 
The common point between  a structure, an union or an enumerated type is that each of these
types has a name and a list of members. In addition, these 
 members can be represented as entities, because they have name,
type and eventually initial value. There are two possibilities to
represent these new types
\begin{enumerate}
\item Each type is represented separately:

\verb/type +=  struct:entity* + union:entity* + enum:entity*/ 
\item They are grouped into a composed type

\verb/type +=  composed/ 

\verb/composed =  members:entity* x kind/

\verb/kind = struct:unit + union:unit + enum:unit/
\end{enumerate}
The first solution is chosen because it is simpler and more direct when
we want to access a special type. 

In addition, in order to be homogeneous with \verb/pointer/, these
new types can be added to \verb/basic/:

\verb/basic +=  struct:entity* + union:entity* + enum:entity*/ 

but adding them to \verb/type/ makes the traversal much shorter. We do
not have to pass through \verb/variable/ each time we want to refer to an
entity of struct/union/enum type. 

The storage class of the struct/union/enum entities and their
members is \verb/rom/. The \verb/initial value/ of a member entity can be used to represent
the offset of the member in the struct.   

All the above discussions talk about the entity related to the
struct/union/enum declaration (this template about the shape of a
structure create no storage) such as 
\begin{verbatim}
struct key {
  int tab[3];
  int keycount;
};
\end{verbatim}
This entity \verb/key/ is of type \verb/struct/ and is associated to a list
of members.\\

In the declaration like \verb/struct key var = {{1,2,3}, 3}/, the entity
\verb/var/ is of type \verb/struct key/ and there are two possibilities to
represent its type: 
\begin{enumerate}
\item Associate directly  to the entity \verb/key/
\item Associate it to a composed type which contains the name of the structure (\verb/key/)
and the list of
members, etc. This solution is redundant because we have to store the same
information for each variable of type
\verb/struct key/. 
\end{enumerate}
So to represent a variable whose type is struct/union/enum, we add to \verb/basic/ the \verb/derived/ type to point to
these new types. 

\verb/basic += derived:entity/

The initial value of \verb/var/ in this case is a list of lists, which is
not representable actually in Newgen. But since we know statically the
size of the array \verb/tab/ in the \verb/struct key/, we can represent
this value as a normal list \{1,2,3,3\} and the information can be extracted
when needed. 
\subsubsection{Structure declaration}
\begin{verbatim}
struct key {
  char *keyword;
  int keycount;
};

name = key
type = struct 

     name = keyword
     type = variable
          dimension = NIL
          basic =  pointer of type variable
                  basic = char
                  dimension = NIL
     storage = rom
     initial = 8

     name = keycount
     type = variable
          dimension = NIL
          basic = int
     storage = rom
     initial = 8
\end{verbatim}
Nga's comments: the traversal is much shorter if 
\verb/pointer, array, basic/ are added to \verb/type/ as in the other
solution (which is more logic ...).
\begin{verbatim}
     name = keyword
     type = pointer of type basic = int

     name = keycount
     type = basic = int
\end{verbatim}
\subsubsection{Pointer to structures}
\begin{verbatim}
struct key *p;

name = p
type = variable
    dimension = NIL
    basic =  pointer of type variable
          basic = derived = entity key
          dimension = NIL
\end{verbatim}
\subsubsection{Array of structures}
\begin{verbatim}
struct key keytab[10];

name = keytab
type = variable
     dimension = [0:9]
     basic = derived = entity key
\end{verbatim}
\subsubsection{Pointer to structures}
\begin{verbatim}
struct key *p;

name = p
type = variable
    dimension = NIL
    basic =  pointer of type variable
          basic = derived = entity key
          dimension = NIL
\end{verbatim}
\subsubsection{Self-referential Structure}
\begin{verbatim}
struct node {
  char word[10];
  struct node * next;
};

name = node
type = struct 

      name = word
      type = variable
          basic = char
          dimension = [0:9] 

      name = next
      type = variable
           dimension = NIL
           basic = pointer of type variable
                  basic = derived = entity node
                  dimension = NIL
\end{verbatim}
 
\subsection{Typedef}
\begin{verbatim}
typedef char *STRING;
typedef int A[2][3];
typedef int (*PFI)();
typedef struct {} TREE,*TREEPRT;
\end{verbatim}
\verb/STRING, A, PFI, TREE, TREEPRT/ are entities with:
\begin{itemize}
\item Global name = \verb/TYPEDEF_PREFIX/ + local name. The \verb/TYPEDEF_PREFIX/ is
  used to regenerate code. 
\item Storage = rom 
\item Type 
\item Initial value
\end{itemize}
It is more logic if the initial value of a typedef entity is
\verb/type/. But in this case, we have to modify the \verb/value/ Newgen
structure, so it is better to put the type directly in the \verb/type/ of
the entity.

To represent variables whose type is a \verb/typedef/ entity, we add to
\verb/basic/ the \verb/typedef/ structure. 

\verb/basic += typedef:entity/ 

\begin{verbatim}
typedef struct key {...} key;
key k1;
struct key k2; 

name = STRUCT_PREFIX:key
type = struct  
storage = rom 

name = TYPEDEF_PREFIX:key 
type = variable 
     dimension = NIL
     basic = derived = entity STRUCT_PREFIX:key
storage = rom

name = k1
type = variable 
     dimension = NIL
     basic = typedef = entity TYPEDEF_PREFIX:key

name = k2
type = variable 
     dimension = NIL
     basic = derived = entity STRUCT_PREFIX:key
\end{verbatim}  
\subsection{Functional Type}
Functional type has already been treated for Fortran, except some small details.

\verb/type = functional + .../

\verb/functional = parameters:parameter* x result:type/

\verb/parameter = type x mode /

\begin{itemize}
\item For \verb/extern int f(void);/, \verb/parameters/ is a list of one element of type
\verb/void/.
\item How to represent \verb/extern int f();/ (the same for Fortran with
\verb/EXTERNAL F/ ?) \verb/parameters_undefined/ creates
consistency/updating problems ?
\item We can add a Newgen structure for the function qualifier \verb/inline/ but
it is not necessary for this moment (no inline function in SPEC 2000).
\end{itemize}
\subsection{Type qualifiers : Const, Restrict, Volatile}
\label{type_qualifiers}
\begin{verbatim}
const int *p;
void func(const a)
void h(int * const restrict p)
\end{verbatim}
Although only a small percent of variables are declared with these
qualifiers and it is expensive, we choose to create a new Newgen structure for them. Attempts
to put these information in existing Newgen
structures, such as a \verb/rom/
storage for local variables qualified with \verb/const/, \verb/shared/
field of \verb/ram/ for \verb/restrict/ed local variables are not
successful because they do not handle all possible
cases. For example, \verb/rom/ cannot be used for a formal variable declared \verb/const/ (which
can be found in SPEC 2000 benchmarks),  \verb/shared/ cannot be used for a
formal variable declared \verb/restrict/.  

We can add a new field \verb/qualifiers/ for \verb/type/, at the
\verb/variable/ level. This qualifier can also contain the \verb/register/
and \verb/auto/ cases.

\verb/variable = basic x dimensions x qualifiers:qualifier*/

\verb/qualifier = const:unit+restrict:unit+volatile:unit+register:unit+auto:unit/

There are about 36 make\_variable in PIPS source code to modify.

\newpage
\section{Expressions}

Expressions in Fortran PIPS must be extended to handle new kinds of
expressions with C language.

\verb/expression = syntax x normalized/

\verb/syntax = reference + range + call/

\verb/reference = variable:entity x indices:expression*/

New kinds of syntax are added to handle C language. They are {\it cast},
 {\it sizeof}, {\it subscripting array} and
{\it function application} expressions. The subscripting array expression
is an extension of the reference expression, which includes other
more complicated array objects such as pointer, function, 
structure or union member, ... The same extension is made to call
expression, named function application, because the called function is
not necessarily an entity but can be any expression that denotes a function.   

\verb/syntax += cast + sizeofexpression + subscript + application/

\subsection{Cast}
\verb/cast = type x expression/

\subsection{Sizeof}
\verb/sizeofexpression = type + expression/

\subsection{Subscript}
\verb/subscript = array:expression x indices:expression*/

\subsection{Application}
\verb/application = function:expression x arguments:expression*/

For example, we can have this complicated function call in C:
\verb|(*ctx->Driver.RendererString)()|.
\subsection{Special calls }

\subsubsection{Member references }

There are different solutions:
\begin{enumerate}
\item A special call expression, FIELD\_MEMBER\_CALL(exp1, exp2), to handle for example
\verb|str[1].fld[2]|
\item \verb/reference = variable:entity x indices:expression x offset:entity*/ 
This is not sufficient because we can have foo.x where foo is a function
returning a struct,... 

More explaination here !
\end{enumerate}
 Distinguish between . and \verb/->/ ? The last one can be represented by
 the first one and *. 
 Disadvantages for type checking, program
analyses, transformations ?

\subsubsection{Address of, value of}
Special call expressions for \verb/&/ and *.

\subsubsection{Comma operator - list of expressions}
f(a, (t=3,t+2),c)

Special call expression COMMA\_OPERATOR: n-ary or binary call ?
The same for \verb/a=b=c=d/. 

\subsubsection{Conditional expression}
e1? e2 : e3 

Special call expression CONDITIONAL\_OPERATOR.

\newpage
\section{Control Flow}
\subsection{Module Code}
\verb/code = declarations:entity* x decls_text:string x initializations:sequence/

As discussed in section \ref{external}, external variables that are
declared outside a module can be pretty-printed if the source file is
considered as a module, with the declarations list and the decls\_text string. 

The field \verb/initializations/ is used to represent DATA statements in
Fortran. 
In C, the initializer for a scalar variable is a single expression, for
objects that have aggregate or union types is an initializer list, which
can be not complete and by default, the remaining elements are initialized by zero (for
arithmetic type) or null pointer (for pointer type).

There are two possibilities to represent this initialization information:
\begin{itemize}
\item Represent this initialization in the initial values of
entities. New kinds of value can be added to \verb/value/ such as
\verb/expression/ for scalar variables, 
aggregate or union types (array, structure, ...). String can be used to
  regenerate code, list (list of lists is not permitted in Newgen) can be used  to
  have fine preconditions on array elements, structure member, ... But the
  list length is the array size ?

Other problem: how to represent \verb/int i = j;/ ? The initial value of i
is j ?
\item Treat this initialization as a special kind of statements like DATA
  in Fortran. 
\end{itemize}
The first solution is choosen. 
\subsection{Statement}
\label{statement}
 
\verb/statement = label x number x ordering x comments x instruction/

Since in C, a declaration can appear in any block, not only at the
beginning of a function, we have to associate variable declarations
to blocks. There are two possibilities to perform this in the current internal representation: declaration can be associated to
either a statement or a sequence (a block in fact). But since the true and false branches of a
conditional statement as well as the body of a loop are not necessarily
sequences, declarations in these statements will be lost if
they are associated to sequence. So we choose to associate them to
a statement, although it may be useless for some elementary statements such
as call, test or loop. 

Note: we have to pay attention in coding in
PIPS: we cannot refer to declarations once we have already reach the
instruction.   

\verb/statement += declarations:entity* x decls_text:string/

Member \verb/declarations/ contains a list of entities in the statement
scope (all kinds of entities such as intrinsics, ... or only those that
are effectively declared).  Member \verb/decls_text/ is used to generate source code (as for module code).

\subsection{Instruction}

\verb/sequence + test + loop + whileloop + goto + call + unstructured/

Other kinds of instructions in C such as \verb/switch/, \verb/for/, ... can be
added to \verb/instruction/ or represented through the existing structures. 
In addition, a statement in C can be any expression, not only call
expression, so we have to add \verb/expression/ to \verb/instruction/. 
However, to make PIPS backward compatible, we try to create a call
statement for each call expression. Expression statement is only used for
special cases, such as cast expression.
 
\subsubsection{Switch}
There are two solutions: we can add a new kind of instruction
\verb/multitest/ (the keyword \verb/switch/ cannot be used) or we can
represent a switch through \verb/if/ and \verb/goto/ statements.  

First solution: 

\verb/instruction += multitest/
\verb/multitest = controller:expression x body:statement/

The second solution is choosen: 
 
\verb/case/ and \verb/default/ are two kinds of labeled statements, which
can be treated as \verb/goto/. Their
associated labels are entities which must be unique.  The initial values of these
entities are constant expressions. We can add them to the
declarations list of the switch statement in order to match them to the actual
switch? The entity local name is the constant expression of case, and a
special name for default.  

The \verb/break/ statement can be treated as a \verb/goto/.

\begin{verbatim}
   switch (c) {
     case 1:
     s1;
     case 2: 
     s2;  
     default: 
     sd;
     }

     if (c==1) goto switch_xxx_case_1;
     if (c==2) goto switch_xxx_case_2;
     goto switch_xxx_default;
     switch_xxx_case_1: ;
     s1;
     switch_xxx_case_2: ;
     s2;
     switch_xxx_default: ;
     sd;      

   In si, we can have goto switch_exit_xxx; (which was a break) 

   and switch_exit_xxx: ; is inserted at the end of the switch statement
             
\end{verbatim}
How about code regeneration ? 
\subsubsection{While Loop}
The \verb/while (expression) statement/ and 
\verb/do statement while (expression)/ in C can be represented together
 by adding a new field to distinguish
if the evaluation of the
controlling expression takes place before or after each execution of the
loop body.   

\verb/whileloop=condition:expressionxbody:statementxlabel:entityxevaluation/

\verb/evaluation = before:unit + after:unit/

To keep the readability, evaluation:bool is not used here, as in another
cases in the internal representation (mode, action, ...).

\subsubsection{For Loop}
There is always a trade-off between regrouping different loop structures
and separating them. The first case makes program analyses more compact,
with less code to write but it makes pretty-printing original code
difficult. It is reverse for the second case. 

We have different possibilities to choose
\begin{enumerate}
\item Represent for loop as while loop
\item Represent for loop as loop (do loop in Fortran),but it is not always
  possible. Loops could be pretty-printed as for loops but for loops cannot
  always be represented as loops (several loop indexes).
\item In order to keep the initial program structures, we can treat the for loop
separately from the other loops. 

\verb/forloop = initialization:expression x condition:expression x/

\verb/incrementation:expression x body:statement/
\end{enumerate}

The ISO C standard \cite{ISOC} states that
the initialization of a for loop may contain variable declaration, which
is not the case for \cite{Kern78}. 

\subsubsection{Null statement}
Null statement in C is treated as CONTINUE statement in Fortran. We only need to make the
differences at the prettyprinter level. 

\subsubsection{Return statement}
Return statement in C (return; or return (exp);) is treated as RETURN statement in Fortran, which is
considered as nullary operator but can have 0 or 1 argument, such as the
cases of STOP and PAUSE statements. 

\subsubsection{Break, Continue, Exit, Jump, Interruption}

We can add new kinds of instruction to handle break and continue: 

\verb/instruction += break + continue /

CIL says that leaving \verb/break/ and \verb/continue/ as they are  makes transformations such as code motion
easier? The semantic difference between \verb/continue/ in Fortran and
\verb/continue/ in C?

Another solution is to treat \verb/break/ and \verb/continue/ as
\verb/goto/, which is choosen as solution for the moment. How about code regeneration ?  

\section*{Conclusion}
\nocite{Kern78, Necu02}
\newpage
\section{Appendix A : Current internal representation}
\begin{verbatim}
action = read:unit + write:unit ;
approximation = may:unit + must:unit + exact:unit ;
area = size:int x layout:entity* ;
basic = int:int + float:int + logical:int + overloaded:unit + complex:int + string:value ;
call = function:entity x arguments:expression* ;
callees = callees:string* ;
cell = reference + preference ;
code = declarations:entity* x decls_text:string x initializations:sequence ;
constant = int + litteral:unit + call:entity ;
control = statement x predecessors:control* x successors:control* ;
controlmap = persistant statement->control ;
descriptor = convexunion:Psysteme* + convex:Psysteme + none:unit ;
dimension = lower:expression x upper:expression ;
effect = cell x action x approximation x descriptor ;
effects = effects:effect* ;
effects_classes = classes:effects* ;
entity_effects = entity->effects ;
entity_int = entity->int ;
execution = sequential:unit + parallel:unit ;
expression = syntax x normalized ;
formal = function:entity x offset:int ;
functional = parameters:parameter* x result:type ;
instruction = sequence + test + loop + whileloop + goto:statement + call + unstructured ;
loop = index:entity x range x body:statement x label:entity x execution x locals:entity* ;
mode = value:unit + reference:unit ;
normalized = linear:Pvecteur + complex:unit ;
parameter = type x mode ;
persistant_expression_to_effects = persistant expression -> effects ;
persistant_statement_to_control = persistant statement -> persistant control ;
persistant_statement_to_int = persistant statement -> int ;
persistant_statement_to_statement = persistant statement -> persistant statement ;
predicate = system:Psysteme ;
preference = persistant reference ;
ram = function:entity x section:entity x offset:int x shared:entity* ;
range = lower:expression x upper:expression x increment:expression ;
reference = variable:entity x indices:expression* ;
sequence = statements:statement* ;
statement = label:entity x number:int x ordering:int x comments:string x instruction ;
statement_effects = persistent statement->effects ;
storage = return:entity + ram + formal + rom:unit ;
symbolic = expression x constant ;
syntax = reference + range + call ;
tabulated entity = name:string x type x storage x initial:value ;
test = condition:expression x true:statement x false:statement ;
transformer = arguments:entity* x relation:predicate ;
type = statement:unit + area + variable + functional + varargs:type + unknown:unit + void:unit ;
unstructured = entry:control x exit:control ;
value = code + symbolic + constant + intrinsic:unit + unknown:unit ;
variable = basic x dimensions:dimension* ;
whileloop = condition:expression x body:statement x label:entity ;
\end{verbatim}
\newpage
\section{Appendix B : Proposed internal representation (modified structures only)}
\begin{verbatim}
basic = int:int + float:int + logical:int + overloaded:unit + complex:int 
+ string:value + bit:int + pointer:type + derived:entity + typedef:entity;

instruction = sequence + test + loop + whileloop + goto:statement + call +
unstructured + forloop + expression;

forloop = initialization:expression x condition:expression x
incrementation:expression x body:statement ;

statement = label:entity x number:int x ordering:int x comments:string x
instruction x declarations:entity* x decls_text:string ;

syntax = reference + range + call + cast + sizeofexpression + subscript + application;

cast = type x expression ;

sizeofexpression = type + expression ;

subscript = array:expression x indices:expression* ;

application = function:expression x arguments:expression* ;

type = statement:unit + area + variable + functional + varargs:type +
unknown:unit + void:unit + struct:entity* + union:entity* + enum:entity*;

variable = basic x dimensions:dimension* x qualifiers:qualifier* ;

qualifier = const:unit + restrict:unit + volatile:unit + register:unit + auto:unit;

whileloop = condition:expression x body:statement x label:entity x
evaluation ;

evaluation = before:unit + after:unit ;

value = code + symbolic + constant + intrinsic:unit + unknown:unit + expression;
\end{verbatim}
\newpage
\bibliographystyle{abbrv}
\bibliography{biblio}


\end{document}












