%%
%% $Id$
%%

\documentstyle{article}
%% \input{/usr/share/local/lib/tex/macroslocales/Dimensions.tex}

\title{PIPS: Dependence Graph}
\author{Fran\c{c}ois Irigoin \\
    Pierre Jouvelot \\
    R\'emi Triolet\\
    Yi-Qing Yang \\
\\
    CRI, \'Ecole des mines de Paris}

\newcommand{\domain}[2]{\paragraph{{#1}}\paragraph{}{#2}}

\begin{document}
\maketitle

%% \sloppy

The dependence graph specific data structures, as well as auxiliary data
structures used to compute it, are described in this report. The graph
itself is a generic graph defined in \verb/Newgen/graph.f.tex/. It is
specialized into a dependence graph by using specific arc and vertex
labels. 

The genericity is not supported explicitly by NewGen\cite{JT89,JT90}. Although the arc
and vertex labels are NewGen objects, they are seen as {\em external}
types. Specific procedures are specified in calls to
\verb/gen_init_external/ (see \verb/initialize_newgen/ in Library
\verb/top-level/). C modules using these labels must use explicit
casts to convert the generic labels into dependence graph labels.

Auxiliary data structures are used to compute the strongly connected
components (SCC) of the dependence graph. The SCC routines are not as
generic as could be hoped. They take into account the graph hierarchy
implied by the dependence levels of the Allen \& Kennedy parallelization
algorithm. This explains why the SCC related data structures are declared
here and not with data structure
\htmladdnormallink{\verb+graph+}{http://www.cri.ensmp.fr/pips/graph}.

\begin{comment}
Les structures de donne'es suivantes sont utilise'es par la phase de
construction du graphe des de'pendances. Elles sont construites en
utilisant les structures de donne'es \verb+statement+ et \verb+effect+
qui ont e'te' de'finies dans la repre'sentation interne, ainsi que la
structure de donne'es ge'ne'riques \verb+vertex+ qui fait partie du
package \htmladdnormallink{{\em graph}}{http://www.cri.ensmp.fr/pips/graph}.
\end{comment}

Several functions compute or print a dependence graph. See the
\htmladdnormallink{PipsMake
  manual}{http://www.cri.ensmp.fr/pips/pipsmake-rc} for the list of
user-callable functions. Advanced users have access to more functions
through the PIPS
\htmladdnormallink{properties}{http://www.cri.ensmp.fr/pips/properties-rc}.

It is not possible to save efficiently
the internal representation of a dependence graph onto disk. The
dependence graph structure contains pointers to memory effects and
indirectly to memory references included in code statements. These
references which are part of the code data structure do not have {\em
  names} and would have to be duplicated in the dependence graph
file. Because of the evaluation mechanism used by \htmladdnormallink{PipsMake
  manual}{http://www.cri.ensmp.fr/pips/pipsmake-rc}, it is not possible to
reuse a parallelized code from one PIPS session to the next one. The
parallelized code depends on the dependence graph, and the dependence
graph cannot be retrived. Hence the parallelized code, however correct, is
deemed inconsistent with respect to the workspace.

Section~\ref{section-imported-newgen-types} deals with the usual import of
other NewGen data structures.
Section~\ref{section-data-structures-external-to-newgen} contains the
declarations of data structure external to NewGen.
Section~\ref{section-arc-and-vertex-lables-for-the-dependence-graph}
presents the arc and vertex decorations specific to the dependence graph.
And the last section, Section~\ref{section-strongly-connected-components},
is used to describe the data structures required by the Allen\&Kennedy
parallelization algorithm.

\section{Imported NewGen Types}
\label{section-imported-newgen-types}

The dependence graph points to statement and to memory effects (see
\htmladdnormallink{use-def chains}{http://www.cri.ensmp.fr/pips/ri} for
both data structures).

\begin{comment}
\domain{Import statement from "ri.newgen"}
{}
\end{comment}

\domain{Import effect from "ri.newgen"}
{}

The dependence graph data structure is  a particular instance of the
\verb/graph/ data structure.

\domain{Import vertex from "graph.newgen"}
{}

\section{Data Structures External to NewGen}
\label{section-data-structures-external-to-newgen}

Generating systems are used to abstract a set of dependence arcs by
dependence cones (convex and transitive closure of the dependence set)
and/or dependence polyhedron (convex closure). This data structure,
\verb/generating system/, is
part of the C3 Linear Algebra library.

\domain{External Ptsg}

\section{Arc and Vertex Labels for the Dependence Graph}
\label{section-arc-and-vertex-lables-for-the-dependence-graph}

\subsection{DG Vertex Label}

\domain{dg\_vertex\_label = statement:int x sccflags}
{}

This data type carries information attached to each dependence graph
verted (a.k.a. node). Graph nodes are of type \verb/node/ defined in
{\tt graph.f.tex}.  Field \verb/statement/ contains a
\verb/statement_ordering/, i.e. a statement identifier (see {\tt
ri.f.tex}. The effective statement can be retrieved using mapping
\verb/OrderingToStatement/.

Field \verb/sccs_flags/ is used to compute strongly connected components
of the dependence graph.

\begin{comment}
Ce domaine est utilise' pour contenir les informations qui sont
attache'es a` chaque noeud du graphe de de'pendances (voir le domaine
{\tt graph} dans le fichier {\tt graph.f.tex}). Le sous-domaine {\tt
statement} permet de retrouver l'instruction qui porte la de'pendance;
cet entier est le champ {\tt ordering} de l'instruction concerne'e. Le
sous-domaine {\tt sccflags} contient diverses informations utiles pour
le calcul des composantes fortement connexes.
\end{comment}

\subsection{DG Vertex Label}

\domain{dg\_arc\_label = conflicts:conflict*}
{}

This data type contains information attached to a dependence graph arc.
See Newgen data type \verb/graph/ in file \verb/graph.f.tex/. Each DG arc
points to the list of all conflicts due to references in either the
source or the sink statements, i.e. the statements attached to the sink
or the source vertex.

\begin{comment}
Ce domaine est utilise' pour contenir les informations qui sont
attache'es a` chaque arc du graphe de de'pendances (voir le domaine {\tt
graph} dans le fichier {\tt graph.f.tex}). Chaque arc du GD contient les
conflits entre les deux statements des noeuds du graphe de de'pendance.
\end{comment}

\subsection{Dependence Conflict}

\domain{conflict = persistant source:effect x persistant sink:effect x cone}
{}

A conflict is generated by two effects, one in the source statement and
one in the sink statement. Simple effects are due to references in
statements. More complex effects are generated for procedure
calls. Different kinds of arcs (a.k.a. chains), use-def, def-use,
use-use and def-def, are derived from the different kinds of effects,
read and write. Note that def-def conflicts are computed to provide
locality information. This information is or might be used in phase WP65.

\begin{comment}
Un conflit existe entre deux effets pre'sents dans deux noeuds voisins
du graphe de de'pendance. Les types d'effet (write, read) sert a`
distinguer les conflits ``use-def'', ``def-def'' et ``{def-use}''. Le
cone comprend les informations precises de conflict.
\end{comment}

\subsection{Dependence Levels and Dependence Cone}

\domain{cone = levels:int* x generating\_system:Ptsg}
{}

Dependence arcs do not carry enough information to parallelize and/or
transform loops~\cite{YAI95}. The simplest loop parallelization
information is called {\em dependence levels}. Each level correspond to
one common enclosing loop. A common enclosing loop contains both source
and sink statements. The outermost enclosing loop is denoted 1, and the
innermost loop has the higher level. Non-loop carried dependences have
level {\em number of common enclosing loops plus one}, which makes
impossible to know if a dependence is loop carried or not if the number of
common enclosing loops is unknown.

Intra-statement conflicts, i.e. conflict between two references of the
same statement, are preserved, even if they are not loop-carried. They
are required for instruction level scheduling (not implemented in PIPS)
and for consistency across program transformations. After instruction
atomization (see \verb/pipsmake-rc.tex/), the number of dependence arc
should remain the same.

Loop interchange, for instance, requires more information than
dependence levels, namely {\em dependence direction vectors}
(DDV). DDV's are not computed in PIPS because a more precise
information, dependence cone and/or dependence polyhedron, is made
available. These convex polyhedra are not represented by sets of affine
constraints but by their generating systems because generating systems
are directly useful to constraint schedules (hyperplane methods) and
tilings (supernode partitioning).

The ultimate dependence information is {\em value} and not {\em
location} based. PIPS contains an implementation of Pr. Feautrier's
Array Data Flow Graph (DFG). The DFG is implemented with another data
structure described in \verb/paf_ri.f.tex/.

\begin{comment}
Le domaine co^ne de'finit une approximation polye'drique de l'ensemble
des de'pendances porte'es par un arc. Plusieurs types d'approximation
sont possibles: les niveaux de de'pendance, les vecteurs de direction de
de'pendance, les syste`mes ge'ne'rateurs (aussi
connus sous le nom de {\em co^ne de de'pendance}) et le me'canisme
du {\em Data Flow Graph} n'est pas imple'mente' avec cette structure
de donne'es mais avec \verb|paf_ri|.
\end{comment}

\begin{comment}
Le niveau d'une de'pendance d\'ecrivent le nombre de boucles englobantes
concerne\'ees.  ``{generating\_system}'' est la r\'epresentation de
system generateur.
\end{comment}

\section{Strongly Connected Components}
\label{section-strongly-connected-components}

\subsection{Flags for Strongly Connected Components}

\domain{Sccflags = enclosing\_scc:scc x mark:int x dfnumber:int x lowlink:int}
{}

This type is an auxiliary type used when computing the strongly
connected components of the dependence graph. More information
available in Tarjan's algorithm description?

\begin{comment}
Ce domaine est utilise' par l'algorithme de calcul des composantes
fortement connexes d'un graphe.
\end{comment}

\subsection{Sets of Strongly Connected Components}

\domain{Sccs = sccs:scc*}
{}

This data type is used to return the strongly connected components of
the dependence graph. It is a simple list of SCC's.

\begin{comment}
Ce domaine permet de contenir le re'sultat de l'algorithme de calcul
des composantes fortement connexes d'un graphe. Il s'agit d'une liste de
{\tt scc}, c'est a` dire de composantes fortement connexes.
\end{comment}

\subsection{Strongly Connected Component}

\domain{Scc = vertices:vertex* x indegree:int}
{}

A strongly connected component is a set of vertices, represented as a
list, and an integer field, \verb/indegree/, used for their topological
sort. Allen \& Kennedy algorithm is based on topological sorts of the
dependence graph SCC's.

\begin{comment}
Ce domaine permet de repre'senter une composante fortement connexe d'un
graphe. Une {\tt scc} se compose d'un ensemble de noeuds, et d'un
sous-domaine {\tt indegree} utilise' par l'algorithme de tri topologique.
\end{comment}

\begin{thebibliography}{9}

% pipsmake

\bibitem{YAI95} Y. Yang, C. Ancourt, F. Irigoin,
{\em Minimal Data Dependence Abstractions for Loop Transformations}, 
International Journal of Parallel Programming,
v. 23, n. 4, Aug. 1995, pp.~359-388

\bibitem{JT89} P. Jouvelot, R. Triolet,
{\em NewGen: A Language Independent Program Generator},
Rapport Interne CAII 191, 1989

\bibitem{JT90} P. Jouvelot, R. Triolet,
{\em NewGen User Manual},
Rapport Interne CAII ???, 1990

\end{thebibliography}

\end{document}
\end
