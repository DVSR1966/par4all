\section{Selection of the optimal architectonic solution}

The selection of the optimal architectonic solution and libraries used to create the framework is not an easy task. Constraints imposed on the PaWS framework (see Section \ref{design_contraints}) require creation of the web application. It can be done with various programming languages and libraries. 

\subsubsection{Language selection}

One group of languages used in bigger development projects are Java, C++ and C\#. Using them requires program compilation and creation of binary files. Also the programming process requires more programming work and it is not efficient. What is more, C\# is dedicated to the Microsoft Windows operation system which disqualifies its use (see Design Contraints Section \ref{design_contraints}).

The second group of possible languages are scripting languages like Perl, Python, Ruby and PHP. Perl is an obsolete language, but all of them are popular as a languages for web application creation, being constantly developed, free upward-compatible and with lively community, rich documentation, libraries and support. That explains their popularity.

\subsubsection{WEB support}

There are three aproaches to create web application using scripting languages. The first of them is based on CGI\cite{cgi} - Common Gateway Interface. This technology enables communication between web server software and and other programs located on the server. The main disadvantage of this approach is that CGI usually requires the creation a new process for each request. This is not scalable, makes not possible to use the same context and global variables for several requests and can cause server load over very quickly. What is more, CGI does not provide session mechanisms and the Ajax library is not built in.

The second possibility is to use template engines like Cheetah \cite{cheetah} or Jinja2 \cite{jinja2}. The problem is that this solution is too lightweight - it supports only code generation and not provides other useful mechanisms such as routing, session variables handling etc.

The third approach is the most popular approach. It is based on the Model-View-Controller design pattern (described in Section \ref{mvc}). This paradigm supports reusability, separation of layers of the application, is lightweight but easy to extend and takes care of persistent storage if it is needed. There are a lot of framework using the MVC pattern like Pylons \cite{pylons}, Django \cite{django} and Turbogears \cite{turbogears} in Python, Ruby on Rails \cite{rubyonrails} in Ruby and Symfony \cite{symfony}, CakePHP \cite{cakephp} and Yii \cite{yii} in PHP.

\subsubsection{WEB frameworks}

PHP frameworks are more heavyweight and harder to learn than Python and Ruby ones. The most popular PHP web framework - Symphony also has problems with compatibility between its versions which makes difficult to use its support.

Ruby on Rails is a very good framework, but has less support than Pythonic frameworks, because Ruby is younger language than Python. Also Ruby frameworks have performance problems.

The use of Python tools has also other advantages. The main one is the availability of a Python binding in PIPS framework - PyPS, which can be use to dynamically perform PIPS operations instead of creating Tpips (see Section \ref{tpips_interface}) scripts on-the-fly. The selection of Python enables using PyPS in easy way (by importing it as a Python module) but also does not exclude usage of Tpips.

Three Python web frameworks were taken into consideration: Pylons, Django and Turbogears. The last one was rejected because of stability problems\footnote{TurboGears is based on CherryPy \cite{cherrypy} unstable server.}. Despite the fact that Django is more popular than Pylons, the second one was chosen due to greater flexibility. Pylons can be extended with any component the user needs, while Django has a set of its own preferred components. A good example of this problem is the usage of ORM (Object-relational mapping\cite{orm}). Django does not allow to use all of the available toolkits, like the most popular one - SQLAlchemy\cite{sqlalchemy}.

Choosing the Pylons tool and its approach to architecture and structure of the application paid off very quickly - after less than seven days of work, the first prototype of the application was ready and able to communicate with PIPS. 
%% Very significant fact is that PaWS is not the first CRI project based on Pylons and I could count on strong support from more experienced developers.

