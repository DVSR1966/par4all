
The major aim of the project was to create lightweight web application, which allows to use PIPS without installing it. Other goals were to provide different types of ways of using PIPS, which were depending on user's level of advance. Whole list of constraints and functional requirements can be found in the Sections \ref{design_contraints} and \ref{design_requirements}.

To achieve those goals, web application was developed, which presents PIPS in two modes, described in the Section \ref{paws_project}. Content of all the pages of those modes is generated dynamically and is based on the structure of the directories and files with descriptions. Thanks to this solution, configuration of the PaWS is very easy and does not require knowledge of the Pylons framework.

Implemented functionality contains also a lot of user-friendly features (see requirements in the Section \ref{design_requirements} and components used \ref{components_used}), like creating, displaying and zooming graphs, resizing font size (according to user preferences), uploading user files (also archive .zip files), saving results of the operations on user's machine or printing them.

Main goal for the future is to implement third mode, full control, which will enable user to graphically create his/her own PIPS script. PaWS should also have more tests - currently there are no tests for its graphical interface. Other new utility would be Pyps/Tpips code generation with all the properties, analyses and phases used (for advanced mode). Also possibility of saving and printing dependence graphs would be useful.

Other small, possible improvements are:
\begin{itemize}
  \item Adding the scale for a slider in demonstation module.
  \item Control the time of the response from the server.
  \item Integration of the IR Navigator \cite{irnavigator}.
\end{itemize}

Current version of the PaWS framework meets all the design constraints and requirements imposed at the beginning. It can also be easily extended with the new functionality.