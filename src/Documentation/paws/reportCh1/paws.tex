\section{PaWS - Towards Pips as a Web Service}

For the time being, PaWS is a WEB interface. Some more work is needed
to make it a WEB service.


\subsection{What is PaWS?}

The initial aim of the PaWS project is to create a customizable, light and
user-friendly WEB graphical interface for the PIPS framework.

\subsubsection{PIPS and PyPS}
\label{pips_and_pyps}

{\bf PIPS}, a.k.a. \emph{Parallel Integrated Programming System} is a
source-to-source compilation framework for analyzing and transforming
C and Fortran programs~\cite{pips4u}. It can be used to verify, optimize and
parallelize programs at source level. PIPS helps the users to
understand compilation process and to optimize their programs
(especially in reducing the execution time and latency). It can also
provide assistance in debugging, maintaining and reingeneering code.

{\bf PyPS} is the Python API for PIPS framework. PyPS can be used as a
scripting language for creating high-level scripts to perform a
sequence of PIPS operations. It also provides a kind of shell for
interactive usage of PIPS, with on-line help and documentation. Another
advantage of Pyps is that it does not require deep knowledge of
Python, so it can be used even by not advanced users.

PIPS framework has other line interfaces such as tpips (see
\ref{tpips_interface}), a shell interface, Pips, and a Java interface,
jpips.

%% the programable pass manager for pips

\subsubsection{Problems}

PIPS is a very powerful but complex framework. Firstly, this makes the
installation process complicated. It often requires also installation
of other packagies, and their own dependencies. The second problem is
the lack of user-friendly graphical interface which would allow to use
PIPS in an intuitive way.

\subsubsection{Solution - PaWS}

PaWS (Pips as Web Service) is a WEB application, which was designed
and implemented as a solution to the problems mentioned above. It
enables people to experiment with PIPS without installing it on their
machines. It presents not only elementary analyses or transformations
that are available in PIPS, but also allows the user to experiment
with more complex combination of PIPS passes. Besides that, the user is
able to generate and view dependence graphs from source code.

Depending on the user level, PaWS provides different ways of using
PIPS - from an intuitive demonstration to the ability to change PIPS
scripts code. Each user can decide which approach is most appropriate for
him/her.

PaWS provides ready examples which might help to understand how
PIPS works, but it also lets user try analysis and transformations of
his/her own code. There are also other user-friendly features
available, like displaying graphs, printing result code or saving it
on users machine.

For the time being, the elementary analyses and transformations are
preconditions, OpenMP and Array, IN and OUT regions, all with examples
available. There are also three tpips tutorials: acca, aile-excerpt
and convol.
