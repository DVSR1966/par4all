\section{Process of creating PAWS}

PaWS framework was created with a Agile Methodology \cite{agilemethodology}. Whole process had several short iterations, which were incrementing PaWS functionalities. Each iteration had its own list of the requirements refering to the general PaWS design requirements and constraints (see Sections \ref{design_contraints} and \ref{design_requirements}). After completing the iteration, it was summed up and, according to it result, main goals might have been slightly modified. List of the encountered problems is in the Section \ref{encountered_problems}.

The goal of the first iteration was to create working skeleton of the framework, which was linking Pylons and Pyps technologies together. It was performing \emph{preconditions} analysis. Next steps included adding new PIPS tools and customizing pages by implementing user-friendly features (saving, printing results, source code colorization). Further action was to extend PaWS by the second mode - demonstration and by possibility of creating graphs. 

At the same time, consistency and stability of the PaWS framework (see Section \ref{encountered_problems}) has been being improved.

The goal of the last step of the project was to improve existing framework by writing tests, scripts for administrators to add new tool or demo, refactor code and framework structure and implementing several user-friendly features (such as possibility of uploading archive files). PaWS also was included in the PIPS building process.

%% TO DO: About process of creating PaWS application - from what we started, what was added, what was not good idea.
%% trial and error process
%% no ... agile programming
%% V CYCLE, AGILE

%% exploratory process, agile development, validation, evaluation


