\documentstyle[a4,psfig]{report}

\input{wpips-epips-declarations}

\begin{document}

\title{WPips \& EPips User Manual\\ (Parall�liseur Interproc�dural de
  Programmes Scientifiques) --- a Linear Algebra based Automatic
  Parallelizer \& Program Transformer}

\author{Ronan Keryell}

\maketitle

\chapter{Introduction}
\label{cha:introduction}

PIPS is an automatic parallelizer for scientific programs that takes as
input Fortran 77 and HPF codes.  Since its inception in 1988, it
emphasizes interprocedural techniques based on linear algebra
techniques.

PIPS is a highly modular framework which can be used to test and
implement independently various compilation schemes, program analyses
and transformations, such as automatic vectorization and
parallelization (Cray Fortran, CRAFT, CM Fortran, F90 \& PVM output),
HPF compilation, loop transformations, partial evaluation,\ldots

The compiler is made of independent phases that are called on demand to
perform the analyses or transformations required by the user.
Consistency and interprocedurality issues are dealt with
automatically.

The phase chaining is driven by an interprocedural make system {\em
pipsmake\/} and internal data structures are automatically stored in a
database {\em pipsdbm\/} wich manages persistence through different
runs. Persistence is used to avoid the memory bottleneck when large
programs are analyzed interprocedurally. It is also used by newcommers
to develop new applications without interferring with existing code.

PIPS is built on top of two tools. The first one is data structure
description language, {\em Newgen\/}, which manages allocation,
de-allocation, storage for persistent objects, walks thru complex data
structures, and provides basic classes, such as list, set and mapping.
{\em Newgen\/} is embedded in C, CommonLISP and SML.
All PIPS data-types are declared with {\em Newgen\/} descriptions and used with the C or the Common LISP embedding.

The second tool is the {\em Linear $C^3$\/} library which handles
linear formul\ae{} and structures based on these, such as vectors,
constraints, polyhedra, generating system. This library is used to
analyze codes (preconditions, regions, dependence test) and to generate
new versions of codes (partial evaluation, unimodular transformations,
tiling, send/receive, temporary array allocation,...)

Three interfaces are available: a shell interface ({\em Pips\/}), an X-WindowS interface
({\tt wpips\/}) and a hypertextual interface ({\tt epips\/}).

To have more information: {\tt http://www.cri.ensmp.fr/$\sim$pips}

This report give a user manual on the X-WindowS (XView) interface
\WPIPS{} and the hypertextual GNU-Emacs based interface \EPIPS{}.

With these interfaces, a user can use most of the \PIPS{} project
environment features:
\begin{itemize}
\item create workspaces from Fortran sources;
\item close and re-open these workspace later by exploiting the
  \PNewGen{} persistent data types;
\item edit the Fortran sources;
\item analyse the programs by displaying various interprocedural
  informations based on the interprodural linear algebra framework
  developped in the \PIPS{} project, such as the known execution
  context for each instruction (the \Ppreconditions);
\item apply interactively by using menus various program
  transformations such as partial evaluation, loop transformations,
  etc, to improve program efficiency;
\item parallelize the code and target various parallel Fortran
  dialects including one emulating a shared memory in \PVM{};
\item compile \HPF{} code to Fortran~77 with a \PVM{} back-end
\end{itemize}

The \PIPS{} environment is also configurable and quite modular at the
developper level and allow easily some various extensions.

\chapter{Getting started}
\label{cha:getting_started}


\chapter{User manual}



\section{The PIPS Unix commands}
\label{sec:unix_commands}


\subsection{WPips}
\label{sec:unix_wpips}

The {\tt wpips} command is used to run the OpenLook/X11 interface for
\PIPS. {\tt wpips} does not need any OpenWindows specific feature. You
need to have a correctly initialized {\tt DISPLAY} variable with
enough access rights to the X11 display.

The default directory is the one where {\tt wpips} is launched.



\subsection{EPips}
\label{sec:unix_epips}

For more GNU Emacs familiar users, there is an extension to {\tt
  wpips} that use some Emacs windows to display various \PIPS{}
informations. You can bennefit various Emacs advanced features, like
couloured prettyprinted Fortran code, hypertextual interaction on
the code, etc.

If you have already an Emacs running, {\tt M-X epips} launches a
special {\tt wpips} instance from Emacs. You need to load some E-Lisp
stuff before, for example by modifying your {\tt .emacs} file
according to the \PIPS{} {\tt README}. The default directory is the
one of the Emacs buffer where {\tt wpips} is launched.

You can also launch a separate Emacs that deals with 

\section{Some basics about the OpenLook interface}
\label{sec:openlook}

The most useful button is the {\bf right mouse button} since it is
used to select everything you want in the menus or the panels.

The left button is used to pick a default selection as a short cut if
you want exactly what you want. By using the control+right mouse
button you can change the default selection of a menu as you want.

Some menus have a ``push pin''. If you click on it, the menu is
changed in a panel window you can place as you want. It is useful when
you often use a menu.

Some menu items or display items may be shaded. That means that they
cannot be selected by the user according to the current situation.


\section{Basic Pips containers: workspaces \& modules}
\label{sec:workspace_module}

In order to analyse Fortran programs, \PIPS{} create a workspace where
it puts all the information about the source, the transformated code,
some compiled code, some executables for \PVM{} output, etc. Thus the
first thing to begin with \PIPS{} is to ask for a workspace creation
in the current directory.

Each source code is splitted in modules that are the functions and
procedures of the program. Most of the \PIPS{} transformations can
deal with a module but some other ones, like interprocedural analyses,
deal with all the modules at once, that is with all the modules of the
workspace.

Workspaces and modules are thus the basic containers used in \PIPS{}.

\section{WPips: the main panel}
\label{sec:main_panel}

\psfig{file=images/main_panel.eps,width=\hsize}

The main panel contains most of the menus usable in \PIPS{} and is the
window that appears first when \PIPS{} begins. It also give various
informations on the \PIPS{} current state.


\subsection{Message}
\label{sec:main_message}

The line give you the last information message given by \PIPS, such as
a warning, a log or an error message. Usually, this line is the same
as the last line of the log window (see~\ref{sec:log_window}) but is
useful since this may be closed or hidden.


\subsection{Directory}
\label{sec:main_directory}

This line display the current directory and can be user-edited to
change directory (but only when there is no workspace currently
open). By using the small directory menu, one can change the directory
by browsing the tree structure.



\subsection{Workspace}
\label{sec:main_workspace}

This line of the main panel display the current open workspace. It can
be edited to open an old workspace or create a new one if a workspace
with this name does not already exist in the current directory.  If a
workspace is already open, it is first closed befor opening or
creating a new one.

There is also a small workspace menu that allows to open a workspace
from the current directory, to close the current one or to delete a
workspace. If one try to delete the current workspace, it is closed
first. 

Creating a workspace ask for a workspace name. If a workspace with
this name already exists, \PIPS{} ask for its deletion.
Then, \PIPS{} pop a window with the list of all the Fortran programs
of the current directory. One can give a list of file names separated
with space(s) or more easily by selecting the files with the mouse in
the scrolling list.



\subsection{Module}
\label{sec:main_module}

After selecting a workspace, the module line and menu should become active.

The module line display the current module selected that is the main
module (the module from a {\tt PROGRAM} statement) if any.

The user can type a module name to select one or use the small menu to
select one quickly (if there is not too many modules to fit in the
screen).



\subsection{Memory}
\label{sec:main_memory}

This line display an idea of the {\tt break} limit of the \PIPS{}
process, that is the memory used in megabytes by the code and the data
(but not by the stack). This information cannot be modified by the
user\footnote{It is not possible to give some memory back...}.


\subsection{CPU time}
\label{sec:main_CPU_time}

Another useful information for experimentation is the \CPU{} user
time given by this line.


\subsection{Number of display windows}
\label{sec:main_number_display_windows}

This item shows the number of active windows at this time. The default
number is 2, that means that when some code is displayed it uses,
cyclicly the 2 windows availables (see also \ref{sec:display_windows} to
retains some windows).

The number of active windows can be modified by editing the line or
easier by clicking up or down the small arrows.



\subsection{PIPS icon button}
\label{sec:pips_icon_button}

This button is used to interrupt the current PIPS work. It is taken
into account by \PIPSMAKE{} at a phase boundary. That means that if
you are doing a compute intensive phase, you will wait until the end
of this one, since it is the only way to have clean and easy data
coherence.

%\section{Select menu}
%\label{sec:select_menu}

%\psfig{file=images/select_menu.eps,width=\hsize}


\section{View menu}
\label{sec:view_menu}

It is used to display some code or some informations from \PIPS. To
display these, \PIPS{} may execute lots of analyses or code
transformations according to \PIPSMAKE.

\begin{center}
  \mbox{\psfig{file=images/view_menu.eps,width=0.3\hsize}  }
\end{center}

\begin{description}
\item[Lasts/No selection:] in \WPIPS{} only, to open all the last
  display windows;

\item[\PSequentialView:] it is the basic view of a module : the code is
  displayed as \PIPS{} understand it. It may be decorated with some
  internal informations by using options ``\PSequentialView''
  (section~\ref{sec:option_sequential_view});

\item[\PUserView:] since the Sequential View is a prettyprinted version
  of the code, some program details may have been modified an
  ``\PUserView'' give the code before parsing by \PIPS{}. It is closer
  to the original code. Of course, after transforming the code by the
  user, this code may be meaningles... As for the Sequential View, the
  code can be decorated with some internal informations by using
  options ``\PUserView'' (section~\ref{sec:option_user_view});

\item[\PSequentialViewControlGraph:] this button is used to
  display the \Pcontrolgraph{} of the \Pcurrentmodule{} with the graph
  editor \PdaVinci{}. Each node contains some code statements that can
  be decorated as for the \PSequentialView{} by using options
  ``\PSequentialViewControlGraph''
  (section~\ref{sec:option_sequential_view_control_graph}).
  
  Displaying the control flow graph is interesting to precisely
  analyze some codes in order to figure out what structural
  optimizations to apply. This kind of graph view is also used to
  display the interprocedural control flow graph (see
  section~\ref{sec:ICFG_view}) and the more classical call graph (see
  section~\ref{sec:call_graph_view}) the linking the various
  procedures and functions of a program.
  
  In \PIPS{}, the control graph is represented with a hierachical
  control flow graph, as explained about the \Punstructured{} in the
  \Pri{}. The controf flow graph is hierarchical since a statement can
  contain an unstructured graph to represent another control graph
  with no edge outside of the statement.

  
  The control flow is represented in a directed graph of
  \Punstructured{} nodes. For example, a \PGOTO{} leads to an edge
  from the source to the destination, an \PIF{} with some \PGOTO{}s
  leads to one edge to the \PTHEN{} branch and another one to the
  \PELSE{} branch.
  
  As a consequence a node without predecessor is unreachable and can
  be discarded (see section~\ref{sec:unspaghettify}).
  
  In the \PdaVinci{} output, the following style hints are used:
  \begin{itemize}
  \item the first statement block of the programm is yellow;
  \item the entry node of an \Punstructured{} is a light green ellipse;
  \item the exit node of an \Punstructured{} is a light grey ellipse;
  \item an unstructured \PIF{} (that is an \PIF{} with some \PGOTO{}s)
    is a cyan rhombus (if it is the entry of the \Punstructured{} it is
    a light green one);
  \item a \PTHEN{} branch is blue;
  \item a \PELSE{} branch is red.
  \end{itemize}

\item[\PDependenceGraphView:] displays the dependence graph view. \ref{NeedFI};
  
\item[\PArrayDataFlowGraphView:] displays the array dataflow graph of
  the code, that is the information used to track array data flowing
  in the program as used in the method also known as {\em
    Feautrier\/}'s one. Just an example to explain a little bit the
  output:
\begin{verbatim}
INS_100:
********
 Execution Domain for 100:
{
  J - 10 <= 0 ,
- J + I + 1 <= 0 ,
  I - 10 <= 0 ,
- I + 1 <= 0 ,
} 

 ---Def-Use---> ins_130:
  Reference: X
  Transformation: [I,J]
  Governing predicate:
{
  K - 1 <= 0 ,
} 
  Execution Domain for 130:
{
- I + K + 1 <= 0 ,
- K + 1 <= 0 ,
  J - 10 <= 0 ,
- J + I + 1 <= 0 ,
  I - 10 <= 0 ,
- I + 1 <= 0 ,
}
\end{verbatim}
  It first describes the data generated by the instruction 0 of line
  10, that is {\tt INS100} with its execution domain. There is a
  use-def dependence with instruction 0 of line 13, that is {\tt
    INS130}, about array {\tt X} only if the governing predicate is
  true, that is $K - 1 \leq 0$ here. Then, we have the execution
  domain on {\tt X(I,J)} that used the data previously defined in {\tt
    INS100}. {\tt Transformation: [I,J]} means that {\tt X(I,J)}
  is defined in the loop-nest {\tt INS100(i,j)} with $i=I,j=J$;

\item[\PSchedulingView:] give a tiem base for each instruction. For example 
\begin{verbatim}
ins_140:
     pred: TRUE
     dims: -1+3*I
\end{verbatim}
  means that if the predicate is true (here it is of course always
  true...), these instruction ins executed at time $-1+3\times I$;

\item[\PPlacementView:] gives where an instruction is executed. For example
\begin{verbatim}
Ins_140 :I , J
\end{verbatim}
  means that these instruction is executed on processor $(I,J)$;
  
\item[\PCallgraphView:] display a tree of all the functions or
  procedures called in the current module. The code can also be
  decorated according to the options ``\PCallgraphView''
  (section~\ref{sec:option_call_graph});
  
\item[\PICFGView:] display a more precise tree than the
  \PCallgraphView. It is the Interprocedural COntrol Flow Graph, where
  each call, {\tt do}-loops and \PIF{} added according to the options
  ``\PICFGView'' (section~\ref{sec:option_ICFG_view});
  
\item[\PDistributedView:] ask \PIPS{} for parallelizing the code with
  the WP65/PUMA method, {\bf with all the prerequisite of this method
    on the input code}. The output is in fact not 1 code but 2 ones:
  \begin{itemize}
  \item the computational code;
  \item the memory bank code that does the parallel memory feature;
  \end{itemize}

\item[\PParallelView:] ask \PIPS{} for parallelizing the code and
  displaying it according the parallel dialect given in the options 
  ``\PParallelView'' (section~\ref{sec:option_parallel_view});

\item[\PFlintView:] launch a Fortran {\tt lint} on the module and give
  the information back to the user;

\item[Close:] in \WPIPS{} only, to close all display windows.
\end{description}


\section{Transform/Edit}
\label{sec:transform/edit_menu}

The Transform/Edit menu is used to apply various transformations on
the current module in \PIPS. Furthermore, the user can edit the code
of the module as a special transformation.

\begin{center}
  \mbox{\psfig{file=images/transform_edit_menu.eps,width=0.3\hsize}}
\end{center}

\input{wpips-epips-transform-menu.tex}

\begin{description}
\item[\PDistributeLoops:] distribute all the loops of the module;

\item[\PFullLoopUnroll:] unroll all the iterations of a loop with
  constant loop bounds (that means that some ``\PPartialEval'' may be
  useful before);

\item[\PLoopUnroll:] unroll a loop by a given number of iterations;
  
\item[\PLoopInterchange:] interchange a loop nest by pointing the
  outermost loop;
  
\item[\PLoopNormalize:] normalize all the loop of the module.
  Afterward, all loops have a step of 1 and begin at 1, if possible;

\item[\PStripMining:] replace a loop by a loop nest;

\item[\PPrivatizeScalars:] try to find loop private scalar variables;

\item[\PPrivatizeScalarsAndArrays:] try to find loop private scalar and
  array variables by using the IN/OUT regions;
  
\item[\PDeadCodeElimination::] use the semantical anylisis of \PIPS{}
  to remove some use less code, such as \PIF{} alway true, loops never
  executed, etc. Useful to simplify the code after code
  transformations;
  
\item[\PPartialEval:] use the semantical anylisis of \PIPS{} to
  compute values known at compile time. Useful to simplify the code
  before applying some code transformations;

\item[\PRestructureWithSTF:] use the STF restructrurer from ToolPack
  to restructure the code;

\item[\PUnspaghettifyTheControlGraph:] simplify the internal
  representation of the control graph and try to restructure some
  \PIF;

\item[\PUseDefElimination:] remove the instructions that are not
  involved in any IO nor are necessary for any IO related result;

\item[\PAtomize:] transform the code in simpler one such as {\tt A = B
    op C};

\item[\PNewAtomize:] another version of ``\Patomize'', more robust;

\item[\PStaticControlize:] try to transform the code in a more
  suitable form for the polyhedral method: normalize the loops, create
  an internal ressource, etc;
  
\item[\PEdit:] a special transformation : the user one! Load the
  original code of the module. Do not forget to save your modification
  after you have finished (the menu {\em File/Save Current File\/} in
  a \WPIPS{} Edit window or the menu {\em Save the file after edit in
    the seminal .f} in \EPIPS).

\end{description}


\section{Compile}
\label{sec:compile_menu}

\begin{center}
  \mbox{\psfig{file=images/compile_menu.eps,width=0.3\hsize}}
\end{center}

\begin{description}
\item[\PCompileAnHPFProgram:] compile all the modules with the \HPF{}
  compiler;

\item[\PMakeAnHPFProgram:] run {\tt make} on the \HPF{} program;

\item[\PRunAnHPFProgram:] go a step further by trying to run the
  Fortran 77 output of the \HPF{} compiler;

\item[\PViewTheHPFCompilerOutput:] this menu allows you to view one of
  the files generated by the \HPF{} compiler. For each module, the
  main files are the {\tt \_host.f} file for the scalar code and the
  {\tt \_node.f}  file for the parallel code.

\end{description}

\section{Options}
\label{sec:options_menu}

\begin{center}
  \mbox{\psfig{file=images/options_menu.eps,width=0.3\hsize}}
\end{center}


\chapter{Conclusion}
\label{cha:conclusion}

\end{document}
