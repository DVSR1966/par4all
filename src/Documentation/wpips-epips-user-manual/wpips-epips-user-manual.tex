\documentclass[a4paper]{report}
\usepackage[latin9]{inputenc}
\usepackage[backref,pagebackref]{hyperref}
\usepackage{xspace}
\usepackage{url}
\usepackage{epsfig,xr,fancybox,amstext,alltt,pstricks}
%% Since it includes some parts of pipsmake-rc.tex, add some external
%% references from it:
%\externallabels{http://www.cri.ensmp.fr/pips/pipsmake-rc}{/projects/Pips/Documentation/public_html/pipsmake-rc/labels.pl}
%\externaldocument{../pipsmake/pipsmake-rc}
%% Idem for LaTeX with "xr":
%% Je voudrais bien trouver un moyen de rajouter un pr�fixe � ces
%% labels... :-(

\newenvironment{PipsExample}{\VerbatimEnvironment
\begin{Verbatim}}{\end{Verbatim}}

\newcommand{\PipsWEpipsPDF}{\url{http://www.cri.ensmp.fr/pips/wpips-epips-user-manual.htdoc/wpips-epips-user-manual.pdf}}
\newcommand{\PipsWEpipsHTDOC}{\url{http://www.cri.ensmp.fr/pips/wpips-epips-user-manual.htdoc}}

\input{wpips-epips-declarations.tex}

\begin{document}

\title{WPips \& EPips User Manual\\ (Parall�liseur Interproc�dural de
  Programmes Scientifiques) \\
  --- \\
  Linear Algebra based Automatic
  Parallelizer \& Program Transformer}

\author{Ronan Keryell}

\maketitle

A WWW version of this report can be shown at \PipsWEpipsHTDOC and a
printable version of this report at \PipsWEpipsPDF.

\setcounter{tocdepth}{6}
\tableofcontents

\centerline{\psfig{file=images/logo-pips-embossed.eps,width=0.25\hsize}}


\chapter{Introduction}
\label{cha:introduction}

\PIPS{} is an automatic parallelizer for scientific programs that takes as
input Fortran 77 and HPF codes.  Since its inception in 1988, it
emphasizes interprocedural techniques based on linear algebra
techniques.

\PIPS{} is a highly modular framework which can be used to test and
implement independently various compilation schemes, program analyses
and transformations, such as automatic vectorization and
parallelization (Cray Fortran, CRAFT, CM Fortran, F90 \& PVM output),
HPF compilation, loop transformations, partial evaluation,\ldots

The compiler is made of independent phases that are called on demand to
perform the analyses or transformations required by the user.
Consistency and interprocedurality issues are dealt with
automatically.

The phase chaining is driven by an interprocedural make system
\PIPSMAKE{} and internal data structures are automatically stored in a
database {\em pipsdbm\/} wich manages persistence through different
runs. Persistence is used to avoid the memory bottleneck when large
programs are analyzed interprocedurally. It is also used by newcommers
to develop new applications without interferring with existing code.

\PIPS{} is built on top of two tools. The first one is data structure
description language, \PNEWGEN{}, which manages allocation,
de-allocation, storage for persistent objects, walks thru complex data
structures, and provides basic classes, such as list, set and mapping.
\PNEWGEN{} is embedded in C, CommonLISP and SML.
All \PIPS{} data-types are declared with \PNEWGEN{} descriptions and used with the C or the Common LISP embedding.

The second tool is the \PLinear{} library which handles
linear formul\ae{} and structures based on these, such as vectors,
constraints, polyhedra, generating system. This library is used to
analyze codes (preconditions, regions, dependence test) and to generate
new versions of codes (partial evaluation, unimodular transformations,
tiling, send/receive, temporary array allocation,...)

Three interfaces are available: a shell interface ({\em Pips\/}), an X-WindowS interface
({\tt wpips\/}) and a hypertextual interface ({\tt epips\/}).

To have more information:
\url{http://www.cri.ensmp.fr/pips}.

This report give a user manual on the X-WindowS (XView) interface
\WPIPS{} and the hypertextual GNU-Emacs based interface \EPIPS{}.

With these interfaces, a user can use most of the \PIPS{} project
environment features:
\begin{itemize}
\item create workspaces from Fortran sources;
\item close and re-open these workspace later by exploiting the
  \PNEWGEN{} persistent data types;
\item edit the Fortran sources;
\item analyse the programs by displaying various interprocedural
  informations based on the interprodural linear algebra framework
  developped in the \PIPS{} project, such as the known execution
  context for each instruction (the \Ppreconditions);
\item apply interactively by using menus various program
  transformations such as partial evaluation, loop transformations,
  etc, to improve program efficiency;
\item parallelize the code and target various parallel Fortran
  dialects including one emulating a shared memory in \PVM{};
\item compile \HPF{} code to Fortran~77 with a \PVM{} back-end
\end{itemize}

The \PIPS{} environment is also configurable and quite modular at the
developper level and allow easily some various extensions.

\chapter{Getting started}
\label{cha:getting_started}


In the {\tt GettingStarted} directory, there is a collection of small
programs to start up with the \PIPS{} environment.

\section{Parallelizing a small program}
\label{sec:parallelizing_a_small_program}

We begin here by using 4 different parallelizing methods on a small
program, a matrix multiplication.


\subsection{Parallelize or vectorize the code}
\label{sec:parallelize_or_vectorize_the_code}


\subsection{Generate parallel PVM code from HPF}
\label{sec:generate_parallel_PVM_code_from_HPF}


\subsection{Generate distributed code for processors and memory banks}
\label{sec:generate_distributed_code}

{\em This section comes mainly from Corinne A{\sc ncourt}.}

%%
%% $Id$
%%
%% Copyright 1989-2010 MINES ParisTech
%%
%% This file is part of PIPS.
%%
%% PIPS is free software: you can redistribute it and/or modify it
%% under the terms of the GNU General Public License as published by
%% the Free Software Foundation, either version 3 of the License, or
%% any later version.
%%
%% PIPS is distributed in the hope that it will be useful, but WITHOUT ANY
%% WARRANTY; without even the implied warranty of MERCHANTABILITY or
%% FITNESS FOR A PARTICULAR PURPOSE.
%%
%% See the GNU General Public License for more details.
%%
%% You should have received a copy of the GNU General Public License
%% along with PIPS.  If not, see <http://www.gnu.org/licenses/>.
%%

%\documentstyle[psfig,11pt]{article}

%\title{Code Generation for distributed memory machines}
%\author{Corinne Ancourt}
%\begin{document}
%\maketitle


To provide accurate results in reasonable time, real applications such
as seismic, fluid mechanics and structure computation applications (that
use large set of data and costly algorithms) need to be widely
parallelized. Distributed memory machines are good candidates for these
applications because they support large number of processors without
the shared memory bottle neck problem.  However, distributed memory machines
are much more difficult to program efficiently than shared memory machines.  Shared memories are
often managed by the hardware. Conscientious programmer must only
restructure its program in order to take care of  cache use.
In distributed memory machines, the distribution of data onto the local 
memories must be designed  carefully in order to keep good performance and 
to avoid too much communications.

Many approaches to generate distributed code have been suggested.
Some are language-based. The programmer has to specify the data
distribution pattern and the parallelism but he does not have to
generate processes nor send/receive pairs between processors.
Processes are automatically derived from the data distribution using
the so-called {\em owner computes rule} and a SPMD model: each
processor executes instruction producing a piece of data located in
its memory.  Send/recei\-ve pairs are derived from the data
distribution and from the instruction reference patterns.


Other approaches are  based on the operating system or/and 
hardware support. A virtual shared memory provides the uniform address space.
Some software mechanisms or complex cache systems maintains the memory consistency.


Three different approaches have been implemented in Pips. The first one
is language-based. High Performance Fortran is designed to express data
parallel applications for distributed memory machines. It provides the
user with  a set of directives to specify data distribution and parallelism. A prototype  HPF compiler  is implemented in Pips.

The second  approach is also language-based but the data  distribution and the scheduling are 
automatically computed. Designed for static control programs, this technique
 generates SIMD programs expressed  in CRAFT (Fortran CRAY-T3D) or CM-Fortran.

Finally, the third approach  suggests the emulation of a shared memory onto a
distributed memory machine. Classical parallelization techniques are used
to generate SPMD code. The compiler manages the emulated shared memory and the maintenance of memory coherency.

The following section introduces some characteristics of  these three
 approaches. It focuses on the scheduling differences. More details are 
given   in the associated presentation papers.
 Example presented  in Figure~\ref{ex1} illustrates the approaches.


\begin{figure}[htp]
\begin{verbatim}
    DO I=1,10
S1       B(I,I)=1
    ENDDO
    DO I=1,10
      DO J= I+1,10
S2        B(I,J)=1
S3        A(I,J)=B(I,J-1)
      ENDDO
    ENDDO  
\end{verbatim}
\caption{Program 1}
\label{ex1}
\end{figure}

\subsubsection{HPF program}

In HPF, the  data distribution and the parallelism are specified
 via directives. The
parallel execution of the distributed application is guided by the {\it
owner computes} rule: a processor can update only its local variables. Let's take our example. 

\begin{figure}[htp]
\psfig{file=figures/hpfc1.idraw}
\caption{Data distribution}
\label{hpfc}
\end{figure}

Due to the data dependences existing on the second dimension of Array B, 
the data distribution that minimizes
the communications groups the array elements by rows. Here,
 blocks of 2 rows are distributed onto the 5 processor local memories as 
depicted in Figure~\ref{direct}.

\begin{figure}[htp]
\begin{verbatim}
CHPF$ ALIGN A WITH B
CHPF$ PROCESSORS p1(5)
CHPF$ DISTRIBUTE B(block,*) ONTO p1
\end{verbatim}
\caption{ HPF Directives}
\label{direct}
\end{figure}

According to the {\it owner computes} rule,  2 blocks of
iterations of \verb+J+ are executed on each processor. The corresponding 
generated code is presented in Figure \ref{hpfcc}.

\begin{figure}[htp]
\begin{verbatim}
C
    DO I=2*PROC_ID, 2*PROC_ID+1
S1       B(I,I)=1
    ENDDO
    DO I=2*PROC_ID, 2*PROC_ID+1
      DO J= I+1,10
S2        B(I,J)=1
S3        A(I,J)=B(I,J-1)
      ENDDO
    ENDDO  
\end{verbatim}
\caption{Code generated from HPF}
\label{hpfcc}
\end{figure}

The "HPFC" presentation details all the characteristics of our HPF compiler.

\subsubsection{Automatic placement}

The problem of solving automatically data and control distributions is
NP-complet. Thus, they are sequentially solved. In the suggested approach,
scheduling is first computed. Then, the data distribution computes the mapping
of each array onto a virtual processor grid so as to minimize 
the communication cost.

First, the array data flow graph is built. It characterizes the 
 precedence relations between instruction instances and contains only true 
dependences because the program is put into a single assignment form. 
 Figure \ref{plac} presents these precedence relations for program 1. 
Instructions \verb+S1+ and \verb+S2+ that assign array elements of B, should 
be executed before instruction \verb+S3+. Some array elements are assigned
by \verb+S1+ and others by \verb+S2+. The resulting code generated from the 
DFG is presented in Figure \ref{plac2}

\begin{figure}[htp]
\psfig{file=figures/plac.idraw}
\caption{Precedence Relations}
\label{plac}
\end{figure}

\begin{figure}[htp]
\begin{verbatim}
CDIR$ SHARED B1(:BLOCK, :BLOCK)
CDIR$ SHARED B2(:BLOCK, :BLOCK)
CDIR$ SHARED A(:BLOCK, :BLOCK)

CDIR$ DOSHARED(I) ON B1(I,I)
    DO I=1,9
       B1(I,I)=1
    ENDDO

CDIR$ DOSHARED(I,J) ON B2(I,J)
    DO I=1,10
      DO J= I+1,10
        B2(I,J)=1
      ENDDO
    ENDDO  
CDIR$ DOSHARED(I) ON A(I,I+1)
    DO I=1,9
       J= I+1
        A(I,J)=B1(I,J-1)
      ENDDO
    ENDDO  

CDIR$ DOSHARED(I,J) ON A(I,J)
    DO I=1,8
       DO J= I+2,10
        A(I,J)=B2(I,J-1)
      ENDDO
    ENDDO  
\end{verbatim}
\caption{CRAFT Code}
\label{plac2}
\end{figure}

This technique is detailed in the "Polyhedric method" presentation.

\subsubsection{Emulated Shared Memory}


Since the management of data  distribution is complex, this approach suggests
 the emulation of a shared memory onto a
distributed memory machine. Control distribution is applied through {\it 
tiling}. The iteration domain is partitioned into tiles that can be executed 
concurrently on different processors. Figure \ref{wp65d} illustrates this 
tiling and the assignment to the processors.  Tiles  are distributed in
 a cyclic way. 

\begin{figure}[htp]
\psfig{file=figures/wp651.idraw}
\caption{Control distribution}
\label{wp65d}
\end{figure}


Data distribution is implicit.  One half of the processors perform
computations and the other half emulate memory banks.
Figure \ref{wp65c} presents the corresponding generated code for the 
computations. The necessary communications are inserted in the generated code 
automatically by the compiler. The "Distributed code generation" presentation 
 details all the compilation phases of this approach.


\begin{figure}[htp]
\begin{verbatim}

    DO I=PROC_ID,10,5
S1       B(I,I)=1
    ENDDO
    DO I=PROC_ID,10,5
      DO J= I+1,10
S2        B(I,J)=1
S3        A(I,J)=B(I,J-1)
      ENDDO
    ENDDO  
\end{verbatim}
\caption{Wp65 code}
\label{wp65c}
\end{figure}

%\end{document}



\subsection{Parallelize the code with a polyhedric method}
\label{sec:parallelize_polyhedric}




\section{Optimizing a small program}
\label{sec:optimizing_a_small_program}


\section{Array Regions for Interprocedural Parallelization and Array
  Privatization}
\label{sec:parallelizing_a_program_with_aray_privatization}

{\em This is mainly the internal report A/279/CRI from B�atrice C{\sc
    reusillet} you can get at
  \htmladdnormallink{http://www.cri.ensmp.fr/doc/A-279.ps.Z}{http://www.cri.ensmp.fr/doc/A-279.ps.Z}.}

%%
%% $Id$
%%
%% Copyright 1989-2009 MINES ParisTech
%%
%% This file is part of PIPS.
%%
%% PIPS is free software: you can redistribute it and/or modify it
%% under the terms of the GNU General Public License as published by
%% the Free Software Foundation, either version 3 of the License, or
%% any later version.
%%
%% PIPS is distributed in the hope that it will be useful, but WITHOUT ANY
%% WARRANTY; without even the implied warranty of MERCHANTABILITY or
%% FITNESS FOR A PARTICULAR PURPOSE.
%%
%% See the GNU General Public License for more details.
%%
%% You should have received a copy of the GNU General Public License
%% along with PIPS.  If not, see <http://www.gnu.org/licenses/>.
%%

%\documentstyle[11pt,a4]{article}

%\newcommand{\mytilde}{}
%\newsavebox{\boite}
%\setbox\boite=\hbox{\verb+~+}

%\title{Array Regions for Interprocedural Parallelization and Array
%Privatization}
%\author{B�atrice {\sc Creusillet}\thanks{{\tt e-mail: <creusil@cri.ensmp.fr>,
%web: <http://www.cri.ensmp.fr/\usebox{\boite}creusil>}}\\ \\
%Centre de Recherche en Informatique \\ {\'E}cole des mines de Paris \\
%Internal Report A/279/CRI}
%\date{}

%%
%% $Id$
%%
%% Copyright 1989-2014 MINES ParisTech
%%
%% This file is part of PIPS.
%%
%% PIPS is free software: you can redistribute it and/or modify it
%% under the terms of the GNU General Public License as published by
%% the Free Software Foundation, either version 3 of the License, or
%% any later version.
%%
%% PIPS is distributed in the hope that it will be useful, but WITHOUT ANY
%% WARRANTY; without even the implied warranty of MERCHANTABILITY or
%% FITNESS FOR A PARTICULAR PURPOSE.
%%
%% See the GNU General Public License for more details.
%%
%% You should have received a copy of the GNU General Public License
%% along with PIPS.  If not, see <http://www.gnu.org/licenses/>.
%%

\newcommand{\references}[1]{{\em #1~references}}

\newcommand{\mybfitem}[1]{\item[{\bf#1}]}
\newcommand{\myttitem}[1]{\item[{\tt#1}]}
\newenvironment{mylist}{\begin{list}%
        {}{}}{\end{list}}

\newcommand{\smallverb}[1]{{\small \verb+#1+}}
\newcommand{\smalltt}[1]{{\small \tt #1}}

\newenvironment{csmalltt} {\begin{center} \small \tt}
                           {\end{center}}
%\newenvironment{regions} {\begin{center} \small \tt \begin{tabular}{l}}
%                           {\end{tabular} \end{center}}
\newenvironment{regions} {\[\begin{array}{l}}
                           {\end{array} \]}


\newenvironment{videitemize}{\renewcommand{\labelitemi}{}
\renewcommand{\labelitemii}{} \renewcommand{\labelitemiii}{}
\renewcommand{\labelitemiv}{}
\begin{itemize}}
{\end{itemize}}

%noms
\newcommand{\pips}[0]{{\sc pips}}
\newcommand{\PIPS}[0]{{\sc pips}}
\newcommand{\fortran}[0]{{\sc fortran}}

% pour les figures encadr{\'e}es en haut et en bas
\newcommand{\drawline}[0]{\leavevmode\hrulefill}
\newenvironment{myfigure}{\begin{figure} 
                         \drawline }
                            { \drawline 
                        \end{figure}}

\newenvironment{myfigurep}{\begin{figure}[p] 
                         \drawline }
                            { \drawline 
                        \end{figure}}

% symboles math{\'e}matiques
\newcommand{\union}[0]{\cup}
\newcommand{\U}[0]{\cup}
\newcommand{\inter}[0]{\cap}
\newcommand{\loand}[0]{\wedge}
\newcommand{\loor}[0]{\vee}
\newcommand{\tq}[0]{\:|\:}
\newcommand{\ilexiste}[0]{\exists\:}
\newcommand{\implique}[0]{\Longrightarrow}
\newcommand{\rond}[0]{\circ}
\newcommand{\meet}[0]{\wedge}
\newcommand{\Meet}[0]{\bigwedge}
\newcommand{\qqsoit}[0]{\forall \:}
\newcommand{\Umust}[0]{\cup_{must}}
\newcommand{\Umay}[0]{\cup_{may}}
\newcommand{\rminus}[0]{\circleddash}

% r{\'e}gions 
%\newcommand{\convexhull}[0]{convexhull}
%\newcommand{\must}[0]{{\tt MUST}}
%\newcommand{\may}[0]{{\tt MAY}}
\newcommand{\Tinverse}[0]{\widetilde{T}_k^{-1}}
\newcommand{\Ttilde}[2]{{\cal T}_{\sigma_{#1} \rightarrow \sigma_{#2}}}
\newcommand{\Ttildepriv}[3]{{\cal T}_{\sigma_{#1} \rightarrow
\sigma_{#2} \backslash {#3} }}
%\newcommand{\phik}[1]{\ifmmode\else$\fi\phi_{#1}\ifmmode\else$\fi}
\newcommand{\rread}[0]{{\tt  READ}}
\newcommand{\rwrite}[0]{{\tt  WRITE}}
\newcommand{\rin}[0]{{\tt   IN}}
\newcommand{\rout}[0]{{\tt OUT}}
\newcommand{\rw}[0]{{\tt READ/WRITE}}


% th{\'e}or\`{e}mes et similaires
\newtheorem{property}{Property}
\newtheorem{algorithm}{Algorithm}
%\newtheorem{definition}{Definition}
\newtheorem{notation}{Notation}
\newtheorem{criterium}{Criterium}
%\newtheorem{corollary}{Corollary}
\newenvironment{proff}{\paragraph{Proof:}}
                        { {\hfill\ \hfill \(\square\)} \paragraph{} }



\newcommand{\matheq}[1]{$#1$}


\newcommand{\tab}{\hspace{5mm}}


%\begin{document}

%\maketitle 

%\begin{abstract}
%  \em Three demonstrations are presented, that highlight the need for
%  interprocedural analyses such as {\em preconditions\/} and exact {\em
%  array regions\/}, in order to parallelize loops that contain subroutine
%  calls or temporary arrays. These analyses are provided by PIPS in an
%  interactive environment.
%\end{abstract}


\subsection{Interprocedural Parallelization}

\verb+AILE+ is an application from the ONERA, the French institute of
aerospatial research. It has more than 3000 lines of FORTRAN code. It has
been slightly modified to test the coherence of some input values.

The aim of this demonstration is to show that interprocedural analyses are
necessary for an automatic parallelization. 


For that purpose, we have chosen the subroutine (or {\em module\/})
\verb+EXTR+, which is called by the module \verb+GEOM+, itself called by
the main routine \verb+AILE+. An excerpt is given in Figure~\ref{fig:AILE}
(without the intermediate call to \verb+GEOM+).


\begin{figure}[htbp] 
  \begin{center} \footnotesize
    \leavevmode
\begin{minipage}{12cm}
\begin{verbatim}
      PROGRAM AILE
      DIMENSION T(52,21,60)
      COMMON/CT/T
      COMMON/CI/I1,I2,IMAX,I1P1,I1P2,I2M1,I2M2,IBF
      COMMON/CJ/J1,J2,JMAX,J1P1,J1P2,J2M1,J2M2,JA,JB,JAM1,JBP1
      COMMON/CK/K1,K2,KMAX,K1P1,K1P2,K2M1,K2M2
      COMMON/CNI/L      
      ...
      READ(NXYZ) I1,I2,J1,JA,K1,K2
C     
      IF(J1.GE.1.AND.K1.GE.1) THEN
         N4=4
         J1=J1+1
         J2=2*JA+1
         JA=JA+1
         K1=K1+1
         ...
         CALL EXTR(NI,NC)
      ENDIF
      END

      SUBROUTINE EXTR(NI,NC)
      DIMENSION T(52,21,60)
      COMMON/CT/T
      COMMON/CI/I1,I2,IMAX,I1P1,I1P2,I2M1,I2M2,IBF
      COMMON/CJ/J1,J2,JMAX,J1P1,J1P2,J2M1,J2M2,JA,JB,JAM1,JBP1
      COMMON/CK/K1,K2,KMAX,K1P1,K1P2,K2M1,K2M2
      COMMON/CNI/L
      L=NI
      K=K1
      DO 300 J=J1,JA
         S1=D(J,K  ,J,K+1)
         S2=D(J,K+1,J,K+2)+S1
         S3=D(J,K+2,J,K+3)+S2
         T(J,1,NC+3)=S2*S3/((S1-S2)*(S1-S3))
         T(J,1,NC+4)=S3*S1/((S2-S3)*(S2-S1))
         T(J,1,NC+5)=S1*S2/((S3-S1)*(S3-S2))
         JH=J1+J2-J
         T(JH,1,NC+3)=T(J,1,NC+3)
         T(JH,1,NC+4)=T(J,1,NC+4)
         T(JH,1,NC+5)=T(J,1,NC+5)
 300  CONTINUE      
      END

      REAL FUNCTION D(J,K,JP,KP)
      DIMENSION T(52,21,60)
      COMMON/CT/T
      COMMON/CNI/L
C     
      D=SQRT((T(J,K,L  )-T(JP,KP,L  ))**2
     1     +(T(J,K,L+1)-T(JP,KP,L+1))**2
     2     +(T(J,K,L+2)-T(JP,KP,L+2))**2)
      END

\end{verbatim}
  \end{minipage}
  \end{center}
  \caption{Excerpt from program {\tt AILE}.}
  \label{fig:AILE}
\end{figure}

\subsubsection{EXTR}

\verb+EXTR+ contains a \verb+DO+ loop that has several characteristics:
\begin{enumerate}
\item There are several read and write references to elements of the array
  \verb+T+. This induces dependences that cannot be disproved if we don't
  know the relations between index expressions, and more precisely between
  \verb+J+ and \verb+JH+. We already know that \verb|JH=J1+J2-J|, but we
  don't know the values of \verb+J1+, \verb+J2+ and \verb+JA+, which are
  global variables initialized in \verb+AILE+. Thus, we can disprove the
  loop-carried dependences between \verb|T(J,1,NC+3)| and
  \verb|T(JH,1,NC+3)| for instance, only if we interprocedurally propagate
  the values of \verb+J1+, \verb+J2+ and \verb+JA+ from \verb+AILE+. This
  type of information is called {\em precondition\/} in PIPS~\cite{Irig:91,cnrs-nsf92:pips}.

\item There are three calls to the function \verb+D+ in \verb+EXTR+.
  \verb+D+ contains several read references to the global array \verb+T+.
  So, we must assume that the whole array is potentially read by each call
  to \verb+D+. This induces dependences in \verb+EXTR+ between the calls to
  \verb+D+ and the other statements. In order to disprove these dependences,
  we need a way to represent the set of array elements read by any
  invocation of \verb+D+, and be able to use this information at each call
  site. These sets are called {\em array regions\/} in PIPS~\cite{Creu:95a}.

\item \verb+S1+, \verb+S2+, \verb+S3+ and \verb+JH+ are defined and used at
  each iteration. This induces loop-carried dependences. But we may notice
  that each use is preceded by a definition in the same iteration. These
  variables can be privatized (this means that a local copy is assigned to
  each iteration) to remove the spurious dependences.
\end{enumerate}


\subsubsection{D}

As written before, there are several references to elements of the array
\verb+T+ in \verb+D+. Our aim is to represent this set of elements, such
that it can be used at each call site to help disproving dependences. 

If we know nothing about the relations between the values of \verb+K+ and
\verb+KP+ or between \verb+J+ and \verb+JP+, all we can deduce is that the
third index of all the array elements ranges between \verb+L+ and
\verb|L+2|. This is represented by the region:
\begin{regions} 
<T(\phik{1},\phik{2},\phik{3})-R-MAY-\{L<=\phik{3}<=L+2\}>
\end{regions}
The $\phi$ variables represent the dimensions of the array; \verb+R+ means
that we consider the {\em read\/} effects on the variable; and \verb+MAY+
means that the region is an over-approximation of the set of elements that
are actually read.

The relations between the values of \verb+K+ and \verb+KP+ or \verb+J+ and
\verb+JP+ are those that exist between the real arguments. At each call
site, we have {\tt JP==J} and {\tt KP==K+1}. These contidions hold true
before each execution of {\tt D}; we call them {\em preconditions\/}. Under
these conditions, we can now recompute the region associated to the array
{\tt T}:
\begin{regions} 
<T(\phik{1},\phik{2},\phik{3})-R-MUST-\{\phik{1}==J, K<=\phik{2}<=K+1, L<=\phik{3}<=L+2\}>
\end{regions}
Notice that this is a {\tt MUST} region, because it exactly represents the
set of array elements read by any invocation of function {\tt D}. 

\subsubsection{Parallelisation of EXTR}

We can now parallelize {\tt EXTR} by:
\begin{enumerate}
\item privatizing the scalar variables;
\item using {\em array regions\/} to summarize the read effects on the array
  {\tt T} by each invocation of {\tt D};
\item using the {\em preconditions\/} induced by the initialization of
  global scalar variables (in {\tt AILE}) to disprove the remaining dependences.
\end{enumerate}
This leads to the parallelized version of Figure~\ref{fig:EXTR_para}.

\begin{figure}[htbp]
  \begin{center} \footnotesize
    \leavevmode
    \begin{minipage}{12cm}
\begin{verbatim}
      SUBROUTINE EXTR(NI,NC)
      DIMENSION T(52,21,60)
      COMMON/CT/T
      COMMON/CI/I1,I2,IMAX,I1P1,I1P2,I2M1,I2M2,IBF
      COMMON/CJ/J1,J2,JMAX,J1P1,J1P2,J2M1,J2M2,JA,JB,JAM1,JBP1
      COMMON/CK/K1,K2,KMAX,K1P1,K1P2,K2M1,K2M2
      COMMON/CNI/L
      L = NI                                                        
      K = K1                                                        
      DOALL J = J1, JA
         PRIVATE S1,S2,S3
         S1 = D(J, K, J, K+1)                                       
         S2 = D(J, K+1, J, K+2)+S1                                  
         S3 = D(J, K+2, J, K+3)+S2                                  
         T(J,1,NC+3) = S2*S3/((S1-S2)*(S1-S3))                      
         T(J,1,NC+4) = S3*S1/((S2-S3)*(S2-S1))                      
         T(J,1,NC+5) = S1*S2/((S3-S1)*(S3-S2))                      
      ENDDO
      DOALL J = J1, JA
         PRIVATE JH
         JH = J1+J2-J                                               
         T(JH,1,NC+3) = T(J,1,NC+3)                                 
         T(JH,1,NC+4) = T(J,1,NC+4)                                 
         T(JH,1,NC+5) = T(J,1,NC+5)                                 
      ENDDO
      END
\end{verbatim}
    \end{minipage}
  \end{center}
  \caption{Parallelized version of {\tt EXTR}.}
  \label{fig:EXTR_para}
\end{figure}


\subsection{Array Privatization}

Array privatization is not yet implemented in PIPS, but the information
needed to perform the transformation is already available: {\em IN and OUT
regions}~\cite{Creu:95a,Creu:95b}.

To illustrate the characteritics of these regions, we will consider two
examples: {\tt NORM} is another excerpt from {\tt AILE}, and {\tt RENPAR6}
is a contrived example that highlights some details of the computation of
regions and the possibilities opened up by IN and OUT regions.


\subsubsection{NORM}


\begin{figure}[htbp]
  \begin{center} \footnotesize
    \leavevmode
    \begin{minipage}{12cm}
\begin{verbatim}
      PROGRAM AILE
      DIMENSION T(52,21,60)
      COMMON/CT/T
      COMMON/CI/I1,I2,IMAX,I1P1,I1P2,I2M1,I2M2,IBF
      COMMON/CJ/J1,J2,JMAX,J1P1,J1P2,J2M1,J2M2,JA,JB,JAM1,JBP1
      COMMON/CK/K1,K2,KMAX,K1P1,K1P2,K2M1,K2M2
      COMMON/CNI/L
      DATA N1,N3,N4,N7,N10,N14,N17/1,3,4,7,10,14,17/

      READ(NXYZ) I1,I2,J1,JA,K1,K2
C     
      IF(J1.GE.1.AND.K1.GE.1) THEN
         N4=4
         J1=J1+1
         J2=2*JA+1
         JA=JA+1
         K1=K1+1
         CALL NORM(N10,N7,N4,N14,N17,I2)
      ENDIF
      END

      SUBROUTINE NORM(LI,NI,MI,NN,NC ,I) 
      DIMENSION T(52,21,60)
      DIMENSION TI(3)

      COMMON/T/T
      COMMON/I/I1,I2,IMAX,I1P1,I1P2,I2M1,I2M2,IBF
      COMMON/J/J1,J2,JMAX,J1P1,J1P2,J2M1,J2M2,JA,JB,JAM1,JBP1
      COMMON/K/K1,K2,KMAX,K1P1,K1P2,K2M1,K2M2
      COMMON/IO/LEC ,IMP,KIMP,NXYZ,NGEO,NDIST

C ....
DO 300 K=K1,K2
      DO 300 J=J1,JA

      CALL PVNMUT(TI)
      T(J,K,NN  )=S*TI(1)
      T(J,K,NN+1)=S*TI(2)
      T(J,K,NN+2)=S*TI(3)
  300 CONTINUE
C ....
      END

      SUBROUTINE PVNMUT(C)
      DIMENSION C(3), CX(3)
      CX(1)= 1
      CX(2)= 2
      CX(3)= 3
      R=SQRT(CX(1)*CX(1)+CX(2)*CX(2)+CX(3)*CX(3))
      IF(R.LT.1.E-12) R=1.
      DO I = 1,3
      C(I) = CX(I)/R
      ENDDO
      RETURN
      END
\end{verbatim}
    \end{minipage}
  \end{center}
  \caption{Another excerpt from {\tt AILE}: {\tt NORM}}
  \label{fig:NORM}
\end{figure}

This is a very simple example (see Figure~\ref{fig:NORM}) that shows the
necessity of array privatization, and the need for IN and OUT array regions.

In the loop of subroutine {\tt NORM}, the references to the array {\tt T} do
not induce loop-carried dependences.  Furthermore, there are only read-read
dependences on {\tt S}.  However, notice that the array {\tt TI} is a real
argument in the call to {\tt PVNMUT}, and that there are 3 read references
to array {\tt TI}.  This induces potential interprocedural dependences. We
have seen with the previous example that these dependences can sometimes be
disproved with array regions.

We must first compute the regions of array {\tt TI} that are referenced in
{\tt {\tt PVNMUT}}. In {\tt PVNMUT}, {\tt TI} is called {\tt C}. And the 3
elements of {\tt C} are written, but not read. This leads to:
\begin{regions}
<C(\phik{1})-W-MUST-\{1<=\phik{1}<=3\}>
\end{regions}
({\tt W} means that this  is a {\em write\/} effect)

At the call site, {\tt C} is translated into {\tt TI}, which gives the
region:
\begin{regions}
<TI(\phik{1})-W-MUST-\{1<=\phik{1}<=3\}>
\end{regions}
And finally, the regions corresponding to the whole body of the loop nest
are:
\begin{regions}
<TI(\phik{1})-W-MUST-\{1<=\phik{1}<=3\}> \\
<TI(\phik{1})-R-MUST-\{1<=\phik{1}<=3\}>
\end{regions}

These regions are identical, which means that each iteration of loops {\tt K} and
{\tt J} reads and writes to the same memory locations of array {\tt TI}. Thus, there are
loop-carried dependences, and the loop cannot be parallelized.

However, these dependences are false dependences, because if we allocate a
copy of array {\tt TI} to each iteration (in fact to each processor), there are no
more dependences. This is what is called array privatization. In order to
privatize an array, we must be sure that, in each iteration, no element is
read before being written in the same iteration. Thus, there are no
loop-carried producer-consumer dependences.  

This last property cannot be verified by using READ regions, because they
contain all the elements that are read, and not only those that are read
before being written. This is represented in PIPS by IN regions. In our case,
we must verify that no element of {\tt TI} belongs to the IN region corresponding
to the loop body, which is the case. 

We must also be sure that no element of {\tt TI} that is initialized by a single
iteration is used in the subsequent iterations or after the loops.
This information is provided in PIPS by the OUT regions. They represent
the set of live array elements, that is to say those that are used in
the continuation.

We can now parallelize {\tt NORM} by:
\begin{enumerate}
\item using {\em array regions\/} to perform the dependence analysis;
\item using {\em IN\/} and {\em OUT array regions\/} to privatize the array
  {\tt TI}.
\end{enumerate}


This leads to the parallelized version of Figure~\ref{fig:NORM_para}.

\begin{figure}[htbp]
  \begin{center} \footnotesize
    \leavevmode
    \begin{minipage}{12cm}
\begin{verbatim}
      SUBROUTINE NORM(LI,NI,MI,NN,NC ,I) 
      DIMENSION T(52,21,60)
      DIMENSION TI(3)

      COMMON/CT/T
      COMMON/I/I1,I2,IMAX,I1P1,I1P2,I2M1,I2M2,IBF
      COMMON/J/J1,J2,JMAX,J1P1,J1P2,J2M1,J2M2,JA,JB,JAM1,JBP1
      COMMON/K/K1,K2,KMAX,K1P1,K1P2,K2M1,K2M2
      COMMON/IO/LEC ,IMP,KIMP,NXYZ,NGEO,NDIST

C     ....
      DOALL K = K1, K2
         PRIVATE J
         DOALL J = J1, JA
            PRIVATE TI
            CALL PVNMUT(TI)                                             
            T(J,K,NN) = S*TI(1)                                         
            T(J,K,NN+1) = S*TI(2)                                       
            T(J,K,NN+2) = S*TI(3)                                       
         ENDDO
      ENDDO
C     ....
      END
\end{verbatim}
    \end{minipage}
  \end{center}
  \caption{Parallelized version of {\tt NORM}.}
  \label{fig:NORM_para}
\end{figure}

\subsubsection{RENPAR6}


\begin{figure}[htbp]
  \begin{center} \footnotesize
    \leavevmode
    \begin{minipage}{12cm}
\begin{verbatim}
      SUBROUTINE RENPAR6(A,N,K,M)
      INTEGER N,K,M,A(N)
      DIMENSION WORK(100,100)
      K = M * M
      DO I = 1,N
         DO J = 1,N
            WORK(J,K) = J + K
         ENDDO

         CALL INC1(K)

         DO J = 1,N
            WORK(J,K) = J * J - K * K
            A(I) = A(I) + WORK(J,K) + WORK(J,K-1)
         ENDDO
      ENDDO
      END
      
      SUBROUTINE INC1(I)
      I = I + 1
      END
\end{verbatim}
    \end{minipage}
  \end{center}
  \caption{Contrived example: {\tt RENPAR6}}
  \label{fig:RENPAR6}
\end{figure}

{\tt RENPAR6} is a contrived example (see Figure~\ref{fig:RENPAR6}) designed
to show on a very simple program the power of READ, WRITE, IN and OUT
regions, and some particular details of their computations, especially when
integer scalar variables that appear in array indices are modified.

The main purpose is to see that array {\tt WORK} is only a temporary and can
be privatized.  Notice that the value of {\tt K} is unknown on entry to the
loop {\tt I}, and that its value is modified by a call to {\tt INC1} at each
iteration ({\tt INC1} simply increments its value by 1).


We are interested in the sets of array elements that are referenced in each
iteration. However, since the value of {\tt K} is not the same in the two written
references, we cannot summarize the write accesses if we do not know the
relation that exists between the two values of {\tt K}. This is achieved in PIPS
by using transformers, that here show how the new value of {\tt K} is related to the
value before the {\tt CALL} ({\tt K\#init}):
\begin{regions}
T(K) \{K==K\#init+1\}
\end{regions}
And the transformer of the loop shows how the value of {\tt K} at each step
is related to the values of {\tt I} and {\tt K\#init} (value of {\tt K}
before the loop):
\begin{regions}
T(I,K) \{K==I+K\#init-1\}
\end{regions}

This previous information is used to summarize the sets of elements that are
read or written by each program structure. In order to compute the summary
for the loop {\tt I}, we must merge the sets for the two {\tt J} loops. Be
careful that the value of {\tt K} is not the same for these two loops. We
must use the transformer of the {\tt CALL} to translate the value of {\tt K}
in the second region into the value of {\tt K} before the CALL. At this
step, we have a summary of what is done by a single iteration. We then
compute the regions for the whole loop {\tt I}. This is done with the help
of the transformer of the loop that gives the relation between {\tt K} and
{\tt I}.

However, as we have seen with {\tt NORM}, READ and WRITE regions are not
sufficient for array privatization, because we must verify that every
element of {\tt WORK} that is read by an iteration is previously written in
the same iteration. This is achieved by the IN region.  Then OUT regions
allow us to verify that no element of {\tt WORK} is used in the subsequent
iterations or in the continuation of the loop. 




We can now try to parallelize {\tt RENPAR6} by:
\begin{enumerate}
\item using {\em transfomers\/} to compute {\em array regions\/};
\item using {\em array regions\/} to perform the dependence analysis;
\item using {\em IN\/} and {\em OUT array regions\/} to privatize the array
  {\tt WORK}.
\end{enumerate}


This leads to the parallelized version of Figure~\ref{fig:RENPAR6_para}. The
array {\tt WORK} is privatized in loop {\tt I}. However, the loop is not
parallelized, because automatic induction variable substitution is not
available in PIPS. This transformation has been performed by hand. This
leads to the subroutine {\tt RENPAR6\_2} in figure~\ref{fig:RENPAR6_2}.
And after array privatization, PIPS is able to parallelize the loop {\tt I}
(see Figure~\ref{fig:RENPAR6_2_para}).

\begin{figure}[htbp]
  \begin{center} \footnotesize
    \leavevmode
    \begin{minipage}{12cm}
\begin{verbatim}
      SUBROUTINE RENPAR6(A,N,K,M)
      INTEGER N,K,M,A(N)
      DIMENSION WORK(100,100)
      K = M*M                                                      
      DO I = 1, N
         PRIVATE WORK,I
         DOALL J = 1, N
            PRIVATE J
            WORK(J,K) = J+K                                         
         ENDDO
         CALL INC1(K)                                              
         DOALL J = 1, N
            PRIVATE J
            WORK(J,K) = J*J-K*K                                     
         ENDDO
         DO J = 1, N
            PRIVATE J
            A(I) = A(I)+WORK(J,K)+WORK(J,K-1)                        
         ENDDO
      ENDDO
      END
\end{verbatim}
    \end{minipage}
  \end{center}
  \caption{Parallelized version of {\tt RENPAR6}.}
  \label{fig:RENPAR6_para}
\end{figure}


\begin{figure}[htbp]
  \begin{center} \footnotesize
    \leavevmode
    \begin{minipage}{12cm}
\begin{verbatim}
      SUBROUTINE RENPAR6_2(A,N,K,M)
      INTEGER N,K,M,A(N)
      DIMENSION WORK(100,100)
      K0 = M * M
      DO I = 1,N
         K = K0+I-1
         DO J = 1,N
            WORK(J,K) = J + K
         ENDDO

         CALL INC1(K)

         DO J = 1,N
            WORK(J,K) = J * J - K * K
            A(I) = A(I) + WORK(J,K) + WORK(J,K-1)
         ENDDO
      ENDDO
      END
\end{verbatim}
    \end{minipage}
  \end{center}
  \caption{{\tt RENPAR6\_2}.}
  \label{fig:RENPAR6_2}
\end{figure}

\begin{figure}[htbp]
  \begin{center} \footnotesize
    \leavevmode
    \begin{minipage}{12cm}
\begin{verbatim}
      SUBROUTINE RENPAR6_2(A,N,K,M)
      INTEGER N,K,M,A(N)
      DIMENSION WORK(100,100)
      K0 = M*M                                                 
      DOALL I = 1, N
         PRIVATE WORK,J,K,I
         K = K0+I-1                                                
         DOALL J = 1, N
            PRIVATE J
            WORK(J,K) = J+K                                        
         ENDDO
         CALL INC1(K)                                               
         DOALL J = 1, N
            PRIVATE J
            WORK(J,K) = J*J-K*K                                       
         ENDDO
         DO J = 1, N
            PRIVATE J
            A(I) = A(I)+WORK(J,K)+WORK(J,K-1)                         
         ENDDO
      ENDDO
      END
\end{verbatim}
    \end{minipage}
  \end{center}
  \caption{Parallelized version of {\tt RENPAR6\_2}.}
  \label{fig:RENPAR6_2_para}
\end{figure}






In fact, IN and OUT regions could also be used to reduce the set of elements
of array {\tt WORK} to allocate to each processor, because each iteration
only accesses  a sub-array. These regions provide an exact representation of the
set of elements that are actually needed.

%{\small
%\bibliographystyle{plain}
%\bibliography{mybib}

%}\end{document}









\chapter{User manual}


\section{PIPS input language}
\label{sec:input_language}

To be defined... :-(

F77, implicit none, include, .F (cpp) \& .f

What is not in the \PIPS input language...


\section{The PIPS Unix commands}
\label{sec:unix_commands}


\subsection{WPips}
\label{sec:unix_wpips}

The {\tt wpips} command is used to run the OpenLook/X11 interface for
\PIPS. {\tt wpips} does not need any OpenWindows specific feature. You
need to have a correctly initialized {\tt DISPLAY} variable with
enough access rights to the X11 display.

The default directory is the one where {\tt wpips} is launched.



\subsection{EPips}
\label{sec:unix_epips}

For more GNU Emacs familiar users, there is an extension to {\tt
  wpips} that use some Emacs windows to display various \PIPS{}
informations. You can bennefit various Emacs advanced features, like
couloured prettyprinted Fortran code, hypertextual interaction on
the code, etc.

If you have already an Emacs running, {\tt M-X epips} launches a
special {\tt wpips} instance from Emacs. You need to load some E-Lisp
stuff before, for example by modifying your {\tt .emacs} file
according to the \PIPS{} {\tt README}. The default directory is the
one of the Emacs buffer where {\tt wpips} is launched.

You can also launch a separate Emacs that deals with 

\section{Some basics about the OpenLook interface}
\label{sec:openlook}

The most useful button is the {\bf right mouse button} since it is
used to select everything you want in the menus or the panels.

The left button is used to pick a default selection as a short cut if
you want exactly what you want. By using the control+right mouse
button you can change the default selection of a menu as you want.

Some menus have a ``push pin''. If you click on it, the menu is
changed in a panel window you can place as you want. It is useful when
you often use a menu.

Some menu items or display items may be shaded. That means that they
cannot be selected by the user according to the current situation.


\section{Basic Pips containers: workspaces \& modules}
\label{sec:workspace_module}

In order to analyse Fortran programs, \PIPS{} create a workspace where
it puts all the information about the source, the transformated code,
some compiled code, some executables for \PVM{} output, etc. Thus the
first thing to begin with \PIPS{} is to ask for a workspace creation
in the current directory.

Each source code is splitted in modules that are the functions and
procedures of the program. Most of the \PIPS{} transformations can
deal with a module but some other ones, like interprocedural analyses,
deal with all the modules at once, that is with all the modules of the
workspace.

Workspaces and modules are thus the basic containers used in \PIPS{}.

\section{WPips: the main panel}
\label{sec:main_panel}

\psfig{file=images/main_panel.eps,width=\hsize}

The main panel contains most of the menus usable in \PIPS{} and is the
window that appears first when \PIPS{} begins. It also give various
informations on the \PIPS{} current state.


\subsection{Message}
\label{sec:main_message}

The line give you the last information message given by \PIPS, such as
a warning, a log or an error message. Usually, this line is the same
as the last line of the log window (see~\ref{sec:log_window}) but is
useful since this may be closed or hidden.


\subsection{Directory}
\label{sec:main_directory}

This line display the current directory and can be user-edited to
change directory (but only when there is no workspace currently
open). By using the small directory menu, one can change the directory
by browsing the tree structure.



\subsection{Workspace}
\label{sec:main_workspace}

This line of the main panel display the current open workspace. It can
be edited to open an old workspace or create a new one if a workspace
with this name does not already exist in the current directory.  If a
workspace is already open, it is first closed befor opening or
creating a new one.

There is also a small workspace menu that allows to open a workspace
from the current directory, to close the current one or to delete a
workspace. If one try to delete the current workspace, it is closed
first. 

Creating a workspace ask for a workspace name. If a workspace with
this name already exists, \PIPS{} ask for its deletion.
Then, \PIPS{} pop a window with the list of all the Fortran programs
of the current directory. One can give a list of file names separated
with space(s) or more easily by selecting the files with the mouse in
the scrolling list.



\subsection{Module}
\label{sec:main_module}

After selecting a workspace, the module line and menu should become active.

The module line display the current module selected that is the main
module (the module from a {\tt PROGRAM} statement) if any.

The user can type a module name to select one or use the small menu to
select one quickly (if there is not too many modules to fit in the
screen).



\subsection{Memory}
\label{sec:main_memory}

This line display an idea of the {\tt break} limit of the \PIPS{}
process, that is the memory used in megabytes by the code and the data
(but not by the stack). This information cannot be modified by the
user\footnote{It is not possible to give some memory back...}.


\subsection{CPU time}
\label{sec:main_CPU_time}

Another useful information for experimentation is the \CPU{} user
time given by this line.


\subsection{Number of display windows}
\label{sec:main_number_display_windows}

This item shows the number of active windows at this time. The default
number is 2, that means that when some code is displayed it uses,
cyclicly the 2 windows availables (see also \ref{sec:display_windows} to
retains some windows).

The number of active windows can be modified by editing the line or
easier by clicking up or down the small arrows.



\subsection{PIPS icon button}
\label{sec:pips_icon_button}

This button is used to interrupt the current \PIPS{} work. It is taken
into account by \PIPSMAKE{} at a phase boundary. That means that if
you are doing a compute intensive phase, you will wait until the end
of this one, since it is the only way to have clean and easy data
coherence.

%\section{Select menu}
%\label{sec:select_menu}

%\psfig{file=images/select_menu.eps,width=\hsize}



\section{Log window}
\label{sec:log_window}

Since \PIPS is an interactive environment, some information about what
is hapening or what failed, etc. is important. A special window is
allocated to this purpose.

Since dealing with big programs can lead to huge log information, think
to empty the log windows from time to time.


\subsection{WPips log window}
\label{sec:wpips_log_window}

In the \WPIPS mode, an XView {\tt textedit}-like window is used.
It can be opened, closed or emptied from the main panel.


\subsection{EPips log window}
\label{sec:Epips_log_window}

In the Emacs mode, the log window is naturally an Emacs buffer. This
able to display different messages with various text properties (such
as colors, shape, etc.) according to the importance of each messages:
a user warning, a user error, a pips error, etc.

There is also \PIPS-specific menus added to the buffer.


\section{View windows}
\label{sec:display_windows}

According to the mode, that is \EPIPS or \WPIPS, different kind of
windows are used to give information to the user, such as display the
code, display the data dependences, edit the code, etc.

A window to be diplayed is chosen among a pool of $n$ windows, $n$
given by the number of display windows in the main panel
(Section~\ref{sec:main_number_display_windows}). To do some advance
usage, it is often useful to increase the number of available windows
at the same time from the default value of 2 to a greater value.

Since it is useful to keep some information in a window, such as for
example the original code when the user is applying various
transformations on the code, windows can be frozen and retained unused
when they have a retain attribute. According to the mode, different
methods are used to change the retain state.

\subsection{Wpips display windows}
\label{sec:wpips_display_windows}


\subsubsection{Default XView windows}

By default a XView textedit window is open to display user
information. 

Functionalities of this kind of windows can be found in the manual
page with {\tt man textedit}. The most interesting thing to know about
is that you can display a menu with the right button to do operations
on files, etc.

At the bottom of the window there is a mark box to retain a window.
Its retain state can be changer by clicking on this button.

\subsubsection{Alternate view/edit windows}

Some users found useful to be able to chose their editors. For this
feature, an environment variable is used: \verb|PIPS_WPIPS_EDITOR|.
If this variable is set, its value is the name of the command to
execute to display or edit a file. The name of the file to display is
concatenated to that value before being executed.
For example, in a \verb|csh|-like shell,
\begin{verbatim}
setenv PIPS_WPIPS_EDITOR 'xterm -e vi'
\end{verbatim}
will use a \verb|vi| editor in an \verb|xterm|.

Note that with this feature, the control of these windows are under
the control of their users, that means that the retain mode is
meaningless and there is no pool of windows. A new editor windows is
used to display each new data and the user is responsible to remove no
longer used windows.


\subsection{Epips display windows}
\label{sec:epips_display_windows}

In the Emacs mode, an Emacs buffer is of course used to inherit of the
editing power, language-orented editing, hypertextual interactivity,
colored highlighting, etc.

Each \EPIPS window has some \PIPS specific menus and a special keymap.

As for the XView windows of \WPIPS, the number of windows in the pool
is choosen according to the number of display windows in the main panel
(Section~\ref{sec:main_number_display_windows}).

Note that the visited file name of the buffer is set to the displayed
file but since a file cannot be directly visited in different buffer
in Emacs, when \EPIPS is asked to display twice the same version of
the same resource, a confirmation is asked to the user and the file
name is not set in the last buffer.

\section{View menu}
\label{sec:epips_view_menu}

It is used to display some code or some informations from \PIPS. To
display these, \PIPS{} may execute lots of analyses or code
transformations according to \PIPSMAKE.

\begin{center}
  \mbox{\psfig{file=images/view_menu.eps,width=0.3\hsize}  }
\end{center}

\begin{description}
\item[Lasts/No selection:] in \WPIPS{} only, to open all the last
  display windows;

\item[\PSequentialView:] it is the basic view of a module : the code is
  displayed as \PIPS{} understands it. It may be decorated with some
  internal informations by using options ``\PSequentialView''
  (section~\ref{sec:option_sequential_view});

\item[\PUserView:] since the Sequential View is a prettyprinted version
  of the code, some program details may have been modified an
  ``\PUserView'' give the code before parsing by \PIPS{}. It is closer
  to the original code. Of course, after transforming the code by the
  user, this code may be meaningles... As for the Sequential View, the
  code can be decorated with some internal informations by using
  options ``\PUserView'' (section~\ref{sec:option_user_view});

\item[\PSequentialViewControlGraph:] this button is used to
  display the \Pcontrolgraph{} of the \Pcurrentmodule{} with the graph
  editor \PdaVinci{}. Each node contains some code statements that can
  be decorated as for the \PSequentialView{} by using options
  ``\PSequentialViewControlGraph''
  (section~\ref{sec:option_sequential_view_control_graph}).
  
  Displaying the control flow graph is interesting to precisely
  analyze some codes in order to figure out what structural
  optimizations to apply. This kind of graph view is also used to
  display the interprocedural control flow graph (see
  section~\ref{sec:ICFG_view}) and the more classical call graph (see
  section~\ref{sec:call_graph_view}) the linking the various
  procedures and functions of a program.
  
  In \PIPS{}, the control graph is represented with a hierachical
  control flow graph, as explained about the \Punstructured{} in the
  \Pri{}. The controf flow graph is hierarchical since a statement can
  contain an unstructured graph to represent another control graph
  with no edge outside of the statement.

  
  The control flow is represented in a directed graph of
  \Punstructured{} nodes. For example, a \PGOTO{} leads to an edge
  from the source to the destination, an \PIF{} with some \PGOTO{}s
  leads to one edge to the \PTHEN{} branch and another one to the
  \PELSE{} branch.
  
  As a consequence a node without predecessor is unreachable and can
  be discarded (see section~\ref{sec:unspaghettify}).
  
  In the \PdaVinci{} output, the following style hints are used:
  \begin{itemize}
  \item the first statement block of the programm is yellow;
  \item the entry node of an \Punstructured{} is a light green ellipse;
  \item the exit node of an \Punstructured{} is a light grey ellipse;
  \item an unstructured \PIF{} (that is an \PIF{} with some \PGOTO{}s)
    is a cyan rhombus (if it is the entry of the \Punstructured{} it is
    a light green one);
  \item a \PTHEN{} branch is blue;
  \item a \PELSE{} branch is red.
  \end{itemize}

\item[\PDependenceGraphView:] displays the dependence graph view. \ref{NeedFI};
  
\item[\PArrayDataFlowGraphView:] displays the array dataflow graph of
  the code, that is the information used to track array data flowing
  in the program as used in the method also known as {\em
    Feautrier\/}'s one. Just an example to explain a little bit the
  output:
\begin{verbatim}
INS_100:
********
 Execution Domain for 100:
{
  J - 10 <= 0 ,
- J + I + 1 <= 0 ,
  I - 10 <= 0 ,
- I + 1 <= 0 ,
} 

 ---Def-Use---> ins_130:
  Reference: X
  Transformation: [I,J]
  Governing predicate:
{
  K - 1 <= 0 ,
} 
  Execution Domain for 130:
{
- I + K + 1 <= 0 ,
- K + 1 <= 0 ,
  J - 10 <= 0 ,
- J + I + 1 <= 0 ,
  I - 10 <= 0 ,
- I + 1 <= 0 ,
}
\end{verbatim}
  It first describes the data generated by the instruction 0 of line
  10, that is {\tt INS100} with its execution domain. There is a
  use-def dependence with instruction 0 of line 13, that is {\tt
    INS130}, about array {\tt X} only if the governing predicate is
  true, that is $K - 1 \leq 0$ here. Then, we have the execution
  domain on {\tt X(I,J)} that used the data previously defined in {\tt
    INS100}. {\tt Transformation: [I,J]} means that {\tt X(I,J)}
  is defined in the loop-nest {\tt INS100(i,j)} with $i=I,j=J$;

\item[\PSchedulingView:] give a tiem base for each instruction. For example 
\begin{verbatim}
ins_140:
     pred: TRUE
     dims: -1+3*I
\end{verbatim}
  means that if the predicate is true (here it is of course always
  true...), these instruction ins executed at time $-1+3\times I$;

\item[\PPlacementView:] gives where an instruction is executed. For example
\begin{verbatim}
Ins_140 :I , J
\end{verbatim}
  means that these instruction is executed on processor $(I,J)$;
  
\item[\PCallgraphView:] display a tree of all the functions or
  procedures called in the current module. The code can also be
  decorated according to the options ``\PCallgraphView''
  (section~\ref{sec:option_call_graph});
  
\item[\PICFGView:] display a more precise tree than the
  \PCallgraphView. It is the Interprocedural COntrol Flow Graph, where
  each call, {\tt do}-loops and \PIF{} added according to the options
  ``\PICFGView'' (section~\ref{sec:option_ICFG_view});
  
\item[\PDistributedView:] ask \PIPS{} for parallelizing the code with
  the WP65/PUMA method, {\bf with all the prerequisite of this method
    on the input code}. The output is in fact not 1 code but 2 ones:
  \begin{itemize}
  \item the computational code;
  \item the memory bank code that does the parallel memory feature;
  \end{itemize}

\item[\PParallelView:] ask \PIPS{} for parallelizing the code and
  displaying it according the parallel dialect given in the options 
  ``\PParallelView'' (section~\ref{sec:option_parallel_view});

\item[\PFlintView:] launch a Fortran {\tt lint} on the module and give
  the information back to the user;

\item[Close:] in \WPIPS{} only, to close all display windows.
\end{description}


\section{Transform/Edit}
\label{sec:transform/edit_menu}

The Transform/Edit menu is used to apply various transformations on
the current module in \PIPS. Furthermore, the user can edit the code
of the module as a special transformation.

\begin{center}
  \mbox{\psfig{file=images/transform_edit_menu.eps,width=0.6\hsize}}
\end{center}

\input{wpips-epips-transform-menu.tex}

  
\begin{description}
\item[\PEdit:] a special transformation : the user one! Load the
  original code of the module. Do not forget to save your modification
  after you have finished (the menu {\em File/Save Current File\/} in
  a \WPIPS{} Edit window or the menu {\em Save the file after edit in
    the seminal .f} in \EPIPS).
\end{description}

%% Get a more precise comment about the transformations:
\input{wpips-epips-transform-end.tex}


\section{Compile}
\label{sec:compile_menu}

\begin{center}
  \mbox{\psfig{file=images/compile_menu.eps,width=0.3\hsize}}
\end{center}

\begin{description}
\item[\PCompileAnHPFProgram:] compile all the modules with the \HPF{}
  compiler;

\item[\PMakeAnHPFProgram:] run {\tt make} on the \HPF{} program;

\item[\PRunAnHPFProgram:] go a step further by trying to run the
  Fortran 77 output of the \HPF{} compiler;

\item[\PViewTheHPFCompilerOutput:] this menu allows you to view one of
  the files generated by the \HPF{} compiler. For each module, the
  main files are the {\tt \_host.f} file for the scalar code and the
  {\tt \_node.f}  file for the parallel code.

\end{description}

\section{Options}
\label{sec:options_menu}

\begin{center}
  \mbox{\psfig{file=images/options_menu.eps,width=0.3\hsize}}
\end{center}

\input{wpips-epips-options-menu.tex}

\chapter{Conclusion}
\label{cha:conclusion}

{\small
\bibliographystyle{plain}
\bibliography{mybib-beatrice}
}

\end{document}
