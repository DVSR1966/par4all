%%
%% $Id$
%%
%% Copyright 1989-2009 MINES ParisTech
%%
%% This file is part of PIPS.
%%
%% PIPS is free software: you can redistribute it and/or modify it
%% under the terms of the GNU General Public License as published by
%% the Free Software Foundation, either version 3 of the License, or
%% any later version.
%%
%% PIPS is distributed in the hope that it will be useful, but WITHOUT ANY
%% WARRANTY; without even the implied warranty of MERCHANTABILITY or
%% FITNESS FOR A PARTICULAR PURPOSE.
%%
%% See the GNU General Public License for more details.
%%
%% You should have received a copy of the GNU General Public License
%% along with PIPS.  If not, see <http://www.gnu.org/licenses/>.
%%
\slidepart{Synth�se de code}


\begin{trans}{Compilation de sp�cification}
  \begin{itemizer}
  \item \hypertarget{Synth}{Application} traitement du signal syst�matique
  \item Langage de type Fortran 90, Alpha
  \item Temps infini
  \item S�quence de nids de boucles parall�les
  \item Assignation unique
  \item Modulo
  \end{itemizer}
\end{trans}


\begin{trans}{Veille � Bande Large}
  %%/users/cri/ancourt/RAPPORT/SLIDES/INRIA/talk97.tex
  {\small
  \begin{verbatim}
doall t,h
  call FFT(t,h)
enddo
doall t,f,v 
  call BeamForming(t,f,v)
enddo
doall t,f,v
  call Energy(t,f,v)
enddo 
doall t,v
  call ShortIntegration(t,v)
enddo
doall t,v
  call AzimutStabilization(t,v)
enddo
doall t,v
  call LongIntegration(t,v)
enddo
\end{verbatim}
    }
\end{trans}


\begin{trans}{Machines cibles}
  \begin{itemizer}
  \item Multiprocesseurs � m�moire distribu�e 
  \item Pas de topologie particuli�re
  \item Programmation SIMD/SPMD
    \begin{itemizet}
    \item Une t�che de calculs � la fois (temps d'ex�cution fixe) 
    \item Temps de synchronisation et communications
    \end{itemizet}
  \item Possibilit� recouvrement calculs/communications
  \end{itemizer}
  \belleboite{Programmation Logique par Contraintes}
\end{trans}


\begin{trans}{Ordonnancement maquette VBL}
{\footnotesize
\begin{alltt}
 do t8 = 0, infini
   do t7 = 8*t8, 8*t8+7 {\orange
     FFT
     FFT
     FORMATION DE VOIES ; ENERGIE }
     do t5 = 8*t7,8*t7+7 {\orange
        FFT
        FORMATION DE VOIES ;  ENERGIE
        REGROUPEMENT LARGE BANDE }
     enddo {\orange
     FORMATION DE VOIES ; ENERGIE
     REGROUPEMENT LARGE BANDE
     REGROUPEMENT LARGE BANDE
     INTEGRATION COURTE
     STABILISATION EN AZIMUT }
   enddo {\orange
  INTEGRATION LONGUE 
  STABILISATION EN AZIMUT }
 enddo
\end{alltt}
}
\end{trans}


\begin{trans}{FFT}
\begin{alltt}
  do t=0,infinity
   do h=0,511
{\orange{}c          Read/IN Region:   
c                HYDRO(h,512*t:512*t+511)}
{\green{}c          Write/OUT Region: 
c                TABFFT(h,0:255,t)}
    call FFTDbl({\orange{}HYDRO(h,512*t:512*t+511)}, {\green{}TABFFT(h,0:255,t)})
   enddo
  enddo
\end{alltt}
\end{trans}


\newcommand{\boitea}[1]{\psframebox*[fillcolor=wheat,framesep=1pt]{\black #1}}
\newcommand{\boiteb}[1]{\psframebox*[fillcolor=salmon,framesep=1pt]{\black
    #1}}
\newcommand{\boitec}[1]{\psframebox*[fillcolor=red,framesep=1pt]{\black
    #1}}
\newcommand{\boited}[1]{\psframebox*[fillcolor=yellow,framesep=1pt]{\black #1}}
\newcommand{\boitee}[1]{\psframebox*[fillcolor=springgreen,framesep=1pt]{\black #1}}
\newcommand{\boitef}[1]{\psframebox*[fillcolor=green,framesep=1pt]{\black
    #1}}
\newcommand{\boiteg}[1]{\psframebox*[fillcolor=cyan,framesep=1pt]{\black #1}}
\newcommand{\boiteh}[1]{\psframebox*[fillcolor=violet,framesep=1pt]{\black
    #1}}

\begin{trans}{Contribution de PIPS : (r�)allocation de tableaux}
{\footnotesize
\begin{alltt}    
real \boitea{Hydro}(0:N,0:511)
real \boiteb{Frequence}(0:N,0:255,0:511)
real \boitec{Voies}(0:N,0:199,0:127)
real \boited{Energie}(0:N,0:199,0:127)
real \boitee{BL}(0:N,0:127)
real \boitef{INTC}(0:N,0:127)
real \boiteg{VSTAB}(0:N,0:127)
real \boiteh{OUTPUT}(0:N,0:127)
DO t8 = 0, N
   DO t7 = 8*t8, 8*t8+7
      DO t5 = 8*t7, 8*t7 + 7
         DO t1 = 512*t5, 512*t5+511
            read 5, (\boitea{Hydro}(t1,h1), h1 = 0 , 511)
         DO h2 = 0, 511
            call FFTr(\boiteb{Frequence},t5, h2, \boitea{Hydro}, N)
\newslide
         DO v3 = 0, 127
            DO f3 = 0, 199
               call FV(\boitec{Voies}, t5, f3, v3, \boiteb{Frequence}, N)
               call MOD2(\boited{Energie}(t5,f3,v3), \boitec{Voies}(t5,f3,v3))
         DO v5 = 0, 127
            call RtBL(\boitee{BL}, \boited{Energie},t5,v5,N)
      DO v6 = 0, 127
         call INTnL(\boitef{INTC},t7,v6, \boitee{BL}, N)
      DO v7 = 0, 127
         call STABAz(\boiteg{VSTAB}, \boitef{INTC},t7,v7,N)
   DO v8 = 0, 127
      call INTnL(\boiteh{OUTPUT},t8,v8, \boiteg{VSTAB}, N)
   print 6, (\boiteh{OUTPUT}(t8,i), i = 0, 127)
  \end{alltt}
}  
\end{trans}


\begin{trans}{VBL r�allou�e}
{\footnotesize
\begin{alltt}
DO t8 = 0, N
   real \boiteg{VSTAB}(0:7,0:127)
   real OUTPUT(0:0,0:127)
   DO t7 = 8*t8, 8*t8+7
      real \boitee{BL}(0:7,0:127)
      real \boitef{INTC}(0:0,0:127)
      DO t5 = 8*t7, 8*t7 + 7
         real \boitea{Hydro}(0:511,0:511)
         real \boiteb{Frequence}(0:0,0:255,0:511)
         real \boited{Energie}(0:0,0:199,0:127)
         DO t1 = 0, 511
            read 5, (\boitea{Hydro}(t1,h1), h1 = 0 , 511)
         DO h2 = 0, 511
            call FFTr(\boiteb{Frequence},0, h2, \boitea{Hydro}, N)
\newslide
         DO v3 = 0, 127
            DO f3 = 0, 199
               real \boitec{Voies}(0:0,0:199,0:127)
               call FV(\boitec{Voies}, 0, f3, v3, \boiteb{Frequence}, N)
               call MOD2(\boited{Energie}(0,f3,v3), \boitec{Voies}(0,f3,v3))
         DO v5 = 0, 127
            call RtBL(\boitee{BL}, \boited{Energie},t5,v5,N)
      DO v6 = 0, 127
         call INTnL(\boitef{INTC},0,v6, \boitee{BL}, N)
      DO v7 = 0, 127
         call STABAz(\boiteg{VSTAB}, \boitef{INTC},t7,v7,N)
   DO v8 = 0, 127
      call INTnL(OUTPUT,t8,v8, \boiteg{VSTAB}, N)
   print 6, (\boiteh{OUTPUT}(0,i), i = 0, 127)
  \end{alltt}
}  
\end{trans}


\begin{trans}{Programmation logique concurrente par contraintes}
  \psfighautcentre{dessins/relation2.idraw}
\end{trans}


\begin{trans}{Compilation de sp�cifications fonctionnelles}
  \begin{itemizer}
  \item Programmation concurrente par contraintes 
  \item Traitement global du probl�me du placement (non-lin�aire)
  \item D�monstration de faisabilit�
  \item �tudes de mod�lisation (communications dans PSP RBE2)
  \item Extension en robustesse et applicabilit� (th�se en cours)
  \item Projet recherche exploratoire RNRT PROMPT
  \item plan d'�tude amont DGA NARVAL action Paradis
  \end{itemizer}
\end{trans}


%%% Local Variables: 
%%% mode: latex
%%% TeX-master: "trans"
%%% End: 
