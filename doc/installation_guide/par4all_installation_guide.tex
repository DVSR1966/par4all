\documentclass[a4paper]{article}
\usepackage[utf8]{inputenc}

\usepackage{a4wide}
\usepackage{url}

\usepackage{alltt}
% Hmmm... Do not work with UTF-8:
% \usepackage{verbatim}

\usepackage{listings}%[hyper,procnames]
\lstset{extendedchars=true, language=C++, basicstyle=\scriptsize\ttfamily, numbers=left,
  numberstyle=\tiny, stepnumber=5, numberfirstline=true,
  tabsize=8, tab=\rightarrowfill, keywordstyle=\bf,
  stringstyle=\rmfamily, commentstyle=\rmfamily\itshape}

\usepackage{abbrev_reactive}
\let\OldRightarrow=\Rightarrow
\RequirePackage{marvosym}
\let\MarvosymRightarrow=\Rightarrow
\let\Rightarrow=\OldRightarrow
\RequirePackage{wasysym}
\let\Lightning\UnTrucIndefini% Car conflit entre ifsym et marvosym
\let\Sun\UnTrucIndefini%
\RequirePackage[weather]{ifsym}


\newcommand{\LINK}[1]{\url{#1}\xspace}
\newcommand{\PfaInstallationPDF}{\LINK{http://download.par4all.org/doc/installation/par4all_installation_guide.pdf}}
\newcommand{\PfaAllInstallationHTDOC}{\LINK{http://download.par4all.org/doc/installation/par4all_installation_guide.htdoc}}

\sloppy

% Number everything in the TOC:
\setcounter{secnumdepth}{10}
\setcounter{tocdepth}{10}

\begin{document}

\title{\protect\Apfa Installation Guide\\
  ---\\
  HPC Project}

\author{\Apfa \textsc{team}}

\maketitle

% The version is here and not in the title to avoid triggering a bug in
% tex4ht:
\noindent\textbf{This installation guide is for \Apfa version ../../VERSION}
\bigskip

This document can be found in PDF format on \PfaInstallationPDF and in HTML
on \PfaAllInstallationHTDOC.

% To automatically build reports from this content:
%%ContentBegin

\section{Introduction}
\label{sec:introduction}


You can install \Apfa in different ways, more or less automatic. The easiest
way to install \Apfa is to use update manager or tarball files. If you are
interested in a more advanced installation from the sources, you can clone 
the \Apfa \Agit repository.

This document describes these different ways of installing \Apfa as well as 
the prerequisites of the installation and the way to verify that your installation
is correctly done.

\section{Requirements}
\label{sec:requirements}

Current version of \Apfa should work on any \texttt{GNU/Linux} distribution, especially Debian, 
Ubuntu and Fedora.

\subsection{Packages needed to build and run Par4All}
\label{sec:pack-need-build}

% To ease copy/paste:
% cproto indent flex bison automake libtool autoconf libreadline5-dev python-dev swig python-ply libgmp3-dev libmpfr-dev gfortran subversion git

To compile \Apfa with Debian or Ubuntu, the following additional
packages are needed:
\begin{quote}
  \texttt{cproto}, \texttt{indent}, \texttt{flex}, \texttt{bison},
  \texttt{automake}, \texttt{libtool}, \texttt{autoconf},
  \texttt{libreadline5-dev}, \texttt{python-dev}, \texttt{swig},
  \texttt{python-ply}, \texttt{libgmp3-dev},
  \texttt{libmpfr-dev}, \texttt{gfortran}, \texttt{subversion},
  \texttt{git} \texttt{wget}
\end{quote}

To compile \Apfa with Fedora Core 15, 14 or 13, the following additional
packages are needed:
\begin{quote}
  \texttt{cproto}, \texttt{indent}, \texttt{flex}, \texttt{bison},
  \texttt{automake}, \texttt{libtool}, \texttt{autoconf},
  \texttt{readline-devel}, \texttt{python-devel}, \texttt{swig},
  \texttt{python-ply}, \texttt{gmp-devel},
  \texttt{mpfr-devel}, \texttt{gcc-gfortran}, \texttt{subversion},
  \texttt{git} \texttt{wget}
\end{quote}

% To ease copy/paste:
% libncurses5 libreadline5 python python-ply gfortran

To run \Apfa with Debian or Ubuntu, the following packages are needed:
\begin{quote}
  \texttt{libmpfr1ldbl},
  \texttt{libncurses5}, \texttt{libreadline5}, \texttt{python},
  \texttt{python-ply}, \texttt{gfortran},
  \texttt{ipython}.
\end{quote}

To run \Apfa with Fedora Core 15, 14 or 13, the following packages are needed:
\begin{quote}
  \texttt{ncurses-5}, \texttt{readline}, \texttt{python},
  \texttt{python-ply}, \texttt{python3}, \texttt{gcc-gfortran},
  \texttt{ipython}.
\end{quote}

To compile the documentation with Debian or Ubuntu, the following
additional packages are needed:
\begin{quote}
  \texttt{python-docutils}, \texttt{texlive-full}, \texttt{tex4ht}
\end{quote}

To compile the documentation with Fedora Core 15, 14 or 13, the following
additional packages are needed:
\begin{quote}
  \texttt{texlive}, \texttt{tex4ht}
\end{quote}

\subsection{CUDA environment to compile and execute on NVidia GPU}

To compile and to run the \Acuda and \Aopencl output, you should have the following
environment variables set:
\begin{itemize}
\item \verb|CUDA_DIR| to the directory where \Acuda has been installed (default to
  \texttt{/usr/local/cuda}
\item \verb|LD_LIBRARY_PATH| should contains at least \verb|$CUDA_DIR/lib64|
\end{itemize}

\section{Installations}
\label{sec:installation}

This section represents different ways to install \Apfa.

The installation is done into \texttt{/usr/local/par4all}. If you want \Apfa
installed in another location, use the compilation way of life at 
section~\ref{sec:options}.

\subsection{Installation using update manager}

The best way if you are on \texttt{GNU/Linux} Debian or Ubuntu is to use
our package repository. This way, when a new version is out, your
classical package manager can automatically install it.

To use our package repository, pick \texttt{one} of the following lines, and
add it graphically with the \texttt{Update Manager} with
\texttt{Settings.../Third-Party Software} or append it with a text editor to
your \texttt{/etc/apt/sources.list}, if you are using Ubuntu::

\begin{verbatim}
  deb http://download.par4all.org/apt/ubuntu releases main
  # --OR--
  deb http://download.par4all.org/apt/ubuntu development main
\end{verbatim}

or if you are running Debian::
\begin{verbatim}
  deb http://download.par4all.org/apt/debian releases main
  # --OR--
  deb http://download.par4all.org/apt/debian development main
\end{verbatim}

So you need to choose between \texttt{releases} or \texttt{development}
versions. Development packages are generated often, may be unstable, and
are best suited if you want to track more closely the \Apfa development.

Once this is done, run your favorite graphics package tool
(synaptic...) or:
\begin{verbatim}
  sudo aptitude update
  sudo aptitude install par4all
\end{verbatim}

And you are (almost) good to go! To set your environment up and test your \Apfa
installation, please refer to the section~\ref{sec:testing}, 
or log out and log in again for the environment to get properly set
up for \Apfa.

\subsection{Installation using tarball files}

A less automatic way on Debian or Ubuntu is to install the Par4All
\texttt{.deb} package you have found on \url{http://download.par4all.org} with:
\begin{verbatim}
  sudo gdebi <the_package>.deb
\end{verbatim}
or 
\begin{verbatim}
  sudo dpkg -i <the_package>.deb 
\end{verbatim}

The second one would also work but does not
automatically install dependencies you should install later.

An even less automatic way is to use a tarball \texttt{.tar.gz} file. It
contains the binaries as built on a stable Ubuntu or unstable Debian
distribution.

Once you have downloaded one of these \texttt{.tar.gz} packages from
\url{http://download.par4all.org}, extract it with the following command::
\begin{verbatim}
  tar xvzf <the_package>.tar.gz
\end{verbatim}

It will create a directory named \texttt{par4all}. Move this directory to its final
location:
\begin{verbatim}
  sudo mv par4all /usr/local
\end{verbatim}

Then please go to the section~\ref{sec:testing} to get information on how to test 
your installation.

\subsection{ Installation using \protect\Agit}

From the \Apfa source directory, \Apfa is compiled and configured
by running \verb|src/simple_tools/p4a_setup.py|. See
\url{http://www.par4all.org/documentation/par4all_organization} for details.

To download and compile \Apfa from the \Agit, do the following:
\begin{alltt}
# Get a working copy of the Par4All public read-only git repository:
git clone --branch p4a git://git.hpc-project.com/git/par4all.git
# Go into the working copy:
cd par4all
# Build everything using 8 processes to speed up things:
src/simple_tools/p4a_setup.py [--prefix=\emph{/install/dir}] [-v[v[v]]] --jobs=8 [...]
\end{alltt}%

\Apfa will be installed into \verb|/usr/local/par4all| by default.
The target directory must be writable by the installer, either by running
as \texttt{root} with \texttt{sudo} or by creating first a
writable target directory as follows:
\begin{alltt}
mkdir /usr/local/par4all
chown \emph{your_login_name} /usr/local/par4all
\end{alltt}
In general, it is less dangerous to limit the number of commands
executed as \texttt{root}, therefore, the latter approach to \Apfa
installation is preferable.

To install in another location, the \verb|--prefix| option can be used,
for example to choose \texttt{/opt/par4all}. Warning: do not use plain
system directory names such as \texttt{/usr} or \texttt{/usr/local}
because some system files such as \texttt{/usr/include/assert.h} may be
overwritten and havoc may happen...

To see more installation options (including specifying the locations for
other packages), run \verb|p4a_setup.py -h| or see
section~\ref{sec:installation_options}.

To pull a new version, do:
\begin{verbatim}
git pull origin p4a
\end{verbatim}

and run \verb|p4a_setup.py| again. The
\verb|--rebuild| and \verb|--clean| parameters should be used to
ensure that all sources are recompiled
(since \Apips is a frequently-updated project, incremental build is
not always guaranteed to succeed). Removing the \texttt{build}
directory
when the \verb|--prefix| directory changes is also recommended
in order to remove obsolete
information about the installation directory that may remain in
the \texttt{build} directory and cause the compilation to fail.

\subsection{Installation options}
\label{sec:installation_options}

The compilation and installation of \Apfa is controlled by the
\verb|p4a_setup.py| script, with the usage and options described in
this section.

\input{p4a_setup-help}

\subsection{Advanced installation}

\section{Setting up environment and testing \protect\Apfa}
\label{sec:testing}

In any case, you will then need to source one of the following shell
scripts which set up the environment variables for proper \Apfa
execution:
\begin{itemize}
\item if you use \texttt{bash, sh, dash}, etc.:
\begin{verbatim}
      source /usr/local/par4all/etc/par4all-rc.sh
\end{verbatim}
\item if you use \texttt{csh, tcsh}, etc.:
\begin{verbatim}
      source /usr/local/par4all/etc/par4all-rc.csh
\end{verbatim}
\end{itemize}

\subsection{Testing and verification}

Once you set your environment up, you can, for example, run \texttt{p4a -h}
to get help about the usage of the \texttt{p4a} frontend script.

\subsubsection{Examples and demos}

In the examples directory there are few examples or demos showing some \texttt{p4a,
tpips, pyps} use cases. You can use these examples to test your installation.
For example, to test \texttt{p4a} go to the directory \texttt{par4all/example/P4A/Hyantes}
and run:
\begin{verbatim}
      make demo
\end{verbatim}
if you got all environment installed, included \Acuda environment.

If you do not have \Acuda environment in your machine, you can test your installation with
\Aopenmp, as the following:
\begin{verbatim}
      make display_openmp
\end{verbatim}
which will transform the Hyantes sequential codes to the \Aopenmp, compile the generated codes,
run and display the results.
To get more information on the examples, have a look at \texttt{examples/README.txt}

Further you can look at the manual of \texttt{p4a} on
\url{http://www.par4all.org/documentation} for information on how to use Par4All.

\end{document}



