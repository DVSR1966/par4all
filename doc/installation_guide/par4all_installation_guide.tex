\documentclass[a4paper]{article}
\usepackage[utf8]{inputenc}

\usepackage{a4wide}
\usepackage{url}

\usepackage{alltt}
% Hmmm... Do not work with UTF-8:
% \usepackage{verbatim}

\usepackage{listings}%[hyper,procnames]
\lstset{extendedchars=true, language=C++, basicstyle=\scriptsize\ttfamily, numbers=left,
  numberstyle=\tiny, stepnumber=5, numberfirstline=true,
  tabsize=8, tab=\rightarrowfill, keywordstyle=\bf,
  stringstyle=\rmfamily, commentstyle=\rmfamily\itshape}

\usepackage{abbrev_reactive}
\let\OldRightarrow=\Rightarrow
\RequirePackage{marvosym}
\let\MarvosymRightarrow=\Rightarrow
\let\Rightarrow=\OldRightarrow
\RequirePackage{wasysym}
\let\Lightning\UnTrucIndefini% Car conflit entre ifsym et marvosym
\let\Sun\UnTrucIndefini%
\RequirePackage[weather]{ifsym}


\newcommand{\LINK}[1]{\url{#1}\xspace}
\newcommand{\PfaInstallationPDF}{\LINK{http://download.par4all.org/doc/installation/par4all_installation_guide.pdf}}
\newcommand{\PfaAllInstallationHTDOC}{\LINK{http://download.par4all.org/doc/installation/par4all_installation_guide.htdoc}}

\sloppy

% Number everything in the TOC:
\setcounter{secnumdepth}{10}
\setcounter{tocdepth}{10}

\begin{document}

\title{\protect\Apfa Installation Guide\\
  ---\\
  SILKAN}

\author{\Apfa \textsc{team}}

\maketitle

% The version is here and not in the title to avoid triggering a bug in
% tex4ht:
\noindent\textbf{This installation guide is for \Apfa version ../../VERSION}
\bigskip

This document can be found in PDF format on \PfaInstallationPDF and in HTML
on \PfaAllInstallationHTDOC.

% To automatically build reports from this content:
%%ContentBegin

\section{Introduction}
\label{sec:introduction}


You can install \Apfa in different ways, more or less automatic. The easiest
way to install \Apfa is to use update manager or tarball files. If you are
interested in a more advanced installation from the sources, you can clone
the \Apfa{} \Agit repository.

This document describes these different ways of installing \Apfa as well
as the prerequisites of the installation and the way to verify that your
installation is correctly done.


\section{Requirements}
\label{sec:requirements}

Current version of \Apfa should work on any \texttt{GNU/Linux}
distribution, especially Debian, Ubuntu and Fedora.


\subsection{Packages needed to build and run Par4All}
\label{sec:pack-need-build}

% To ease copy/paste:
% libncurses5 libreadline5 python python-ply gfortran

To run \Apfa with Debian or Ubuntu, the following packages are needed:
\begin{quote}
  \texttt{cproto} \texttt{ed} \texttt{gfortran} \texttt{gnuplot}
  \texttt{ipython} \texttt{libmpfr-dev} \texttt{libncurses5}
  \texttt{libpython2.7} \texttt{libreadline5} \texttt{python}
  \texttt{python-ply}
\end{quote}

Some examples shipped with \Apfa needs other packages such as:
\begin{quote}
  \texttt{freeglut3-dev} \texttt{libfftw3-dev} \texttt{libgtk2.0-dev}
  \texttt{time}
\end{quote}
and even the nVidia CUDA environment with:
\begin{quote}
  \texttt{nvidia-cuda-toolkit} \texttt{nvidia-cuda-dev}
\end{quote}


If you get a precompiled version of \Apfa, you may select a more specific
Python version.

To run \Apfa with Fedora Core 15, 14 or 13, the following packages are needed:
\begin{quote}
  \texttt{cproto} \texttt{gnuplot} \texttt{ncurses-5}
  \texttt{readline} \texttt{python} \texttt{python-ply}
  \texttt{python3} \texttt{gcc-gfortran} \texttt{ipython}
  \texttt{libncurses5-dev}
\end{quote}

The following packages need to be installed especially if you use a less
automatic way of installation such as a tarball file, or if you install
from the sources.

% To ease copy/paste:
% cproto indent flex bison automake libtool autoconf libreadline6-dev python-dev swig python-ply libgmp3-dev libmpfr-dev gfortran subversion git wget

To compile \Apfa with Debian or Ubuntu, the following additional
packages are needed:
\begin{quote}
  \texttt{autoconf} \texttt{automake} \texttt{bison} \texttt{cproto}
  \texttt{gfortran} \texttt{indent} \texttt{flex} \texttt{git}
  \texttt{gnulib} \texttt{libgmp3-dev} \texttt{libmpfr-dev}
  \texttt{libncurses5-dev} \texttt{libreadline6-dev} \texttt{libtool}
  \texttt{python-dev} \texttt{python-ply} \texttt{subversion}
  \texttt{swig} \texttt{wget}
\end{quote}
In Ubuntu 11.10 there is a bug in the \texttt{libtinfo-dev} and
\texttt{libtinfo5} packages that prevents \texttt{tpips} from being
compiled. If you are really brave, you can try to download manually the
Debian packages and force their installation with: {\small
\begin{verbatim}
wget -nd http://ftp.us.debian.org/debian/pool/main/n/ncurses/libtinfo-dev_5.9-4_amd64.deb
sudo dpkg --force-depends -i libtinfo-dev_5.9-4_amd64.deb
wget -nd http://http.us.debian.org/debian/pool/main/n/ncurses/libtinfo5_5.9-4_amd64.deb
sudo dpkg -i libtinfo5_5.9-4_amd64.deb
\end{verbatim}
}%
The links from Debian/unstable are likely to change. So look at
\url{http://packages.debian.org/sid/amd64/libtinfo5/download} and
\url{http://packages.debian.org/sid/amd64/libtinfo-dev/download}.  See
\url{https://bugs.launchpad.net/ubuntu/+source/ncurses/+bug/900635} for
the bug report.


To compile \Apfa with Fedora Core 15, 14 or 13, the following additional
packages are needed:
\begin{quote}
  \texttt{cproto} \texttt{indent} \texttt{flex} \texttt{bison}
  \texttt{automake} \texttt{libtool} \texttt{autoconf}
  \texttt{readline-devel} \texttt{python-devel} \texttt{swig}
  \texttt{python-ply} \texttt{gmp-devel}
  \texttt{mpfr-devel} \texttt{gcc-gfortran} \texttt{subversion}
  \texttt{git} \texttt{wget}
\end{quote}
On OpenSuse it looks like you need roughly the same packages, such as
\texttt{mpfr-devel}.

To compile the documentation with Debian or Ubuntu, the following
additional packages are needed:
\begin{quote}
  \texttt{python-docutils} \texttt{tex4ht} \texttt{texlive-full}
\end{quote}

To compile the documentation with Fedora Core 15, 14 or 13, the following
additional packages are needed:
\begin{quote}
  \texttt{texlive} \texttt{tex4ht}
\end{quote}


\subsection{CUDA environment to compile and execute on NVIDIA GPU}

To compile and to run the \Acuda and \Aopencl output, you should have the following
environment variables set:
\begin{itemize}
\item \verb|CUDA_DIR| to the directory where \Acuda has been installed (default to
  \texttt{/usr/local/cuda}
\item \verb|LD_LIBRARY_PATH| should contains at least \verb|$CUDA_DIR/lib64|
\end{itemize}


\section{Installations}
\label{sec:installation}

This section represents different ways to install \Apfa.

The installation is done into \texttt{/usr/local/par4all}. If you want
\Apfa installed in another location, have a loot at the
section~\ref{sec:relocation} for a binary installation and at the
section~\ref{sec:installation_options} for the installation from the
sources.


\subsection{Installation using \protect\Apfa package repository}

The best way if you are on \texttt{GNU/Linux} Debian or Ubuntu is to use
our package repository. This way, when a new version is out, your
classical package manager can automatically install it.

To use our package repository, pick \texttt{one} of the following lines, and
add it graphically with the \texttt{Update Manager} with
\texttt{Settings/Third-Party Software} or append it with a text editor to
your \texttt{/etc/apt/sources.list}, if you are using Ubuntu:

\begin{verbatim}
  deb http://download.par4all.org/apt/ubuntu releases main
  # --OR--
  deb http://download.par4all.org/apt/ubuntu development main
\end{verbatim}

or if you are running Debian::
\begin{verbatim}
  deb http://download.par4all.org/apt/debian releases main
  # --OR--
  deb http://download.par4all.org/apt/debian development main
\end{verbatim}

So you need to choose between \texttt{releases} or \texttt{development}
versions. Development packages are generated often, may be unstable, and
are best suited if you want to track more closely the \Apfa development.

Once this is done, run your favorite graphics package tool
(synaptic...) or:
\begin{verbatim}
  sudo aptitude update
  sudo aptitude install par4all
\end{verbatim}

To set your environment up and test your \Apfa
installation, please refer to the section~\ref{sec:testing}.


\subsection{Installation using a package}

A less automatic way on Debian or Ubuntu is to install a Par4All
\texttt{.deb} package manually.

\subsubsection{Getting the package}

For release versions, according to your OS and architecture, download a package from:
\begin{itemize}
\item \url{http://download.par4all.org/releases/debian/i686}
\item \url{http://download.par4all.org/releases/debian/x86_64}
\item \url{http://download.par4all.org/releases/ubuntu/i686}
\item \url{http://download.par4all.org/releases/ubuntu/x86_64}
\end{itemize}


For development versions, according to your OS and architecture, download a package from:
\begin{itemize}
\item \url{http://download.par4all.org/development/debian/i686}
\item \url{http://download.par4all.org/development/debian/x86_64}
\item \url{http://download.par4all.org/development/ubuntu/i686}
\item \url{http://download.par4all.org/development/ubuntu/x86_64}
\end{itemize}

\subsubsection{Installing the package}
You can then install the package with:
\begin{verbatim}
sudo gdebi <the_package>.deb
\end{verbatim}
or
\begin{verbatim}
sudo dpkg -i <the_package>.deb
\end{verbatim}

The second one would also work but does not
automatically install dependencies you should install later.


\subsection{Installation using tar.gz binary distribution}

An even less automatic way is to use a tarball \texttt{.tar.gz} file. It
contains the binaries as built on a stable Ubuntu or unstable Debian
distribution.

It should work also on any GNU/Linux distribution with the following
libraries installed: (a fairly recent) \texttt{libc.so.6},
\texttt{libncurses.so.5}, \texttt{libreadline.so.6}, etc. and Python
2.7. We chose this Python version because it is recent enough to provide
nice features for Par4All and not too recent to be absent from most Linux
distributions. See the section~\ref{sec:requirements} to have the list of
some needed packages.

If you are not on Debian on Ubuntu, try one of the following, it may work.


\subsubsection{Getting the distribution}

For the 64-bit x86 architecture, according to the OS and version you want,
download a \texttt{...\_x86\_64.tar.gz} file from

\begin{itemize}
\item \protect\url{http://download.par4all.org/releases/debian/x86_64}
\item \protect\url{http://download.par4all.org/releases/ubuntu/x86_64}
\item \protect\url{http://download.par4all.org/development/ubuntu/x86_64}
\item \protect\url{http://download.par4all.org/development/debian/x86_64}
\end{itemize}

For the 32-bit x86 architecture, according to the OS and version you want,
download a \texttt{...\_i686.tar.gz} file from
\begin{itemize}
\item \protect\url{http://download.par4all.org/releases/debian/i686}
\item \protect\url{http://download.par4all.org/releases/ubuntu/i686}
\item \protect\url{http://download.par4all.org/development/debian/i686}
\item \protect\url{http://download.par4all.org/development/ubuntu/i686}
\end{itemize}


\subsubsection{Installing the distribution}

Once you have downloaded one of these \texttt{.tar.gz} packages from
\url{http://download.par4all.org}, extract it with the following command::
\begin{alltt}
tar xvzf \emph{<the_package>}.tar.gz
\end{alltt}

It will create a directory named \texttt{par4all}. Move this directory to its final
location:
\begin{verbatim}
sudo mv par4all /usr/local
\end{verbatim}

Then please go to the section~\ref{sec:testing} to get information on how
to set up your environment and to test your installation.


\subsection{Installation using the sources}

This is not the preferred way to work, but it can be useful for people who
cannot use a precompiled version.


\subsubsection{Get the source distribution}
First get a source tar ball in the following directories (Ubuntu or Debian
do not matter here) :
\begin{itemize}
\item \protect\url{http://download.par4all.org/releases}
\item \protect\url{http://download.par4all.org/development}
\end{itemize}

Pick up a file with a name ending with \texttt{\_src.tar.gz}. You can
decompress it with a \texttt{tar zxvf}.


\subsubsection{Get a source snapshot from the git repository}

On \url{https://git.par4all.org/cgit/par4all} you can find the
snapshots in the Download section in \texttt{.zip}, \texttt{.tar.gz} or
\texttt{.tar.bz2}.

By tweaking the URL of a tag snapshot you can even get a snapshot from any
branch or commit, not only tag. For example the last snapshot of the
\texttt{p4a} integrated branch is in:

\url{https://git.par4all.org/cgit/par4all/snapshot/p4a.tar.bz2}


\subsubsection{Installing from the sources}
Please refer to the section~\ref{sec:installation_using git} to get to
know how to install \Apfa from the sources.


\subsection{Installation using \protect\Agit}
\label{sec:installation_using git}

From the \Apfa source directory, \Apfa is compiled and configured
by running \verb|src/simple_tools/p4a_setup.py|. See
\url{http://www.par4all.org/documentation/par4all_developer_guide} for details.

To download and compile \Apfa from the \Agit, do the following:
\begin{alltt}
# Get a working copy of the Par4All public read-only git repository:
git clone --branch p4a git://git.par4all.org/git/par4all.git
# Go into the working copy:
cd par4all
# Build everything using 8 processes to speed up things:
src/simple_tools/p4a_setup.py [--prefix=\emph{/install/dir}] [-v[v[v]]] --jobs=8 [...]
\end{alltt}%

\Apfa will be installed into \verb|/usr/local/par4all| by default.
The target directory must be writable by the installer, either by running
as \texttt{root} with \texttt{sudo} or by creating first a
writable target directory as follows:
\begin{alltt}
mkdir /usr/local/par4all
chown \emph{your_login_name} /usr/local/par4all
\end{alltt}
In general, it is less dangerous to limit the number of commands
executed as \texttt{root}, therefore, the latter approach to \Apfa
installation is preferable.

To install in another location, the \verb|--prefix| option can be used,
for example to choose \texttt{/opt/par4all}.

\begin{alltt}
src/simple_tools/p4a_setup.py --prefix=/opt/par4all
\end{alltt}

Warning: do not use plain
system directory names such as \texttt{/usr} or \texttt{/usr/local}
because some system files such as \texttt{/usr/include/assert.h} may be
overwritten and havoc may happen...

To see more installation options (including specifying the locations for
other packages), run \verb|p4a_setup.py -h| or see the \Apfa developer guide}.

To pull a new version, do:
\begin{verbatim}
git pull origin p4a
\end{verbatim}

and run \verb|p4a_setup.py| again. The
\verb|--rebuild| and \verb|--clean| parameters should be used to
ensure that all sources are recompiled
(since \Apips is a frequently-updated project, incremental build is
not always guaranteed to succeed). Removing the \texttt{build}
directory
when the \verb|--prefix| directory changes is also recommended
in order to remove obsolete
information about the installation directory that may remain in
the \texttt{build} directory and cause the compilation to fail.


\subsection{Relocating binary installation}
\label{sec:relocation}
If needed a \Apfa binary installation can be relocated. There are a few ways to relocate it:

\begin{enumerate}
\item Copy or move all \Apfa binary installation to another directory, for
  example to copy from \texttt{/usr/local/par4all} to \texttt{/opt}:
\begin{alltt}
cp -av \emph{/usr/local/par4all} \emph{/opt}
\end{alltt}

\item Extract from a Debian package, for example to install
  \Apfa in \texttt{/opt} directory:
\begin{alltt}
sudo dpkg -x par4all-1.3.1-xxx.deb \emph{/opt}
\end{alltt}

\end{enumerate}

After having installed \Apfa in a new installation directory (not the
default one), \verb|par4all-rc.sh| should be updated.  \verb|P4A_DIST| and
\verb|P4A_ROOT| variables in the \verb|par4all-rc.sh| should be changed
according to this new installation directory.  The following examples are
for the cases 1 and 2.

Relocation has been done by copying all the binary installation (case 1):
\begin{alltt}
P4A_DIST='/opt/par4all'
P4A_ROOT='/opt/par4all'
\end{alltt}

Relocation has been done using \texttt{dpkg -x} command (case 2) :
\begin{alltt}
P4A_DIST='/opt/usr/local/par4all'
P4A_ROOT='/opt/usr/local/par4all'
\end{alltt}


\subsection{Advanced installation}

To get information on a more advanced installation please see
\url{http://www.par4all.org/documentation/par4all_developer_guide}


\section{Setting up environment and testing \protect\Apfa}
\label{sec:testing}

\subsection{Setting up environment variables}
In any case, you will then need to source one of the following shell
scripts which set up the environment variables for proper \Apfa
execution:
\begin{itemize}
\item if you use \texttt{bash, sh, dash}, etc.:
\begin{verbatim}
source /usr/local/par4all/etc/par4all-rc.sh
\end{verbatim}
\item if you use \texttt{csh, tcsh}, etc.:
\begin{verbatim}
source /usr/local/par4all/etc/par4all-rc.csh
\end{verbatim}
\end{itemize}


\subsection{Testing and verification}

Once you set your environment up, you can, for example, run \texttt{p4a -h}
to get help about the usage of the \texttt{p4a} frontend script.


\subsubsection{Examples and demos}

In the examples directory there are few examples or demos showing some
\texttt{p4a}, \texttt{tpips}, \texttt{pyps} use cases. You can use these
examples to test your installation.  For example, to test \texttt{p4a}
using one of the examples in the \texttt{P4A} directory go to one of the
directories in \texttt{P4A} and run:
\begin{verbatim}
make demo
\end{verbatim}
if you got all needed environments installed, included \Acuda. This will
chain all available demos included:
\begin{itemize}
\item sequential execution;
\item automatic parallelization with \texttt{p4a} for parallel execution
  on multi-cores with \Aopenmp;
\item automatic parallelization with \verb|p4a --cuda| for parallel
  execution on nVidia \Agpu;
\item automatic parallelization with \verb|p4a --accel| for an \Aopenmp
  parallel emulation of a \Agpu-like accelerator;
\item automatic parallelization with \verb|p4a --opencl| for the \Aopencl
  parallel execution on nVidia \Agpu.
\end{itemize}

If you do not have \Acuda environment in your machine, you can test your
installation with \Aopenmp, as the following:
\begin{verbatim}
make display_openmp
\end{verbatim}
which will transform the sequential codes to the \Aopenmp, compile the
generated codes, run and display the results.  To get more information on
the examples, have a look at \texttt{examples/README.txt}.

Further you can look at the manual of \texttt{p4a} on
\url{http://www.par4all.org/documentation} for information on how to use Par4All.

\end{document}



%%% Local Variables:
%%% mode: latex
%%% ispell-local-dictionary: "american"
%%% End:
