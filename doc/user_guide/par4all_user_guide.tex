\documentclass[a4paper]{article}
\usepackage[utf8]{inputenc}
\usepackage{a4wide}
\usepackage{url}
\usepackage{alltt}
\usepackage{hyperref}
\usepackage{abbrev_reactive}
\usepackage{listings}
\usepackage{color}

\begin{document}

\title{Par4All programming user guide\\
  ---\\
  HPC Project}

\author{Mehdi \textsc{Amini} \and Amaury \textsc{Darsch} \and
  Stéphanie \textsc{Even} \and Serge \textsc{Guelton} \and Ronan
  \textsc{Keryell} \and Grégoire \textsc{Péan} \and Claire
  \textsc{Seguin} \and Mickaël \textsc{Thievent} \and Pierre
  \textsc{Villalon}}

\maketitle

\tableofcontents

\section{Introduction}
\label{sec:introduction}

The backend of \Apfa defines a common interface for target languages
and architecture such as C/OpenMP on CPUs or C for CUDA and OpenCL for
GPUs. Based on macros, this backend can also be used independantly of
the whole \Apfa platform, that comprises various open source tools to
automate the migration of sequential software to multicore and other
parallel processors.  The set of macros of the backend of \Apfa can be
used inside a program in a very simple way, allowing the target
architecture and/or language to be tested by changing only one
compilation option. The main difference with using the complete \Apfa
suite is that i) the automatic analysis to optimize the loop (prevent
data dependence, data structuring) have to be manually performed by
the user, ii) the source language must be C/C++ compatible.

\section{The set of macros}
\label{sec:macros}

\subsection{Begin and End actions}

\begin{description}
\item[\texttt{P4A\_init\_accel}] Initialize the use of the hardware
  accelerator ;
\item[\texttt{P4A\_release\_accel}] Release the hardware accelerator.
\end{description}

Those two commands must be placed once at the beginning and once at
the end of the program. In addition, the \texttt{p4a\_accel.h} must be
included from the main file.

\subsection{Memory allocation and copy}

\begin{description}
\item[\texttt{P4A\_accel\_malloc(void **address, size\_t size)}]
  Allocate memory on the hardware accelerator.

  \begin{description}
  \item[address] Address of the allocated memory zone;
  \item[size] Size of the memory zone to be allocated.
  \end{description}

\item[\texttt{P4A\_accel\_free(void *address)}]
  Free memory on the hardware accelerator.

   \begin{description}
  \item[address] Address of the memory zone to be free.
  \end{description}
 

\item[\texttt{P4A\_copy\_to\_accel(size\_t element\_size, void
    *host\_address, const void *accel\_address)}] Copy a scalar or a
  compact memory zone from the host to the hardware accelerator;

  \begin{description}
  \item[element\_size] Total size in byte of the memory that must be copied;
  \item[host\_address] Address of the host memory zone from where to copy;
  \item[accel\_address] Address of the device memory where to copy.
  \end{description}


\item[\texttt{P4A\_copy\_from\_accel(size\_t element\_size, void
    *host\_address, const void *accel\_address)}] Copy a scalar or a
  compact memory zone from the hardware accelerator to the host.  Do
  not change the place of the pointers in the API. The host address is
  always first.

  \begin{description}
  \item[element\_size] Total size of the memory that must be copied;
  \item[host\_address] Address of the host memory zone from where to copy;
  \item[accel\_address] Address of the device memory where to copy.
  \end{description}



\item[\texttt{P4A\_copy\_to\_accel\_1d(size\_t element\_size, size\_t
    d1\_size,size\_t d1\_block\_size, size\_t d1\_offset, void
    *host\_address, const void *accel\_address)}] Function for copying
  a 1D memory zone from the host to a compact memory zone in the
  hardware accelerator.

  \begin{description}
  \item[element\_size] The size of one element of the array in byte;
  \item[d1\_size] The number of elements in the array;
  \item[d1\_block\_size] The number of element to transfer;
  \item[d1\_offset] The element order to start the transfer from (host
    side);
  \item[host\_address] Address of the host memory zone from where to copy;
  \item[accel\_address] Address of the device memory where to copy.
  \end{description}


\item[\texttt{P4A\_copy\_from\_accel\_1d(size\_t element\_size,
    size\_t d1\_size,size\_t d1\_block\_size, size\_t d1\_offset, void
    *host\_address, const void *accel\_address)}] Function for copying
  a compact memory zone from the hardware accelerator to a 1D array in the
  host.

  \begin{description}
  \item[element\_size] The size of one element of the array in byte;
  \item[d1\_size] The number of elements in the array;
  \item[d1\_block\_size] The number of element to transfer;
  \item[d1\_offset] The element order to start the transfer to (host
    side);
  \item[host\_address] Address of the host memory zone where to copy;
  \item[accel\_address] Address of the device memory from where to copy.
  \end{description}


\item[\texttt{P4A\_copy\_to\_accel\_2d(size\_t element\_size, size\_t
    d1\_size, size\_t d2\_size, size\_t d1\_block\_size, size\_t
    d2\_block\_size, size\_t d1\_offset, size\_t d2\_offset, void
    *host\_address, const void *accel\_address)}] Function for copying
  a 2D memory zone from the host to a compact memory zone in the
  hardware accelerator.

  \begin{description}
  \item[element\_size] The size of one element of the array in byte;
  \item[d1\_size] The number of elements in a row;
  \item[d2\_size] The number of rows;
  \item[d1\_block\_size] The number of element to transfer in a row;
  \item[d2\_block\_size] The number of rows to transfer;
  \item[d1\_offset] The element order to start the transfer from (host
    side);
  \item[d2\_offset] The number of rows to be skipped (host side);
  \item[host\_address] Address of the host memory zone from where to copy;
  \item[accel\_address] Address of the device memory where to copy.
  \end{description}


\item[\texttt{P4A\_copy\_from\_accel\_2d(size\_t element\_size, size\_t
    d1\_size, size\_t d2\_size, size\_t d1\_block\_size, size\_t
    d2\_block\_size, size\_t d1\_offset, size\_t d2\_offset, void
    *host\_address, const void *accel\_address)}] Function for copying
  a 2D memory zone from the accelerator (compact zone) to the host.

  \begin{description}
  \item[element\_size] The size of one element of the array in byte;
  \item[d1\_size] The number of elements in a row;
  \item[d2\_size] The number of rows;
  \item[d1\_block\_size] The number of element to transfer in a row;
  \item[d2\_block\_size] The number of rows to transfer;
  \item[d1\_offset] The element order to start the transfer
    to (host side);
  \item[d2\_offset] The number of rows to be skipped (host side);
  \item[host\_address] Address of the host memory zone where to copy;
  \item[accel\_address] Address of the device memory from where to copy.
  \end{description}

\item[\texttt{P4A\_copy\_to\_accel\_3d(size\_t element\_size, size\_t
    d1\_size, size\_t d2\_size, size\_t d2\_size, size\_t
    d1\_block\_size, size\_t d2\_block\_size, size\_t d3\_block\_size,
    size\_t d1\_offset, size\_t d2\_offset, size\_t d3\_offset, void
    *host\_address, const void *accel\_address)}] Function for copying
  a 3D memory zone from the host to a compact memory zone in the
  hardware accelerator.

  \begin{description}
  \item[element\_size] The size of one element of the array in byte;
  \item[d1\_size] The number of elements in a row;
  \item[d2\_size] The number of rows;
  \item[d3\_size] The number of elements in the third dimension;
  \item[d1\_block\_size] The number of element to transfer in a row;
  \item[d2\_block\_size] The number of rows to transfer;
  \item[d3\_block\_size] The number of element in the third
    dimension to transfer;
  \item[d1\_offset] The element order to start the transfer from (host
    side);
  \item[d2\_offset] The number of rows to be skipped (host side);
  \item[d2\_offset] The number of element in the third dimension to
    be skipped (host side);
  \item[host\_address] Address of the host memory zone from where to copy;
  \item[accel\_address] Address of the device memory where to copy.
  \end{description}


\item[\texttt{P4A\_copy\_from\_accel\_3d(size\_t element\_size,
    size\_t d1\_size, size\_t d2\_size, size\_t d1\_block\_size,
    size\_t d2\_block\_size, size\_t d1\_offset, size\_t d2\_offset,
    void *host\_address, const void *accel\_address)}] Function for
  copying memory from the hardware accelerator to a 3D array in the
  host.

  \begin{description}
  \item[element\_size] The size of one element of the array in byte;
  \item[d1\_size] The number of elements in a row;
  \item[d2\_size] The number of rows;
  \item[d3\_size] The number of elements in the third dimension;
  \item[d1\_block\_size] The number of element to transfer in a row;
  \item[d2\_block\_size] The number of rows to transfer;
  \item[d3\_block\_size] The number of element in the third dimension
    to transfer;
  \item[d1\_offset] The element order to start the transfer to (host
    side);
  \item[d2\_offset] The number of rows to be skipped (host side);
  \item[d2\_offset] The number of element in the third dimension to be
    skipped (host side);
  \item[host\_address] Address of the host memory zone where to copy;
  \item[accel\_address] Adress of the device memory from where to copy.
  \end{description}

\end{description}

\subsection{The call to the kernel}

The core of the computation, typically a loop, will constitute the kernel.
The call to the kernel must follows many rules :

\begin{enumerate}
\item the kernel must be isolated is a separate file. For many kernel,
  each kernel must be in a separate file;
\item the kernel file must contain a kernel\_wrapper and the kernel itself;
\item the kernel\_wrapper is the function that is called from the main
  file. It must be declared from there (prototype);
\item the name of the file containing the kernel function and the
  kernel\_wrapper must be equal to the kernel\_wrapper name;
\item absolutely avoid the name <<kernel>> for the kernel. This a key
  word for some languages;
\item the \texttt{p4a\_accel\_wrapper.h} file must be included in each
  kernel file;
\item apart from \texttt{p4a\_accel\_wrapper.h}, avoid inclusion of files
  in the kernel file. That means that internal struct, typedef or
  specific macros must be explicitly replicated.
\end{enumerate}

\begin{description}
\item[\texttt{P4A\_wrapper\_proto(kernel, ...)}] To automatically
  declare the prototype of a kernel\_wrapper. Must be placed at the
  head of the main file.

  \begin{description}
  \item[kernel] The name of the kernel wrapper to be called from the main file;
  \item[...] List of the arguments to the kernel wrapper.
  \end{description}

\item[\texttt{P4A\_call\_accel\_kernel\_1d(kernel,
    P4A\_n\_iter\_0,...)}] Call a kernel in a 1-dimension parallel
  loop.

  \begin{description}
  \item[kernel] The name of the kernel wrapper called from the main file;
  \item[P4A\_n\_iter\_0] Maximum  value of the iteration indice in the loop;
  \item[...] List of the arguments to the kernel wrapper.
  \end{description}

\item[\texttt{P4A\_call\_accel\_kernel\_2d(kernel,
    P4A\_n\_iter\_0,P4A\_n\_iter\_1,...)}] Call a kernel in a
  2-dimension parallel loop.

  \begin{description}
  \item[kernel] The name of the kernel wrapper called from the main file;
  \item[P4A\_n\_iter\_0] Maximum value of the external iteration
    indice in the loop;
  \item[P4A\_n\_iter\_1] Maximum value of the internal iteration
    indice in the loop;
  \item[...] List of the arguments to the kernel wrapper.
  \end{description}

\end{description}

Appart from the prototype and call macros, \Apfa defines some general
key words for functions and arguments types :

\begin{description}
\item[P4A\_accel\_kernel] A declaration attribute for the
  hardware-accelerated kernel. The type must be specified in addition;
\item[P4A\_accel\_kernel\_wrapper] A declaration attribute for the
  hardware-accelerated kernel wrapper. As the type must be void, it is
  included in the macro.
\item[P4A\_accel\_constant\_address] Some constant variable, even in
  the argument list;
\item[P4A\_accel\_global\_address] Address of a variable placed in the global
  memory pool in the hardware-accelerated;
\item[P4A\_accel\_local\_address] Address of a variable placed in the local
  memory pool in the hardware-accelerated;
\item[P4A\_vp\_0] The local external indice value in the loop;
\item[P4A\_vp\_1] The local first internal indice value in the loop;
\item[P4A\_vp\_2] The local second internal indice value in the loop.
\end{description}

\subsection{The timer functions}

The timer functions is composed of the calls :

\begin{description}
\item[\texttt{P4A\_accel\_timer\_start}] To start the timer;
\item[\texttt{P4A\_accel\_timer\_stop\_and\_float\_measure()}] To stop
  the timer. Returns the elapsed time since the last
  P4A\_accel\_timer\_start call (in seconds);
\end{description}


\section{Example}
\lstset{language=C,stepnumber=1,rulesepcolor=\color{blue}}

\subsection{Original version}
The example is based on the Jacobi calculated over an
image.  Appart from the read and write of the results, the original
loop, written in C, looks like :

\begin{lstlisting}
typedef float float_t;
#define SIZE 501
#define T 400

float_t space[SIZE][SIZE];
float_t save[SIZE][SIZE];

void compute() {
  int i, j;

  /* Use 2 array in flip-flop to have dataparallel forall semantics. I
     could use also a flip-flop dimension instead... */
  for(i = 1;i < SIZE - 1; i++)
    for(j = 1;j < SIZE - 1; j++) {
      save[i][j] = 0.25*(space[i - 1][j] + space[i + 1][j]
			  + space[i][j - 1] + space[i][j + 1]);
    }
  for(i = 1;i < SIZE - 1; i++)
    for(j = 1;j < SIZE - 1; j++) {
      space[i][j] = 0.25*(save[i - 1][j] + save[i + 1][j]
			  + save[i][j - 1] + save[i][j + 1]);
    }
}


int main(int argc, char *argv[]) {
  
  for(t = 0; t < T; t++) {
    compute();
  }
}
\end{lstlisting}

\subsection{P4A transformations}

\subsubsection{The begin and End actions}
\begin{lstlisting}
int main(int argc, char *argv[]) {
  P4A_init_accel;
  for(t = 0; t < T; t++) {
    compute();
  }
  P4A_release_accel;
}

\end{lstlisting}

\subsubsection{Memory allocation and copy with timer}
\begin{lstlisting}
int main(int argc, char *argv[]) {
  P4A_init_accel;

  P4A_accel_timer_start;
  float_t (*p4a_var_space)[SIZE][SIZE];
  P4A_accel_malloc((void **) &p4a_var_space, sizeof(space));
  P4A_copy_to_accel(sizeof(space), space, p4a_var_space);

  float_t (*p4a_var_save)[SIZE][SIZE];
  P4A_accel_malloc((void **) &p4a_var_save, sizeof(save));
  P4A_copy_to_accel(sizeof(space), save, p4a_var_save);
  double copy_time = P4A_accel_timer_stop_and_float_measure();

  for(t = 0; t < T; t++) {
    compute();
  }

  P4A_accel_timer_start;
  P4A_copy_from_accel((size_t)sizeof(space), (void *)space, (void *)p4a_var_space);
  copy_time += P4A_accel_timer_stop_and_float_measure();
  fprintf(stderr, "Time for copy : %f s\n", copy_time);

  P4A_accel_free(p4a_var_space);
  P4A_accel_free(p4a_var_save);
  P4A_release_accel;
}

\end{lstlisting}

\subsubsection{Loop transformation}

There are two kernels. Two files must be created, one for each kernel.


The file for the first kernel must be called
\texttt{kernel1\_wrapper.c}, in correspondence with the name given to
the kernel1 wrapper function :

\begin{lstlisting}
#include "p4a_accel_wrapper.h"
typedef float float_t;
#define SIZE 501


P4A_accel_kernel void kernel1(P4A_accel_global_address float_t space[SIZE][SIZE],P4A_accel_global_address float_t save[SIZE][SIZE],int i, int j) 
{ 
  save[i][j] = 0.25*(space[i-1][j]+space[i+1][j]+space[i][j-1]+space[i][j+1]);
}

P4A_accel_kernel_wrapper kernel1_wrapper(P4A_accel_global_address float_t space[SIZE][SIZE], P4A_accel_global_address float_t save[SIZE][SIZE])
{
    
  int i = P4A_vp_0;
  int j = P4A_vp_1;
  if(i >= 1 && i <= SIZE - 2 && j >= 1 && j <= SIZE - 2)
    kernel1(space, save, i, j);
}
\end{lstlisting}

In the same way, the \texttt{kernel2\_wrapper.c} file contains :

\begin{lstlisting}
#include "p4a_accel_wrapper.h"
typedef float float_t;
#define SIZE 501

P4A_accel_kernel void kernel2(P4A_accel_global_address float_t space[SIZE][SIZE], P4A_accel_global_address float_t save[SIZE][SIZE], int i, int j)
{
   space[i][j] = 0.25*(save[i-1][j]+save[i+1][j]+save[i][j-1]+save[i][j+1]);
}

P4A_accel_kernel_wrapper kernel2_wrapper(P4A_accel_global_address float_t space[SIZE][SIZE], P4A_accel_global_address float_t save[SIZE][SIZE])
{
  int j = P4A_vp_1;
  int i = P4A_vp_0;

  if (i >= 1 && i <= SIZE - 2 && j >= 1 && j <= SIZE - 2)
    kernel2(space, save, i, j);
}
\end{lstlisting}

The transformations in the main are i) adding the kernels wrapper prototypes, ii) call to the kernels wrappers via \texttt{P4A\_call\_accel\_kernel\_2d()}.


\begin{lstlisting}
#include <p4a_accel.h>
#include <stdio.h>
#include <stdlib.h>
typedef float float_t;
#define SIZE 501
#define T 400

float_t space[SIZE][SIZE];
// For the dataparallel semantics:
float_t save[SIZE][SIZE];

// Prototypes for the two kernels of jacobi
P4A_wrapper_proto(kernel1_wrapper,P4A_accel_global_address float_t space[SIZE][SIZE],P4A_accel_global_address float_t save[SIZE][SIZE]);
P4A_wrapper_proto(kernel2_wrapper,P4A_accel_global_address float_t space[SIZE][SIZE],P4A_accel_global_address float_t save[SIZE][SIZE]);

void compute(float_t space[SIZE][SIZE], float_t save[SIZE][SIZE]) {
  int i, j;

  /* Use 2 array in flip-flop to have dataparallel forall semantics. I
     could use also a flip-flop dimension instead... */
 P4A_call_accel_kernel_2d(kernel1_wrapper, SIZE, SIZE, space, save);
 P4A_call_accel_kernel_2d(kernel2_wrapper, SIZE, SIZE, space, save);
}

int main(int argc, char *argv[]) {
  ...

  P4A_accel_timer_start;
  for(t = 0; t < T; t++)
    compute(*p4a_var_space, *p4a_var_save);
  double execution_time = P4A_accel_timer_stop_and_float_measure();
  fprintf(stderr, "Execution time : %f s\n", execution_time);

  ...
}

\end{lstlisting}

\subsubsection{To compile}

A specific \texttt{p4a\_accel.mk} file has been created to help
compiling. The Makefile created from the user must contain the
definition of \texttt{TARGET, CFILES, KERNELFILES} and
\texttt{P4A\_ACCEL\_DIR}.  \texttt{DUMMYFILE} is added to the make
clean and can be used to define some output, non permanent, files.
Additional compilation options are added in the \texttt{BASEFLAGS}
field. Typically, two profiling options are proposed

\begin{enumerate}
\item \texttt{P4A\_PROFILING} enables the user to place timer makers everywhere he wants as described in the upper part ;
\item \texttt{P4A\_TIMING} will display automated P4A internal timer.
\end{enumerate}

\begin{lstlisting}
TARGET = jacobi
CFILES = jacobi-p4a.c
KERNELFILES = kernel1_wrapper.c kernel2_wrapper.c
DUMMYFILE = output.pgm

P4A_ACCEL_DIR = ../../..
include $(P4A_ACCEL_DIR)/p4a_accel.mk
\end{lstlisting}

The command \texttt{make} will make the seq, openmp, cuda and opencl binaries.
But \texttt{make seq}, \texttt{make openmp}, \texttt{make cuda} and \texttt{make opencl} are working as well.

\end{document}
