\documentclass[a4paper]{article}
\usepackage[latin9]{inputenc}
\usepackage{url}
\usepackage{abbrev_reactive}
\let\OldRightarrow=\Rightarrow
\RequirePackage{marvosym}
\let\MarvosymRightarrow=\Rightarrow
\let\Rightarrow=\OldRightarrow
\RequirePackage{wasysym}
\let\Lightning\UnTrucIndefini% Car conflit entre ifsym et marvosym
\let\Sun\UnTrucIndefini%
\RequirePackage[weather]{ifsym}


\sloppy

\begin{document}

\title{Par4All organization\\
  \textsc{draft}\\
  ---\\
  HPC Project}

\author{Serge \textsc{Guelton} \and Ronan \textsc{Keryell} \and Gr�goire
  \textsc{P�an} \and Claire \textsc{Seguin} \and Micka�l \textsc{Thievent}
  \and Pierre \textsc{Villalon}}

\maketitle

\section{Introduction}
\label{sec:introduction}

\Apfa is a platform that merges various open source developments
to aim at achieving the migration of software to multicore and other
parallel processors.

It is mainly developed by \Ahpcp, MINES ParisTech/CRI, Institut
T�l�com/T�l�com Bretagne and others.

This document describes the internal organization of \Apfa and how its
construction relies on \Agit repositories, \Asvn repositories and other
projects.

Since it relies on other tool projects, the documentation of these other
projects should also be taken into account.

Before using PIPS, once it is installed, you should do, according to your
shell, a
\begin{verbatim}
source build/etc/par4all-rc.sh
\end{verbatim}
or a
\begin{verbatim}
source build/etc/par4all-rc.csh
\end{verbatim}


\section{Collaborative repositories}
\label{sec:coll-repos}

\subsection{Public repositories}
\label{sec:public-repositories}

There are few \Agit repositories used by the project.

To have access without authentication and only for reading/cloning, you
can use the \texttt{git:} prefix instead of \texttt{ssh:}.

The main repository for the project is:
\url{ssh://git.hpc-project.com/git/par4all.git}

It can be seen with a \Awww browser at
\url{https://git.hpc-project.com/cgit/par4all}

To get directly involved into the project with full commit right in the
repositories, ask \Ahpcp.

There are also ancillary \Agit repositories to offer a \Agit interface to
the trunk of the \Asvn repositories for the \Apips components from \Acri:
\begin{itemize}
\item \url{ssh://git.hpc-project.com/git/svn-linear.git}
\item \url{ssh://git.hpc-project.com/git/svn-newgen.git}
\item \url{ssh://git.hpc-project.com/git/svn-nlpmake.git} is a top-level
  repository with \texttt{trunk}, \texttt{branch} and \texttt{tag} because
  it started without this standard layout that was added later around
  revision 750;
\item \url{ssh://git.hpc-project.com/git/svn-pips.git}
\item \url{ssh://git.hpc-project.com/git/svn-validation.git}
\end{itemize}
These ancillary gateways only include the \texttt{trunk} history since at
CRI branches are not public.


\subsection{Private repositories}
\label{sec:private-repositories}

There is a private directory shared between core developers used mainly
for validation of the project on non public codes, benchmarks, demos, for
developing private reports, phases, scripts and so on:
\url{ssh://git.hpc-project.com/git/par4all-private.git}

For \Ahpcp-confidential information,
\url{ssh://git.hpc-project.com/git/par4all-private-hpc.git} is used.

According to the evolving private collaborations, other repositories can
be created and used on demand.


\section{Packages}
\label{sec:packages}

\Apfa integrate different tools from different projects. Right now \Apfa
is composed by \Apips, \Apipsgfc, \Apolylib, with some extensions. Each
project is included in \Apfa as a package and is placed in a directory
inside the \texttt{package} top-level directory.


\section{Directory organization}
\label{sec:direct-organ}

\begin{description}
\item[\texttt{build}] is the directory created with the various productions
  of the compilation of all the \Apfa packages: binaries, header files,
  documentation;
  \marginpar{Right now it is still mainly in \texttt{package/PIPS} since
    I'm waiting for the new \Apips compilation infrastructure. RK}
  \begin{description}
  \item[\texttt{bin}] contains the executable programs from \Apfa;
  \item[\texttt{doc}] is the generated documentation for the \Apfa
    infrastructure;
  \item[\texttt{etc}] contains some generated configuration files;
  \item[\texttt{include}] contains the include files used for the
    compilation of \Apfa;
  \item[\texttt{lib}] owns the libraries used to run \Apfa;
  \item[\texttt{share}] contains shared files for run-time and
    configuration files

    \marginpar{what is the difference of this \texttt{pipsrc.sh} and the one in
      \texttt{etc}? \texttt{RK->FC}}

  \item[\texttt{utils}] gathers various utility programs used by \Apfa
    infrastructure (validation...) and other internal tools;
  \end{description}
\item[\texttt{doc}] are the sources of the \Apfa infrastructure;
\item[\texttt{examples}] comes with some examples to exercise \Apfa;
\item[\texttt{packages}] contains the different components of \Apfa; of
  all the \Apfa packages: binaries, header files, documentation;
  \begin{description}
  \item[\texttt{PIPS}] contains the components of \Apips framework itself
    with
    \begin{description}
    \item[\texttt{linear}] is the \Apips linear library of \Apips;
    \item[\texttt{newgen}] is the object management infrastructure used by
      \Apips;
    \item[\texttt{nlpmake}] is the makefile common infrastructure used by
      all the \Apips components
    \item[\texttt{pips}] is the \Apips core;
    \item[\texttt{validation}] is the validation of \Apips
    \end{description}
  \item[\texttt{pips-gfc}] contains a \Agcc 4.4 source patched to be
    compiled and linked with \Apips to add a Fortran 95+ parser to \Apips;
  \item[\texttt{polylib}] contains the \Apolylib library source
  \end{description}
\item[\texttt{src}] contains source of tools used for the internal
  organization of \Apfa itself, such as repository and product management,
  publication process.
\end{description}


\section{Repositories and work-flow}
\label{sec:repos-workfl}

For history tracking and collaborative development, \Apfa relies on a main
\Agit repository that is accessed by \Apfa developers, users, integrators
and the product and quality team.

Since \Apfa intimately extends some tools such as \Apips, their are some
ancillary repositories to ease the impedance matching between \Apfa and
those other project.

Since \Apolylib is already a \Agit repository, it is simply fetched as a
remote \Agit into the \Apfa at the right place.

But this is far more complex with \Apips which is a project split in 5
independent \Asvn repositories, so its organization needs more massage to
be presented in a coherent way, with some \Asvn-\Agit gateways. We can
take advantage of these gateways to offer more free developments with
light branches stored into a common \Agit view of the \Apips{} \Asvn
repositories that can be pushed back into the \Apips{} \Asvn .


\subsection{PolyLib work-flow}
\label{sec:polylib-workflow}

The basic workflow for \Apolylib is to develop new features into the
original \Apolylib{} \Agit, to fetch it into the \Apfa{} \Agit and select what
feature we want to have into the \texttt{packages/polylib} of \Apfa.

Since the \Apolylib is already in a \Agit, the \texttt{polylib} is simply
a remote in the \Apfa{} \Agit.

So to import the latest \Apolylib development into the \Apfa{} \Agit for
inspection, choice for inclusion, you fetch \Apolylib with:
\begin{verbatim}
git fetch polylib
\end{verbatim}
This command is also done by a \verb|p4a_fetch_all| that fetches also the
\Apips parts.

You can then merge the feature you want with
\begin{verbatim}
git merge -s subtree remotes/polylib/master
\end{verbatim}
or from any tree
identifier. The \texttt{-s subtree} is necessary since in \Apfa the
\Apolylib files are not at the top-level directory.


\subsection{PIPS work-flow}
\label{sec:pips-workflow}

The work-flow related to \Apips is quite more complex since we must
consolidate data from 5 different \Asvn repositories.

The basic \Apips work-flow is to develop into the 5 original \Apips{}
\Asvn repositories at MINES ParisTech/CRI and to import the selected
developments into the \Apfa{} \Agit.

For this, we use 5 \Agit repositories that are gateways with these \Asvn
repositories. For technical reasons, it is better that this kind of subtle
gateway exists in only one place and this is currently done on the laptop
of Ronan \textsc{Keryell}. Then these 5 gateways are used as remotes into
the \Apfa{} \Agit. To synchronize these gateways to the latest version of
the \Apips{} \Asvn repositories, use a \verb|pips_git| in the directory
owning these \Agit-\Asvn repositories.

But since we have then a \Agit interface of the original \Apips{} \Asvn
repositories and \Agit is quite more powerful than \Asvn, some people may
want to develop into \Apips with \Agit by using these gateways. For
example they may want to develop with common branches easily and when they
are right with them push them back into the \Apips{} \Asvn trunk. To have
a public interface for them, these \Agit gateways are pushed to public
\Agit repository programmers can play with. Regularly these public
gateways are synchronized manually with the \Apips{} \Asvn. Nevertheless,
\Asvn may not be credited with the right owner of the commit but with the
one running the gateway, which is not fair. So the preferred way is to
directly develop in the original \Apips \Asvn repository (eventually with
her own \Agit-\Asvn gateway).

The 5 public gateway \Agit repositories are also defined as 5 remotes into
the \Apfa{} \Agit:
\begin{description}
\item[\texttt{remotes/CRI/linear}]
\item[\texttt{remotes/CRI/newgen}]
\item[\texttt{remotes/CRI/nlpmake}]
\item[\texttt{remotes/CRI/pips}]
\item[\texttt{remotes/CRI/validation}]
\end{description}

These remote repositories are merged into \Apfa in the following
respective directories:
\begin{description}
\item[\texttt{packages/PIPS/linear}]
\item[\texttt{packages/PIPS/newgen}]
\item[\texttt{packages/PIPS/nlpmake}]
\item[\texttt{packages/PIPS/pips}]
\item[\texttt{packages/PIPS/validation}]
\end{description}

To import the latest \Apips development into the \Apfa{} \Agit for
inspection, choice for inclusion, you fetch the repositories you want
with:
\begin{verbatim}
git fetch CRI/linear
git fetch CRI/newgen
git fetch CRI/nlpmake
git fetch CRI/pips
git fetch CRI/validation
\end{verbatim}
These commands are also done by a \verb|p4a_fetch_all| that fetches also the
\Apolylib part.

You can then merge the feature you want with a \texttt{git merge -s
  subtree} from \texttt{remotes/CRI/.../master} as:
\begin{verbatim}
git merge -strategy=subtree remotes/CRI/linear/master
git merge -strategy=subtree remotes/CRI/newgen/master
git merge -strategy=subtree remotes/CRI/nlpmake/master
git merge -strategy=subtree remotes/CRI/pips/master
git merge -strategy=subtree remotes/CRI/validation/master
\end{verbatim}
or from any tree identifier to do more precise version selection. The
\texttt{-s subtree} is necessary since in \Apfa the \Apips files are not
at the top-level directory and you do not want them to appear at the
top-level directory.

To pull everything at one for testing, this is done with the
\verb|p4a_pull_all|. It may useful to create a branch with a \texttt{git
  checkout -b} before to test it, so the branch can be deleted for an easy
rollback.


\subsection{PIPS-GFC extension workflow}
\label{sec:pips-gfc-workflow}

This part is into the \texttt{package/pips-gfc} directory. It contains
plain \Agcc core and Fortran 4.4.1 distribution, pointed by the branch
\texttt{gcc-4.4.1}

The development of \Apipsgfc should be done in the branch
\texttt{pips-gfc-4.4.1} and this branch should be merged with the
branch \texttt{gcc-4.4.1} into a branch \texttt{pips-gfc+gcc-4.4.1} with a
more global name \texttt{pips-gfc+gcc} tracking which version is to be
merged into the global \texttt{master} branch.

So in \texttt{pips-gfc-4.4.1} there should be only files different from
the \Agcc distribution. In this way, if we want to have more subtle
construction methods, later, it will be clearer how to get the real
content.

There is the same variation in branches for version 4.4.2.

So to develop and test the \Apipsgfc extension, you get into for example
\texttt{pips-gfc-4.4.1} with
\begin{verbatim}
git checkout pips-gfc-4.4.1
\end{verbatim}
and develop your code in this branch.

To test, you commit and change to the \texttt{pips-gfc+gcc-4.4.1} branch with
\begin{verbatim}
git checkout pips-gfc+gcc-4.4.1
\end{verbatim}
where you merge with a
\begin{verbatim}
git merge pips-gfc-4.4.1
\end{verbatim}
and compile. If you are happy, you commit or you revert and then go back
into branch \texttt{pips-gfc-4.4.1}.

If you want to avoid spoiling the branches \texttt{pips-gfc-4.4.1} and
\texttt{pips-gfc+gcc-4.4.1}, you can create sub-branches of them, commit in
them and merge this work in the former one back (with the \verb|--slashed|
if you want to be modest about your gory hesitations \smiley{} and delete
these branches).


\subsection{Setting and running the infrastructure up}
\label{sec:setup}

The installation has been tested on Ubuntu/Linux 9.10 and you need to have
at least these packages installed:
MICKAEL !


There are scripts to ease the developer and user life.


\subsubsection{Scripts for an everyday work}
\label{sec:an-everyday-work}


\begin{itemize}
\item to pull everything at one for testing, this is done with the
  \verb|p4a_pull_all|. It may useful to create a branch with a \texttt{git
    checkout -b} before to test it, so the branch can be deleted for an
  easy rollback;
\item \verb|p4a_setup| is used to compile all the \Apfa infrastructure and
  setting up many things. It should be used at least for the first
  compilation.
\end{itemize}


\subsubsection{Scripts used to setup the infrastructure}
\label{sec:scripts-used-setup}


Some scripts used to setup the infrastructure are located in
\texttt{src/dev}, that should be used in this order:
\begin{itemize}
\item \verb/p4a_create_CRI_git_svn/ is used once to create the \Apips
  \Asvn-\Agit gateways. This script is here to show how it can be done but
  also to keep track of the exact parameters used to create them in the
  case we loose them and want to recreate them with exactly with the same
  identifiers (that depends from the creation parameters);
\item \verb/p4a_import_external_gits/ imports all the external \Agit
  repositories into the \Apfa \Agit repository. It should be used only
  once but are included as example for other projects or to help adding
  other repositories later;
\item \verb/p4a_apply_pips_patches/ is used to patch the original \Apips
  files to fit the \Apfa architecture. It should be used only once, after
  external \Agit import;
\end{itemize}


\subsubsection{Scripts for debugging}
\label{sec:scripts-debugging}

\begin{itemize}
\item \verb|p4a_valgrind| launch a command with Valgrind with a memory
  checker in paranoid mode, mainly with the options described in the
  \Apips development guide.
\end{itemize}


\section{Examples and demos}
\label{sec:examples-demos}

The \texttt{examples} directory comes with examples to show some aspects
of \Apfa. Of course there are far more examples in the validation of \Apfa
(\S~\ref{sec:validation}) but it is often less pedagogical.

\section{Compilation}
\label{sec:compilation}

Once you have an image of the main \Agit repository content, \Apfa is
compiled and configured by running \verb|src/dev/p4a_setup|.

To be sure you do not break the compilation process, you should compile
your work before committing it on the central repositories. By a nice side
effect, it allows you to test your code. \smiley

More seriously, it is not obvious that it may compile on your account but
not for other people. For example because you forgot to commit some files.

A nice feature of \Agit over \Asvn is that since you split the commit from
the publication, you can test your own committed state before it hurts the
team.

For example, you can create a light\footnote{Because the objects are
  shared with symbolic links and not copied since we did not use the
  \texttt{file://} syntax.} clone with a
\begin{verbatim}
git clone par4all par4all-compile
\end{verbatim}
and after you have tested and committed your modifications inside the
\texttt{par4all} working copy, you do the same into the
\texttt{par4all-compile} working copy after a \texttt{git pull}.
If some files are lacking from the commit, it will be detected.

Afterwards you can do a \texttt{git push} into the central \Apfa
repository with less risks.


\section{Validation}
\label{sec:validation}


Validation of \Apfa is done inside the \texttt{validation} directory with
\texttt{make validate}.

But right now it is still mostly done in the
\texttt{package/PIPS/validation} directory.

To ease interacting with the validation, the \verb|p4a_validate| is a
script that adds the concept of validation classes to the \Apips
validation. A class is a set of validation cases. For example we may have
a class for \texttt{ALL} the validation, the \texttt{CHANGED},
\texttt{FAILED}, \texttt{PASS}, \verb|PREVIOUS_ALL|,
\verb|PREVIOUS_CHANGED|, \verb|PREVIOUS_FAILED|, \verb|PREVIOUS_PASS|,
whatever user class and it is possible to make actions with them, combine
them with operators (unions, intersections...).

Some other classes can be defined directly in the validation directories
with \texttt{.vclass} line-oriented regexp filter lines or
generic Python code \texttt{.vclasspy}.

\verb|p4a_validate| has a small script interface, but the power user
should use the Python classes directly, for example from \texttt{iPython}
to add interactivity.

The first use in \Apfa is to select from \Apips only the test cases that
pass the validation (futuristic cases are interesting for \Apips but from
least interest for \Apfa...), defining smaller validation classes such as
a \Amat (Minimal Acceptance Test).


\section{Branches}
\label{sec:branches}

Since there are restrictions on the use of \texttt{/} in branch names, we
prefer to use \texttt{-} to add hierarchy.

To ease the developments and the organization, there are some already
defined branches:
\begin{description}
\item[\texttt{gcc-4.4.1}] is the original \Agcc 4.4.1 core \& Fortran in
  \texttt{package/pips-gfc};
\item[\texttt{gcc-4.4.2}] is the original \Agcc 4.4.2 core \& Fortran in
  \texttt{package/pips-gfc};
\item[\texttt{linear}] points to the import of \Apips \texttt{linear} part;
\item[\texttt{master}] is the branch for the main latest developments in
  \Apfa;
\item[\texttt{newgen}] points to the import of \Apips \texttt{newgen}
  part;
\item[\texttt{nlpmake}] points to the import of \Apips \texttt{nlpnake}
  part;
\item[\texttt{pips}] points to the import of \Apips \texttt{pips} part;
\item[\texttt{pips-gfc+gcc}] points to a working \Apipsgfc implementation
  of the Fortran 95 extension for \Apips in \texttt{package/pips-gfc};
\item[\texttt{pips-gfc-4.4.1}] points to the original developments of
  Rapha�l in \Agcc in \texttt{package/pips-gfc};
\item[\texttt{validation}] points to the import of \Apips
  \texttt{validation} part.
\end{description}


\section{Making releases}
\label{sec:releases}

After validating, a release is done by making some branch at a given state
a tag.


\end{document}


%%% Local Variables: 
%%% mode: latex
%%% ispell-local-dictionary: "american"
%%% End: 
