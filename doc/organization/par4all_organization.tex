\documentclass[a4paper]{article}
\usepackage[latin9]{inputenc}
\usepackage{url}
\usepackage{abbrev_reactive}

\sloppy

\begin{document}

\title{Par4All organization\\
  draft
}

\author{Ronan Keryell, HPC Project}

\maketitle

\section{Introduction}
\label{sec:introduction}

\Apfa is a platform that merges various open source developments
to aim at achieving the migration of software to multicore and other
parallel processors.

It is mainly developed by \Ahpcp, MINES ParisTech/CRI, Institut
T�l�com/T�l�com Bretagne and others.

This document describes the internal organization of \Apfa and how its
construction relies on \Agit repositories, \Asvn repositories and other
projects.

Since it relies on other tool projects, the documentation of these other
projects should also be taken into account.


\subsection{Collaborative repositories}
\label{sec:coll-repos}

There are few \Agit repositories used by the project.

The main repository for the project is:
\url{ssh://git.hpc-project.com/git/par4all.git}

There are also ancillary \Agit repositories to offer a \Agit interface to
the trunk of the \Asvn repositories for the \Apips components from \Acri:
\begin{itemize}
\item \url{ssh://git.hpc-project.com/git/svn-linear.git}
\item \url{ssh://git.hpc-project.com/git/svn-newgen.git}
\item \url{ssh://git.hpc-project.com/git/svn-nlpmake.git} is a top-level
  repository with \texttt{trunk}, \texttt{branch} and \texttt{tag} because
  it started without this standard layout that was added later around
  revision 750;
\item \url{ssh://git.hpc-project.com/git/svn-pips.git}
\item \url{ssh://git.hpc-project.com/git/svn-validation.git}
\end{itemize}
These ancillary gateways only include \texttt{trunk} history since at CRI
branches are not public.

To have access without authentication and only for reading/cloning, you
can use the \texttt{git:} prefix instead of \texttt{ssh:}.


\subsection{Compilation}
\label{sec:compilation}

Once you have an image of the main \Agit repository content, \Apfa is
compiled and configured by running \verb|src/dev/p4a_setup|.


\section{Packages}
\label{sec:packages}

\Apfa integrate different tools from different projects. Right now \Apfa
is composed by \Apips and \Apolylib, with some extensions. Each project is
included in \Apfa as a package and is placed in a directory inside the
\texttt{package} top-level directory.


\section{Directory organization}
\label{sec:direct-organ}

\begin{description}
\item[\texttt{build}] is the directory created with the various productions
  of the compilation of all the \Apfa packages: binaries, header files,
  documentation;
  \begin{description}
  \item[\texttt{bin}] contains the executable programs from \Apfa;
  \item[\texttt{doc}] is the generated documentation for the \Apfa
    infrastructure;
  \item[\texttt{etc}] contains some generated configuration files;
  \item[\texttt{include}] contains the include files used for the
    compilation of \Apfa;
  \item[\texttt{lib}] owns the libraries used to run \Apfa;
  \item[\texttt{share}] contains shared files for run-time and
    configuration files

    \marginpar{what is the difference of this pipsrc.sh and the one in
      etc?}

  \item[\texttt{utils}] gathers various utility programs used by \Apfa
    infrastructure (validation...) and other internal tools;
  \end{description}
\item[\texttt{doc}] are the sources of the \Apfa infrastructure;
\item[\texttt{packages}] contains the different components of \Apfa; of
  all the \Apfa packages: binaries, header files, documentation;
  \begin{description}
  \item[\texttt{PIPS}] contains the components of \Apips framework itself
    with
    \begin{description}
    \item[\texttt{linear}] is the \Apips linear library of \Apips;
    \item[\texttt{newgen}] is the object management infrastructure used by
      \Apips;
    \item[\texttt{nlpmake}] is the makefile common infrastructure used by
      all the \Apips components
    \item[\texttt{pips}] is the \Apips core;
    \item[\texttt{validation}] is the validation of \Apips
    \end{description}
  \item[\texttt{polylib}] contains the \Apolylib library source
  \end{description}
\item[\texttt{src}] contains source of tools used for the internal
  organization of \Apfa itself, such as repository and product management,
  publication process.
\end{description}


\section{Repositories and work-flow}
\label{sec:repos-workfl}

For history tracking and collaborative development, \Apfa relies on a main
\Agit repository that is accessed by \Apfa developers, users, integrators
and the product and quality team.

Since \Apfa intimately extends some tools such as \Apips, their are some
ancillary repositories to ease the impedance matching between \Apfa and
those other project.

Since \Apolylib is already a \Agit repository, it is simply fetched as a
remote \Agit into the \Apfa at the right place.

But this is far more complex with \Apips which is a project split in 5
independent \Asvn repositories, so its organization needs more massage to
be presented in a coherent way, with some \Asvn-\Agit gateways. We can
take advantage of these gateways to offer more free developments with
light branches stored into a common \Agit view of the \Apips{} \Asvn
repositories that can be pushed back into the \Apips{} \Asvn .


\subsection{PolyLib work-flow}
\label{sec:polylib-workflow}

The basic workflow for \Apolylib is to develop new features into the
original \Apolylib{} \Agit, to fetch it into the \Apfa{} \Agit and select what
feature we want to have into the \texttt{packages/polylib} of \Apfa.

Since the \Apolylib is already in a \Agit, the \texttt{polylib} is simply
a remote in the \Apfa{} \Agit.

So to import the latest \Apolylib development into the \Apfa{} \Agit for
inspection, choice for inclusion, you fetch \Apolylib with:
\begin{verbatim}
git fetch polylib
\end{verbatim}
This command is also done by a \verb|p4a_fetch_all| that fetches also the
\Apips parts.

You can then merge the feature you want with
\begin{verbatim}
git merge -s subtree remotes/polylib/master
\end{verbatim}
or from any tree
identifier. The \texttt{-s subtree} is necessary since in \Apfa the
\Apolylib files are not at the top-level directory.


\subsection{PIPS work-flow}
\label{sec:pips-workflow}

The work-flow related to \Apips is quite more complex since we must
consolidate data from 5 different \Asvn repositories.

The basic \Apips work-flow is to develop into the 5 original \Apips{}
\Asvn repositories at MINES ParisTech/CRI and to import the selected
developments into the \Apfa{} \Agit.

For this, we use 5 \Agit repositories that are gateways with these \Asvn
repositories. For technical reasons, it is better that this kind of subtle
gateway exists in only one place and this is currently done on the laptop
of Ronan \textsc{Keryell}. Then these 5 gateways are used as remotes into
the \Apfa{} \Agit. To synchronize these gateways to the latest version of
the \Apips{} \Asvn repositories, use a \verb|pips_git| in the directory
owning these \Agit-\Asvn repositories.

But since we have then a \Agit interface of the original \Apips{} \Asvn
repositories and \Agit is quite more powerful than \Asvn, some people may
want to develop into \Apips with \Agit by using these gateways. For
example they may want to develop with common branches easily and when they
are right with them push them back into the \Apips{} \Asvn trunk. To have
a public interface for them, these \Agit gateways are pushed to public
\Agit repository programmers can play with. Regularly these public
gateways are synchronized manually with the \Apips{} \Asvn. Nevertheless,
\Asvn may not be credited with the right owner of the commit but with the
one running the gateway, which is not fair. So the preferred way is to
directly develop in the original \Apips \Asvn repository (eventually with
her own \Agit-\Asvn gateway).

The 5 public gateway \Agit repositories are also defined as 5 remotes into
the \Apfa{} \Agit:
\begin{description}
\item[\texttt{remotes/CRI/linear}]
\item[\texttt{remotes/CRI/newgen}]
\item[\texttt{remotes/CRI/nlpmake}]
\item[\texttt{remotes/CRI/pips}]
\item[\texttt{remotes/CRI/validation}]
\end{description}

These remote repositories are merged into \Apfa in the following
respective directories:
\begin{description}
\item[\texttt{packages/PIPS/linear}]
\item[\texttt{packages/PIPS/newgen}]
\item[\texttt{packages/PIPS/nlpmake}]
\item[\texttt{packages/PIPS/pips}]
\item[\texttt{packages/PIPS/validation}]
\end{description}

To import the latest \Apips development into the \Apfa{} \Agit for
inspection, choice for inclusion, you fetch the repositories you want
with:
\begin{verbatim}
git fetch CRI/linear
git fetch CRI/newgen
git fetch CRI/nlpmake
git fetch CRI/pips
git fetch CRI/validation
\end{verbatim}
These commands are also done by a \verb|p4a_fetch_all| that fetches also the
\Apolylib part.

You can then merge the feature you want with a \texttt{git merge -s
  subtree} from \texttt{remotes/CRI/.../master} as:
\begin{verbatim}
git merge -strategy=subtree remotes/CRI/linear/master
git merge -strategy=subtree remotes/CRI/newgen/master
git merge -strategy=subtree remotes/CRI/nlpmake/master
git merge -strategy=subtree remotes/CRI/pips/master
git merge -strategy=subtree remotes/CRI/validation/master
\end{verbatim}
or from any tree identifier to do more precise version selection. The
\texttt{-s subtree} is necessary since in \Apfa the \Apips files are not
at the top-level directory and you do not want them to appear at the
top-level directory.

To pull everything at one for testing, this is done with the
\verb|p4a_pull_all|. It may useful to create a branch with a \texttt{git
  checkout -b} before to test it, so the branch can be deleted for an easy
rollback.


\subsection{Setting the infrastructure up}
\label{sec:setup}

Some scripts used to setup the infrastructure are located in
\texttt{src/dev}, that should be used in this order:
\begin{itemize}
\item \verb/p4a_create_CRI_git_svn/ is used once to create the \Apips
  \Asvn-\Agit gateways. This script is here to show how it can be done but
  also to keep track of the exact parameters used to create them in the
  case we loose them and want to recreate them with exactly with the same
  identifiers (that depends from the creation parameters);
\item \verb/p4a_import_external_gits/ imports all the external \Agit
  repositories into the \Apfa \Agit repository. It should be used only
  once but are included as example for other projects or to help adding
  other repositories later;
\item \verb/p4a_apply_pips_patches/ is used to patch the original \Apips
  files to fit the \Apfa architecture. It should be used only once, after
  external \Agit import;
\item \verb|src/dev/p4a_setup| is used to compile all the \Apfa
  infrastructure and setting up many things. It should be used at least
  for the first compilation.
\end{itemize}


\section{Validation}
\label{sec:validation}


Validation of \Apfa is done inside the \texttt{validation} directory with
\texttt{make}.


\section{Making releases}
\label{sec:releases}

After validating, a release is done by making some branch at a given state
a tag.


\end{document}


%%% Local Variables: 
%%% mode: latex
%%% ispell-local-dictionary: "american"
%%% End: 
