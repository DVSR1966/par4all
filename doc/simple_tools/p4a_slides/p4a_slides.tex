%% To deal with the colors and number of slides per page according to the
%% \VersionPapier switch :
\ifx\VersionExpose\UnTrucInexistant%
% Version �l�ve
\documentclass[handout,compress,10pt,hyperref={hyperindex},xcolor={svgnames,x11names}]{beamer}
\usepackage{pgfpages}
\pgfpagesuselayout{4 on 1}[a4paper,landscape,border shrink=5mm]
\else
% Version enseignant
\documentclass[compress,10pt,hyperref={hyperindex},xcolor={svgnames,x11names}]{beamer}
\fi
%\documentclass[11pt,a4]{powersem}
\usepackage[latin9]{inputenc}
\usepackage[T1]{fontenc}
\usepackage{amsmath,wasysym,marvosym,soul}
\usepackage[hyper,procnames]{listings}
\usepackage{alltt}%verbatim,
%\usepackage[linkbordercolor={1 1 1},bookmarks=true,colorlinks=false,urlbordercolor={1 1 1},runbordercolor={1 1 1},citebordercolor={1 1 1},bookmarksnumbered=true,hypertexnames=false,hyperindex]{hyperref}
%usepackage{tabularx}

%\usepackage{beamer_rk}
\usepackage[LeftSideBar,ShadedBackground]{beamer_rk}
%\usepackage[WhiteBackground]{beamer_rk}


\usepackage{perso2e,bbold2e,getVersion,exemple}
%\usepackage{beamer_rk,perso2e,bbold2e,LienPDF,example}
\usepackage{makeidx}
%\usepackage{multicol,tabularx}
\usepackage{pst-node}

\usepackage[greek,francais,english]{babel}
\frenchbsetup{og=�, fg=�, ThinColonSpace=true,
  % ShowOptions,
  % �vite de casser les r�f�rences bibliographiques du style harvard avec
  % des ":" dedans :
  %AutoSpacePunctuation=false
}
% S'arrange pour avoir une virgule qui ne rajoute pas de blanc dans les
% nombres � virgule :
%\iffalse
\DecimalMathComma
% M�me pas besoin pour les ligatures � 2 accents en fait :
%\languageattribute{greek}{polutoniko}

% Et pour avoir \nombre qui marche :
\usepackage[autolanguage]{numprint}

\lstset{extendedchars=true, language=C++, basicstyle=\footnotesize\ttfamily, numbers=left,
  numberstyle=\tiny, stepnumber=2, numberfirstline=true,
  tabsize=8, tab=\rightarrowfill, keywordstyle=\color{orange}\bf,
  stringstyle=\rmfamily, commentstyle=\rmfamily\itshape}
% Un peu violent :
%  index=[1][keywords],indexprocnames=true%\lstloadlanguages{Basic}

%\usepackage{beamerthemesplit}
%{\usebeamercolor{palette primary}}
%\setbeamertemplate{sidebar canvas}[vertical shading][top=palette primary.bg,middle=white,bottom=palette primary.bg]

\makeindex

\getRCSDollarsVersion$Revision: 185 $

\usepackage[dvips]{preview}
\PreviewEnvironment{trans}
\PreviewEnvironment{frame}

\newcommand{\FigCentreTourne}[1]{\centerline{\includegraphics[width=\vsize,angle=-90]{#1}}}
\newcommand{\boiteblanc}[1]{\psframebox*[fillcolor=papayawhip]{#1}}

%\DealWithTwoUp

%\newcommand{\reddrawline}{{\red \noindent\leavevmode\hrulefill}}


\usetheme{gtc}
\usecolortheme{gtc}


\title[Par4All --- \texttt{p4a}]{Par4All\\
---\\
\texttt{p4a}\\
Simple command line interface}


\subtitle{\LienPDF{http://www.par4all.org}}

\author[\textsc{Keryell} \& \textsc{P�an}]{Ronan \textsc{Keryell} \&
  Gr�goire \textsc{P�an}}

\newcommand{\HPCP}[0]{\href{http://hpc-project.com}{HPC Project}}

\newcommand{\TB}[0]{\href{http://hpcas.enstb.org}{Institut T�L�COM/T�L�COM
    Bretagne/HPCAS}}

\newcommand{\CRI}[0]{\href{http://www.cri.mines-paristech.fr}{Mines ParisTech/CRI}}

\newcommand{\RPI}[0]{\href{http://www.cs.rpi.edu/}{Rensselaer Polytechnic Institute/CS}}

\institute[\HPCP]{\HPCP}


%\logo{\href{http://hpc-project.com}{\pgfuseimage{logo-HPC-Project}}\href{http://www.par4all.org}{\pgfuseimage{logo-Par4All}}}
\pgfdeclareimage[width=0.75cm]{logo-HPC-Project}{Logo_HPC-Project-512x247-crop}
\pgfdeclareimage[width=0.75cm]{logo-Wild-Systems}{WildSystems-site-logo}
\pgfdeclareimage[width=0.75cm]{logo-Par4All}{Par4All-logo}
\logo{\hbox to 0.75cm{\vbox{\href{http://hpc-project.com}{\pgfuseimage{logo-HPC-Project}}\\\href{http://wild-systems.com}{\pgfuseimage{logo-Wild-Systems}}\\\href{http://www.par4all.org}{\pgfuseimage{logo-Par4All}}\vss}}}
%\logo{\hbox to 0.75cm{\vbox{\href{http://wild-systems.com}{\pgfuseimage{logo-Wild-Systems}}\\\href{http://www.par4all.org}{\pgfuseimage{logo-Par4All}}\vss}}}
%\logo{\hbox to 0.75cm{\vbox{\pgfuseimage{logo-HPC-Project}\\\pgfuseimage{logo-telecom-bretagne}\\\pgfuseimage{logo-HPCAS}}\hss}}


%\subject{}


%\date{26/05/2010\\
%  \LienPDF{http://www.par4all.org}}

\newcommand{\Idee}{{\red\Handwash}\xspace}
%\Stopsign \Laserbeam \Estatically \Attention \Idee
%\FilledRainCloud\Coffeecup

\newcommand{\Cnn}{C$_{99}$\xspace}

%% Plan plus petit :
%\renewcommand{\OutlineFontSize}{\scriptsize}
% Plan en une colonne :
\renewcommand{\OutlineColumnNumbers}{1}
\renewcommand\outlinename{Outline}


\begin{document}

%\DealWithColors

\begin{frame}%[plain]
  \titlepage
\end{frame}

%\slidepagestyle{CRI}

%\LicenseTrans

%\AtBeginSection[]
%{
%}
%\AtBeginSubsection[]
%{}

\newcommand{\HPCIN}{%\protect\usebeamercolor{normal text}
  \textcolor{red}{\st{\textsc{hpc}}}plain computing\xspace}


\section*{Introduction}


\begin{frame}{Source-to-source compilation}
  \begin{itemize}
  \item Multicore (r)evolution
  \item Huge need of parallel programs
  \item Some tools are needed to leverage parallel programming
  \item Par4All based on PIPS source-to-source transformation framework
  \item Par4All aims at a usable distribution for PIPS
  \end{itemize}
  \vavers \texttt{p4a}: simple command line interface
\end{frame}


\begin{frame}{\texttt{p4a} capabilities}
  \begin{itemize}
  \item Automatic parallelization with advanced semantics analysis
  \item C and Fortran input language
  \item OpenMP code generation (C \& Fortran)
  \item Code generation for heterogeneous accelerators
    \begin{itemize}
    \item CUDA for GPU
    \end{itemize}
  \item Source-to-source: capitalize on real value \& surf on best
    back-ends
  \item Manage back-end invocation for simplicity
  \item Makefile generation
  \end{itemize}
\end{frame}


\begin{trans}{\texttt{p4a} in a nutshell}
  \begin{BoiteA}{Parallelisation}
\begin{verbatim}
p4a matmul.f
\end{verbatim}
    generates an OpenMP program in \texttt{matmul.p4a.f}
  \end{BoiteA}
  {\tiny
\begin{lstlisting}
!$omp parallel do private(I, K, X)
C multiply the two square matrices of ones
      DO J = 1, N                                                       0016
!$omp parallel do private(K, X)
         DO I = 1, N                                                    0017
            X = 0                                                       0018
!$omp parallel do reduction(+:X)
            DO K = 1, N                                                 0019
               X = X+A(I,K)*B(K,J)                                      0020
            ENDDO
!$omp end parallel do
            C(I,J) = X                                                  0022
         ENDDO
!$omp end parallel do
      ENDDO
!$omp end parallel do
\end{lstlisting}
  }

  \begin{BoiteB}{Parallelisation with compilation}
\begin{verbatim}
p4a matmul.f -o matmul
\end{verbatim}
    generates an OpenMP program \texttt{matmul.p4a.f} that is compiled
    with \texttt{gcc} into \texttt{matmul}
  \end{BoiteB}

  \begin{BoiteC}{CUDA generation with compilation}
\begin{verbatim}
p4a --cuda saxpy.c -o s
\end{verbatim}
    generates a CUDA program that is compiled with \texttt{nvcc}
  \end{BoiteC}
\end{trans}

\begin{frame}{Parallelization options}
\begin{alltt}
p4a [\emph{options}] \emph{files}
\end{alltt}
  \begin{itemize}
  \item \texttt{-{}-openmp} generates OpenMP output. This is the default
    behaviour
  \item \texttt{-{}-cuda} generates CUDA output
  \item \texttt{-{}-accel} \texttt{-{}-openmp} generates OpenMP output that
    simulates the CUDA behaviour
  \end{itemize}
\end{frame}


\input{p4a-help}


\begin{trans}{Current limitations for CUDA}
  \begin{itemize}
  \item Some C99 are not yet managed in kernels by \texttt{nvcc}
\begin{lstlisting}
__global__ fct(int n , int array[n]) {
  [...]
}
\end{lstlisting}
    But PIPS trusts \texttt{nvcc} \smiley{} and generates this code on
    some examples
  \item There are limits on the size of blocks of threads and grids of
    blocks but PIPS may generate some size bigger than what \texttt{nvcc}
    and its run-time can cope
  \item No code generation for shared memory
  \item No communication optimization
  \end{itemize}
\end{trans}


\begin{trans}{Do and don't}
  \begin{itemize}
  \item Program in clean C99
  \item Use array with dynamic sizes instead of \texttt{malloc()} spamming
  \item Do not think that if there are pointers in C you need to use
    them \smiley{} (same in Fortran95)
  \item Array linearization is often wicked. There are multidimensional
    arrays, so use them \smiley
  \item You can pass multidimensional arrays in C
\begin{lstlisting}
int f(size_t n, size_t m, double matrix[n][m]) {
}
\end{lstlisting}
    It was in Fortran 77 but only in C99... \smiley
  \end{itemize}
\end{trans}


\begin{frame}{Conclusion}
  \begin{itemize}
  \item Motto: keep things simple
  \item Easy way to begin with parallel programming
  \item Source-to-source
    \begin{itemize}
    \item Can be used to give some programming examples
    \item Good start that can be reworked upon
    \end{itemize}
  \item Open Source for community network effect
  \item For more finer control, possible to directly use \texttt{tpips}
    and (i)PyPS
  \end{itemize}
\end{frame}


\begin{frame}{Par4All is currently supported by...}
  \begin{itemize}
  \item HPC Project
  \item Mines ParisTech
  \item Institut T�L�COM/T�L�COM Bretagne
  \item Rensselaer Polytechnic Institute
  \item European ARTEMIS SCALOPES project
  \item European ARTEMIS SMECY project
  \item French NSF (ANR) FREIA project
  \item French NSF (ANR) MediaGPU project
  \item French Images and Networks research cluster TransMedi@ project
  \item French System@TIC research cluster OpenGPU project
    % \item nVidia
  \end{itemize}
\end{frame}


\section*{Table of content}

\begin{multicols}{2}
  \tiny
  \tableofcontents[frametitles]
  \textbf{You are here!}\hfill\expandafter\the\csname c@page\endcsname
\end{multicols}

\end{document}

%%% Local Variables:
%%% mode: latex
%%% ispell-local-dictionary: "american"
%%% End:
