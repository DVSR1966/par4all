\documentclass[a4paper]{article}
\usepackage[utf8]{inputenc}
\usepackage{url}
\usepackage{alltt}
\usepackage{../libs/tex/abbrev_reactive}

\let\OldRightarrow=\Rightarrow
\RequirePackage{marvosym}
\let\MarvosymRightarrow=\Rightarrow
\let\Rightarrow=\OldRightarrow
\RequirePackage{wasysym}
\let\Lightning\UnTrucIndefini% Car conflit entre ifsym et marvosym
\let\Sun\UnTrucIndefini%
\RequirePackage[weather]{ifsym}


\sloppy

\begin{document}

\title{Par4All script\\
  \textsc{draft}\\
  ---\\
  HPC Project}

\author{Ronan \textsc{Keryell}}

\maketitle

\tableofcontents{}

\section{Introduction}
\label{sec:introduction}

\texttt{p4a} is the basic script interface to quickly use \Apfa for people
not interested in all the \Apfa details.

This script can take C or Fortran source files and generate \Aopenmp or
\Acuda output to run respectively on shared memory multicore processors or
\Agpu.

The output is created in a \texttt{\emph{x}.database} directory.


\section{Examples}
\label{sec:examples}

You want to generate some \Aopenmp code:
\begin{verbatim}
p4a --openmp example.f
\end{verbatim}
and the output should be in a \texttt{\emph{random-name}.database/Src}
directory.

To generate a \Acuda executable:
\begin{verbatim}
p4a --accel --cuda --compile --database a example.c
\end{verbatim}
there should be a \texttt{example} binary executable file in
\texttt{a.database/P4A}

To generate an \Aopenmp emulation executable of \Agpu like code, try:
\begin{verbatim}
p4a --accel --openmp --compile example.c
\end{verbatim}
there should be a \texttt{example} binary executable file in
\texttt{\emph{random-name}.database/P4A}

To generate accelerated source code, use:
\begin{verbatim}
p4a --accel --database b example.c
\end{verbatim}
then you will have in \texttt{b.database/P4A} a parallelized source code
you can use later by compiling it or improving it. When compiling if you
define the
\begin{itemize}
\item \verb|P4A_ACCEL_CUDA| preprocessor symbol, the source is to be
  compiled as \Acuda;
\item \verb|P4A_ACCEL_OPENMP| preprocessor symbol, the source is to be
  compiled as \Aopenmp or sequential emulation code.
\end{itemize}


\section{Options}
\label{sec:options}

The basic usage is \texttt{p4a [\emph{options}] <\emph{source files}>}
with the following options:
\begin{description}
\item[\texttt{--openmp}:] parallelize with \Aopenmp output. If combined
  with the \texttt{--accel} option, generate \Apfa \texttt{accel} run-time
  call with \Aopenmp implementation instead of native \Aopenmp output;

\item[\texttt{--accel}:] parallelize with output using the \Apfa
  \texttt{accel} run-time that can execute code for various
  hardware accelerators such as \Agpu or even \Aopenmp emulation;

\item[\emph{--cuda}:] ask for \Acuda output, to be combined with
  \texttt{--accel};

\item[\texttt{--compile}:] call the compiler to generate an executable,
  \texttt{gcc} in the case of sequential or \Aopenmp code, \texttt{nvcc}
  in the case of \Acuda output code.

\item[\texttt{--database \emph{name}}:] choose to create a
  \texttt{\emph{name}.database} database instead of one with a random name;

\item[\texttt{-h} or \texttt{--help}:] show this help message and exit.
\end{description}


\end{document}


%%% Local Variables: 
%%% mode: latex
%%% ispell-local-dictionary: "american"
%%% End:

