%%
%% $Id: ri.tex 17253 2010-05-31 08:43:19Z creusillet $
%%
%% Copyright 1989-2010 MINES ParisTech
%% Copyright 2009-2010 HPC Project
%%
%% This file is part of PIPS.
%%
%% PIPS is free software: you can redistribute it and/or modify it
%% under the terms of the GNU General Public License as published by
%% the Free Software Foundation, either version 3 of the License, or
%% any later version.
%%
%% PIPS is distributed in the hope that it will be useful, but WITHOUT ANY
%% WARRANTY; without even the implied warranty of MERCHANTABILITY or
%% FITNESS FOR A PARTICULAR PURPOSE.
%%
%% See the GNU General Public License for more details.
%%
%% You should have received a copy of the GNU General Public License
%% along with PIPS.  If not, see <http://www.gnu.org/licenses/>.
%%
\documentclass[a4paper]{article}

\usepackage[latin1]{inputenc}
\usepackage{verbatim,comment,newgen_domain}
\usepackage{graphicx,ifpdf}
% Use classical figure extension if we are in classical LaTeX (instead of
% pdflatex), necessary for TeX4ht (htlatex):
\ifpdf
\else
  \DeclareGraphicsExtensions{.idraw,.eps}
\fi
\usepackage[backref,pagebackref]{hyperref}
\usepackage[nofancy]{svninfo}

%% To generate an index:
\usepackage{makeidx}
\makeindex

\title{PIPS: Memory effects of Statements}
\author{Fabien Coelho\\
  Fran�ois Irigoin \\
  Pierre Jouvelot \\
  R�mi Triolet\\
  \\
  CRI, M\&S, MINES ParisTech  \\
  \\
  B�atrice Creusillet\\
  Ronan Keryell \\
  \\
  HPC Project
}
\date{\svnInfoLongDate{}, revision r\svnInfoRevision}

\begin{document}
\svnInfo $Id: ri.tex 17253 2010-05-31 08:43:19Z creusillet $
\maketitle
\sloppy
\newpage
\tableofcontents
\newpage
\section*{Introduction}

\section{Imported domains}
\domain{import entity from "ri.newgen"}
{}
\domain{import reference from "ri.newgen"}
{}
\domain{import preference from "ri.newgen"}
{}
\domain{import expression from "ri.newgen"}
{}
\domain{import statement from "ri.newgen"}
{}
\domain{External Psysteme}
{}

\section{Memory access paths}

\domain{Cell = reference + preference}
{}
Cells are used in effects to specify memory access paths. See Section~\ref{effects} for more details.

% \section{Effets des instructions}
\section{Effects of statements and instructions}
\label{effects}

Each statement reads and writes several memory locations to retrieve
stored values and to store new values. Understanding the relationship between
statements and memory is one of the many keys to restructure and
parallelize programs. Several analyses provide different approximations
of the statement effects on memory.

Each program variable is a unique set of memory locations. Effects can
be expressed as effects on these sets. They are called \emph{atomic}
effects, because a whole data structure is seen as read or written as
soon as one element is read or written. Note that indirect effects due
to pointer uses are related to unknown memory locations.

The memory location representation can be refined for arrays. Certain
sets of array locations are handled, for instance intervals like
\verb/A[I:J]/ or even the so-called regular sections or Fortran~90
triplets, like \verb/A[0:N:2]/, which adds a stride to the concept of
interval. PIPS is able to handle polyhedral sets, called \emph{array
regions}. Extension to non-convex sets, intersections of a lattice and a
polyhedron, is under investigation.

Memory effects are not always perfectly known. It is undecidable in the
general case. Effects can be labelled as \texttt{MAY} if they might happen,
\texttt{MUST} if they always happen, and \texttt{EXACT} if the abstract set
used to represent them is equal to the dynamic set of effects. For
historical reasons, exact effects are labelled \texttt{MUST} and must
effects are not computed.

Finally, read and write effects are not precise enough for compound
statements. It is more interesting to known if the \emph{initial} value
of a memory location is \emph{used} by a statement, which is called a
\texttt{IN} effect, or if the memory location only is used for temporary
storage. In the same way, it is important to know if the value left by a
statement in a memory location is dead when leaving a statement or if it
is used later by another statement execution. In the later case, it is
called an \texttt{OUT} effect.

Note that spurious effects are added in loop bodies to avoid... and/or
simulate later a control dependence by a data dependence \marginpar{see
PJ}.  These effects are \verb/read/ effects on the loop bounds
(variables used in loop bound expressions?).

Note also that \verb/read/ effects of Fortran \verb/DO/ loop indices due
to the incrementations are ignored because they are never upward
exposed. This is due to the \emph{compound} nature of the PIPS \verb/DO/
construct. If it was decomposed into elementary parts, there would be no
such surprising approximations. Note that read effects which might be due
to bound or increment expressions as in \verb/DO I = I, 10*I, I/ must be
preserved.

\subsection{Effects}
\label{subsection-effects}
\index{Effects}

The \texttt{effects} domain is a list of individual effects. Each
effect can only be a read or a write and is related to only one
entity, most of the time a variable entity\footnote{In Fortran, the
  entity is always a variable entity.}, but possibly an area entity or
even an entity representing a set of areas\footnote{Areas and set of
  areas are used to express imprecise memory effects due to pointer
  dereferencing.}. Lots of individual effects are linked to each
statements, especially compound statements like \texttt{blocks},
\texttt{tests}, \texttt{loops} and \texttt{unstructured}.

Control effects, such as \verb/STOP/ or exception, are not
computed. PIPS only deals with memory effects. Fortran exceptions like
overflows or zero divides are considered programm errors and the error
behaviors are not taken into account in PIPS analyses.

Effects and the types defined in the following subsections are not used to
represent code, but to store analysis result. These types are declared in
the \emph{internal representation\/} for historical reasons.

\domain{Effects = effects:effect*}

The next domain can be used to store summary effects of callees.

\domain{entity\_effects = entity->effects}
{}

The next domain can be used to store a statement to effects mapping.
Should be used for proper and cumulated effects and references.

\domain{statement\_effects = persistent statement->effects}
{}


\subsection{Effects Classes}
\label{subsection-effects-classes}
\index{Effects Classes}

\domain{Effects\_classes = classes:effects*}
{}

The type \texttt{effects\_classes} is used to store equivalence classes of
dynamic aliases, i.e. aliases created at call sites. \texttt{Effects\_classes}
are lists of effects, i.e. lists of lists of regions.

\subsection{Effect}
\label{subsection-effect}
\index{Effect}\index{Region}

\domain{Effect = cell x action x approximation x descriptor}
{}
\domain{Descriptor = convexunion:Psysteme* + convex:Psysteme + none:unit}
{}

Type \verb/effect/ is used to represent a read or write access to a
variable or through a pointer, i.e. to abstract a reference in a
statement. Statement effects are used to compute array regions,
transformers and preconditions, to build use-def chains, dependence
%% http://www.cri.ensmp.fr/pips/pipsmake-rc.html}
graphs, Summary Data Flow Information (SDFI), known as summary effects
at the module level, and as cumulated effects at the statement level,
and array regions. Proper effects, cumulated effects, summary effects
and array regions are all of type \verb/effect/ but they are
distinguished by pipsmake as difference resources. Proper effects,
cumulated effects and summary effects are called simple effects and
they do not store information in the last field, \verb/descriptor/.

Field \verb/cell/ specifies which memory locations are accessed. In
Fortran, variables, scalar or array, are accessed directly. In C, it
may be a pointer specifying an indirect adressing. A \verb/cell/ is
either a \verb/reference/ or a persistant\footnote{Persitant is a
  Newgen attribute carried by a type or by a field. It means that
  recursive Newgen procedures must stop there. This impacts especially
  the free and copy functions.} \verb/preference/. Attribute
\verb/persistant/ is used for so-called \emph{reference} and \emph{simple} 
effects to keep track of actual program references, that should not be modified or freed. This feature is useful for several program analyses or transformations(see the pipsmake-rc documentation).


This persistant attribute is not welcome for more advanced effects,
such as regions, which use pseudo-references based on \verb/PHI/
variables. Memory allocation is even more difficult to manage when the
persistant attribute is declared at the type level and not at the
object level. This explains why the field \verb/reference/ was moved
down and accessed now through the field \verb/cell/.

Field \verb/action/ specifies if the memory access is a read or a write,
for read/write effects, and if the memory value is read from the statement
store for a \verb/in/ effect or region, and if the memory value written by
the statement is later read for a \verb/out/ effect or region. So,
\verb/read/ and \verb/in/ effects and regions have action \verb/read/ while
\verb/write/ and \verb/out/ effects and regions have action \verb/write/.


Field \verb/approximation/ is used to specify if all the memory addresses
of the reference for simple effects, or of the set of array elements
defined by the \verb/descriptor/ for array regions, are or not
accessed for sure. For instance, a conditional is going to generate
may effects, and a sequence, must effects\footnote{Needless to say,
  reality is much more complex. This oversimplified statement only is
  written to support some intuition about may and must and exact
  information.}.

Simple effects and regions may reference a global variable which is in
the scope of the callee but not in the scope of the caller or a static
variable of the callee declared in a \verb/SAVE/ statement. In the first
case, the effect translation process from the callee to the caller must
use a unique canonical name for such a variable, although the caller
does not provide one. In order to define a canonical name, a module
whose scope the variable belongs to is arbitrarily chosen and its name
is used to prefix the variable name. There is no known trivial choice
for this module. Currently, the module name of the first variable in the
common variable list is used:
\begin{quote}
 \texttt{ram\_section(storage\_ram(entity\_storage(<my\_common>)))}
\end{quote}
Unfortunately, this name depends on the module parsing order. It would
be much better to use the lexicographic order among callees, assuming
that callees are known before any analysis is started, which is true,
and assuming that scopes are known, which is not true because some
modules may be analyzed before some other ones are parsed (see pipsmake
in \cite{Trio90}\cite{Baro91}).

Field \verb/reference/ can be used to specify that an effect is limited
to a sub-array since a \verb/range/ can be used as subscript expression
of a reference. This facility is used when the cumulated effects of a
callee are translated into proper effects of the CALL site in the caller
scope. For regions, field \verb/reference/ defines the accessed variable
as well as pseudo-variables, known as \verb/PHI/ variables. There is one
\verb/PHI/ variable per array dimension.

Field \verb/context/ only is used for effects known as \emph{array
regions}. They were defined by Rémi Triolet in~\cite{Trio84} and
extended by Béatrice Creusillet in~\cite{Creu96}.

There should not be much strict aliasing between effects in an effect
list, but this is (was?) not checked and enforced. Some efforts are
made when translating the summary effects of a callee into the
caller's frame to avoid this problem.

\subsection{Nature of an Effect}
\label{subsection-action}
\index{Action}

\domain{Action = read:unit + write:unit}
{}

Two kinds of memory effects are used in Bernstein parallelization
conditions \cite{Bern66} and in other program transformation conditions:
read and write. \verb/IN/ regions are represented by \verb/read/ effects
and \verb/OUT/ regions by \verb/write/ effects.


\subsection{Approximation of an Effect}
\label{subsection-approximation}
\index{Approximation}

\domain{Approximation = may:unit + must:unit + exact:unit}
{}

It is not always possible to determine statically if a statement guarded
by control structures such as loops and tests is always executed. Thus,
it is not possible to know for sure that a variable is read or written
by such a compound statement. Some executions may always access it, some
other ones may never access it, and some may access it or not dependeing
on the statement occurence. Such effects are of the \verb/may/ kind.

\begin{comment}
La présence de tests et boucles ne permet pas de déterminer en général
si une variable est effectivement lue ou écrite lors de l'exécution
d'un \texttt{statement}. Il se peut même que certaines exécutions
y accèdent et que d'autres n'y fassent pas référence. Les effets
calculés sont alors de type \texttt{may}.
\end{comment}

Sometimes, a simple statement, such as an assignment like \texttt{J = 2},
has a known effect. Here \texttt{J} \verb/must/ be written. Such
\verb/must/ effects can be used in \emph{use/def chains} analysis to 
\emph{kill} some scalar variables. A simple effect, proper, cumulated or
summary, with a \verb/must/ attribute does not mean that the \emph{whole}
array is read or written, but that at least one of its elements is read
or written. Such information is not \emph{kill information}. Region
effects must be used to that effect.

Must effects may be detected for tests and loops. For instance, a
variable may be read in both test branches, or a loop range may be
numerically known.

\begin{comment}
Dans quelques cas particuliers, comme une affectation simple \texttt{I = 2},
l'effet est certain (\emph{must}). Il peut alors être utilisé dans
le calcul des \emph{use-def chains} pour effectuer un \emph{kill} sur les
variables scalaires. Un effet \emph{must} sur un tableau ne signifie pas
que tout le tableau est lu ou écrit mais qu'au moins un de ses
éléments l'est.
\end{comment}

\subsection{Mapping from Statements to Effects}

A different mapping from reachable statements to effect lists is
computed by each effect analysis. Because NewGen did not offer the 
\emph{map} type construct, there is no NewGen type for these mappings. They
are encoded as hash tables and used with primitives provided by the
NewGen library. They are stored on and read from disk by PIPS
interprocedural database manager, \emph{pipsdbm}.

For details about effect analyses available see the Effects Section in the
PIPS phase descriptions).
%%{http://www.cri.ensmp.fr/pips/pipsmake-rc.html}

\subsection{Mapping between Expressions and effects}

\domain{persistant\_expression\_to\_effects = persistant expression -> effects}


\section{Conclusion}

The PIPS internal representation is a relatively small set of data
structures, which has very slowly increased since the project inception.
Various mappings have been added. It was not possible to declare them
with NewGen in 1988 and quite a few implicit mappings exist.

NewGen data types can be walked with two generic iterators,
\verb/gen_recurse()/ and \verb/gen_multi_recurse()/. These two iterators
have been added to NewGen. They are not systematically used.

\newpage

\section*{Annexe: NewGen Declarations -- ri.newgen --}
\verbatiminput{effects.newgen}

\newpage

\begin{thebibliography}{9}

% pipsmake
\bibitem{Baro91} B. Baron,
\emph{Construction flexible et cohérente pour la compilation
interprocédurale}, 
Rapport interne EMP-CRI-E157, juillet 1991

\bibitem{Bern66} A. J. Bernstein, \emph{Analysis of Programs for
Parallel Processing}, IEEE Transactions on Electronic Computers,
Vol.~15, n.~5, pp. 757-763, Oct. 1966.

\bibitem{Creu96} B. Creusillet,
\emph{Analyses de régions de tableaux et applications}, Thèse de
Doctorat, Ecole des mines de Paris, Décembre 1996

\bibitem{JT89} P. Jouvelot, R. Triolet,
\emph{NewGen: A Language Independent Program Generator},
Rapport Interne CAII 191, 1989

\bibitem{JT90} P. Jouvelot, R. Triolet,
\emph{NewGen User Manual},
Rapport Interne CAII ???, 1990

\bibitem{Trio90} R. Triolet,
\emph{PIPSMAKE and PIPSDBM: Motivations et fonctionalités},
Rapport Interne CAII TR~E/133

\bibitem{Trio84} R. Triolet,
\emph{Contribution à la parallélisation automatique de programmes
Fortran comportant des appels de procédures}, Thèse de
Docteur-Ingénieur, Université Pierre et Marie Curie, décembre 1984.

\bibitem{TrIr91} R. Triolet,  F. Irigoin,
\emph{PIPS High-Level Software Interface: Pipsmake}
Documentation PIPS

\bibitem{ZhIr91} L. Zhou, F. Irigoin,
\emph{Properties: Low Level Tuning of PIPS},
PIPS Documentation

\end{thebibliography}


%\newpage

% Cross-references for points and keywords

\printindex

\end{document}
\end

%%% Local Variables: 
%%% mode: latex
%%% TeX-master: t
%%% ispell-local-dictionary: "american"
%%% End: 
