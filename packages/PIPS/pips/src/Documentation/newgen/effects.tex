%%
%% $Id$
%%
%% Copyright 1989-2014 MINES ParisTech
%% Copyright 2009-2010 HPC Project
%%
%% This file is part of PIPS.
%%
%% PIPS is free software: you can redistribute it and/or modify it
%% under the terms of the GNU General Public License as published by
%% the Free Software Foundation, either version 3 of the License, or
%% any later version.
%%
%% PIPS is distributed in the hope that it will be useful, but WITHOUT ANY
%% WARRANTY; without even the implied warranty of MERCHANTABILITY or
%% FITNESS FOR A PARTICULAR PURPOSE.
%%
%% See the GNU General Public License for more details.
%%
%% You should have received a copy of the GNU General Public License
%% along with PIPS.  If not, see <http://www.gnu.org/licenses/>.
%%
\documentclass[a4paper]{article}

\usepackage[latin1]{inputenc}
\usepackage{verbatim,comment,newgen_domain}
\usepackage{graphicx,ifpdf}
% Use classical figure extension if we are in classical LaTeX (instead of
% pdflatex), necessary for TeX4ht (htlatex):
\ifpdf
\else
  \DeclareGraphicsExtensions{.idraw,.eps}
\fi
\usepackage[backref,pagebackref]{hyperref}
\usepackage[nofancy]{svninfo}
\usepackage{array}
\usepackage{listings}
%% To generate an index:
\usepackage{makeidx}
\makeindex

\title{PIPS: Memory effects of Statements}
\author{Fabien Coelho\\
  Fran�ois Irigoin \\
  Pierre Jouvelot \\
  R�mi Triolet\\
  \\
  CRI, M\&S, MINES ParisTech  \\
  \\
  B�atrice Creusillet\\
  Ronan Keryell \\
  \\
  HPC Project
}
\date{\svnInfoLongDate{}, revision r\svnInfoRevision}

\begin{document}
\svnInfo $Id$

\maketitle
\sloppy
\newpage
\tableofcontents
\newpage
\lstset{language=C}
\section*{Introduction}
Each statement reads and writes several memory locations to retrieve
stored values and to store new values. Understanding the relationship
between statements and memory is one of the many keys to restructure
and parallelize programs. Several analyses provide different
approximations of the statement effects on memory. This document
gathers the internal data structures used to represent memory
locations and the effects of statements~\footnote{Several of the
  newgen domains declared in this document were originally part of the
  \texttt{ri.tex} file. They have been moved here because, strictly
  speaking, they do not deal with the internal representation of
  programs, and because the extension to C programs required the
  extension of the \texttt{cell} domain.}.

\section{Imported Domains}

\domain{import entity from "ri.newgen"}
{}
\domain{import reference from "ri.newgen"}
{}
\domain{import preference from "ri.newgen"}
{}
\domain{import expression from "ri.newgen"}
{}
\domain{import statement from "ri.newgen"}
{}
\domain{External Psysteme}
{}

\section{Representing Memory Locations}

A first solution to represent memory locations would be to use dummy
memory addresses, potentially one for each memory cell. However
analyses results would be equivalent modulo a renaming of dummy
addresses, making non-regression tests more dificult. Moreover,
analyses results would be difficult to interpret for a human operator,
and program transformations would lead to code difficult to read for
an end-user. This would not be a problem in a traditional
compiler. However, PIPS is a source-to-source compiler, and aims at
generating user-readable code, as close as possible to the user input
code. The first target language being Fortran~77, this primarily led
to represent memory locations as they are represented in the internal
program representation, that is to say as symbolic references on
program variables. Hence a read of variable \texttt{a} is represented
as an effect on reference \texttt{a}, and a read of array element
\texttt{t[5]} by an effect on reference \texttt{t[5]}. \texttt{t[*]}
is used to represent a set of unknown locations in array \texttt{t} or
to obtain a representation independent of the memory store.

However, this representation is highly imprecise to deal with sets of
array elements, which is necessary for effective dependence testing
and program transformations. Polyhedral sets (also called \emph{array
  regions}~\cite{Creu96}\cite{Trio84}) were introduced to refine the memory location
representation for arrays. Internally, the \texttt{reference} domain
is still used, but parameterized by dummy\texttt{phi} variables. An
additional convex descriptor describes the relations between the
\texttt{phi} variables and the program variables values in the
considered memory store. 

It has been chosen not to append the descriptor to the memory location
representation, but to integrate it in the client domains. This
seems more adapted to alias-like analyses for which we intend to use a
single descriptor to describe the relationships between the sink and
source memory cells representations.

For C programs, memory accesses are not solely represented by objects
of the \texttt{reference} domain (see~\texttt{ri.pdf}), in particular
when dereferencing pointers or refering to aggregate data
structures. Subsection~\ref{subsec:pointers_and_structs} presents the
choices made to represent memory accesses through pointers and
non-recursive data structures. Subsection~\ref{subsec:gaps} introduces
new domains to enable analyses of recursive data structures.

\subsection{Pointers, non-recursive structs and unions}
\label{subsec:pointers_and_structs}

Many question arose with the introduction of C as an PIPS input
language. For memory location representation, most of them were due to
pointers. A discussion of the different issues encountered and the
different solutions that were considered can be found
in~\verb+ri_C.pdf+. As it stretches over several pages, we do not
move or reproduce it here, but only present the final option.

It has been chosen to preserve the present represent representation of
memory locations as symbolic \texttt{reference}s. This has the
invaluable advantage of backward compatibilty for the existing
analyses. But it also enables the use of existing operators and
functions dealing with memory locations represented as references, and
thus enables a quicker integration of new functionalities necessary do
deal with C data structures.

This is not to the cost of the representation
precision. Table~\ref{tab:references} shows that memory accesses
involving pointers, arrays, structs and mix of them, can be
accuratly represented using symbolic references.


\begin{table}[htbp]
\begin{center}
\begin{tabular}{| >{\tt}l | >{\tt}l | >{\tt}l |}

\hline
declarations & reference & effects \\ \hline

int a, *p; &   & \\
      & a & a\\
      &*p & p[0] \\\hline

int t[N], *p, (*q)[N], *u[N], **v; & & \\ 
        & *t & t[0] \\
        & t[i] & t[i] \\
        & *p & p[0] \\
         & p[i]& p[i]\\ 
        & (*q)[i] & q[0][i] \\
        & *u[i]& u[i][0]\\
        & *v[i]& v[i][0]\\\hline

typedef struct \{ & & \\
int num; && \\
int tab1[N] ; && \\
int *tab2; \} mys;&& \\
&& \\

mys a, b[N], *c, **d; & & \\
                  & a.num  & a[.num] \\
                  & a.tab1[j] & a[.tab1][j] \\
                  & a.tab2[k] & a[.tab2][k] \\
&& \\
                  & b[i].num & b[i][.num] \\
                  & b[i].tab1[j] & b[i][.tab1][j]\\
                  & b[i].tab2[k]& b[i][.tab2][k]\\
&& \\
                  & c->num & c[0][1]\\
                  & c->tab1[j] & c[0][.tab1][j] \\
                  & c->tab2[k] & c[0][.tab2][k] \\
&& \\
                  & d[i]->num & d[i][0][.num]\\
                  & d[i]->tab1[j] & d[i][0][.tab1][j]\\
                  & d[i]->tab2[k] & d[i][0][.tab2][k]\\ \hline
                  
\end{tabular}
\end{center}
\caption{Representing memory locations using references}\label{tab:references}
\end{table}

Unions are more difficult to track because they create an equivalence
between two structures located at the same address. We intend to
represent accesses as we would do it for structs, delaying the burden
to disambiguate them to client analyses.

\subsection{Recursive data structures : GAPs}
\label{subsec:gaps}

Recursive data structures such as linked lists or trees are frequent
in C programs, and being able to deal with them is essential. However,
the \texttt{reference} domain is not adapted to the collective
representation of sets of recursive data structures elements, because
it can only describe fixed length access paths (that is to say the
number of indices of the reference must be numerically known and
finite). 

A quick extension could have been the introduction of a special flag
index meaning that up to a certain depth the path is
recursive. However, some intended program transformations could not
sufficiently benefit from this approach, and we have searched for a
new data structure to hold the necessary information. 

Experiments on real codes, and the need of a maximal compatibility
with existing data structures, led us to chose an extension of Alain
Deutsch's Symbolic Access Paths (SAPs)~\cite{Deutsch94}, which we call
Generic Access Paths or GAPs. GAPs are very similar to SAPs, but for
path selectors, which, instead of being limited to variable names and
struture field names, can additionally be any subscript expression,
thus allowing the representation of elements of arrays of linked lists
of arrays of linked lists... 

For instance \lstinline!p[i]([0][.next])^(c)[.tab][j]! represents the
j-th element of the \texttt{tab} field of the c-th element of the
linked list towards which \texttt{p[i]} points.

For backward compatibility reasons, we intend to use GAPs internally
only for recursive data structures, but they could also be used for
single scalar variables, stack or heap arrays, non-recursive aggregate
data structures...


\domain{Gap = variable:entity x path_selectors} 
{} 
\domain{Path_selectors = path_selector*} 
{} 
\domain{Path_selector = expression + recursive_selector} 
{} 
\domain{Recursive_selector = basis:path_selectors* x coefficient:expression} 
{} 



\subsection{Cell domain}

The domain~\texttt{cell} is used to represent memory
locations. Instead of the~\texttt{reference} domain,
\texttt{preference} can be used to retain a link to the actual program
reference. This is mainly used for Fortran programs where all memory
locations references are internally represented as objects of domain
\texttt{reference}. However this does not make much sense for C
programs because more complex expressions are often used to refer to
memory locations. GAPs have been added for recursive data structures
(see~ref{subsec:gaps} for more details).


\domain{Cell = reference + preference + gap} 
{} 

% \section{Effets des instructions}
\section{Effects of statements and instructions}
\label{effects}


Each program variable is a unique set of memory locations. Effects can
be expressed as effects on these sets. They are called \emph{atomic}
effects, because a whole data structure is seen as read or written as
soon as one element is read or written. Note that indirect effects due
to pointer uses are related to unknown memory locations.

The memory location representation can be refined for arrays. Certain
sets of array locations are handled, for instance intervals like
\verb/A[I:J]/ or even the so-called regular sections or Fortran~90
triplets, like \verb/A[0:N:2]/, which adds a stride to the concept of
interval. PIPS is able to handle polyhedral sets, called \emph{array
regions}. Extension to non-convex sets, intersections of a lattice and a
polyhedron, is under investigation.

Memory effects are not always perfectly known. It is undecidable in
the general case. Effects can be labelled as \texttt{MAY} if they
might happen, \texttt{MUST} if they always happen, and \texttt{EXACT}
if the abstract set used to represent them is equal to the dynamic set
of effects. In fact, must effects are not computed because with
polyhedral sets, they are either equal to the exact set or to the
empty set.

Read and write effects are useful for dependence analyses. However, it
is sometimes interesting to see compound statements as black boxes. In
this case we only want to known if the \emph{initial} value of a
memory location is \emph{used} by the statement, which is called a
\texttt{IN} effect, or if the memory location is only used for
temporary storage. In the same way, it is important to know if the
value left by a statement in a memory location is dead when leaving a
statement or if it is used later by another statement execution. In
the later case, it is called an \texttt{OUT} effect. Currently
\texttt{IN} effects are internally represented as \texttt{READ}
effects, and \texttt{OUT} effects as \texttt{WRITE} effects. The
programmer must be aware of the type of effects he is dealing with.

Note that spurious effects are added in loop bodies to avoid... and/or
simulate later a control dependence by a data dependence
\marginpar{see PJ}. These effects are \verb/read/ effects due to the
loop bounds and increment evaluations.

Note also that \verb/read/ effects of Fortran \verb/DO/ loop indices due
to the incrementations are ignored because they are never upward
exposed. This is due to the \emph{compound} nature of the PIPS \verb/DO/
construct. If it was decomposed into elementary parts, there would be no
such surprising approximations. Note that read effects which might be due
to bound or increment expressions as in \verb/DO I = I, 10*I, I/ must be
preserved.


\subsection{Effect representation}
\label{subsection-effect}
\index{Effect}\index{Region}

\domain{Effect = cell x action x approximation x descriptor}
{}
\domain{Descriptor = convexunion:Psysteme* + convex:Psysteme + none:unit}
{}

Domain \verb/effect/ is used to represent a read or write access to a
variable or through a pointer, i.e. to abstract a reference in a
statement. Statement effects are used to compute array regions,
transformers and preconditions, to build use-def chains, dependence
%% http://www.cri.ensmp.fr/pips/pipsmake-rc.html
graphs, Summary Data Flow Information (SDFI), known as summary effects
at the module level, and as cumulated effects at the statement level,
and array regions. Proper effects, cumulated effects, summary effects
and array regions are all of type \verb/effect/ but they are
distinguished by pipsmake as difference resources. Proper effects,
cumulated effects and summary effects are called simple effects and
they do not store information in the last field, \verb/descriptor/.

Field \verb/cell/ specifies which memory locations are accessed. In
Fortran, variables, scalar or array, are accessed directly. In C, it
may be a pointer specifying an indirect adressing. A \verb/cell/ is
either a \verb/reference/ or a persistant\footnote{Persitant is a
  Newgen attribute carried by a type or by a field. It means that
  recursive Newgen procedures must stop there. This impacts especially
  the free and copy functions.} \verb/preference/. Attribute
\verb/persistant/ is used for so-called \emph{reference} and
\emph{simple} effects to keep track of actual program references, that
should not be modified or freed. This feature is useful for several
program analyses or transformations(see the pipsmake-rc
documentation).


This persistant attribute is not welcome for more advanced effects,
such as regions, which use pseudo-references based on \verb/PHI/
variables. Memory allocation is even more difficult to manage when the
persistant attribute is declared at the type level and not at the
object level. This explains why the field \verb/reference/ was moved
down and accessed now through the field \verb/cell/.

Field \verb/action/ specifies if the memory access is a read or a write,
for read/write effects, and if the memory value is read from the statement
store for a \verb/in/ effect or region, and if the memory value written by
the statement is later read for a \verb/out/ effect or region. So,
\verb/read/ and \verb/in/ effects and regions have action \verb/read/ while
\verb/write/ and \verb/out/ effects and regions have action \verb/write/.


Field \verb/approximation/ is used to specify if all the memory addresses
of the reference for simple effects, or of the set of array elements
defined by the \verb/descriptor/ for array regions, are or not
accessed for sure. For instance, a conditional is going to generate
may effects, and a sequence, exact effects\footnote{Needless to say,
  reality is much more complex. This oversimplified statement is only
  written to support some intuition about may and exact
  information.}.

Simple effects and regions may reference a global variable which is in
the scope of the callee but not in the scope of the caller or a static
variable of the callee declared in a \verb/SAVE/ statement. In the first
case, the effect translation process from the callee to the caller must
use a unique canonical name for such a variable, although the caller
does not provide one. In order to define a canonical name, a module
whose scope the variable belongs to is arbitrarily chosen and its name
is used to prefix the variable name. There is no known trivial choice
for this module. Currently, the module name of the first variable in the
common variable list is used:
\begin{quote}
 \texttt{ram\_section(storage\_ram(entity\_storage(<my\_common>)))}
\end{quote}
Unfortunately, this name depends on the module parsing order. It would
be much better to use the lexicographic order among callees, assuming
that callees are known before any analysis is started, which is true,
and assuming that scopes are known, which is not true because some
modules may be analyzed before some other ones are parsed (see pipsmake
in \cite{Trio90}\cite{Baro91}).

Field \verb/reference/ can be used to specify that an effect is limited
to a sub-array since a \verb/range/ can be used as subscript expression
of a reference. This facility is used when the cumulated effects of a
callee are translated into proper effects of the CALL site in the caller
scope. For regions, field \verb/reference/ defines the accessed variable
as well as pseudo-variables, known as \verb/PHI/ variables. There is one
\verb/PHI/ variable per array dimension.

Field \verb/context/ only is used for effects known as \emph{array
regions}. They were defined by R�mi Triolet in~\cite{Trio84} and
extended by B�atrice Creusillet in~\cite{Creu96}.

There should not be much strict aliasing between effects in an effect
list, but this is (was?) not checked and enforced. Some efforts are
made when translating the summary effects of a callee into the
caller's frame to avoid this problem.

\subsubsection{Nature of an Effect}
\label{subsubsection-action}
\index{Action}

\domain{Action = read:action_kind + write:action_kind}
{}

Two different memory effects are used in Bernstein parallelization
conditions \cite{Bern66} and in other program transformation
conditions: read (a.k.a. use) and write (a.k.a. def).

To generalize use-def chains and dependence graphs to C source code,
several new kinds of state mutations must be taken into account. The
usual state at low level includes only the memory (a.k.a. the store),
but, at C source level, local variable and type declarations must also
be considered before distributing loops or eliminating seemingly
useless statements. So two new kinds of actions are added:
\verb/environment/ for variable declarations and
\verb/type_declaration/ for declarations of types, struct, union and
enum. This information is carried by the \verb/action_kind/ domain.

\domain{Action_kind = store:unit + environment:unit + type_declaration:unit}
{}

\verb/IN/ regions are
represented by \verb/read/ effects and \verb/OUT/ regions by
\verb/write/ effects. The action kind is \verb/store/.


\subsubsection{Approximation of an Effect}
\label{subsubsection-approximation}
\index{Approximation}

\domain{Approximation = may:unit + must:unit + exact:unit}
{}

It is not always possible to determine statically if a statement guarded
by control structures such as loops and tests is always executed. Thus,
it is not possible to know for sure that a variable is read or written
by such a compound statement. Some executions may always access it, some
other ones may never access it, and some may access it or not dependeing
on the statement occurence. Such effects are of the \verb/may/ kind.

\begin{comment}
La pr�sence de tests et boucles ne permet pas de d�terminer en g�n�ral
si une variable est effectivement lue ou �crite lors de l'ex�cution
d'un \texttt{statement}. Il se peut m�me que certaines ex�cutions
y acc�dent et que d'autres n'y fassent pas r�f�rence. Les effets
calcul�s sont alors de type \texttt{may}.
\end{comment}

Sometimes, a simple statement, such as an assignment like \texttt{J = 2},
has a known effect. Here \texttt{J} \emph{is} written. Such
\verb/exact/ effects can be used in \emph{use/def chains} analysis to
\emph{kill} some scalar variables.


\begin{comment}
Dans quelques cas particuliers, comme une affectation simple \texttt{I = 2},
l'effet est certain (\emph{exact}). Il peut alors �tre utilis� dans
le calcul des \emph{use-def chains} pour effectuer un \emph{kill} sur les
variables scalaires. 
\end{comment}

\subsection{Lists of effects}
\label{subsection-effects}
\index{Effects}

The \texttt{effects} domain is a list of individual effects. Each
effect can only be a read or a write and is related to only one
entity, most of the time a variable entity\footnote{In Fortran, the
  entity is always a variable entity.}, but possibly an area entity or
even an entity representing a set of areas\footnote{Areas and set of
  areas are used to express imprecise memory effects due to pointer
  dereferencing.}. Lots of individual effects are linked to each
statements, especially compound statements like \texttt{blocks},
\texttt{tests}, \texttt{loops} and \texttt{unstructured}.

Control effects, such as \verb/STOP/ or exception, are not
computed. PIPS only deals with memory effects. Fortran exceptions like
overflows or zero divides are considered programm errors and the error
behaviors are not taken into account in PIPS analyses.

Effects and the types defined in the following subsections are not used to
represent code, but to store analysis result. These types are declared in
the \emph{internal representation\/} for historical reasons.

\domain{Effects = effects:effect*}
{}



The next domain can be used to store summary effects of callees.

\domain{entity\_effects = entity->effects}
{}

The next domain can be used to store a statement to effects mapping.
Should be used for proper and cumulated effects and references.

\domain{statement\_effects = persistent statement->effects}
{}


\subsubsection{Effects Classes}
\label{subsubsubsection-effects-classes}
\index{Effects Classes}

\domain{Effects\_classes = classes:effects*}
{}

The type \texttt{effects\_classes} is used to store equivalence classes of
dynamic aliases, i.e. aliases created at call sites. \texttt{Effects\_classes}
are lists of effects, i.e. lists of lists of regions.

\subsubsection{Mapping from Statements to Effects}

A different mapping from reachable statements to effect lists is
computed by each effect analysis. Because NewGen did not offer the 
\emph{map} type construct, there is no NewGen type for these mappings. They
are encoded as hash tables and used with primitives provided by the
NewGen library. They are stored on and read from disk by PIPS
interprocedural database manager, \emph{pipsdbm}.

For details about effect analyses available see the Effects Section in the
PIPS phase descriptions).
%%{http://www.cri.ensmp.fr/pips/pipsmake-rc.html}

\subsubsection{Mapping from Expressions to effects}

\domain{persistant_expression_to_effects = persistent expression->effects}
{}

\section{Effects and pointer analyses}

In languages involving pointers, elements of symbolic references
which are store-dependent are not anymore confined to array indices or
polyhedra describing array indices relations with values of program
variables, but include the whole memory access path description. 

\lstset{numbers=left, numberstyle=\tiny, captionpos=b, framexleftmargin=5mm, frame=shadowbox}
\lstset{escapeinside={(*@}{@*)}}

\begin{lstlisting}[float,caption=Varying paths,label=lst:varying_paths]
int **p, **q, *a, *b;
a = (int *) malloc(10*sizeof(int)); (*@\label{lst:varying_paths_la}@*) 
b = (int *) malloc(10*sizeof(int)); (*@\label{lst:varying_paths_lb}@*)
p = &a;
p[0][2] = 0; (*@\label{lst:varying_paths_l1}@*)
p = &b;
q = p;
p[0][2] = 0; (*@\label{lst:varying_paths_l2}@*)
q[0][2] = 0; (*@\label{lst:varying_paths_l3}@*)
\end{lstlisting}
%
For instance, in Listing~\ref{lst:varying_paths}, \lstinline!p[0][2]!
on line~\ref{lst:varying_paths_l1} refers to \lstinline!a[2]! whereas
on line~\ref{lst:varying_paths_l2} it refers to \lstinline!b[2]!: in
the symbolic reference \lstinline!p[0][2]!, \lstinline!p[0]! is a
varying part that depends on the memory store. On the contrary,
\lstinline!p[0][2]! on line~\ref{lst:varying_paths_l2} and
\lstinline!q[0][2]!  on line~\ref{lst:varying_paths_l3} refer to the
same memory location whereas their symbolic references are
syntactically different.

Thus taking into account some kind of pointer analysis is necessary to
be able to disambiguate symbolic references during effect analyses and
all kind of dependence tests used throughout the parallelization
processes.

Pointer analyses try to gather the \emph{relations} that exist between
memory locations, due to pointer values. May alias analyses for
instance describe the possible overlaps between memory
locations. Points-to analyses establish the relations existing between
pointers and the memory locations they point to. We also intend to
design another analysis (called \emph{Pointer Values}) to gather the
relations existing between the values of pointers as well as the
memory locations represented by pointer values.

For all this kind of analyses\footnote{Points-to analyses are
  currenlty under development using alternative domains described in
  \texttt{points\_to\_private.pdf}.} we need to know the cells whose
values, addresses or locations are described, the type of information
each cell represents (value, address or location), the approximation
of the relation (exact, may or must) and a compulsory descriptor. It's
still unclear if some kind of additional information is needed to
handle casts.


\domain{Interpreted_cell = cell x cell_interpretation}
{}

\domain{Cell_interpretation = value_of:unit + address_of:unit}
{}

a \texttt{location} tag could be added for analyses describing relations
between memory locations, but they are not planned for the moment.


\domain{Cell_relation = first:interpreted_cell x second:interpreted_cell x approximation x descriptor}
{}

\domain{Cell_relations = list:cell_relation*}
{}

\domain{Statement_cell_relations = persistent statement->cell_relations}
{}


\newpage


\section*{Annexe: NewGen Declarations -- effects.newgen --}
\verbatiminput{effects.newgen}

\newpage

\begin{thebibliography}{9}

\bibitem{Baro91} B. Baron,
\emph{Construction flexible et coh�rente pour la compilation
interproc�durale},
Rapport interne EMP-CRI-E157, juillet 1991

\bibitem{Bern66} A. J. Bernstein, \emph{Analysis of Programs for
Parallel Processing}, IEEE Transactions on Electronic Computers,
Vol.~15, n.~5, pp. 757-763, Oct. 1966.


\bibitem{Creu96} B. Creusillet,
\emph{Analyses de r�gions de tableaux et applications}, Th�se de
Doctorat, Ecole des mines de Paris, D�cembre 1996.

\bibitem{Deutsch94} A. Deutsch,
\emph{Interprocedural may-alias analysis for
  pointers: beyond k-limiting}, Conference on Programming Language
Design and Implementation Proceedings of the ACM SIGPLAN 1994
conference on Programming language design and implementation 1994 ,
Orlando, Florida, United States.


\bibitem{JT89} P. Jouvelot, R. Triolet,
\emph{NewGen: A Language Independent Program Generator},
Rapport Interne CAII 191, 1989.

\bibitem{JT90} P. Jouvelot, R. Triolet,
\emph{NewGen User Manual},
Rapport Interne CAII ???, 1990

\bibitem{Trio84} R. Triolet,
\emph{Contribution � la parall�lisation automatique de programmes
Fortran comportant des appels de proc�dures}, Th�se de
Docteur-Ing�nieur, Universit� Pierre et Marie Curie, d�cembre 1984.

\bibitem{Trio90} R. Triolet,
\emph{PIPSMAKE and PIPSDBM: Motivations et fonctionalit�s},
Rapport Interne CAII TR~E/133

\end{thebibliography}


%\newpage

% Cross-references for points and keywords

\printindex

\end{document}
\end

%%% Local Variables:
%%% mode: latex
%%% TeX-master: t
%%% ispell-local-dictionary: "american"
%%% End:
