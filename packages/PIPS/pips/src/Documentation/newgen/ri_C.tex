%%
%% $Id$
%%
%% Copyright 1989-2010 MINES ParisTech
%%
%% This file is part of PIPS.
%%
%% PIPS is free software: you can redistribute it and/or modify it
%% under the terms of the GNU General Public License as published by
%% the Free Software Foundation, either version 3 of the License, or
%% any later version.
%%
%% PIPS is distributed in the hope that it will be useful, but WITHOUT ANY
%% WARRANTY; without even the implied warranty of MERCHANTABILITY or
%% FITNESS FOR A PARTICULAR PURPOSE.
%%
%% See the GNU General Public License for more details.
%%
%% You should have received a copy of the GNU General Public License
%% along with PIPS.  If not, see <http://www.gnu.org/licenses/>.
%%

\documentclass[a4paper]{report}
\usepackage[nofancy]{svninfo}
\usepackage{verbatim,newgen_domain}
\usepackage{listings}
\usepackage{array}
\usepackage[backref,pagebackref]{hyperref}

\lstset{extendedchars=true, language=C}

% Ou est le fichier de biblio ?

\title{PIPS: Extension of the Internal Representation for C}
\author{
  Fabien Coelho \\
  Fran\c{c}ois Irigoin \\
  Pierre Jouvelot \\
  Thi Viet Nga Nguyen \\
  CRI, MINES ParisTech}
\date{\svnInfoLongDate{}, revision \svnInfoRevision}

\begin{document}
\svnInfo $Id$

\noindent
\maketitle
\tableofcontents


\chapter*{Introduction}

This document discusses the implementation choices of the high-level
description of data structures used in PIPS as internal representations
(RI) of C programs.

These data structures are declared using the Newgen Data Definition
Language in the Abstract Syntax Tree \texttt{ri.tex} companion
document and shared with the Fortran77 parser..

Here are the goals of our work:
\begin{itemize}
\item Enough information must be preserved to prettyprint source code
  from the internal representation: this is the ultimate goal we must
  meet.
\item Whole program must be stored in the internal
  representation. Modules written in different languages such as
  Fortran and C must be stored in memory as part of one application,
  using the very same data structures.
\item Make the internal representation compatible with the SPEC2000
  CFP, SPEC1995 INT and SPEC2002 HPC as a first step, and extend it
  later if necessary to cover all of C ISO 99.
\end{itemize}
The reader is assumed knowledgeable in Newgen \cite{Jouv90} and the
internal representation used for Fortran \cite{Coel01}. Another
important reference is CIL, a C Intermediate Language developed at
University of California \cite{Necu02}, which is used for comparison
purposes.

The standard initially used was ISO C89, but C99 extensions must be
supported too.

In Section~\ref{chpater:naming}, we deal with naming issues. Then in
Section~\ref{chapter::storage}, the memory storage of different
classes of variables is presented. Types in C are much more diverse
than in Fortran 77 and numerous extesnsions are presented in
Section\ref{chapter:type}. C expressions are also extended beyond
Fortran 77 with features such as casts, sizeof and address-of. These
issues are handled in Section~\ref{chapter:expressions}. Finally
control flow extensions with respect to Fortran 77 are covered in
Section~\ref{chapter:control-flow}.

Beeware that details about such or such value given in this report may
be outdated. Please check \verb/ri.newgen/ and \verb/ri-util-local.h/
files for programming purposes.


\chapter{Naming}
\label{chpater:naming}

In C, the scope of a local variable is the block where it is declared,
the scope of an external static variable is the source file where it
is declared, not the module as in Fortran. So when it is necessary, we
have to add all information such as the current source file, module
and block to the entity name in order to locate an entity in the
symbol table. Here are several objectives for naming different kinds
of entities:
\begin{itemize}
\item As short as possible;
\item As significant as possible;
\item As uniform as possible;
\item As efficient as possible;
\item To be able to recompile after modifying a file (???);
\item Faithful with respect to the original declarations ;
\item Compatible with the existing internal representation (Fortran).  
\end{itemize}
Before entering the details related to this naming issue, here are some
recalls of the C standard \cite{ISOC}.

\section{ISO C concepts}

An identifier can denote
\begin{itemize}
\item an object (i.e. a variable)
\item a function
\item a tag of a structure, union or enumeration
\item a member of a structure, union or enumeration
\item a typedef name
\item a label name
\end{itemize}

\subsection{Scopes of identifiers (6.2.1)}

There are four kinds of scopes: function definition, file, block and
function prototype, also known as function declaration.
\begin{itemize}
\item A label name has {\it function definition scope}.
\item If the declarator or type specifier that declares the identifier
  appears outside of any block or list of parameters, the identifier has
  {\it file scope}.
\item If the declarator or type specifier that declares the identifier
  appears inside a block, the identifier has
  {\it block scope}.
\item If the declarator or type specifier that declares the identifier
  appears inside the list of formal parameters of a function
  definition, the identifier has {\it function definition scope}.
\item If the declarator or type specifier that declares the identifier
  appears within the list of parameter declarations in a function
  prototype, the identifier has {\it function prototype scope}.
\end{itemize}

\subsection{Linkages of identifiers (6.2.2)}

An identifier declared in different scopes or in the same scope more than
once can be made to refer to the same object or function by the {\it
  linkage} (external, internal and none). 

So the concept of global entity (\verb/TOP_LEVEL/) does not exist for
C, but we can use it to refer to external linkage entities, which
include functions and objects but not tags for structures, unions,
enumerations and typedef names.

\subsection{Name spaces of identifiers (6.2.3)}

There are separate name spaces : 
\begin{itemize}
\item label names
\item tags of structures, unions and enumerations 
\item members of structures or unions
\item ordinary identifiers 
\end{itemize}

Members of enumerations are handled as identifiers.

\section{File name scope}

There may be functions or variables that have the same name but in physically
  different files, such as :

\begin{itemize}
\item in \texttt{Directory1/foo.c}:
  \begin{lstlisting}
    static bar()
  \end{lstlisting}
\item in \texttt{Directory2/foo.c}:
  \begin{lstlisting}
    static bar()
  \end{lstlisting}
\end{itemize}

In order to distinguish these two static functions, the two different
absolute file names must be taken into account\footnote{This may not
  be implemented in August 2008}. The same situation can arrive with
\verb/typedef, struct, union/ and \verb/enum/ types, i.e two
structures with the same name in different files are different.  There
are several possibilities to uniquely name a source file
\begin{enumerate}
\item Use the absolute paths (problem with name lengths and file moves)
\item Use the relative paths (problem when copying/moving the
  database system)
\item Use specific naming in PIPS DBM, with a database that stores the
  correspondences between the actual file name (absolute or relative, FC
  ?) and the specific name. This is the current solution for Fortran.
\item By default: relative and an option for absolute 
\end{enumerate}
There may be problems related to special characters used in a C file
name which can be in conflict with characters used as separators in
the entity name, but on;y the \verb/MODULE_SEPARATOR/ is critical.

Another possible whole program compilation problem: different
compilation units use different names for structurally equivalent
types. CIL \cite{Necu02} resolves this problem by some merging phases
: merge types and tags and rewrite variable declarations and function
bodies.

\section{Block scope}

There are different solutions to handle the blocks:
\begin{enumerate}
\item Flattening like CIL \cite{Necu02} and then ignoring completely
  these block scopes, except for memory allocation and for
  prettyprinting.
\item Flattening the blocks and conserving the scope information
  somewhere to regenerate code. There must have special separators and
  tables to interchange between the internal names and external
  names. Problem for debugging? Problem for controlizer?
\item Dewey indexing for blocks (like trees). It can be associated to
  the above solution ?
\item Multiple symbol tables. Note: the actual solution, a unique
  symbol table for general "entity" may create problem when dealing
  with large-scale programs and prevents PIPS distribution.
\end{enumerate}
We do not use flattening as in CIL because the source programs are
transformed too much, which is not appropriated for a source-to-source
compiler, although the separation of declaration and code makes it
more easier to analyze the program. For example, in CIL, local
variables in inner scopes are pulled to function scope with variable
renaming like this:
\begin{lstlisting}
int main() {
  int x = 6;
  {
    int x = 7;
  }
  return x;
}

int main() {
  int x__0;
  int x__1; 
  {
    x__0 = 6;
    x__1 = 7
  }
  return x__0;
}
\end{lstlisting}
Impact of block scope on analyses of PIPS ? When propagating
transformers or preconditions, we must take into account the block
scope of variables, and pay attention to the different variables that
have the same user name when prettyprinting results of analyses.

\section{Current naming mechanism}

An entity name contains the name of the object, concatenated to a
prefix string and a special separator \verb/MODULE_SEP_STRING/
(":"). The prefix string is the name of the package defining the scope
of the object which can be \verb/"TOP-LEVEL"/
(\verb/TOP_LEVEL_MODULE_NAME/) or a module name.

In Fortran 77, the intrinsic \verb/ABS/ has internal name
\verb/"TOP-LEVEL:ABS"/ while variable \verb/INIT/ of module \verb/FOO/
has \verb/"FOO:INIT"/ as internal name.

There are also special prefixes to distinguish between a main program,
a common, a block data, an identifier or a label. For instance,
\verb/MAIN_PREFIX/ ("\%"), \verb/LABEL_PREFIX/ ("@"). result in
\begin{verbatim}
TOP-LEVEL:%MAIN
TOP-LEVEL:@LAB

Here are the characters corresponding to special separators (attention, \$
can be used in the identifier name): 
\begin{verbatim}
MODULE_SEP_STRING ":"
FILE_SEP_STRING "%" (compilation unit names are suffiexes by "!")
BLOCK_SEP_STRING "~" (redefined as "`")
\end{verbatim}  

So the entity name can be 
\begin{itemize}
\item 
\verb/TOP-LEVEL:name/
\item or 
\verb/[file%][module:][block~]name/ which can be one of the following: 
\begin{verbatim}
FILE%name
MODULE:name
FILE%MODULE:name
MODULE:BLOCK~name
FILE%MODULE:BLOCK~name
\end{verbatim}  
\end{itemize} 
In the current implementation, \verb/FILE%/ is (incorrectly) replaced
 by the compilation unit name, which is not sufficient to eliminate
 all anme conflicts.

The name of a label is MODULE:@label because it has function scope. 

In addition, to distinguish a structure, an union or an enumerator
that has the same name as other program variables, we have to add
special constant characters such as
\verb/STRUCT_PREFIX, UNION_PREFIX, ENUM_PREFIX/ to the name prefix. We
also have to keep the name of the structure and the union in the
global name of its members in order to distinguish these members with
other program variables, and so a \verb/MEMBER_SEP_STRING/ is needed.
It is not necessary for the enum member, because the name of a
variable must be different from the name of an enumerator
member\footnote{Enumerator members are represented as nullary
  functions with symbolic values, like Fortran parameters.}.  A prefix
for typedef \verb/TYPEDEF_PREFIX/ is necessary to distinguish a
defined name and to regenerate code.
\begin{verbatim}
STRUCT_PREFIX #
UNION_PREFIX *
ENUM_PREFIX ?
TYPEDEF_PREFIX $

MEMBER_SEP_STRING ^
\end{verbatim}

All prefixes and separators are defined in
\verb/ri-util-local.h/. They are defined twice as string and as
characters. Hence they are often assumed to be of length one by PIPS
programmers.

Examples: 
\begin{verbatim}
struct node {};            [file%][module:][block~]#node
struct key {int node};     [file%][module:][block~]key^node 
typedef int node;          [file%][module:][block~]$node
int key;                   [file%][module:][block~]key
union key2 {int node};     [file%][module:][block~]*key2 
                           [file%][module:][block~]key2^node 
enum hue {toto,tata};      [file%][module:][block~]?hue
                           [file%][module:][block~]toto 
int hue;                   [file%][module:][block~]hue
\end{verbatim}   

A function formal parameter in a function definition does not contain
 any block information. A function formal parameter in a function
 declaration is named with a special dummy module name.


Note: C distinguishes between uppercases and lowercases while Fortran
 77 considers all characters to be upper case, except in string
 constants.
 
The scope of struct, union, enum and typedef is the current
 compilation unit (current source file), whose name contains a special
 character, defined as \verb/FILE_SEP_STRING/. Struct and union share
 the same space. A struct and a union cannot have the same name within
 one scope, which may make the above distinction between unions and
 structures redundant.
 

\chapter{Storage}
\label{chapter:storage}

The storage class determines the location and lifetime of the storage
associated with a variable. 

\section{External variables}

\label{external}
External variables are defined outside any function, and are thus
 potentially available to any function that declares it
 \verb/extern/. Any function may access an external variable by
 referring to it by name, if the name has been declared somehow. If an
 external variable is to be referred to before it is defined, or if it
 is defined in a different source file, then an \verb/extern/
 declaration is mandatory.
\begin{verbatim}
Name prefix = TOP-LEVEL:
Storage = ram 
Ram_function = TOP-LEVEL-ENTITY 
Ram_section = TOP-LEVEL area 
\end{verbatim}
In Fortran, the scope of a variable is the module, or in other words, 
a global variable is always associated to the list of entities
of a module, which facilitates code regeneration. 

In order not to allocate external variables several times and to
 prettyprint their declarations properly, they are kept in the
 \verb/externs/ field of the \verb/code/ data structure.

In the old versions of C, we can declare (extern is not a real
 declaration, since it doesn't allocate memory per se) the same
 variable in multiple places (files) with no problem at link
 time. This is probably what we had in mind when we designed this part
 of the RI.
%%
This is not working anymore with more recent versions of C. As it
 produces linking error, multiple declarations although allowed by
 some explicit arguments to the compilers (like \verb/-z muldefs/ in
 \textsc{gcc}) are not clean programmin practice. Hence there must be
 some way to know the difference between the declarations of external
 variable and global variables and how to regenerate the declaration
 related to external variables, such as:
\begin{itemize}
\item in \texttt{file1.c}:
\begin{lstlisting}
  int m;
  void func1 { }
\end{lstlisting}
\item in \texttt{file2.c}:
\begin{lstlisting}
  int i;
  extern int m;
  void func2 { }
\end{lstlisting}
in \texttt{file3.c}:
\begin{lstlisting}
  void func3() {
    extern int m;
  }
\end{lstlisting}
\end{itemize}

Multiple declarations are allowed in a compilation unit only, but
 multiple initializations are forbidden even when they are compatible.

A source file can be considered as a module, and in the first case,
 the entity $m$ can be added to the list of entities associated to
 this source file. They are named \verb/"TOP-LEVEL:m"/. In the
 second and third cases there should not be any memory allocation for
 entity $m$. The real problem lies to differentiate between entity $i$
 and entity $m$ for second case and hence allocate memory for \verb/i/ but
 not for \verb/m/. We need to have some information in the internal
 representation: it is carried by the \verb/externs/ field of
 \verb/code/.

\section{Static variables}

\subsection{Internal static variables}

\begin{lstlisting}
int foo() {
  ...
  {
     static int i;
     ...
  } 
}  
\end{lstlisting}

\begin{verbatim}
Name = foo!:0`11`i, where 0`11` is just an example of block numbering 
Storage = ram 
Ram_function = current module or compilation unit, "foo!"
Ram_section = *STATIC* area of current module, "foo:*STATIC*"
\end{verbatim}
The outermost block can be omitted, since in fact it is considered as
 the current module. It is not necessary to generate an entity for
 each block, we only need to number the blocks. The offset of a
 variable is computed from the declarations of variables in the same
 block. So variables with the same offset but in different blocks are
 different.

\subsection{External static variables}

An external static variable is defined outside of any function, and is
 known within the remainder of the source file in which it is
 declared, but not in any other file. The source file is called a
 compilation unit and a pseudo-function is associated to it.

In file \texttt{foo.c}
\begin{lstlisting}
static int i = 0; 
int f() {
}
\end{lstlisting}
we have a corresponding entity to this external static variable:
\begin{verbatim}
Name = foo!:i
Storage = ram 
Ram_function = source file, "TOP-LEVEL:foo!"
Ram_section = *STATIC* area of source file , "foo!:*SSTATIC*"
\end{verbatim}
In addition, other entities are generated for the source file and the 
\texttt{*STATIC*} area of this source file.
\begin{verbatim}
Source file (or compilation unit)
   name = TOP-LEVEL:source_file_name, "TOP-LEVEL:foo!"
   type = functional (parameters = NIL, result = void)
   storage = rom

Area
   name = source_file_name:*STATIC*, "foo!:*SSTATIC*"
   type = area
   storage = rom (as for all area entities)
\end{verbatim}
 The prettyprint of the external static variable is based on the list
 of entities associated to the source file entity. The position of the
 variable declaration in the source file is the position in the
 declaration field of the \verb/code/ data structure.

\section{Automatic variables}

They are handled like dynamic local variables in Fortran \cite{Coel01}. 
The outermost block cannot be omitted to handle conflicts with formal parameter names. 
\begin{verbatim}
Name prefix =  [file%]module:[block~]
Storage = ram 
Ram_function = current module 
\end{verbatim}
If the size in bytes of the object is known at compile time, we use:
\begin{verbatim}
Ram_section = *DYNAMIC* area of current module
\end{verbatim}
Else we use
\begin{verbatim}
Ram_section = *STACK* area of current module
\end{verbatim}
as in Fortran for varying size arrays that are not formal parameters..
The keyword \verb/auto/ is kept in variable qualifiers, presented in
\ref{type_qualifiers}. 

\section{Formal variables}

If they appera in a function definition, they are represented as formal variables in Fortran \cite{Coel01}. 
\begin{verbatim}
Name prefix = [file%]module:
Storage = formal 
\end{verbatim}

Note that no block information appears.

If they appear in function declaration, aka function signatures, aka
 function types, their name prefix is synthesized to avoid any
 conflict (see \verb/DUMMY_PARAMETER_PREFIX/).

\section{Register variables}

A declaration of an identifier for an object with storage-class specifier
\verb/register/ suggests that access to the object be as fast as possible and 
the address of any part of an object declared with register cannot be
computed.  A \verb/register/ declaration can only be applied to automatic 
variables and to the formal parameters of a function. 
 
\begin{lstlisting}
f(c,n)
register int c,n;
{
  register int i;
}
\end{lstlisting}

\begin{verbatim}
storage = return:entity + ram + formal + rom:unit
\end{verbatim}

Since a formal variable can be declared with register, creating
 another type of storage such as \verb/register/ to store this
 information does not work and is not compatible with the pre-existing
 data structuress. Furthermore, for a source-to-source compiler, this
 information is not important, it is only used to regenerate the
 source code. So we only need to add this information some where in
 the type structure. A solution is presented in \ref{type_qualifiers},
 which also deals with \verb/const, volatile/ and \verb/restrict/.


\chapter{Type}
\label{chapter:type}

The type system of C is much more extended than Fortran 77's which
 makes the backward compatibility difficult to ensure.  Beside some
 usual data types such as int, float and char, C also has enumerated
 type and derived types such as array, structure, union, function and
 pointer.  Usually, these kinds of type are added to \verb/type/:
 
\verb/type += basic + array + pointer + struct + union + enum/

\verb/array = type x dimensions/

\verb/basic = int + float + .../

However, since PIPS Newgen internal representation has been designed
 for Fortran where the main data structure is array, the Newgen
 structure \verb/variable/ (\verb/variable = basic x dimensions/),
 used to represent Fortran scalar and array variables, must be handled
 compatibly.  Here are different possible solutions to deal with this:
\begin{enumerate}
\item Try to replace all the functions related to \verb/variable/ by
  macros, then \verb/variable/ is no more used in the new version.

\verb/type += basic + array + pointer + struct + union + enum/

\verb/array = type x dimensions/

\verb/basic = int + float + .../

\item Keep \verb/variable/ and \verb/array/ in parallel 

\verb/type += basic + array + variable + pointer + struct + union + enum/

\verb/array = type x dimensions/

\verb/variable = basic x dimensions/

\verb/basic = int + float + .../

\item To avoid the modifications related to \verb/variable/, which is
 expensive, the new types can be added to \verb/basic/. This method is
 called {\it array-oriented}. The Newgen data structure \verb/basic/
 is modified to enable the recursion among array, pointer, structure,
 ....
  
\verb/type = statement + area + variable + functional + void + .../

\verb/variable = basic x dimensions /

\verb/basic = int + float + logical + string + pointer + .../

However, in this method, the traversal is not always direct, i.e to
 access a pointer, we have to go through \verb/variable/, ... which
 may create bugs with malloc, free and it is not easy for debugging.
\end{enumerate}

The third solution is chosen because it requires less modifications in
 the actual internal representation. However its main impediment is
 that functional \verb/typedef/ appears as \verb/variable/ at firat,
 although the placement of \verb/typedef/ in the \verb/basic/ or
 \verb/type/ data structures does not appear clearly in the above
 dicussion. In particular, unlike \verb/pointer/, \verb/typedef/ does
 not require the dimension field provided by \verb/variable/.

\section{Basic types}

\subsection{Integer type}
\label{int}

C has different kinds of predefined integer type : int, signed int,
 unsigned int, short, long, long long, ... and char, signed char,
 unsigned char. Since \verb/int x/ and \verb/signed int x/
 declarations are implementation-defined (which is found in the file
 stdio.h but not in SPEC 2000), they should be distinguished.

Furthermore, new types are added such as \verb/intptr_t/,
 \verb/int32_t/,...

Currently, PIPS does not have a proper mechanism to adapt to a
 particular architecture. At best, \verb/#define/ are used. They
 define a 32 bit architecture: pointers are assumed stored in 32 bits
 like long int.

There are different solutions to represent all this information:
\begin{enumerate}
\item A compact representation that only uses the basic \verb/int/ and gives different values to
each kind of type:
\begin{verbatim}
char                   = 1
short_int              = 2
int                    = 4
long_int               = 6
long_long_int          = 8

unsigned_char          = 11
unsigned_short_int     = 12
unsigned_int           = 14
unsigned_long_int      = 16
unsigned_long_long_int = 18

signed_char            = 21
signed_short_int       = 22
signed_int             = 24
signed_long_int        = 26
signed_long_long_int   = 28
\end{verbatim}

We could use \verb/mod(int,10)/ to know the basic size and
 \verb/div(int,10)/ to know if the variable is unsigned, signed or
 not. However, this not fully compatible with the Fortran version
 \cite{Coel01} where the value of int is the number of bytes required
 to store one scalar object of this type. And this does not hold for \verb/long_int/.

\item A less compact but simpler solution is to represent three different
  cases: unsigned, signed or not. 

\verb/basic += int:int + signed:int + unsigned:int/

In each case, different values are associated to short int,
int, long int or long long int. 
\begin{verbatim}
short_int              = 1
int                    = 2
long_int               = 4
long_long_int          = 8
\end{verbatim}
\end{enumerate}

The first solution is currently implemented. The numbering scheme can
 be checked in \verb/ri-util/prettyprint.c/ and in \verb/cyacc.y/. See
 also function \verb/SizeOfarray()/.

It is not clear that merging signed and unsigned types is a good idea,
 especially in another data structure, \verb/constant/.

\subsection{Character type} 

\begin{enumerate}
\item A character variable is in fact an integer variable, so it can
 be associated to the basic \verb/int/, as in the first solution of
 \ref{int}.
\begin{verbatim}
unsigned_char          = 11
char                   = 21
signed_char            =  1
\end{verbatim}

\item It may also be better to treat character independently. 

\verb/basic += char:int/

where the value of \verb/int/ is 
\begin{verbatim}
unsigned_char          = 11
char                   = 21
signed_char            = 31
\end{verbatim}

\item Ambiguity between \verb/string/ and array of characters ?
 \verb/string/ is not used for C? Impact on semantic analysis, which
 does not handle arrays but handles strings optionnally?

\end{enumerate}

The first solution is chosen and the basic type \verb/string/ is not
 used for C internal representation.

\subsection{Bit type}

We have to add a basic type \verb/bit/ to represent the integral bit fields
occurred in a structure declaration.   

\verb/basic += bit:int/
\begin{verbatim}
int a:1 
unsigned b:2 
signed c:3
\end{verbatim}

\subsection{Boolean}

A special type, \verb/_Bool/ is used by gcc, according to ISO
 standard. This type is mapped on the logical basic.

\subsection{Varying argument lists}

The gcc compiler uses a special builtin type, \verb/va_list/. It is
 defined like a typedef, with not much information beyond what is
 needed for the prettyprinter.

\section{Arrays}

As in Fortran, we have to represent fixed size array and varying size
 arrays, arrays (and typedef) which are sized by expressions evaluated
 dynamically.

We also have to represent implictly sized arrays, whose sizes are
 implied by their initial values but should not be given in the
 declaration itself.

\section{Pointers}

Pointer can point to any variable: to a scalar variable, to an array variable, to a function, ...
To represent an array of pointers and to keep the initial internal representation, a new type \verb/pointer/ is
added to \verb/basic/.   

\verb/basic +=  pointer:type/

\subsection{Pointer to integers}

\begin{verbatim}
int *p1
name = p1
type = variable
   dimension = NIL
   basic = pointer
         pointer of type variable
             basic = int
             dimension = NIL.
\end{verbatim}
with this dynamic allocation:
\begin{lstlisting}
p = malloc(5*sizeof(int));
\end{lstlisting}
to store the size of the corresponding memory zone, the dimension of the
pointer can be changed to : 
\begin{verbatim}
int *p1
name = p1
type = variable
   dimension = NIL
   basic = pointer
         pointer of type variable
             basic = int
             dimension =5.
\end{verbatim}

But how about most cases, as in
 Section~\ref{subsection:pointer-to-pointer}?

Anyway, the dynamic analysis of pointers is not part of the
 parser. Calls to \verb/malloc()/ are not taken into account in the
 parser to do any memory sizing. The size of a pointer is either 32 or
 64 bits, but only the 32 bit architecture is implemented (see
 \verb/DEFAULT_POINTER_TYPE_SIZE/ in \verb/ri-util-local.h/, but this
 should be improved to chose the architecture dynamically). The first
 approach is used.

\subsection{Pointer to pointer}
\label{subsection:pointer-to-pointer}

\begin{lstlisting}
char **p;
p = malloc(n*sizeof(char *));
for (i=0;i<n;i++)
   p[i] = malloc((i+1)*sizeof(char));
\end{lstlisting}
Initially, with the declaration, we have :
\begin{verbatim}
name = p
type = variable
   dimension = NIL
   basic = pointer
         pointer of type variable
             basic = pointer of type variable (basic = int/char, dimension
             = NIL)
             dimension = NIL.
\end{verbatim}
After the first allocation, p points to an array of n pointers (attention
to the scope of n):
\begin{verbatim}
             basic = pointer of type variable (basic = int/char, dimension
             = NIL)
             dimension = n.
\end{verbatim}

How to know the memory size pointed by this each pointer? And the size
 is i-related ...? \lstinline/p[i]/ represents an entity or \lstinline/p/ only?
 The size of p[i] is lost There is a particular case where
 \lstinline/(i+1)/ is replaced by \lstinline/m/ (arrays of same size), we can
 keep this information in dimension of the second pointer.

This is left aside for future pointer analyses.

\subsection{Pointer to an array}

Declaration \lstinline/int (*p2)[13]/ defines \lstinline/p2/ as a
 pointer to an array of 13 integers:
\begin{verbatim}
name = p2
type = variable
   dimension = NIL
   basic = pointer
         pointer of type variable
             basic = int
             dimension = [13].
\end{verbatim}

\subsection{Array of pointers}

\lstinline/int *a[13]/ is an array of 13 pointers to integers
\begin{verbatim}
name = a
type = variable
    dimension = [13]
    basic = pointer
         pointer of type variable
             basic = int
             dimension = NIL.
\end{verbatim}

\subsection{Function returning a pointer}

\lstinline/char *f(n)/, f is a function that returns a pointer to a character
string
\begin{verbatim}
name = f
type = functional 
      result = variable
          dimension = NIL
          basic = pointer,
              pointer of type variable
                 basic = char
                 dimension = NIL. 
\end{verbatim}

\subsection{Pointer to a function}

\lstinline/int (*p)()/
\begin{verbatim}
name = p
type = variable
      dimension = NIL
      basic = pointer
           pointer of type functional
                parameters = undefined
                result = int.
\end{verbatim}

Note: \lstinline/undefined/ is supposed to be avoided in PIPS internal
 data structure. The \lstinline/gen_defined_p()/ predicate should
 always return true. Unknwon or unspecified would be better. Is
 \lstinline/void/ ending up with one parameter of type
 \lstinline{void}, different from \lstinline/NIL/? See the
 prettyprinter source code?

\section{Structure, Union and Enumerated Types} 

The common point between a structure, an union or an enumerated type
 is that each of these types has a name and a list of members. In
 addition, these members can be represented as entities, because they
 have name, type and eventually initial value. There are two
 possibilities to represent these new derived types:

\begin{enumerate}
\item Each type is represented separately:

\verb/type +=  struct:entity* + union:entity* + enum:entity*/ 
\item They are grouped into a composed type

\verb/type +=  composed/ 

\verb/composed =  members:entity* x kind/

\verb/kind = struct:unit + union:unit + enum:unit/
\end{enumerate}
The first solution is chosen because it is simpler and more direct when
we want to access a special type. 

In addition, in order to be homogeneous with \verb/pointer/, these
new types can be added to \verb/basic/:

\verb/basic +=  struct:entity* + union:entity* + enum:entity*/ 

\noindent
but adding them to \verb/type/ would make the traversal much
 shorter. We do not have to pass through \verb/variable/ each time we
 want to refer to an entity of struct/union/enum type.

The storage class of the struct/union/enum entities and their members
 is \verb/rom/. The \verb/initial value/ of a member entity can be
 used (or is used?) to represent the offset in bytes of the member in
 the struct.

All the above discussions talk about the entity related to the
struct/union/enum declaration (this template about the shape of a
structure create no storage) such as 
\begin{lstlisting}
struct key {
  int tab[3];
  int keycount;
};
\end{lstlisting}

This entity \verb/key/ is of type \verb/struct/ and is (implictly)
 associated to a list of members: \lstinline/"#key"/,
 \lstinline/"key^tab"/, \lstinline/"key^keycount"/. Note that the special
 character \lstinline/#/ is not repeated in the field names.\\

In the declaration like \lstinline/struct key var = {{1,2,3}, 3}/, the
 entity \verb/var/ is of type \lstinline/struct key/ and there are two
 possibilities to represent its type:

\begin{enumerate}
\item Associate it directly  to the entity \verb/key/
\item Associate it to a composed type which contains the name of the
 structure (\verb/key/) and the list of members, etc. This solution is
 redundant because we have to store the same information for each
 variable of type \lstinline/struct key/.
\end{enumerate}

So to represent a variable whose type is struct/union/enum, we add to
 \verb/basic/ the \verb/derived/ type to point to these new types.

\verb/basic += derived:entity/

The initial value of \verb/var/ in this case is a list of lists, which
 is not representable actually in Newgen. But since we know statically
 the size of the array \verb/tab/ in the \lstinline/struct key/, we
 could represent this value as a normal list \{1,2,3,3\} and the
 information can be extracted when needed. A new psesudo-operator,
 \lstinline{BRACE_INTRINSIC}, is added to be able to represent exactly
 the list of lists of expressions.

Structures, unions and enumrations that are not named receive a
 default internal name (see \lstinline/DUMMY_STRUCT_PREFIX/,
 \lstinline/DUMMY_UNION_PREFIX/ and \lstinline/DUMMY_ENUM_PREFIX/ in
 ri-util-local.h).

Whithin one scope, a named structure has a unique name, even if it is
 defined within another structure or union.

\subsection{Structure declaration}

\begin{lstlisting}
struct key {
  char *keyword;
  int keycount;
};
\end{lstlisting}

\begin{verbatim}
name = key
type = struct 

     name = keyword
     type = variable
          dimension = NIL
          basic =  pointer of type variable
                  basic = char
                  dimension = NIL
     storage = rom
     initial = 8

     name = keycount
     type = variable
          dimension = NIL
          basic = int
     storage = rom
     initial = 8
\end{verbatim}
Nga's comments: the traversal is much shorter if
 \verb/pointer, array, basic/ are added to \verb/type/ as in the other
 solution (which is more logic ...but there were too few type
 constructor in Fortran 77 to anticipate correctly the needs of C).
\begin{verbatim}
     name = keyword
     type = pointer of type basic = int

     name = keycount
     type = basic = int
\end{verbatim}

\subsection{Pointer to structures}

\begin{lstlisting}
  struct key *p;
\end{lstlisting}

\begin{verbatim}
name = p
type = variable
    dimension = NIL
    basic =  pointer of type variable
          basic = derived = entity key
          dimension = NIL
\end{verbatim}

\subsection{Array of structures}

\begin{lstlisting}
  struct key keytab[10];
\end{lstlisting}

\begin{verbatim}
name = keytab
type = variable
     dimension = [0:9]
     basic = derived = entity key
\end{verbatim}

\subsection{Self-referential structure (recursive data structure)}

\begin{lstlisting}
struct node {
  char word[10];
  struct node * next;
};
\end{lstlisting}

\begin{verbatim}
name = node
type = struct 

      name = word
      type = variable
          basic = char
          dimension = [0:9] 

      name = next
      type = variable
           dimension = NIL
           basic = pointer of type variable
                  basic = derived = entity node
                  dimension = NIL
\end{verbatim}

\section{Typedef}

\begin{lstlisting}
typedef char *STRING;
typedef int A[2][3];
typedef int (*PFI)();
typedef struct {} TREE,*TREEPTR;
typedef int f(char);
\end{lstlisting}
\verb/STRING, A, PFI, TREE, TREEPTR/ are entities with:
\begin{itemize}
\item Global name = \verb/TYPEDEF_PREFIX/ + local name. The \verb/TYPEDEF_PREFIX/ is
  used to regenerate code. 
\item Storage = rom 
\item Type = type which is named;
\item Initial value = could be the type wich is named, but the ype field is alreay used for this
\end{itemize}
It is more logic if the initial value of a typedef entity is
 \verb/type/. But in this case, we have to modify the \verb/value/
 Newgen structure, so it is better to put the type directly in the
 \verb/type/ of the entity. Too bad for functional typedefs whose type
 is \verb/variable/ at first look. The dimensions field is not useful
 either.

To represent variables whose type is a \verb/typedef/ entity, we add to
\verb/basic/ the \verb/typedef/ structure. 

\verb/basic += typedef:entity/ 

\begin{lstlisting}
typedef struct key {...} key;
key k1;
struct key k2; 
\end{lstlisting}  

\begin{verbatim}
name = STRUCT_PREFIXkey
type = struct  
storage = rom 

name = TYPEDEF_PREFIXkey 
type = variable 
     dimension = NIL
     basic = derived = entity STRUCT_PREFIX:key
storage = rom

name = k1
type = variable 
     dimension = NIL
     basic = typedef = entity TYPEDEF_PREFIXkey

name = k2
type = variable 
     dimension = NIL
     basic = derived = entity STRUCT_PREFIX:key
\end{verbatim}

\section{Functional Type}

Functional types have already been treated for Fortran, except for
 some (small) details.

\verb/type = functional + .../

\verb/functional = parameters:parameter* x result:type/

\verb/parameter = type x mode /

\begin{itemize}
\item For \lstinline/extern int f(void);/, \verb/parameters/ is a list
 of one element of type \lstinline/void/. The number of parameters is
 mislseading in this case.
\item How to represent \lstinline/extern int f();/ (the same for
 Fortran with \verb/EXTERNAL F/ ?) \verb/parameters_undefined/ creates
 consistency/updating problems ?
\item We can add a Newgen structure for the function qualifier
 \verb/inline/ but it was not necessary for during the first parser
 development because no inlined functions appear in SPEC 2000. It is
 snot clear the \lstinline/inline/ should be part of the type since
 two functions can have the same signature and hence the same type,
 but they do not have to be both inlined. Should it be a type
 qualifier? But qualifier a are restricted to variable types...
\end{itemize}

C supports varying argument lists and a special keyword,
 \lstinline/va_arg()/, whose two arguments are a variable and a
 type. As for \verb/sizeof/ which takes either a type or a variable as
 argument, this is not representable with a standard \verb/call/..

\section{Type qualifiers : Const, Restrict, Volatile}
\label{type_qualifiers}

\begin{lstlisting}
const int *p;
void func(const a);
void h(int * const restrict p);
\end{lstlisting}

Although only a small percent of variables are declared with these
 qualifiers and it is expensive, we choose to create a new Newgen
 structure for them. Attempts to put these information in existing
 Newgen structures, such as a \verb/rom/ storage for local variables
 qualified with \verb/const/, \verb/shared/ field of \verb/ram/ for
 \verb/restrict/ed local variables are not successful because they do
 not handle all possible cases. For example, \verb/rom/ cannot be used
 for a formal variable declared \verb/const/ (which can be found in
 SPEC 2000 benchmarks), \verb/shared/ cannot be used for a formal
 variable declared \verb/restrict/.

We can add a new field \verb/qualifiers/ for \verb/type/, at the
 \verb/variable/ level. This qualifier can also contain the
 \verb/register/ and \verb/auto/ cases.

\verb/variable = basic x dimensions x qualifiers:qualifier*/

\verb/qualifier = const:unit+restrict:unit+volatile:unit+register:unit+auto:unit/

There are about 36 make\_variable in PIPS source code to modify.

\section{Conclusion}

Some important information is carried by the entity name and not by
 the type data structure. This is the case for struct, union and
 enum. As a result, numerous characters become reserved and it is not
 always possible to use the external operator name as internal PIPS
 operator name. This old assumption of PIPS internal representation
 not longer holds.

\chapter{Expressions}
\label{chapter:expressions}

Expressions in Fortran PIPS must be extended to handle new kinds of
expressions found in the C language. Here are the Fortan 77 data structures used to represent expressions:

\verb/expression = syntax x normalized/

\verb/syntax = reference + range + call/

\verb/reference = variable:entity x indices:expression*/

New kinds of syntax are added to handle C language. They are {\it cast},
 {\it sizeof}, {\it subscripting array} and
{\it function application} expressions. The subscripting array expression
is an extension of the reference expression, which includes other
more complicated array objects such as pointer, function, 
structure or union member... The same extension is made to call
expression, named function application, because the called function is
not necessarily an entity but can be any expression that denotes a function.   

\verb/syntax += cast + sizeofexpression + subscript + application/

\section{Cast}

\verb/cast = type x expression/

Cast cannot be reprenseted easily as intrinsicss as in Fortran,
 because their number is unbounded due to the typedef mechanism.

\section{Sizeof}
\verb/sizeofexpression = type + expression/

\section{Subscript}

In C, pointer expressions can be subscripted, not arrays only.

\verb/subscript = array:expression x indices:expression*/

\section{Application}

C is mot flexible than Fortran about functional pointer. Such pointers
 can be stored in data structures isntead of being restricted to
 function call.

\verb/application = function:expression x arguments:expression*/

For example, we have such a function call in C::
\begin{lstlisting}
  (*ctx->Driver.RendererString)()
\end{lstlisting}


\section{Special calls}

To represent C construct, specific calls are used

\subsection{Member references }

There are different solutions:
\begin{enumerate}
\item A special call expression, \lstinline/FIELD_MEMBER_CALL(exp1, exp2)/, to handle for example
\lstinline|str[1].fld[2]|
\item
  \verb/reference = variable:entity x indices:expression x offset:entity*/
  This is not sufficient because we can have in C99 \lstinline|foo().x|
  where foo is a function returning a struct... That means that we need to
  extract fields from non l-values too.

More explaination here !
\end{enumerate}
 Distinguish between \texttt{.} and \verb/->/ ? The last one can be represented by
 the first one and \texttt{*}. 
 Disadvantages for type checking, program
analyses, transformations?

See \verb/ri-util-local.h/:
\begin{verbatim}
#define FIELD_OPERATOR_NAME             "."
#define POINT_TO_OPERATOR_NAME          "->"
#define DEREFERENCING_OPERATOR_NAME     "*indirection"
#define ADDRESS_OF_OPERATOR_NAME         "__address-of__"       // &

#define COMMA_OPERATOR_NAME             ","
\end{verbatim}

\subsection{Address of, value of}

Special functional intrinsics are used for \verb/&/ and \texttt{*}.

\subsection{Comma operator - list of expressions}

\lstinline|f(a, (t=3,t+2),c)|

Special call expression COMMA\_OPERATOR: n-ary or binary call ?

The same holds for \lstinline/a=b=c=d/. 

\subsection{Conditional expression}

\lstinline|e1 ? e2 : e3 |

Special call expression based on CONDITIONAL\_OPERATOR. This is the
only operastor with 3 arguments.


\chapter{Control Flow}
\label{chapter:control-flow}

\section{Module Code}

\verb/code = declarations:entity* x decls_text:string x initializations:sequence/

As discussed in section \ref{external}, external variables that are
declared outside a module can be pretty-printed if the source file is
considered as a module, with the declarations list and the decls\_text
string. This is implemented in the data structure \verb/code/ but the
field \verb/decl_text/ does not seem to be initialized by the parser
and is not used by the prettyprinter.

However, this does not let us know if a global variable has been
declared \verb/extern/ or not. For instance, the two declarations:
\begin{verbatim}
extern int i;
\end{verbatim}
et
\begin{verbatim}
int i;
\end{verbatim}
outside of a function definition result both in the declaration of a
global variable, \lstinline/top-level:i/, within a compilation unit.
To remember in which C file the keyword \verb/extern/ appears, a new
field, \verb/externs/ is added to \verb/code/. Note that the
initialization and/or the declaration without extern can appear only
once within an application. However, the sequence:
\begin{verbatim}
extern int i;
...
extern int i;
...
int i;
\end{verbatim}
is legal, probably to simplify the design of include files.

The field \verb/code_declarations/ contains all variables declared
within the function definition. Iternal PIPS entities such as memory
areas are declared first, followed by formal parameters. Other
variables declared within statement blocks are also listed here. So
most variables appear in a \verb/statement_declarations/ field and in
the \verb/code_declarations/ field of the module.

For a compilation unit, which does not really have a body, the
\verb/code_declarations/ field can be used.

The field \verb/initializations/ is used to represent DATA statements
in Fortran. In C, the initializer for a scalar variable is a single
expression, for objects that have aggregate or union types is an
initializer list, which can be not complete and by default, the
remaining elements are initialized by zero (for arithmetic type) or
null pointer (for pointer type).

There are two possibilities to represent this initialization information:
\begin{itemize}
\item Represent this initialization in the initial values of
entities. New kinds of value can be added to \verb/value/ such as
\verb/expression/ for scalar variables, 
aggregate or union types (array, structure, ...). String can be used to
  regenerate code, list (list of lists is not permitted in Newgen) can be used  to
  have fine preconditions on array elements, structure member, ... But the
  list length is the array size ?

Other problem: how to represent \verb/int i = j;/ ? The initial value of i
is j ?
\item Treat this initialization as a special kind of statements like DATA
  in Fortran.
\item since any statement and hence any instruction can include a
  declaration and since ISO C99 allows mix of declarations and
  executable statements, represent \lstinline/int i =j/ as
  \lstinline/int i; i = j;/.
\end{itemize}
The first solution is chosen and the field \verb/initial_value/ leads
to a value containing the initialization expression thru
\verb/value_expression/. Sepcial expressions are built for array and
structure initializations.

\section{Statement}
\label{statement}
 
The Fortran 77 structure of a statement is:
 
\verb/statement = label x number x ordering x comments x instruction/

Since in C, a declaration can appear in any block, not only at the
beginning of a function, we have to associate variable declarations to
blocks. There are two possibilities to perform this in the current
internal representation: declaration can be associated to either a
statement or a sequence (a block in fact). But since the true and
false branches of a conditional statement as well as the body of a
loop are not necessarily sequences, declarations in these statements
will be lost if they are associated to sequence. So we choose to
associate them to a statement, although it may be useless for some
elementary statements such as call, test or loop.

Another approach would have been to consider declarations as
statements (see discussion about initializations in previous section)
and/or to use a NOP instruction such as Fortran \verb/CONTINUE/ to
carry the declarations. This let us also preserve more comments than
other approaches.

Note: we have to pay attention in coding in PIPS that we cannot refer
to declarations once we have already reach the instruction because
there is no upward pointer from instruction to statement.

So we suggests to modify \verb/statement/ in the following way:

\verb/statement += declarations:entity* x decls_text:string/

instead of creating a new instruction, which could be called
declaration. When a new instruction is needed, PIPS designers often
use a new intrinsic rather than a new instruction because fewer PIPS
source code modifications are necessary.

Member \verb/declarations/ contains a list of entities in the
statement scope (all kinds of entities such as intrinsics, ... or only
those that are effectively declared).  Member \verb/decls_text/ could
be used to regenerate source code (as for module code) if the parser
initializes it.

This does not specify which statements carry the declarations. A first
implementation, based on C89, used block statements only to carry
declarations. All other statements had to have an empty declaration
list. This is too restrictive with respect to C99, which allow
declarations to appear anywhere among executable statements, and with
respect to source-to-source constraints. By putting all declarations
of a block in one statement, individual comments and line numbers are
lost.

A second implementation is based on C nop, ";", i.e. Fortran
CONTINUE. This statement has no effect and can be ignored, but for the
possible initial values, by PIPS analyses. PIPS transformations must
preserve it.

\section{Instruction}

The \verb/instruction/ for Fortran 77 is a union:

\verb/sequence + test + loop + whileloop + goto + call + unstructured/

Other kinds of instructions in C such as \verb/switch/, \verb/for/,
... can be added to \verb/instruction/ or represented using the
existing structures declared above.

In addition, a statement in C can be any expression, not only call
expression, so we have to add \verb/expression/ to \verb/instruction/.
However, to make PIPS backward compatible, we try to create a call
statement for each call expression. Expression statement is only used
for special cases, such as cast expression.

Not data structure is added for the \verb/switch/ construct.

A new field is added to \verb/whileloop/ to indicate if the condition
is evaluated before the body or after the body.

To simplify the prettyprinter, a \verb/forloop/ is added.

In each case, a decision must be made between the requirements of the
prettyprinter for the source-to-source use of PIPS (the more
structures the better) and the code complexity of the analyses (the
fewer the structures the better).

\subsection{Switch}

There are two solutions: we can add a new kind of instruction
\verb/multitest/ (the keyword \verb/switch/ cannot be used) or we can
represent a switch through \verb/if/ and \verb/goto/ statements.  

First solution: 

\verb/instruction += multitest/
\verb/multitest = controller:expression x body:statement/

The second solution is choosen: 
 
\verb/case/ and \verb/default/ are two kinds of labeled statements, which
can be treated as \verb/goto/. Their
associated labels are entities which must be unique.  The initial values of these
entities are constant expressions. We can add them to the
declarations list of the switch statement in order to match them to the actual
switch? The entity local name is the constant expression of case, and a
special name for default.  

The \verb/break/ statement can be treated as a \verb/goto/.

\begin{lstlisting}
   switch (c) {
     case 1:
       s1;
     case 2: 
       s2;  
       break;
     default: 
       sd;
   }

   if (c==1) goto switch_xxx_case_1;
   if (c==2) goto switch_xxx_case_2;
   goto switch_xxx_default;
 switch_xxx_case_1: ;
   s1;
 switch_xxx_case_2: ;
   s2;
   goto switch_exit_xxx;
 switch_xxx_default: ;
   sd;      
 switch_exit_xxx:
\end{lstlisting}

Note that this solution assume that the \verb/default/ case always
appears and that it appears after all other cases. Since this is not
always true, a direct syntactic translation is not possible and some
post-processing is required.

How about code regeneration? It might be easier to regenerate nice
code for structured if the internal control structure were based on:

\begin{lstlisting}
if(c!=1) 
  s1;
  goto continue1;
else if(c!=2)
continue_1:
  s2;
  goto continue2;
else if(...)
continue_2:
...
else /* default case */
continue_n:
  sd;
\end{lstlisting}

If \verb/s1/ and \verb/s2/ end with a \verb/break/, the
\verb/goto continuex/ are not reachable and the if-else-if structure
is preserved.

Some pattern-matching could be tried on the resulting
\verb/untructured/ and/or the \verb/unspaghettify/ option of the
controlizer might be able to put things back nicely when it is
possible...

\subsection{While Loop}

The ``\lstinline/while (expression) statement/'' and 
``\lstinline/do statement while (expression)/'' in C can be represented together
 by adding a new field to distinguish
if the evaluation of the
controlling expression takes place before or after each execution of the
loop body.   

\begin{verbatim}
whileloop = condition:expression x body:statement x label:entity x evaluation
evaluation = before:unit + after:unit
\end{verbatim}
To maximise the readability, \verb|evaluation:bool| is not used here,
as in other cases in the internal representation (\verb/mode/,
\verb/action/, ...).

\subsection{For Loop}

There is always a trade-off between regrouping different loop structures
and separating them. The first case makes program analyses more compact,
with less code to write but it makes pretty-printing original code
difficult. It is reverse for the second case. 

We have different possibilities to consider:
\begin{enumerate}
\item Represent for loop as while loop
\item Represent for loop as loop (do loop in Fortran), but it is not always
  possible. Loops could be pretty-printed as for loops but for loops cannot
  always be represented as loops (several loop indexes).
\item In order to keep the initial program structures, we can treat the for loop
separately from the other loops. 

\begin{verbatim}
forloop = initialization:expression x condition:expression
          x incrementation:expression x body:statement
\end{verbatim}
\end{enumerate}

The ISO C standard \cite{ISOC} states that the initialization of a for
loop may contain variable declaration, which is not the case for
\cite{Kern78}. Such declarations could be moved in the statement
containing the for, but for the time being the parser does not accept
such declarations.

Initializations could be put into the very statement containing the
\texttt{for} instruction and naming the block in a way to prettyprint
correctly the declaration into the \texttt{for}.

\subsection{Null statement}
Null statement in C, \verb/";"/, is treated as CONTINUE statement in
Fortran. We only need to make the differences at the prettyprinter
level.

\subsection{Return statement}
Return statement in C (return; or return (exp);) is treated as RETURN statement in Fortran, which is
considered as nullary operator but can have 0 or 1 argument, such as the
cases of STOP and PAUSE statements. 

\subsection{Break, Continue, Exit, Jump, Interruption}

We can add new kinds of instruction to handle break and continue: 

\verb/instruction += break + continue /

CIL says that leaving \verb/break/ and \verb/continue/ as they are
makes transformations such as code motion easier? The semantic
difference between \verb/continue/ in Fortran and \verb/continue/ in
C?

Another solution is to treat \verb/break/ and \verb/continue/ as
\verb/goto/, which is choosen as solution for the moment. How about code
regeneration ? Some semantics can be added into the generated label names.

PIPS analyses are preserved, but the prettyprinter of unstructured
must be improved to pattern-match \verb/continue/ and \verb/break/.

\chapter{Memory Effects}
\label{chapter:memory-effects}

\section{Pointers}

Pointers are a key part of C since parameters are passed by value and
since recursive data structures require pointers. In the Fotran
implementation, the memory effects are represented by the
\verb/effect/ data structure and its fields, and by lists of such data
structures:

\begin{verbatim}
effects = effects:effect* ;
effects_classes = classes:effects* ;
effect = cell x action x approximation x descriptor ;
cell = reference + preference ;
action = read:unit + write:unit ;
approximation = may:unit + must:unit + exact:unit ;
descriptor = convexunion:Psysteme* + convex:Psysteme + none:unit ;
reference = variable:entity x indices:expression* ;
preference = persistant reference ;
\end{verbatim}

In this framework, the effect of a statement like \verb/*p=1;/ can
 only be expressed as a reference to a large memory
 entity, an area or a set of areas, as no information is locally
 available about \verb/p/.

Effects are abstractions of effective memory effects. For instance
\verb/d[i]/ can be captured as \verb/d[i]/ because the data structure
\verb/reference/ let us do so, as \verb/d[*]/ to obtain a constant
effect independent of the current store, or as an array region, with
constant reference \verb/d[phi]/ and store sensitive descriptor
\verb/{phi=i}/.

We need a new abstraction to deal with indirect effects in such a way
that constant pointers can be detected and taken advantage of. For
instance, a C function incrementing an integer, \verb/void inc(*int p)/,
should be analyzed in such a way that the call site effect of
\verb/inc(&i)/ can be derived exactly.

We need to know that \verb/inc/ performs an indirect write through
\verb/p/.  This is not a great new abstraction for pointers. It is the
minimum needed in a first phase.

We need to encode \verb/p->in/, which is equivalent to
 \verb/(*p).in/. Assuming that \verb/in/ is the third field of the
 pointed structure, this could be rewritten \verb/(*p)[3]/.

We also need to encode \verb/(*p)[i][j][k]/ which is an array access to
 a formal array parameter. Pointer \verb/p/ may be a pointer to a
 dynamicically typed variable such as \verb/double x[n][m]/, where
 \verb/n/ and \verb/m/ are formal parameters too.

We do not know if we need to keep track of pointer arrays, as in
 \verb/*(p[3])/: our region framework can handle it effortlessly, but
 what can we do with information about pointer arrays? If our lattice
 is too simple, we won't be able to make a difference between
 \verb/*(p[3])/ and\verb/(*p)[3]/. However, since we transform
 accesses to structure fields into indexing (Section
 \ref{section:data-structures}), we might be interested in the post
 index form to keep or to retrieve information about data structures
 containing pointers on functions.

What is the new information we want to add? How can we add this new
information in the current PIPS data structure? Many solutions are
possible:
\begin{enumerate}

\item \label{variable-level}add nothing in the data structures and
 keep the information at the source code level: have different
 variables for direct and indirect effects;

\item \label{new-level}define a new level of effects, like
\verb/g_effect = direct:effect + indirect:effect + .../ or, at the list
level, \verb/g_effects = direct:effects + indirect:effects + .../

\item \label{new-action}add new kinds of actions such as
 \verb/indirect_read/ or \verb/indirect_write/;

\item re-use the \verb/descriptor/ field; this does seem to make less
 sense than the previous solution;

\item \label{new-effect}define a brand new pointer effect data structure, extending
 solution~\ref{new-level};

\item \label{new-effect-field} add a new effect field, \verb/addressing/,
 in \verb/effect/;

\item \label{new-effect-reference-dimension} represent pointer accesses using additional effect reference dimensions ; for instance an effect on \verb/*p/ could be represented as \verb/p[0]/;

\end{enumerate}

How do we chose given the current implementation and our goals?

Separating indirect effects and direct effects at the variable level
 would require a huge reworking of the current code since functions
 should return a structure containing several lists instead of a
 single list. This would prevent us from giving a meaning to the
 effect order in an order list: it is useful to know that a write
 occurs before a read.

This holds whether lists are separated at the variable level (solution
 \ref{variable-level}) or at the data structure level (solution
 \ref{new-level}).

Adding new actions, Solution~\ref{new-action}, does not respect the
 field semantics and its link with Bernstein's conditions. The number of
 new actions can be great if we take into account indirect pre- and
 post-indexation. However, we may not need pre- and post-
 indexation. And it might make debugging easier with unknown action
 detection. Unfortunately, with only two actions, people do not use
 switch and default when there are only two actions. They assume that
 if it is not the first one, it has to be the second one.

Solution~\ref{new-effect} would require a proper
 survey of most published pointer analyses. It would be nice if the
 old effect data structure could be mapped onto it to minimize source
 code modifications via macros. This could also be done later and
 independently, using simple pointer information gathered with the
 effect data structure as a starting point for more advanced analyses.


Adding a new field to effect, for instance \verb/addressing/ or
 \verb/addressing_mode/, Solution~\ref{new-effect-field}, is nicely
 orthogonal, let us structure it as we want, and avoid a combination
 of attribute with the read and write actions. It requires source code
 modifications for \verb/make_effect/. And the bugs due to a lack of
 access checking will not be syntactically detectable.

However indirect accesses would have to be reduced to standard
 accesses before the dependence test can be used and after the
 semantics analysis has managed to propagate pointer values and
 equality. The region analysis does not expect any indirect
 effects. The easiest way out might be to add a new phase using
 general effects and reducing them all to standard effects, exploiting
 semantics information. This would require a renaming of effects at
 the pipsmake and database level to avoid confusion and to force the
 conversion of resources.

This is the solution which was first chosen. It consisted in adding a new field
 \verb/addressing_mode/ (or \verb/addressing/ to avoid underscore in
 field names?) with three values for pre- and post-indexation:

\begin{verbatim}
effect = cell x action x approximation x descriptor x addressing;
addressing = index:unit + pre:unit + post:unit
\end{verbatim}

All accesses were indexed by default in PIPS internal representation,
 so \verb/post/ and \verb/pre/ were equivalent for scalar accesses.

Mode \verb/index/ could have been called \verb/direct_indexing/ to be more
 homogeneous, but indexing always occured. So \verb/index/ could have been
 called \verb/direct/.

The mode \verb/preindexing/ did not seem to be very useful in the
 short term.


But it proved to lack accuracy to handle real applications which use indirect accesses at several levels of data structures, and not only at the uppermost level. So, lastly, solution \ref{new-effect-reference-dimension} was retained. Table~\ref{tab:effects}, which is not limitative, shows that many cases of indirect accesses can be handled this way. Moreover, an obvious advantage of this approach is to unify all effects into effects on arrays which can be handled by following phases. 

\begin{table}[htbp]
\begin{center}
\begin{tabular}{| >{\tt}l | >{\tt}l | >{\tt}l |}

\hline
declarations & reference & effects \\ \hline

int a, *p; &   & \\
      & a & a\\
      &*p & p[0] \\\hline

int t[N], *p, (*q)[N], *u[N], **v; & & \\ 
        & *t & t[0] \\
        & t[I] & t[I] \\
        & *p & p[0] \\
         & p[I]& p[I]\\ 
        & (*q)[I] & q[0][I] \\
        & *u[I]& u[I][0]\\
        & *v[I]& v[I][0]\\\hline

typedef struct \{ & & \\
int num; && \\
int tab1[N] ; && \\
int *tab2; \} mys;&& \\
&& \\

mys a, b[N], *c, **d; & & \\
                  & a.num  & a[num] \\
                  & a.tab1[J] & a[tab1][J] \\
                  & a.tab2[K] & a[tab2][K] \\
&& \\
                  & b[I].num & b[I][num] \\
                  & b[I].tab1[J] & b[I][tab1][J]\\
                  & b[I].tab2[K]& b[I][tab2][K]\\
&& \\
                  & c->num & c[0][1]\\
                  & c->tab1[J] & c[0][tab1][J] \\
                  & c->tab2[K] & c[0][tab2][K] \\
&& \\
                  & d[I]->num & d[I][0][num]\\
                  & d[I]->tab1[J] & d[I][0][tab1][J]\\
                  & d[I]->tab2[K] & d[I][0][tab2][K]\\ \hline
                  
\end{tabular}
\end{center}
\caption{Representing effects with additional dimensions}\label{tab:effects}
\end{table}


\section{Data structures}
\label{section:data-structures}

PIPS was designed primarily for arrays and this shows in the reference
 data structure, which only support the indexed access mode. Hence, we
 need to map the offset accesses found with data structure references
 such as \verb/c.in/ onto indexed accesses.

Several possibilities come to mind for data structures if the current
 effect data structure is to be preserved:
\begin{enumerate}

\item to mimic the address computation at the byte level, assuming any
 field access can be interpreted as an array access of some byte
 elements; this does not account for bit fields, but those are not frequently used;

\item to rename fields by their ranks and to access them by indexing;
 this is not correct for a compiler, but the access information is
 preserved, which is enough for an analyzer; array fields are a
 natural dimension, inserted at the right place among index
 dimensions;

\item a variant of the previous possibility is to use field entities
  (and not solely their names) as effect reference indices;

\item to build new variable names located at the field address,
 i.e. in equivalence with the whole data structure but not with its
 other fields; for instance \verb/a.b.c/ would be represented by one
 variable of name \verb/a-b-c/ with no name conflict if dash is used
 inside the new name; array fields are would be taken care of as with
 a regular array; for instance \verb/a.b[i].c/ would be represented by
 \verb/a-b-c[i]/;

\item to build new variable names based on the offset values and to
 define some kind array type reflecting the real increments: an
 element size is associated to each dimension, so that the impact of
 any index is exactly reproduced; this might be a cross of the first
 two solutions; the offset based name may be equivalent to using the
 global offset as a unique last dimension instead of using one
 dimension per field and array traversal.

\end{enumerate}

We have to analyze the impact of the choice with respect to effect
 computation, but also with region and semantics analyses, as well as
 dependence testing.

Regions are similar to effects and are not impacted. The semantics
 analyses is not compatible with the first option, for instance
 because the value of an integer will not be analyzed since four
 different bytes are involved. The second option requires the
 semantics analysis to be extended to deal with fixed array elements
 such as \verb/x[1][2]/, which has been a long pending extension. The
 third option requires the semantics analysis to be extended to deal
 with equivalenced variables: currently, the analyses are restricted
 to non aliased variables.

The dependence test should be able to deal with all four options
 because it deals with regions, i.e. set of array elements, and with
 equivalenced variables, i.e. an aliasing test is performed before the
 array dependence test.

The fourth option requires improvements in memory allocation and in
 aliasing management. Data structures for aliasing exist because of
 Fortran, but they are not currently used for C. The scalar analysis
 of semantics can be used right away to analyze structure fields.

The effect prettyprinter is easier with option 2, unless the naming
 scheme is exaclty C scheme, i.e. \verb/a.x/ is equivalenced with a
 variable of name \verb/a.x/. Hopefully, a dot in a variable name is
 not going to create implementation nightmares. But field names may
 conflict: they should include the structure name and \verb/a.x/ would
 become something like \verb/a.a-x/.

All four schemes are minimum. They are sufficient to represent any
 constant address with indexing equivalent to a field access
 expression. And any field expression is represented in only one way.

After testing the second option for a while, we finally changed for the third one in particular because it avoids merging convex effects of different access path types. We have chosen to keep PHI variable numbers equal to their ranks in the effect reference indices list for practical reasons. Functions are provided to convert effects with references containing indices refering to field entities to their rank equivalent versions.

\section{Unions}
\label{section:unions}

Union are more difficult to track. They create an equivalence between
 two structures located at the same address. This reminds us of
 Fortran equivalences, extended to data structures.

For instance, \verb/union {int i; double x;} u;/ could be interpreted
 as three variables called verb/u/, /u-i/ and /u-x/ located at the very same
 address.

Parallelism could be safely detected for arrays of unions and for unions of
 arrays using existing PIPS technology.

It is not yet clear how this could be extended when unions are part of
 more complex data structures. Can we still use a naming scheme
 (option 3 for data structures) and move all index information at the
 end of the reference? It this still a way to denote precisely the
 address accessed such that all analyses are based on a correct
 representation of locations?

If we use a naming scheme (option 3 above), we should be able to
 compute addresses and ranges for combination of union and struct and
 to build aliasing information due to unions. To be refined.

Indexing may be more difficult to deal with as its impact on address
 computation must be precisely reproduced when dealing with
 equivalenced variables.

If we use the basic byte based representation (Solution 1), union can
 be handled like structures and lead to automatic parallelization. But
 difference sets of indices will lead to array linearization.

So we have to find the best trade-off between naming, indexing and
 linearization for dependence testing when unions are used.

\section{Point to Operator}

The point to operator is syntactic sugar. The expression \verb/a->x/
 can be replaced by \verb/(*a).x/.

\section{Address Expressions}

Address expressions are very general as base and offset can be
 provided by function calls and offset by any integer arithmetic
 expressions. Pointers differences are allowed to compute offsets,
 making it difficult to identify a base pointer and an
 offset. Consider for instance \verb/*(p+q-r)/: is it \verb/p/ or
 \verb/q/ the offset?

If we could distinguish a base pointer \verb/p/ and one offset, the address
 expression could be rewritten \verb/*(p[offset])/.

But the basis may be returned by a function, as in \verb/f()->x/.

Multiple indirections, as in \verb/a->x->y/, \verb/M[M[a]+x]+y/ make
 the notions of base and offset ludicrous.

So we need a way to express fuzzy address, for instance using a
 special address value {\em anywhere} equivalent to the whole memory
 \verb/M/.

\section{Indirect Effect Normalization}

When pointers value are known, indirect accesses can be converted
 precisely into direct accesses as in \verb/*(&i)=1/.

Fuzzy pointer values, i.e. sets of pointer values, must be introduced
 to cover other cases, using if possible PIPS support for aliasing,
 a.k.a. equivalence in Fortran. The lattice for pointer values may be
 built with areas, finite sets of areas or special new areas such as
 the area containing all areas. PIPS uses four areas for each module:
 the static area for objects of constant addresses, the dynamic area
 for objects of constant sizes\footnote{They are given constant
 addresses in this area, but they are only constant with respect to
 the frame pointer.}, the stack area for objects of unknown size and
 the heap area for dynamic memory allocation. Currently aliasing is
 only statically handled in the static and dynamic areas, using
 constant addresses.

The potentiel levels are:
\begin{enumerate}

\item at least three as for any constant propagation lattice, with
 {\em anywhere} used as soon as the constant value is unknown;

\item but it could be slightly increased by taking into account the
 scope of a function and the scope of its compilation unit.

\item or it could include all subsets of the area set since the latest
 is bounded by the number of functions and compilation units.

\item or...

\end{enumerate}

Note that \verb/free/ does not let us infer that two pointers are
 different because both of them were malloced. Hence the different
 heap areas\footnote{A separate heap area is used for each function.}
 are in fact a unique heap.

What aliasing assumptions should we make about formal parameters and
 global variables? Do we assume constant value for formal pointers and
 hence no aliasing between them and between local pointers?

How do we implement the lattice? By multiplying the effects in the
 effect list, with one effect per area possibly accessed? By adding
 new super areas including several subareas and using equivalence
 information? By limiting the number of levels in the lattice?

\section{Putting together pointers, structures, unions and C address computations...}

The general case has not been discussed yet as we are driven by the
 Ter@ops project and its requirements:
\begin{enumerate}
\item key information such as array dimensions and loop bounds are
 parts of structures;
\item array dimensions and hence typing are dynamic; formal arrays are
 likely to be accessed via pointers as in \verb/(*p)[i]j]/;
\item ...
\end{enumerate}

This does not require mixes of structures, unions and pointers to be
 dealt with...

\chapter*{Conclusion}

There is no perfect solution, especially before full implementations
tell us how much each solution does really cost.

We need to keep a proper balance between information needed to
prettyprint source code and unifications which streamlines the code
for analyses and transformations.

This document is an evolving document. Information is not always up to
date.with respect to PIPS source code, especially the \verb/ri-util/,
\verb/c_syntax/ and, to a lesser esxtent, \verb/preprocessor/
libraries. In \verb/ri-util/, the files \verb/ri-util-local.h/,
\verb/prettyprint.c/ and \verb/cyacc.y/ are especially relevant.

\nocite{Kern78, Necu02}

The choices that have been done are in the companion file \texttt{ri.tex}


\appendix{}

\chapter{Inital Fortran-oriented internal representation}

\begin{verbatim}
action = read:unit + write:unit ;
approximation = may:unit + must:unit + exact:unit ;
area = size:int x layout:entity* ;
basic = int:int + float:int + logical:int + overloaded:unit + complex:int + string:value ;
call = function:entity x arguments:expression* ;
callees = callees:string* ;
cell = reference + preference ;
code = declarations:entity* x decls_text:string x initializations:sequence ;
constant = int + litteral:unit + call:entity ;
control = statement x predecessors:control* x successors:control* ;
controlmap = persistant statement->control ;
descriptor = convexunion:Psysteme* + convex:Psysteme + none:unit ;
dimension = lower:expression x upper:expression ;
effect = cell x action x approximation x descriptor ;
effects = effects:effect* ;
effects_classes = classes:effects* ;
entity_effects = entity->effects ;
entity_int = entity->int ;
execution = sequential:unit + parallel:unit ;
expression = syntax x normalized ;
formal = function:entity x offset:int ;
functional = parameters:parameter* x result:type ;
instruction = sequence + test + loop + whileloop + goto:statement + call + unstructured ;
loop = index:entity x range x body:statement x label:entity x execution x locals:entity* ;
mode = value:unit + reference:unit ;
normalized = linear:Pvecteur + complex:unit ;
parameter = type x mode ;
persistant_expression_to_effects = persistant expression -> effects ;
persistant_statement_to_control = persistant statement -> persistant control ;
persistant_statement_to_int = persistant statement -> int ;
persistant_statement_to_statement = persistant statement -> persistant statement ;
predicate = system:Psysteme ;
preference = persistant reference ;
ram = function:entity x section:entity x offset:int x shared:entity* ;
range = lower:expression x upper:expression x increment:expression ;
reference = variable:entity x indices:expression* ;
sequence = statements:statement* ;
statement = label:entity x number:int x ordering:int x comments:string x instruction ;
statement_effects = persistent statement->effects ;
storage = return:entity + ram + formal + rom:unit ;
symbolic = expression x constant ;
syntax = reference + range + call ;
tabulated entity = name:string x type x storage x initial:value ;
test = condition:expression x true:statement x false:statement ;
transformer = arguments:entity* x relation:predicate ;
type = statement:unit + area + variable + functional + varargs:type + unknown:unit + void:unit ;
unstructured = entry:control x exit:control ;
value = code + symbolic + constant + intrinsic:unit + unknown:unit ;
variable = basic x dimensions:dimension* ;
whileloop = condition:expression x body:statement x label:entity ;
\end{verbatim}

\chapter{Proposed new internal representation (modified structures only)}

This information may be outdated. Please check \verb/ri.newgen/ for
PIPS current data structures. This information was useful initially to
estimate the changes in the data structure and the impact on the
existing code. It does not seem useful to maintain it now that the C
parser is implemented..

\begin{verbatim}
basic = int:int + float:int + logical:int + overloaded:unit + complex:int 
+ string:value + bit:int + pointer:type + derived:entity + typedef:entity;

instruction = sequence + test + loop + whileloop + goto:statement + call +
unstructured + forloop + expression;

forloop = initialization:expression x condition:expression x
incrementation:expression x body:statement ;

statement = label:entity x number:int x ordering:int x comments:string x
instruction x declarations:entity* x decls_text:string ;

syntax = reference + range + call + cast + sizeofexpression + subscript + application;

cast = type x expression ;

sizeofexpression = type + expression ;

subscript = array:expression x indices:expression* ;

application = function:expression x arguments:expression* ;

type = statement:unit + area + variable + functional + varargs:type +
unknown:unit + void:unit + struct:entity* + union:entity* + enum:entity*;

variable = basic x dimensions:dimension* x qualifiers:qualifier* ;

qualifier = const:unit + restrict:unit + volatile:unit + register:unit + auto:unit;

whileloop = condition:expression x body:statement x label:entity x
evaluation ;

evaluation = before:unit + after:unit ;

value = code + symbolic + constant + intrinsic:unit + unknown:unit + expression;
\end{verbatim}

\bibliographystyle{abbrv}
\bibliography{biblio}


\end{document}

%%% Local Variables: 
%%% mode: latex
%%% TeX-master: t
%%% ispell-local-dictionary: "american"
%%% End: 
