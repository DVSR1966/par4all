%%
%% $Id: alias_private.tex 14503 2009-07-10 15:11:52Z mensi $
%%
%% Copyright 1989-2009 MINES ParisTech
%%
%% This file is part of PIPS.
%%
%% PIPS is free software: you can redistribute it and/or modify it
%% under the terms of the GNU General Public License as published by
%% the Free Software Foundation, either version 3 of the License, or
%% any later version.
%%
%% PIPS is distributed in the hope that it will be useful, but WITHOUT ANY
%% WARRANTY; without even the implied warranty of MERCHANTABILITY or
%% FITNESS FOR A PARTICULAR PURPOSE.
%%
%% See the GNU General Public License for more details.
%%
%% You should have received a copy of the GNU General Public License
%% along with PIPS.  If not, see <http://www.gnu.org/licenses/>.
%%

\documentclass{article}

\usepackage{newgen_domain}
\usepackage[backref,pagebackref]{hyperref}

\title{Points\_to Analysis}
\author{Amira Mensi}
\begin{document}
\maketitle

\section{Introduction}

\domain{import cell from "effects.newgen"}
{}
\domain{import descriptor from "ri.newgen"}
{}
\domain{import approximation from "ri.newgen"}
{}
\domain{import statement from "ri.newgen"}
{}
% These newgen data structure are used to propagate interprocedural points_to
% information. A points_to relation involved two entities :source and
% sink and an approximation of their relation.  
 


 
The domains points-to indicates that an abstract memory location
called {\em source} points to another abstract memory location called
{\em sink}. The arc between the two locations may either always exist,
and the approximation is EXACT, or possibly exist, and the
approximation is MAY. The abstract memory locations may contain
program variables or PIPS entities in their subscript expressions, and
their values are constrained by the {\em descriptor}.

\domain{points\_to = source:cell x sink:cell x approximation x descriptor}
{}
Note: the domains {\em approximation} and {\em descriptor} are
documented in ri.tex.

% \domain{access = referencing:points\_to\_path + addressing:points\_to\_path + dereferencing:points\_to\_path}{}
%\domain{points\_to\_path = reference}{}
%\domain{points\_to\_graph = arcs:points\_to*}{}

The points-to information is a list of points-to relations:
\domain{points\_to\_list = list : points\_to*}
{}

This domain is used to associate its points-to information to each statement:
\domain{statement\_points\_to = persistant statement -> points\_to\_list}
{}

\end{document}
