%%
%% $Id$
%%
%% Copyright 1989-2014 MINES ParisTech
%%
%% This file is part of PIPS.
%%
%% PIPS is free software: you can redistribute it and/or modify it
%% under the terms of the GNU General Public License as published by
%% the Free Software Foundation, either version 3 of the License, or
%% any later version.
%%
%% PIPS is distributed in the hope that it will be useful, but WITHOUT ANY
%% WARRANTY; without even the implied warranty of MERCHANTABILITY or
%% FITNESS FOR A PARTICULAR PURPOSE.
%%
%% See the GNU General Public License for more details.
%%
%% You should have received a copy of the GNU General Public License
%% along with PIPS.  If not, see <http://www.gnu.org/licenses/>.
%%
\documentclass[a4paper]{article}
\usepackage{newgen_domain}
\usepackage[backref,pagebackref]{hyperref}

\title{PIPS: Complementary sections}
\author{Manjunathaiah M and Be'atrice Creusillet}
\date{April 19th, 1996}

\begin{document}
\sloppy

This document contains the newgen structures for Manjunathaiah M
complementary sections.

\domain{External Pvecteur}
{}

\domain{Import reference from "ri.newgen"}
{}
\domain{Import action from "ri.newgen"}
{}

\domain{simple\_section = context:context\_info x dad:dad\_struct }
{}
A simple section has two parts. a) context information and 
b) the section descriptor

\domain{context\_info = line:int x rank:int x nest:int }
{}
Context includes the line number of the source, the dimensionality
of the array(denoted as rank) and the scope of the array reference

\domain{dad\_struct = rtemps:ref\_temp* x shape:bound\_pair*}
{}
The descriptor is a two tuple. The first element is used when
applying translation algorithm. The second element describes
the access shape a.k.a convex polyhedron.

\domain{ref\_temp = index:int x rtype }
{}
Actually we want an array of ints of size = Rank. Hence a partial
function index -> rtype. I don't like this roundabout but I don't
have the time to look at different options. 

\domain{rtype = nonlinear:unit + linvariant:unit + lininvariant:unit}
{}
The different values that each element of reftemps can hold.
I would have prefered enumerated types.

\domain{bound\_pair = index:int x lb:Pvecteur x ub:Pvecteur}
{}
Again a partial function to simulate arrays.

\domain{comp\_sec = hull:simple\_section x complements:simple\_section*}
{}
A complementary section is a list of simple sections.
It is structured as a pair (E,{C}) for conceptual reasons.
If the second element of this pair is empty then Compsec = simple section!

\domain{comp\_desc = persistant reference x action x section:comp\_sec }
{}
A descriptor includes the array entity and the section information

\domain{comp\_desc\_set = comp\_descs:comp\_desc*}
{}
And finally the set of descriptors.
\end{document}
\end
