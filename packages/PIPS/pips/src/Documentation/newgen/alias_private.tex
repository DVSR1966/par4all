%%
%% $Id$
%%
%% Copyright 1989-2010 MINES ParisTech
%%
%% This file is part of PIPS.
%%
%% PIPS is free software: you can redistribute it and/or modify it
%% under the terms of the GNU General Public License as published by
%% the Free Software Foundation, either version 3 of the License, or
%% any later version.
%%
%% PIPS is distributed in the hope that it will be useful, but WITHOUT ANY
%% WARRANTY; without even the implied warranty of MERCHANTABILITY or
%% FITNESS FOR A PARTICULAR PURPOSE.
%%
%% See the GNU General Public License for more details.
%%
%% You should have received a copy of the GNU General Public License
%% along with PIPS.  If not, see <http://www.gnu.org/licenses/>.
%%

\documentclass{article}

\usepackage{newgen_domain}
\usepackage[backref,pagebackref]{hyperref}

\title{Alias Analysis}
\author{Nga Nguyen}
\begin{document}
\maketitle

\section{Introduction}

\domain{import entity from "ri.newgen"}
{}
\domain{import expression from "ri.newgen"}
{}

% These newgen data structure are used to propagate interprocedural alias
% information. When a formal parameter is associated with a actual
% variable by a certain call path, we assign the section and the offset of
% the actual variable to the formal one. 

% Lower and upper offsets are kept to give more information to the alias
% check phase, but when the exact offset can be translated to the callee's
% frame, we only keep this exact one. 

% 13/06/2001 NN : as we can not use preconditions (because we may have an
% alias violation in the code, so computed precondition are not true), we
% need only offset, not lower or upper offset.

% Call-site ordering is used to localize the call in the caller function. 

% 13/11/2001 NN: offset is not enough to remove some useless checks
% (because the section is not unique, useless tests on varibales in the
% same common block although there is no intersection of storage sequence,
% or in the same SUB*DYNAMIC* section, for example). So lower and upper
% bounds of the offset are added in order to simplify some false
% tests. These bounds are integers, for scalar variable: lower = upper =
% offset, for array variable: lower = &A <= offset <= upper = &A +
% size_of_array(A) 
 
\domain{call\_site = function:entity x ordering:int}
{}
\domain{alias\_association = variable:entity x section:entity x offset:expression x lower\_offset:int x upper\_offset:int x call\_chain:call\_site*}
{}
\domain{alias\_associations = list:alias\_association*}
{}
\domain{dynamic\_check = first:entity x second:entity x checked:bool}
{}

\end{document}
