%%
%% $Id$
%%
%% Copyright 1989-2010 MINES ParisTech
%%
%% This file is part of PIPS.
%%
%% PIPS is free software: you can redistribute it and/or modify it
%% under the terms of the GNU General Public License as published by
%% the Free Software Foundation, either version 3 of the License, or
%% any later version.
%%
%% PIPS is distributed in the hope that it will be useful, but WITHOUT ANY
%% WARRANTY; without even the implied warranty of MERCHANTABILITY or
%% FITNESS FOR A PARTICULAR PURPOSE.
%%
%% See the GNU General Public License for more details.
%%
%% You should have received a copy of the GNU General Public License
%% along with PIPS.  If not, see <http://www.gnu.org/licenses/>.
%%

\documentclass[a4paper]{article}

\usepackage[latin1]{inputenc}
\usepackage{newgen_domain}
\usepackage[backref,pagebackref]{hyperref}

\title{PIPS/SAC: SIMD Architecture Compiler}
\author{Fran�ois Ferrand \\
\\
        ENST Bretagne}


\begin{document}
\maketitle
\sloppy

\section*{Introduction}

This document defines and describes the data structures used by SAC, the
SIMD Architecture Compiler. SAC is a new PIPS phase, which allows to
generate code optimized for architectures supporting multimedia
instruction sets, such as MMX, SSE or VIS.

\domain{import entity from "ri.newgen"}
{}
\domain{import expression from "ri.newgen"}
{}
\domain{import statement from "ri.newgen"}
{}
\domain{import reference from "ri.newgen"}
{}
\domain{import reduction from "reductions_private.newgen"}
{}

\domain{reductionInfo = persistent reduction x count:int x persistent vector:entity}
{}

\section*{Opcodes and opcode classes}

\domain{opcode = name:string x vectorSize:int x argType:int* x cost:float}
{}

\domain{tabulated opcodeClass = name:string x nbArgs:int x opcodes:opcode*}
{}

\section*{Statement matching}

Statement matching is used to detect ``patterns'' in the code. It
works on the expression tree representing the program. The actual
patterns are read from a file, to create a matchTree that is used to
efficiently parse the expression tree. This process returns a list of
matches, indicating the various opcodes that can thus be generated,
and with which arguments.


A \verb+patternArg+ specifies how an argument is to be generated. It
can be an integer constant, with the specified value, or extracted
from the actual expression tree.

\domain{patternArg = static:int + dynamic:unit}
{}

A \verb+pattern+ identifies what to generate. It specifies the opcode
class corresponding to the pattern, as well as a list that can be used
to translate from original statement references or constants to
arguments for the opcode.

\domain{patternx = class:opcodeClass x args:patternArg*}
{}

\domain{matchTreeSons = int->matchTree}
{}

\verb+matchTree+ is a structure used to efficiently identify
patterns corresponding to a statement. When traversed, it can thus map
a statement to a list of patterns that can be used.

\domain{matchTree = patterns:patternx* x sons:matchTreeSons}
{}

A \verb+pattern+ can be translated into a \verb+match+ by mapping the
arguments properly. The argument list in a \verb+match+ is constructed
from the arguments of the statement, following the rules of pattern
arguments (list of \verb+patternArg+).

\domain{match = type:opcodeClass x args:expression*}
{}

\section*{Statement information}

\domain{referenceInfo = reference x nbDimensions:int x lExp:expression* x lIndex:reference* x lOffset:int*}
{}

The \verb+referenceInfo+ domain reprensents an easier to use
representation of reference of the form \verb+A[I+5]+. The reference 
expression has already been parsed to find all the items.


\domain{vectorElement = statement:simdStatementInfo x vectorIndex:int x element:int}
{}

\domain{statementArgument = expression x dependances:vectorElement*}
{}

\domain{simdStatementInfo = opcode x nbArgs:int x vectors:entity[16] x arguments:statementArgument[48]}
{}

\domain{statementInfo = simd:simdStatementInfo + nonsimd:statement}
{}

\domain{argumentInfo = expression x placesAvailable:vectorElement*}
{}

\section*{Transformation}

\domain{transformation = name:string x vectorLengthOut:int x subwordSizeOut:int x vectorLengthIn:int x subwordSizeIn:int x nbArgs:int x mapping:int[16]}
{}

\end{document}
\end
