%%
%% $Id$
%%
%% Copyright 1989-2010 MINES ParisTech
%%
%% This file is part of PIPS.
%%
%% PIPS is free software: you can redistribute it and/or modify it
%% under the terms of the GNU General Public License as published by
%% the Free Software Foundation, either version 3 of the License, or
%% any later version.
%%
%% PIPS is distributed in the hope that it will be useful, but WITHOUT ANY
%% WARRANTY; without even the implied warranty of MERCHANTABILITY or
%% FITNESS FOR A PARTICULAR PURPOSE.
%%
%% See the GNU General Public License for more details.
%%
%% You should have received a copy of the GNU General Public License
%% along with PIPS.  If not, see <http://www.gnu.org/licenses/>.
%%

%% PIPS faq

%% titlepage,
\documentclass[a4paper]{article}

\usepackage{epsfig,amstext,alltt,hyperref,xspace}

\title{\bf {\Huge PIPS} \\ Frequently Asked Questions}

\author{
\begin{tabular}{rl}
  Fran�ois & IRIGOIN
\end{tabular}
}

\date{September 1997}

\renewcommand{\indexname}{Index}
\makeindex

\newcommand{\PIPSMAKERC}{\htmladdnormallink{\texttt{pipsmake-rc}}{http://www.cri.ensmp.fr/pips/pipsmake-rc}\xspace}

\newcommand{\PIPS}{\htmladdnormallink{{\em PIPS}}{http://www.cri.ensmp.fr/pips}\xspace}
\newcommand{\WPIPS}{\htmladdnormallink{{\em WPIPS}}{http://www.cri.ensmp.fr/pips/wpips-epips-user-manual/wpips-epips-user-manual.html}\xspace}
\newcommand{\TPIPS}{\htmladdnormallink{\texttt{tpips}}{http://www.cri.ensmp.fr/pips/line-interface.html}\xspace}
\newcommand{\PROPERTIES}{\htmladdnormallink{\emph{properties}}{http://www.cri.ensmp.fr/pips/properties-rc}\xspace}

\sloppy

\begin{document}
\maketitle

\clearpage
\tableofcontents

\newpage
\section*{Welcome}

Welcome to the not so frequently asked questions about PIPS. Not so
frequently, because the PIPS user population, however growing, still is
small. Please send more questions to \verb/pips-support@cri.ensmp.fr/.

\section{Some of my source code is missing. What can I do?}
\index{Source File}

PIPS is designed as a whole application compiler and it complains about
missing source code for leaf modules in the call graph. In fact PIPS may
even generate a user error and stop if the missing piece of code is needed
to answer the user request. For instance, PIPS cannot compute the side
effects of procedure \verb/P/ without analyzing the code procedure
\verb/Q/ called by \verb/P/.

However, it is possible to write a minimal version of the missing routine,
called a {\em stub}, with the proper effects on the arguments and possibly
on global variables in COMMON.

If memory effects are not required, PIPS can generate stubs for the
missing routines if this option is requested with property
\verb+PREPROCESSOR_MISSING_FILE_HANDLING+ set to \verb+"generate"+.

This also is useful if some modules are not written in Fortran but in C or
in another language than is not parsed by PIPS.

\section{Why is PIPS parser so limited?}
\index{Parser}
\index{Input Language}
\index{Fortran PIPS}

Parsing is divided in multiple stages and extensive preprocessing is
available to convert Fortran features which are not directly supported by
PIPS into equivalent forms. For instance, ALTERNATE RETURN, ENTRY and
ASSIGNED GO TO statements can be converted into equivalent constructs.
Explicit return code variables and tests of these variables are added to
mimic the alternate return mechanism. Each entry is converted into a first
class SUBROUTINE or FUNCTION and a common is added to store static
variables. ASSIGN statements are converted into standard assignments to an
integer variable whose value can be tested to control a branch.

Statement functions can also be substituted.

These program transformations modify the user code and most of them are
not applied by default. The corresponding properties must be set (see \PROPERTIES{}):

\begin{itemize}

\item \verb+PARSER_SUBSTITUTE_ALTERNATE_RETURNS+

\item \verb+PARSER_SUBSTITUTE_ENTRIES+

\item \verb+PARSER_SUBSTITUTE_ASSIGNED_GOTO+

\item \verb+PARSER_EXPAND_STATEMENT_FUNCTIONS+

\end{itemize}

Formal functional are still unsupported ($Date: 1999/01/12 21:14:43 $) and
the internal representation of DATA statements is unsatisfactory.

Recursivity, which is not in Fortran 77 standard, is not supported.

\section{My include files are in a different directory}
\index{Include}

Use the Shell environment variable \verb/PIPS_CPPFLAGS/ and the usual
\verb/-I/ option to specify directories to lookup for include files.
For instance, assuming that csh is used:

\begin{quote}
\verb|setenv PIPS_CPP_FLAGS "-I ../includes -I ../../generic/includes"|
\end{quote}

The current directory is searched first. For more information, see the
preprocessing sections in \PIPSMAKERC{}.

\section{How can I execute the transformed code?}

The transformed code is stored in the current workspace by an explicit
close or by exiting \PIPS{}. The best way to do it is to open the
workspace with \TPIPS{} and to type:

\begin{quote}
{\tt make UNSPLIT}
\end{quote}

See \PIPSMAKERC for more information about the {\tt UNSPLIT}
pseudo-resource.

Otherwise, the various files can be retrieved, on a one-by-one basis, from
the workspace and renamed with a {\tt .f} suffix required by {\tt f77}.

\section{How can I replay a session?}
\index{logfile\_to\_tpips}

Each PIPS workspace contains a log. This log can be processed using the
{\tt logfile\_to\_tpips} Shell script. The output is a {\tt tpips} script
which can be replayed with the {\tt tpips} interface.

Some commands, such as {\tt cd}, {\tt open}, {\tt close}, {\tt delete},
may destroy the log file or change the current workspace. It is not
possible to recreate a global log from different workspaces.

\section{What should I read to use \PIPS{}?}

You might decide to get started using the \WPIPS{} X-Window interface and a
small Fortran program. On-line help is available.

\section{How can I process several Fortran files?}

The answer depends on the PIPS interface used. With the basic Shell
interface, the argument of the {\tt -f} option can be quoted to use Shell
filename expansion. For instance, the command

\begin{quote}
{\tt Init -d -f "*.f Src/*.f" foo}
\end{quote}

deletes a possibly existing {\tt foo} workspace and creates a new one with
all Fortran files in the current directory and in the {\tt Src}
subdirectory.

The window interface, {\tt wpips}, let you select a number of Fortran
files when you create a workspace but these files must all belong to the
current directory.

The script interface, {\tt tpips}, let you select any number of Fortran
files using the standard Shell syntax. For instance, the above {\tt Init}
command would be:

\begin{quote}
{\tt delete foo}\\
{\tt create foo *.f}
\end{quote}

The standard convention is used for include files.

\section{How can I get the dependence graphs of several modules in one command?}
\index{Dependence Graph}
\index{DG}

The script interface, {\tt tpips}, let you require several data structures
in one command. For instance, you can compute the dependence graphs for
all modules in the current workspace with:

\begin{quote}
{\tt display DG\_FILE[\%ALL]}
\end{quote}

It also is possible to specify a list of relevant modules:

\begin{quote}
{\tt display DG\_FILE[A B C]}
\end{quote}

It is not possible to submit such requests with the other PIPS user interfaces.

\section{PIPS tells me to set a property? What is it? How can I do it}
\index{Property}

Properties are used to encapsulate global variables and to fine tune the
behavior of PIPS. This fine tuning was not intended for PIPS users but for
PIPS developpers. However the property mechanism has been used to extend
PIPS parser and users should be able to define {\em some} properties (see
the
\htmladdnormallink{\emph{property}}{http://www.cri.ensmp.fr/pips/properties-rc}
documentation).

Default values for properties are defined in file
\verb+$PIPS_ROOT/Share/properties.rc+. 
% $
when \PIPS{} is installed. However,
they can be modified in every directory by creating a local file called {\tt
  properties.rc}. Properties can also be modified dynamically with the
most recent \PIPS user interfaces, \TPIPS{} and jpips.

\TPIPS users can also modify the properties in their home directory using
a file called \verb+.tpipsrc+.

\section{PIPS has just core dumped...}
\index{core dump}

It is hard to believe but not impossible. The most effective way to
receive support is to {\tt tar} the current workspace and to mail it to
{\tt pips-support@cri.ensmp.fr} after compression (e.g. gzip) and encoding
(e.g. uudencode).

\section{How can I optimize my code?}
\index{Program transformation}
\index{Code transformation}
\index{Code optimization}

\PIPS{} does not support {\em automatic} code optimization for any target. PIPS
can apply automatically some loop transformations but none of these
transformations is uniformly good for all pieces of code and for all
target machines. So it is better to let the user decide what is good and
to perform the transformation for him. This process is tedious but it can
be replayed automatically with \TPIPS{}.

For the time being, the legality of a transformation requested by the
user is not checked with respect to dependencies.

The loop transformations include loop interchange, loop unrolling, full
unrolling, hyperplane method, tiling,... See \PIPSMAKERC{} for the
exhaustive code transformation list.

Some transformations are applied in a systematic way: the different
parallelization phases, loop distribution (although it is not an
optimizing transformation), statement atomization (because it is a
pre-pass to some parallelization phase), forward subsitution (same reason)
and scalar and array privatizations because they cannot degrade
performance. Dead-code elimination and constant propagation also are
performed systematically.

The complexity analysis, which is automatic like all analyses because they
cannot degrade performance, provides some static symbolic information
about the impact of a given code transformation on the number of executed
operations.

\section{This loop should be parallelized!}
\index{Parallelization}

Many conditions must be met for a loop to be declared parallel and PIPS,
as a research tool, offers (too) many options. By default, the simplest
algorithms are used but this may be changed by customizing
\verb/pipsmake.rc/ and \verb/properties.rc/.

Is your loop pretty simple and legal for Allen\&Kennedy parallelization
algorithm? This means no test, no branch within its body, and no label on
the DO statement because the parallel code generation ignores control
dependences and regenerates the loops around each strongly-connected
component (SCC) of the dependence graph.

If your loop body is not a simple sequence of assignments, you should
select another parallelization algorithm such as
\verb/coarse_grain_parallelization/ (explore the option menus of \verb/jpips/ and
wpips or use the \verb/activate/ command with \verb/tpips/).

Some trivial dependences may require a preliminary privatization. For
temporary scalars, apply the code transformation
\verb/privatize_module/. If temporary arrays are used, an array
privatization is necessary. The privatization information is based on IN
and OUT regions which are computed using MUST READ and MUST WRITE
regions... PIPS does not offer compound commands and, regardless of the
PIPS user interface you are used to, several commands are necessary to
perform a scalar or array privatization followed by a parallelization.

Finally, dependence tests use more or less information. If contextual
information is necessary to find out that a loop is parallel, a dependence
test using preconditions (\verb/rice_semantics_dependence_graph/) or
regions (\verb/rice_regions_dependence_graph/) should be selected.
Furthermore, regions depend on transformers which, like preconditions, may
be computed in different ways. The most accurate information is gathered
with the two selections \verb/transformers_inter_full/ and
\verb/preconditions_inter_full/. Induction variables are found and can be
visualized in preconditions and used in dependence tests. But they are not
substituted and may carry the last dependence arc which prevents a loop
parallelization.

\section{This loop should not be parallelized!}
\index{Parallelization}

A sequential loop may be parallelized by PIPS if it is never executed
because it is part of some dead code. The information is carried by the
preconditions or by the loop bounds and used in the dependence systems. As
a result, no dependence is found because there are no iterations and the
loop is declared parallel.

A loop with only one iteration also is always parallel, regardless of the
references in its body.

This is surprising for the user and we might change PIPS behavior so as
not to parallelize useless loops. But this would require an extra
feasibility test which will slow down PIPS and this would keep sequential
trivially parallel loops which is another source of surprise!

\section{How can I clean up my code?}
\index{Declaration}
\index{Use-def elimination}
\index{Dead code elimination}
\index{Constant propagation}
\index{Partial evaluation}

After several transformations, a piece of code may contain useless
declarations and statements. PIPS does not offer a magic clean-up function
because the process is iterative and lengthy. Four code transformations
are available for cleanup:

\begin{itemize}

\item \verb/clean_declarations/: get rid of unreferenced variables

\item \verb/use_def_elimination/: get rid of useless definitions

\item \verb/suppress_dead_code/: dead code elimination

\item \verb/partial_eval/: constant propagation

\end{itemize}

There is no best order of application but \verb/clean_declarations/ should
be performed last. If many modules must be processed, it is necessary to
perform a concurrent apply to avoid an unbearable number of recomputations
(\verb/capply/ command for \TPIPS{}). Each time a module code is touched,
effects, transformers and preconditions should be recomputed across the
whole application to maintain a full consistency. Since these
transformations do not modify the semantics of the modules, a relaxed
consistency can be used and a concurrent application of the
transformations is legal.

Note that constant propagation can improve the preconditions by making
some non-linear expressions linear. The best result is obtained with a fix
point of these transformations. If a polynomial of degree $k$ can be
replaced by an affine expression, no more than $k-1$ rounds of
transformations are necessary.

\newpage


{\small
\bibliographystyle{plain}
\bibliography{faq}
}

%\printindex

\end{document}
\end
