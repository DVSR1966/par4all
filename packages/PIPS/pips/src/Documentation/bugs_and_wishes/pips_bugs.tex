%%
%% $Id$
%%
%% Copyright 1989-2014 MINES ParisTech
%%
%% This file is part of PIPS.
%%
%% PIPS is free software: you can redistribute it and/or modify it
%% under the terms of the GNU General Public License as published by
%% the Free Software Foundation, either version 3 of the License, or
%% any later version.
%%
%% PIPS is distributed in the hope that it will be useful, but WITHOUT ANY
%% WARRANTY; without even the implied warranty of MERCHANTABILITY or
%% FITNESS FOR A PARTICULAR PURPOSE.
%%
%% See the GNU General Public License for more details.
%%
%% You should have received a copy of the GNU General Public License
%% along with PIPS.  If not, see <http://www.gnu.org/licenses/>.
%%
\documentstyle[12pt]{article}
%\input{/usr/share/local/lib/tex/macroslocales/Dimensions.tex}
\newcommand{\titre}{PIPS BUGS and WISHES \\
                    Version 1.1 \\
                        Draft (November 21, 1991)}
\newcommand{\auteur}{Franc,ois Irigoin\\
                et al.}
\newcommand{\docdate}{10 January 1992}
\newcommand{\numero}{Exxx}

\begin{document}

%%\input{/usr/share/local/lib/tex/macroslocales/PageTitre.tex}

\tableofcontents

\newpage

\section*{Introduction}

Should be augmented with meeting results: solved problems and new
problems and new priorities.

\paragraph{Priorities as defined in May 1991:}

\begin{itemize}

  \item programming environment \marginpar{(PJ/FI/BB)} 

  \item parallelization correctness \marginpar{(PJ)} 

  \item parser robustness \marginpar{(FI)} 

  \item pips process size: pipsdbm and pipsmake 

  \item evaluation and quantification and benchmarking \marginpar{(YY)}

\end{itemize}

\paragraph{Goals for 1992:}

\begin{itemize}

  \item PIPS-CRAY

  \item Yi-Qing's PhD;

  \item Lei's PhD;

  \item CEA's goals;

  \item PIPS evaluation: interprocedural analysis, interprocedural
        parallelization;

\end{itemize}

Non-regression: please fill in Validate and Bugs in a systematic way!

\section{Atomizer (Alexis Platonoff)}

\section{Bootstrap}

\begin{enumerate}

  \item intrinsic \verb+LOG+ is missing; 27 February 1992; to be added
        in effects library too; fixed by FI, 10 March 1992;

  \item same thing for \verb+LOG10+; fixed by FI, 10 March 1992;

  \item intrinsics with a varying number of arguments are encoded
        as having 0 arguments; no good solution has yet been found;
        FI, 18 March 1991 (this would also solve the next problem);

  \item \verb+STOP+ was declared as always having one parameter;
        this was changed to {\em zero} parameters; modified by
        FI, 19 March 1992

\end{enumerate}


\subsection{Wishes}

\begin{itemize}

  \item information about intrinsics, scattered in syntax, ri-util,
        bootstrap and effects should be centralized, probably in
        ri-util; FI, 27 March 1992;

\end{itemize}

\section{Call Graph (Lei Zhou)}

\begin{enumerate}

  \item make a difference between the Call Graph based on symbol tables
        and the elided interprocedural Control Flow Graph, which is
        currently called {\em call graph}

  \item present\marginpar{(LZ/FI)} {\em call graph} is broken by
        unstructured; no relation with the source program; use {\em
        parsed\_code} instead? Faster... but not as fast as a real call
        graph based on declarations; 

  \item property \verb+ALL_DO+ to see nesting levels regardless of procedure
        calls; i.e. display another kind of squeletton for the ICFG:
        all loop control structure, DO-loops and while, every control
        structures, including IFs;

  \item unstructured are not displayed and so WHILE loops are not
        visible (i.e. cycles in the control graph);

  \item function calls are not taken into account; FI, 29 July 1992;

\end{enumerate}

\section{Chains}

Pierre Jouvelot.

\subsection{Bugs}

\begin{enumerate}

  \item local variables should be handled as global ones; else,
        statement grouped by a private declaration may be distributed;
        this doesn't explain the erronneous behavior observed; problems with
        B should also occur;

  \item \verb+KEEP_READ_READ_DEPENDENCE+: \verb+pips_error in dept_type+;

  \item test added so as never not taking into account proper effects or
        regions of blocks; this is due to semantics difference between
        proper effects and regions which look a bit more like cumulated
        effects; FI, 5 August 1992;

\end{enumerate}

\subsection{Modifications}

\begin{enumerate}

  \item two new routines added to compute base chain computation on
        proper effects (as before) or on regions; FI, 5 August 1992;

\end{enumerate}

\section{Code-Generate}

This library was merged into library Transformations by Bruno Baron.

\subsection{Wishes}

\begin{enumerate}

  \item Perform priv is required to detect that the inner loop is
parallel in the following case:
\begin{verbatim}
      program dist
      real a(100,100)

      do 10 i=1,100
         m = j
         do 20 j=1,m
            a(i,j) = m
 20      continue
 10   continue
      end
\end{verbatim}
The fast privatize routine is unable to detect that J is local.  But
this may not be correct. Depends of the semantics of DOALL.
      
\end{enumerate}

\section{Complexity}

The complexity library is managed by Lei.

\subsection{Bugs}

\begin{enumerate}

  \item Do not substract negative execution time for loops with
        non-satisfied conditions

  \item The callee's complexity result which contains the COMMON
        variables as its component CANNOT be added to the caller's
        complexity which contains the same COMMON variables.
        LZ thinks it is beacuse of the restriction of module name.
        like: FOO:IL and TOTO:IL , they can't be added. 24/02/92

  \item pips core dumps when writing ZPROAX.COMPLEXITY after an
        interprocedural analysis;
        see \verb+zproax.f+ in \verb+Bugs/Complexity+;
        FI, 9 August 1992;

  \item Fail to pass FFT algorithm:
        The sample version is kept in Bugs/Complexity/fstr2.f .
        There are two problems: 1) N is unknown in Loop 12, that is to
        say, the precondition is not available.
        2) MMAX is unknown, it is no surprise beacuse it is updated
        on each iteration. Then MMAX has the value ``UR'' as final
        complexity results. Loop 13 and Loop 12 both have MMAX as 
        loop bound and ``UR'' is kept out of C3/polynomial library,
        so two polynomials can not be added, let alone multiplication.
        What we can get is ``Bad internal polynomial representation''.
        LZ, Oct. 8, 92

  \item For tmines.f program in ~pips/Pips/Expe*/Complexity,
        when {\bf setenv COMPLEXITY\_DEBUG\_LEVEL 9} is set,
        {\bf Display -m calcg comp} core dumped.
        If the debug level is not set, it works.
        LZ, 08/06/04

\end{enumerate}

\subsection{Wishes}

\begin{enumerate}

  \item The loop bound cannot be the value of an array.
        When the inner loop's bound are independent, we have no way to
        evaluate the outer loop's body complexity.
        That is, we cannot use a polynomial form to represent the inner
        loop's complexity. 
        Though it's not a normal case. FI: Example?

  \item How should target machines be modeled? Define a NewGen structure
        and an external format? See \verb+model.rc+ in WP65.

  \item Where should model files be stored? How
        should they be accessed? Everything should be in Documentation
        in a LATEX form...

  \item Evaluate parallel complexity for different kind of processors
        (unbounded PRAM machine, multiprocessor, vector processor,...)

  \item Evaluate space and communication complexities

  \item Evaluate dynamic complexity: how much does each statement contribute
        to the total complexity? Proper complexity times number of executions.

  \item Propagate complexities downwards so as to know how many times a
        statement is executed

  \item set up cost tables to compute the number of (visible) floating point
        operations and/or memory requests (loads and stores); intrinsics
        like \verb+SIN+are not taken into account;

  \item define a mechanism to select a particular cost table;

  \item build and test a cost table for Sparcstation 1;

\end{enumerate}

\section{Control}

Build the control graph and generate the \verb+CONTROLIZED_CODE+.

\begin{enumerate}

  \item FORMATs which are unreachable are dead-code eliminated;
        see \verb+Bug/Control/tformat.f+

  \item unstructured code seems to unstructure nearby structured code;
        see \verb+dyna/vt.f+; the bug hasn't been reproduced
        on smaller cases; FI, 5 August 1992;

\end{enumerate}

\subsection{Wishes}

\begin{enumerate}

  \item \marginpar{PJ} make unstructureds as small as possible;
        see TMINES and its semantic analysis for instance or \verb+split.f+
        and the complexity evaluation; BAS should be preserved; prologue
        and postlogue should be moved out of the unstructured;

  \item recognize structured constructs like
        IF/THEN/ELSE? Should be performed by an external tool: 

  \item STOPs in subroutines and functions are not handled at 
        all (core dumped); control effects?

\end{enumerate}

\section{Conversion}

Who's responsible?

\section{Dg-util (Utilities for the dependence graph)}

\section{Effects}

\subsection{Bugs}

\begin{enumerate}

  \item improve \marginpar{(FI)} reshaping handling for numerical array
        bounds: T(961) and U(31,31) in spec77 for instance 

  \item recompute
        effects a second time, using information about constant
        propagation \item document the origin of {\em bad array reshaping} 

  \item
        add a property to cope silently with * dimension declaration; CEA
        suggestion; is it possible to trust the user better?  

  \item bug with spec77/del4 (dim transformed in foodim):
        \verb+pips_assertion_failed in reference_ith_index() +

  \item use order on cumulated effects to
        destroy useless read effects; should be also useful for proper effects
        of composite instruction like DO; may have been done since 
        \verb=Bugs/Effects/effect.f= is correctly parallelized (FI, 17 January
        1992)

  \item add an array of logical
        units (this entity should be created at parse time) and generate
        write effects on this entity for IO statements

  \item core dumps in PerfectClub/W.WS with function {\em gfidi}:\\
        \verb+pips assertion failed in reference_ith_index+

  \item demo/ifp-pips.f, module onde24 with {\tt PIPSDBM\_DEBUG\_LEVEL=5}:
\\ 
\verb+[db_put_resource] Checking Newgen Consistency for resource
PROPER_EFFECTS(ONDE24):+ \\
\verb+Newgen User Error -- gen_check: Type clash (expecting entity,
getting unit)+ \\
Could probably be illustrated with another code.

  \item generate a \verb+MAY+ effect instead of a \verb+MUST+ effect when
        an array element is referenced; this produce wrong \verb+array kill+
        information for subroutine; see \verb+Bugs/Effects/effect2.f+
        (FI, 6 May 1991);

  \item WRITE has a \verb+read+ effects on its IO-list; see for instance
        \verb+ker22.f+

  \item \verb+TOP-LEVEL:PAUSE+ is not recognized by 
        \verb+proper_effects_of_call()+;
        see \verb+Bugs/Syntax/pset.f+; FI, 7 February 1992;
        the bug was in \verb+bootstrap+ where \verb+PAUSE+ was not
        recognized as intrinsics; \verb+PAUSE+ was not recognized
        in effects either;

  \item control effects are ignored in \verb+PAUSE+ and \verb+STOP+;
        FI, 7 February 1992

  \item effects of IO lists are redundant; the same item is counted
        n times, where n is its position in the io list;
        this has a dramatic effect on use-def chain counts;
        see \verb+Bugs/Effects/udread.f+; YY et FI, 23 June 1992;
        fixed by FI, 23 June 1992;

  \item typo for \verb+OPEN+ with \verb+ACCESS+ option; observed
        with \verb+dyna.f+; fixed by FI in \verb+io.c+, 30 July 1992;

  \item output buffers in \verb+store_text_line+ were filled in without
        checking for overflow; they were allocated in the stack so
        as to create havoc on overflow; observed and fixed - size
        increase and overflow check - by FI,
        31 July 1992;

  \item print stars '*' for unknown array indices when prettyprinting
        effects; FI, 2 August 1992;

  \item This last modification seems to put the same array many times in
        the same list; maybe, this was an old bug; when are the stars
        added? When the effect is built? When the effect is printed?
        This shows in cumulated effects: each proper effect is
        transformed into a cumulated effect without reduction for
        redundant cumulated effects; FI, 27 April 1993;

  \item wrong reshaping on \verb+Y+ in \verb+zproax+; a macro was
        called with a functional argument expansed twice: too bad
        for the side effects; FI, 11 August 1992;

  \item proper effects of DO's include a MAY-READ of the loop index
        due to its incrementation; this does not let the MUST-WRITE
        of the loop index kill use-def chains that should be killed
        and legal privatizations are not possible and parallel loops
        are kept serial; this is due to the compound nature of the PIPS DO
        instruction;

        It is difficult to define a semantics for the effects ordering
        as the masking does not always occur (see test instructions for
instance).
        It is hard to add a masking system to the effect computation.
        It was not chosen to give a copy-in copy-out semantics to
        private variables.
        It was chosen to handle the DO as a special case and to
        suppress the MAY-READ effect on the loop index.

        See \verb+Validation/Effects/loop.f+. FI, RK, March 12, 1993.

  \item the effect print buffer may be too small; size increased to
        2048; overflow check added; FI, 26 May 1993;

  \item \verb+effects_conflict_with_entity()+ modified to call
        \verb+entity_conflict_p()+ instead of performing a pointer
        comparison; \verb+effects_read_or_write_entity()+ added;
        FI, 29 JUne 1993
        
\end{enumerate}

\subsection{Wishes}

\begin{enumerate}

  \item bad reshaping wrongly found for \verb+A+ in \verb+zproax.f+; function
        \verb+translate_array_effect+ does not handle arrays of dimension
        greater than or equal to three because the direct test is not
        linear; a new test should be developped to test that the first
        k-1 dimensions have the same sized in the formal and actual arguments
        and that formal dimension k is less then actual dimension k;
        the formal dimension is k; the actual dimension must be greater
        than or equal to k; equality between two expressions can be proved
        by checking that their difference evaluates to 0 (\verb+sc_eval+);
        FI, 11 August 1992;

  \item effects should be alphabetically sorted by variable names to
        improve readability; FI, 1 Dec. 1992;

  \item \verb+MUST+ effects should mask \verb+MAY+ effects; 
        for instance, the same
        statement should not have a \verb+MUST WRITE T+ and a
        \verb+MAY WRITE T+ simultaneously; the \verb+MAY+ effect
        should be deleted; let say that different kind of effects could
        be computed;

  \item control effects (stop/exit) should also be computed;

  \item effects should take into account the declarations; see
        \$PIPSDIR/Test/Bugs/Semantics/contrex3.f for an example. FI, BA,
        9 february 1994.
        

        
\end{enumerate}

\section{Evaluation of PIPS}

\begin{enumerate}

  \item see what happens in OASSEM (cr2cnf) after privatization;

  \item parser core dumps on SPHOT (PerfectClub)

  \item PerfectClub

  \item Riceps

  \item SIDOLO (YY)

\end{enumerate}

\section{Generation}

Who's responsible?

\section{Hyperplane}

Hyperplane method, meaning linear scheduling, i.e. unimodular transformations.
Include loop interchange.

\begin{enumerate}

  \item \marginpar{YY} should include a test for dependences

  \item \marginpar{BB} should include a machine dependent cost 
function for interchange decision

  \item should handle not perfectly nested loops
\end{enumerate}

\section{ICFG}

Interprocedural Control Flow Graph.

\section{Linear}

\verb+linear.h+ was suppressed. It was redundant with ri.h because
of NewGen requirements for external types. It was not properly
installed as not derived from any particular library. FI, December 91.

\subsection{Vecteur}

\begin{enumerate}

  \item integer overflow are usually not detected; one test was
        added in \verb+vect_cl2()+

\end{enumerate}

\subsection{Contrainte}

\begin{enumerate}

  \item \verb+contrainte_normalize+ did not handle correctly constant terms
in inequalities; the new bound was off by 1, in the safe direction; fixed
by FI, October 91;

  \item coefficients in pairwise elimination were not reduced by their
        GCD; fixed by FI, December 91;

\end{enumerate}

\subsection{Syste`me de Contraintes}

\begin{enumerate}

  \item \verb+sc_rm_empty_constraints()+ did not update the numbers of
        equalities and inequalities (FI, Oct. 30, 1991);

\end{enumerate}

\begin{enumerate}

  \item modify the grammar so as to allow comments linear system
        description files (FI, 8 August 1992)

\end{enumerate}

\subsection{Polye`dres}

\subsubsection{Bugs}

\begin{enumerate}

  \item the convex hull operator core dumps with (Alexis Platonoff,
        6 August 1992, using CEA version):

\begin{verbatim}
        {PHI3<=60, 1<=PHI3, PHI2<=21, 1<=PHI2, PHI1<=52, 1<=PHI1,
         PHI3==1, PHI2==4, PHI1==4}

        {1<=PHI1, PHI1<=52, 1<=PHI2, PHI2<=21, 1<=PHI3, PHI3<=60,
         PHI1==5, PHI2==5, PHI3==1}

Assertion failed: file "binaires.c", line 181
/java/home3/pips/Pips/Utilities/Pips: 14764 Abort - core dumped
Exit code for pips: 134; exiting /java/home3/pips/Pips/Utilities/Pips
\end{verbatim}

  \item the convex hull operator is too slow for large systems as produced
        by semantic analysis on \verb+aile.f+; a new version
        \verb+sc_fast_convex_hull+ was introduced, but it's slower
        for small systems; FI, 9 August 1992;

  \item the convex hull operator does not eliminate redundancy properly;
        see for instance the precondition for the call to \verb+NORM+
        in \verb+geom.f+ (part of \verb+aile.f+)

\end{enumerate}

\subsubsection{Wishes}

\begin{enumerate}

  \item reproduce optimization for small systems that are in 
        \verb+sc_enveloppe+ into \verb+sc_fast_convex_hull+;
        FI, 9 August 1992;

\end{enumerate}


\section{Link(er)}

\begin{enumerate}

  \item could \marginpar{(BB/LZ)} it be executed before the controlizer
to get a nicer interprocedural control flow graph? Remi may have decided
to activate the controlizer just after the parser to save a load/unload
sequence?

  \item LZ: bug linked to variable name change from T to Q

  \item calls in unreachable code seems are not linked; see
        \verb+Bugs/Newgen/simple4.f+; this bug may disappear because the
        link is going to be executed in the parser; 
        Franc,ois Irigoin, 3~February~1992;

  \item core dumps with an illegal branching inside a loop; see
        \verb+Bugs/Link/branch.f+; Bruno Baron, 31 mars 1991.
        Pips is not supposed to work on illegal fortran programs.
        AL, 12 august 1992.

\end{enumerate}

\section{Loop Distribution}

No library. Included in Allen \& Kennedy parallelization algorithm and
as such coded in \verb+rice+.

\begin{enumerate}

  \item induction variables definitions are distributed when they appear
        in inner loop bounds; difficult to correct because of Allen \&
        Kennedy algorithm; PMACS: lots of dependence arcs are added between
        the index definition and the loop body; it was decided not to do
        that in PIPS for performance reasons;

  \item definition of privates variables are sometimes distributed wrt
        their uses; see example mailed to Pierre; corrected

  \item useless privatization after loop distribution; Re'mi is likely to
        reuse the same DO statement...

\end{enumerate}

\section{Loop\_normalize}

\section{Loops\_Interchange}

Mispelled...

\section{Misc(ellaneous)}

\subsection{Bugs}

\begin{enumerate}

  \item \verb+concatenate+ does not check for overflows; observed by FI,
        31 July 1992; fixed by FI, 1 August 1992; wish: \verb+concatenate+
        should automatically malloc'ed a larger buffer when necessary...

\end{enumerate}

Is it a library or a special section?

\begin{enumerate}

  \item define an installation procedure for Paris; how is the PIPS account
        doing? Look at make-tape?

  \item PIPS documentation has to be translated into English...

  \item cycles between libraries: misc (get\_bool\_property) and properties
        (concatenate and pips\_error); what the best solution with these
        low-level libraries? Fuse them? Duplicate some functions with
        diferent names in different libraries? Temporary fix: these libraries
        are called twice;

        Bruno suggests to fuse the two libraries. Properties is a low level
        library. It could be used regardless of PIPS. It contains only
        one file and a few functions. Any disagreement?

        Another solution would be to use a \verb+getenv()+ in \verb+misc+
        to decided whether or not to print messages.

\end{enumerate}

\section{Movements}

\section{NewGen}

Pierre is responsible for NewGen maintenance.

\subsection{Bugs}

\begin{enumerate}

  \item support for mappings over untabulated types?

  \item proper handling of double quotes (onde24.f); fixed, PJ;

  \item consistent code is written using \verb+gen_write()+ and is re-read
        and found not consistent anymore; see onde24.f; summarized
        in \verb+Bugs/Newgen/tformat.f+; further summarized in
        \verb+Bugs/Newgen/simpl4.f+; FI, January 1992; the problem
        is due to Newgen handling of forward referencex; a fixed order
        is now imposed for file input: tabulated domain have to be
        read first; fixed by PJ, February 1992;

  \item SPACE characters are lost in strings delimited by double quotes;
        see the parsed code for tformat (with double quotes used in
        format 1100; FI, January 1992; the bug may be due to PIPS parser;

  \item \verb+gen_free()+ does not go across persistant arcs but tests for
        sharing beyond them; this is highly inefficient when effects
        are freed; fixed by PJ, February 1992;

  \item \verb+gen_read+ does not accept forward references in a tabulated
        domain, entity, with a recursive data structure; FI, 10 February 1992;
        fixed by PJ;

  \item \verb+gen_read+ does accept forward references in a tabulated
        domain, entity, with a recursive data structure; but it does
        not check that they are all filled in at the end; FI, February 1992;
        fixed by PJ by adding \verb+gen_read_and_check_tabulated+; FI:
        I do not understand why this facility is not available at the
        \verb+gen_read+ level? Because the check is already available for
        non-tabulated domain?

  \item \verb+gen_free()+ does dereference a free area and core dumps;
        fixed by PJ, July 1992;

  \item \verb+read.l+ does not check \verb+yytext+ overflows; fixed, FI,
        July 1992;

  \item untested overflow for tabulated domain in \verb+enter_tabulated_def()+;
        FI, 28 July 1992;

\end{enumerate}

\subsection{Wishes}

\begin{enumerate}

  \item support for mappings over untabulated types?

  \item efficient handling of persistant arcs in \verb+gen_free()+
        (and in its call to shared pointers).

  \item better density for NewGen files; for instance, \verb+hash_tables+ are
        stored as dense arrays and, as a consequence, half of entity files
        are likely to be full of \verb+#]null+ to represent empty buckets;

  \item check for disk full condition; 

  \item mapping.h : MAKE\_STATEMENT\_MAPPING should take no argument in
        order to be coherent with the make functions usually generated
        by NewGen; fixed, BA, August 1993. 

\end{enumerate}

\section{Normalize}

Should be merged into Ri-util (20 October 1991). FI in charge? Done,
Dec. 12, 1991.

\newpage

\section{Pips}

Batch interface. The program is named \verb+pips+ and the script shell
used to make it more user-friendly is named \verb+Pips+.

Provides support for \verb+Init+\verb+Select+n, \verb+Build+ and
\verb+Perform+. See also \verb+Delete+. See man(l) for details about options.

\subsection{Bugs:}

\begin{itemize}

  \item varying exit codes; no explicit \verb+exit()+ was executed;
        fixed by Bruno Baron, 27 March 1992.

  \item illegal instruction in exit() avec ifp.f; Bruno Baron, 16 June 96;
        the bug seems to have been an overflow in \verb+Newgen/read.l+;
        the code generated by Lex does not check for overflows or
        \verb=yytext=;
        fixed by FI, 19 July 1992;

  \item no \verb+-v+ option for Perform; non-uniform interface;
        fixed by FI, 23 July 1992;

  \item \verb+Display+ advertizes a non-existing \verb+para+ option
        and do not display the existing \verb+cray+ option;
        found and fixed by FI, 23 July 1992;

  \item modified to accept many \verb+-b+ options so as to compute
        more than one result per call (\verb+tpips+ should be revived
        for large applications); FI, 4 August 1992;

\end{itemize}

\section{Pipsdbm}

\begin{enumerate}

  \item there is no need for absolute file names; they should be relative
        to the location of the database schema file;

  \item xxx.schema and xxx.pipsmake should be hidden in xxx.database.
        ( Done by LZ Sep.91 )
        
  \item pipsdbm should be rewritten; who is interested in it? BB.

  \item builder\_map: should be derived automatically from pipsmake-rc.f.tex

  \item shall we keep the PURE/IMPURE request parameter or move the problem
        upwards in pipsmake? Is it possible? When an IMPURE request is made,
        pipsdbm copy the existing resource on disk, read it back and
        return a pointer towards the copy. This means that all data structures
        based on it are potentially flawed. As an example, suppose you
        get a module code with the impure flag set up. The dependence
        graph that was built before on the module code is not consistent
        with the copy: its nodes point towards the initial copy of the
        code and its arcs point towards the initial references.

        In the same way, statement mappings are likely to be unusable
        after an IMPURE request.

        The solution to this problem is linked to a NewGen extension to
        support absolute naming and some kind of type (or object?) hierarchy.
        All object should be ...?!?

        Fixes (kludges!):

  \begin{enumerate}

    \item to avoid problems with references in \verb+rice+, a special
        test
        is performed in \verb+db_unload()+ so as not to free the
        statements
        of a module;

    \item \verb+Wp65+ modifies the code but performs a \verb+pure+
        get which is OK to generate code in batch mode but kills
        \verb+wpips+

    \item \verb+Privatize+

    \item \verb+Loop Interchange+

    \item \verb+Partial Eval+

  \end{enumerate}

  \item \verb+put+ with \verb+PIPSDBM_DEBUG_LEVEL+ higher than 5 calls
        \verb+gen_consistent_p+ on \verb+DBR_PRECONDITIONS+ which is
        not a Newgen object; fixed, FI, 11 February 1992;
        \verb+DBM_SUMMARY_PRECONDITION+ is also not tested (although
        it should theoretically be tested);

  \item empty lists seem not to be well handled; pipsdbm behaves as
        if the resource was non-existent; see \verb+ifp.f+ and
        \verb+openfile+ effects: the latter are always re-computed
        whenever they are needed; this seems to be due to the fact
        that an empty list is represented by NIL; FI, 22 July 1992;
        problem with \verb+string_undefined+; fixed, FI, 3 August 1992;

  \item calls to unput() from within a MAP over the resources;
        the current cons in the MAP is destroyed by unput()...
        fixed by FI, 20 July 1992;

  \item \verb+REGIONS+ and \verb+SUMMARY_REGIONS+ added; FI, AL,
        4 August 1992;

\end{enumerate}

\subsection{Wishes}

\begin{enumerate}

  \item add a dirty bit to resource description to avoid useless copy back;
        this will introduce a higher risk of non-consistency since 
        resources can be modified although pipsdbm is not notified;
        FI/BB, 21 February 1992;

\end{enumerate}

\newpage

\section{Pipsmake}

Modifications of \verb+pipsmake-rc.f.tex+ are recorded here along with
information regarding library \verb+pipsmake+.

\subsection{Bugs}

\begin{enumerate}

  \item unique type {\tt text} for all displayable objects; why should
        we have special cases like parallel code and call graph?!?
        automatic display as soon as a function producing a text object
        is selected (no more {\tt view}?)? Why should we have to distinguish
        between them? Because they might be parsed again later? One
        reason is to generate wpips menus;

  \item direct display of text objects: no double selection like
        {\em statement + transformer} and {\em view sequential}; it's
        not practical

  \item hash tables on statements are wrong as soon as the code is saved
        on disk (i.e. after a program transformation or a
        parallelization); the code is saved on disk when pipsdbm is called
        with an IMPURE spec. and reloaded in memory at different addresses;
        one could think of unloading every data based on code when code
        is modified (could it be infered automatically by an extended
        NewGen with mappings?); one could also use statement\_ordering as
        a storage invariant key in the statement\_mapping package;
        it is more a pipsdbm problem than a pipsmake problem;

  \item cope with formal functions by assuming that they have no side
        effects?

  \item use network parallelism?

  \item blockdata exploitation?

  \item vfork() for close\_module()? It's a pain to demo pips each time
        you've used the main! Bug generating feature?

  \item a way to make macros, i.e. to chain privatization and
        parallelization would be interesting for some demonstration and
        for naive users; how could we get:
        \begin{quote}
                parallelize(privatise(MODULE.code))
        \end{quote}
        without having to write C code; we need a o operator! on
        pourrait ajouter a pips une option -c pour donner une liste de
        Commandes (select, build et perform); 
        pipsmake is orthogonal wrt Parafrase pass list; 

  \item we need a compile-make to write (temporarily) useless objects
        onto disk and to minimize memory need (see aile.f); similar to
        life span and register allocation?

  \item correctness bug spotted by Bruno; rules should be checked when expansed
        to make sure they still expand in the same way (CALLEES might have
        changed for instance)

  \item correctness problem with \verb+SUMMARY_PRECONDITION+; \verb+pips_make+
        should explictly destroy any resource detected as inconsistent; this
        would prevent \verb+pips_dbm+ from providing stale resources; this
        should never happen with a {\em clean} make mechanism, but we have
        to be not so clean when resources like summary preconditions are
        incrementally computed; see \verb+Bugs/Semantics/eval.f+, unless
        it's in \verb+demo/Vienna+: preconditions are not accurately recomputed
        after a \verb+Perform eval+ which linearized part of the code;
        FI, 6 August 1992;

\end{enumerate}

\subsection{Wishes}

\begin{enumerate}

  \item incremental source file addition: very frustrating when three
        lines are missing out of 4000; at least, a list of missing sources
        should be provided; there is a need for some user interaction:
        \verb+query_user+; FI, each time a new program is analyzed!

  \item user file restoration (same file names, same content, same order);
        keep some additional information: user\_file/source\_file/rank;
        MODULE.user\_file

  \item \verb+builder_map+ should be automatically derived from
        \verb+pipsmake-rc.f.tex+

\end{enumerate}

\subsection{Modifications}

\begin{enumerate}

  \item old rule \verb+chains+ replaced by \verb+atomic_chains+ (which
        is equivalent and \verb+region_chains+ which let pips build the
        use-def chains and the dependence graph on top of regions;
        FI, 5 August 1992;

  \item rules for icfg are modified to enforce recursive calls

\end{enumerate}

\section{Prettyprint(er)}

Managed by Lei Zhou, \verb+<zhou@cri.ensmp.fr>+.

\subsection{Bugs}

\begin{enumerate}

  \item SUBROUTINE and FUNCTION calls with no arguments should not be
        displayed with a parenthesized empty argument list.

  \item redundant parentheses should not be displayed; unbearable with
        complex expressions (i.e. intricate); 
        AC in charge, April 90; fixed by LZ, November 91;

  \item some COMMENTs disappear in onde24.f; maybe a parser problem,
        or a controlizer problem, or...

  \item a function to print all declarations is missing; necessary when some
        code is generated (i.e. Atomizer, wp65); done: new version of the
        prettyprinter written by Lei Zhou (October 1991);

  \item there seems to be no way to know whether a module is a PROGRAM
        or a SUBROUTINE; the problem is usually avoided by keeping all
        declarations in a string called \verb+decls_text+; in doubt,
        SUBROUTINEs with no arguments are deemed to be PROGRAMs (Lei Zhou,
        October 1991);

  \item wrong user line numbering (detected by Bruno Baron, 23 October 1991);
        observed again by BB, November 12; use {\em validate} to get
        the bug; moved to \verb+Syntax+ (FI, 3 January 1992);

  \item \marginpar{LZ}
        no \verb+ENDDO+ after \verb+DOALL+; fixed by LZ 17/01/92. 
        Initial loop label should only be preserved in comments;
        ( Note: when the loop has to be distributed , loop label and
        \verb+CONTINUE+ have been suppressed in 
        \verb+PARALLELIZED_CODE+, How can
        we pick these things up to make them into comments.)

  \item \marginpar{LZ} missing \verb+ENDDO+ after \verb+DOALL+;
        ( Same thing, See previous item LZ , 17/01/92 )

  \item \marginpar{LZ} option needed to avoid printing statement number in
        columns~73-80 (November 8, 1991);

  \item loop increments equal to 1 are not printed anymore (LZ)

  \item statement ordering missed before the IF statement.
        observed by LZ. 20/02/92 in \verb+Bugs/Prettyprint/intpl.f+;
        likely to be a bug in PJ's stuff to prettyprint controls (FI);

  \item remove the confusion between para, para77 and para90; para should
        disappear from pipsmake.rc; Lei is in charge (17 January 1992);

  \item RETURN statements are kept implicit whenever possible;
        FI, AL, 16 July 1992;

  \item the priority between the integer division and the integer
        multiplication is not observed; useful parentheses as in
        \verb+2*(I/2)+ disappear; FI, 10 July 1993

\end{enumerate}

\section{Privatize}

\begin{enumerate}

  \item with nested loops internal private variables are also privatized
        in the outer loop (to be checked);

  \item privatization at the block level, or at least at the procedure
        level would be useful for wp65; at the procedure level, interprocedural
        effects might be enough;

\end{enumerate}

\section{Programming Environment}

See report dret146.

\begin{enumerate}

  \item get PIPS\_LIBS from the shell environment and not from the
        config.makefile; done (PJ);

  \item add a restore: entry in standard PIPS makefile; i.e. modify install
        to preserve the current files in a Backup sub-tree; install:
        mv Production/xxx Backup/xxx; mkdir Production/xxx

  \item archivage des versions successives?

  \item develop new makefiles/shell script in Production to try option
        -O2 or -O3 and -pg; problem with external libraries NewGen and Linear;
        problem with /usr/debug/malloc (add a new environment variable,
        PIPS\_MALLOC?);
        see make-tape for preliminary cleanup; conflicts with
        development activities (available libraries won't be dbx
        compatible); FI, BB: new utility added, {\tt make-pips}.

  \item set core size limit to 0 for demonstration to avoid wasting time;
        how can we do that automatically? On the {\em dret} account? see
        \verb+$PIPSDIR/Tests/demo/demo.rc+;

  \item avoid copying binary codes and libraries twice on {\em install};
        install vs quick-install?!? BB: pips-makemake is modified

  \item add a {\em local pips} entry to library makefile to test library
        changes; the present situation is unbearable if two people work
        at the same time; PJ: a new entry, {\tt test}, is added.

  \item non-regression test: \verb+validate+ added by PJ (November 91);
        bug: validate executes in the Validation directory and
        uses only \verb+Pips/Bin/pips+ (i.e. the pips version visible
        from the Validation directory); fixed by FI, 11 February 1992:
        validate now runs in the current directory;

  \item validate: specific test script shell must be written so as
        to be executable from anywhere, i.e. the user file names must
        be prefixed by \verb+$PIPSDIR/Tests/Validation/xxx+. As a benefit,
        the Validation directories can be write-protected since there
        is no good reason to run validate there. 12 February 1992.

  \item validate generates unpredictable results if two validate
        processes are executed simultaneously in the same directory;
        FI, 12 February 1992.

  \item validate may generate a positive results although some user
        defined tests failed; we do not know if the problem is with
        shell script tests not returning a proper exit code or
        with validate failing to test that a result file exists;
        FI, BB: 12 February 1992.

  \item validate does not always destroy temporary files like diff.nnnnn,
        validate.nnnnn
        and errors.nnnnn in /tmp; maybe interruptions are not trapped;
        FI, 28 February 1992;

\end{enumerate}

\subsection{Wishes:}

\begin{enumerate}

  \item add an option ``-l'' to validate to force \verb+PIPSDBM_DEBUG_LEVEL+ to
        0 (default), 5 or 9.

  \item move PIPS into a SUNOS 4.1 environment

\end{enumerate}

\section{Properties}

\begin{enumerate}

  \item Should they be included in the pipsmake mechanism as variables?
        Wpips could generate automatically their potential values.
        Consistency problem since properties are indirectly used to implement
        pipsmake functions.

  \item \verb+properties-rc.f.tex+ has to be filled in (17 January 1992);
        Lei has to put information about the complexity options; Yi-Qing
        has to explain how her statistics are organized; the present
        content is too sketchy;

\end{enumerate}

\newpage

\section{Regions}

Managed by Beatrice Apvrille, \verb+<apvrille@cri.ensmp.fr>+.

\subsection{Bugs}

\begin{enumerate}

  \item debug Alexis's code; done, June 1992;

  \item elementary statements are decorated with proper regions
        instead of cumulated regions; FI, 2 August 1992;
        useless information is kept but it's hard to decide whether
        a predicate is useful or not (connected components?); {\em
        Franc,ois has found a counter-example that shows that all the 
        information from the predicate must be kept.\/}
        redundant information should be eliminated at least;

  \item three intrinsics were missing: \verb+LOG+, \verb+LOG10+ and
        \verb+PAUSE+; too bad that so many different tables define
        them; FI, 2 August 1992;

  \item \verb+make_phi_entity()+ generated entities with 
        \verb+undefined_storage+; changed to \verb+rom+ to avoid
        later problems; FI, 3 August 1992;

  \item the TCST entity was added in a \verb+sc+ basis in \verb+add_phi_xxx+;
        test added to avoid it; FI, 3 August 1992;

  \item regions could not be computed for io statement with more than
        one element in the io-list; same problem in library effects;
        dealt with in a different way; the keyword \verb+IOLIST==+
        seems ot prefix {\em every} io-list element; FI, 3 August 1992;

  \item debug statements directly and indirectly called by \verb+regions+
        \verb+summary_regions+ do require a value mapping regardless of
        the debug level because arguments of \verb+debug()+ are evaluated
        even if debug does do anything with them; \verb+debug()+ is not
        a macro because varying arguments are not supported by the
        preprocessor; FI, 4 August 1992;

  \item prettyprint modified (as well as \verb+pipsmake-rc.f.tex+) 
        to display the summary region at the beginning of the listing;
        FI, 4 August 1992;

  \item no indices are displayed for regions imported from a called
        procedure; this makes their prettyprinting unreadable and
        the dependence test wrong; fixed by FI and AL, 5 August 1992;

  \item user subscript expressions are substituted by phi variables
        and a polyhedron just before the regions are stored in the
        statement mapping; this is too late for proper combining
        between two (or more) references in the same statement and/or
        expression; see Alexis' mail; FI, 6 August 1992;

  \item dynamic buffer overflow in \verb+words_predicate+; partially
        fixed (buffer size increased, buffer made static and a
        posteriori check); FI, 9 August 1992;

  \item improper profile for function \verb+regions_of_implied_do()+;
        this function expects an expression and an action tag as
        arguments; this is what's passed when it is called from
        \verb+regions_of_iolist()+; but an entity and a list of
        expressions are passed when it is called from 
        \verb+regions_of_intrinsics()+; I cannot think of any easy
        solution to retrieve the action tag, but to set it in a static
        variable when the first call is issued from \verb+regions_of_iolist+;
        see \verb+norm+ in \verb+aile.f+; FI, 9 August 1992;
        I must have fixed it when I debugged {\tt regions\_of\_implied\_do} in
        september or october 1993. But I don't remember how. 
        BA, 10 February 1994.

  \item inadequate optimization in region translation was losing one
        region out of two: each array could have either one READ region
        or one WRITE region; FI, 9 August 1992;

  \item preconditions were not added to regions computed for a user
        call; this bug is likely to have been introduced when FI
        decided to add context to reference predicates as early
        as possible; FI, 7 December 1992; see \verb+Bugs/Regions/vt.f+

  \item the region given for the expression 
                \verb+A(1,1)+ + \verb+A(3,3)+ 
        is a MUST region; it must be a MAY region because the convex
        hull of the two elementary regions is not equal to their strict
        union. BA, September 17, 1993.
        fixed, BA, September 20, 1993 : each time a convex hull is
        calculated, the resulting region is set to MAY. 

  \item the regions given for an implied do are wrong :
        \begin{verbatim}
        C <A-W-MUST-\{PHI1==I\}>
        READ (UNIT=5, FMT=*) (A(I), I=1,5)
        \end{verbatim}
        is displayed instead of :
        \begin{verbatim}
        C <A(PHI1)-W-MUST-\{PHI1<=5, 1<=PHI1\}>
        READ (UNIT=5, FMT=*) (A(I), I=1,5)
        \end{verbatim}
        BA, September 23, 1993
        fixed, even for nested IMPLIED\_DO; BA, September 27, 1993

      \item {\tt sc\_projection} returns a {\tt SC\_EMPTY} predicate when an
        unfeasible system is given as argument. This induces a bug in {\tt
        enlarge\_statement\_regions} because a {\tt sc\_rn} is expected (bug
        detected in {\tt regions\_exact\_convex\_hull}).  Fixed, BA,
        September 28, 1993.

  \item there is a pb during the translation of regions. see mm.f in
        Validation/Regions. BA, February 10, 1994.


\end{enumerate}

\subsection{Whishes}

\begin{enumerate}

  \item use the dependence graph to decide whether or not two elementary
        regions shoud or should not be fused;

  \item the handling of reshaping could benefit from preconditions;
        see \verb+zproax+ in \verb+cr2cnf.f+; Franc,ois Irigoin,
        13 July 1992;

  \item use {Corinne}'s conditions to derive MUST regions? How about MAY/MUST
        preconditions? When preconditions are added within a loop they
        can only *restrain* the region; as long as no (unsafe) projection
        and no convex hull is performed, the region is a {\em must} region
        and it can be used for an array kill if the inclusion of the
        read-may into the write-must can be proved (we could try to use
        the rational generating system, check that it is integer, and
        verify that all vertices are meet the other region constraints);
        this could not be done with the present implementation of regions
        because proper regions are only available for elementary statements
        and not for compound statements; instead of using the \verb+arguments+
        field of transformers, the write effects of the compound statement
        should be used (as for transformers and preconditions);
        FI, 4 August 1992;

  \item in the line with previous paragraph, change the projection
        policy for each control structure: block, test, loop and
        unstructured; the projection should no longer be performed
        uniformely at the statement level in \verb+regions_of_statement+;
        this may create havoc as a major design assumption is changed;
        FI, 5 August 1992; Done, see report E/181/CRI. BA, February 10, 1994.

      \item replace \verb+SC_UNDEFINED+ by \verb+transformer_undefined+ as
        context for scalar variables. Replced by {\tt sc\_rn(..}). BA,
        February 10, 1994.

\end{enumerate}

Note: array kill information is necessary to get rid of dependences on
temporary arrays. The restriction to true dependences does not eliminate
false write-read dependences because there is no way to prove that an
intermediate write rewrites the whole array.

\subsection{Modifications}

\begin{enumerate}

  \item associate to statement cumulated regions instead of summary regions;
        i.e. delay the projection of modified variables as much as
        possible; FI, 5 August 1992; a first step was carried out for
        elementary instruction (i.e. \verb+call+ instruction); well, I
        prettyprint proper regions for elementary statements, but I
        compute cumulated regions to have the regions of a block. See
        report E/181/CRI. BA, February 10, 1994.
        

  \item better computation of MUST vs MAY regions; a new projection procedure
        should be used to know whether the result is exact or not; only
        loop indices can be safely projected (because of the loop semantics);
        loop bounds have to be checked to make sure they are both used
        in the region context before the region is declared MUST; projection
        of variables that are not enclosing loop indices always result
        in a MAY region, regardless of the projection exactness;
        FI, 6 August 1992;

        YY wrote \verb+combiner_with_test(sc, v)+ to check exactness;
        it's in \verb+ricedg/testdep_util.c+ and is cluttered with
        measurement statements; YY, 6 August 1992;

\end{enumerate}

\section{Ri-util}

\begin{enumerate}

  \item break the ri-util.c file into smaller files, one per data type
        (entity, expression, etc...) or at least by data type SCC; we
        need an entity.c for sure; done;

  \item \verb+sc_gen_write+ and \verb+sc_gen_read+ could not handle
        \verb+SC_UNDEFINED+; fixed, FI, 4 August 1992;

        botched: the representation chosen for {\tt SC\_UNDEFINED} is a
        linear system with no equalities and no inequalities over a
        0-dimensional space; this is just the system representing an
        identity transformer;

        suggestion: do not use {\tt SC\_UNDEFINED} in valid data structure;
        modify library region to use \verb+transformer_undefined+ as context
        for scalar variables;

        other suggestion: choose a more complicated scheme to store
        undefined \verb+Psysteme+; for instance \verb+(SC_UNDEFINED)+;
        this implies a test after the opening parenthesis but seems
        pretty easy to implement; FI, 8 August 1992;

  \item dynamic and static areas are declared when a module is created

  \item intrinsic \verb+PAUSE+ was not recognized as an intrinsics:
        there is still another intrinsic table in
        \verb+ri-util/prettyprint.c+;
        FI, 12 March 1993

  \item \verb+entity_conflict_p()+ was deadly wrong; it returned almost
        always false; it contains a static hash table that is going to
        cause trouble in wpips sone day! FI, 29 June 1993;


  \item NORMALIZE\_EXPRESSION is now used instead of
        NormalizeExpression to recurse in normalize.c. RK, FI, 3/12/93.


\end{enumerate}

\section{Rice (parallelization phases)}

\begin{enumerate}

  \item take control dependences into account? Use properly the hierarchical
        control graph? To be studied by BB;

  \item \marginpar{PJ} option to only use data flow dependences
        (\verb+RICE_DG_DATAFLOW_DEPENDENCE_ONLY+); seems to be bugged: only
        output dependencies are ignored (FI);

  \item \marginpar{YY} 
        quantify/summarize parallelization results; comparison with FPP
        should be made easier;

  \item Distribution doesn't work if the loop has a label. pips\_error()
        in local\_name() instead. First loop should keep the label,
        shouldn't it? (BB, 3.1.92)

  \item some parallel loops seem not to be detected in onde24.f and
        its simplification, Wp65/seismic.f

  \item bug in Tests/demo/gewpp.f: combination of tests and loops; 
        GEWPP: Gaussian Elimination With Partial Pivoting;
        a \verb/+1/ had
        been commented out for unknown reasons; fixed by YY, Nov. 8,
        1991;

  \item bug in distribution because of control dependences: if inner
        loop range depends on a statement in the outer loop, it doesn't
        inhibit distribution! See Bugs/Rice/ctrl.f. Bruno Baron,
        6~April~92; this bug was initially detected by Nahid Emad in 1990;

        Basically, the proper effects of statements inside loops now include
        some kind of "control" effects that force sequentialization when loop
        bounds depend on previously defined variables.  The statements' effects
        within a loop have an added read effect on the (possible) variables
        appearing in the loop bounds. The loop body, even if it is a block has
        a proper read effect on these variables. Fix by Pierre Jouvelot, 21 May
        1992;

  \item bug in INTERF from mdg.f; rice does not handle labelled DO loops
like:
\begin{quote}
100 DO 200 I = 1, N
\end{quote}
Label 100 should be reported on an additional previous CONTINUE
statement, because the Allen\&Kennedy algorithm may duplicate loops.
FI, PJ: 16 June 1993; see above, comments from BB on 3 Jan. 92;

\end{enumerate}

\newpage
\section{Ricedg (Dependence Graph Computation)}

This library is managed by Yi-Qing, FI and BA.

\subsection{Bugs}

\begin{enumerate}

  \item \marginpar{YY} normalization problem for:
        \begin{quote}
\begin{verbatim}
A(2*I+4*J) = A(2*I+4*J+1)
\end{verbatim}
        \end{quote}
        Fixed by YY, Nov. 7, 1991; a normalization step was added before
        a call to \verb+sc_minmax_of_variable()+; 
        however, normalization steps
        were later added in \verb+sc_minmax_of_variable()+ by CA for PUMA;
        this should be cleaned up;
  \item \verb+quick_privatize_graph()+ did not work very well. I have
        modified it slightly. It works well now for Wolfe's Cholesky. I can't assure
        it always works for any other programs. If it fails, use
        \verb+Perform priv+ before; (YY)

      \item the types {\tt arc\_label} and {\tt vertex\_label} were not
        defined. They were declared as externals in graph.hand used to
        define vertex; {\tt dg\_vertex\_label} and {\tt dg\_arc\_label} are
        defined in dg.h using vertex.  It didn't appear until PJ added
        castings in dg.h.  fixed by FI and BA, August 1993, by defining the
        types {\tt arc\_label} and {\tt vertex\_label} in {\tt
        ricedg-local.h}.

\end{enumerate}

\subsection{Wishes}

\begin{enumerate}

  \item ddv \marginpar{(YY/FI/PJ)}implementation; necessary for
        PIPS-CRAY and for PUMA; how can we preserve distances (necessary for
        Benoit)? NewGen ddv representation;  can we replace $<$ with
        +1 and say we define a generator of an over-approximation of the
        dependence set?  Decision: dependence cones are implemented;

  \item \marginpar{YY} dependences should be prettyprintable, 
        filterable, and so on; Dominique Cle'ment and visitors in
        Brussels;

  \item the dependence graph should be saved on disk; this is likely
        to become necessary to cooperate with CEA (it was planned
        in FIPEX); it is also very inconvenient when one switches
        from parallel Fortran to Fortran 90 as output style: everything
        has to be recomputed; also, it would make testing easier: now,
        bugs in the dependence graph are likely to be seen as bug in the
        parallelizer; two problems have to be solved, unless one is
        ready to save the statements with it (which may be the easiest
        way): replace pointers from vertices to statements by statement
        ordering (as in mappings), replace pointers from arcs to
        references by some kind of numbering (number of the reference
        in the statement + statement ordering?);

  \item \marginpar{YY}
         an ASCII version of the dependence graph is necessary to validate
         its computation regardless of its use (paralellization, PUMA, Cray);
         its also useful as a debugging tool and as a way of comparison with
         other systems like PAF;

  \item should \marginpar{(PJ)}we introduce control dependences?
        Or rely on our hierarchical control flow graph? should it be
        stored on disk? No NewGen support? 

  \item \marginpar{YY} loops with control structures, structured or not,
       are not analyzed; Remi probably skipped them to gain time because he
       knew that his parallelizer, based on A\&K, would not handle them;
       however, we now need dg information about these loops for PUMA,
       PIPS-CRAY, Yi-Qing's PhD and CEA; the easiest change would be to add a
       property
\verb+RICEDG_ANALYZE_SIMPLE_LOOPS+, which should be set to
\verb+true+ to obtain Remi's behavior; when set to \verb+false+,
its default value, the whole dg would be analyzed; Yi-Qing is in
charge of writing the missing pieces of code; 

The new version was bugged because statements could not be ordered
any more using the low weight part of \verb+statement_ordering+ (YY);
PJ suggests to consider non comparable statements as belonging to
the same compound statement (wich is true). He also suggests to
handle compound statements like user calls, but the use-def chains
are not build that way (so does FI believe).

  \item code related to statistics gathering should not be intimately
        mixed with functional code; FI, 5 August 1992;

  \item the partial evaluation of expression should be used for the
        symbolic loop increment; (YY, 08/92) 
\end{enumerate}

\subsection{Modifications and improvements}

\begin{enumerate}

  \item the dependence cones are implemented which are represented by 
        generating system from \verb+C3/Linear+; (YY, finished in 08/91)

  \item ricedg did not contain the tests of faisabilities for the
        initial system of dependence before the projection to the Di system.The
        modification has been done.  (YY, 18/10/91) 

  \item The module computing
        the levels of dependences gave a false level when di$\leq$0. 
        Thus the case
        of di = NotPositif has been considered in ricedg now.  (YY, 10/91)

  \item auxiliary difference variables have been introduced for all
        variables modified in the current environment (i.e. current
        loop nest); so in this case the dependence test is correct now;
        (YY, 08/92)

  \item the symbolic increment can be accepted by ricedg; the
        independence can be found by GCD test and Di ``all zeros'' test
        in this case; (YY, 08/92) 

  \item statistics \marginpar{(YY)}
        about dependence tests :number of tests, number of dependences, etc...(see
        properties-rc.f.tex also) have been done; its results are stocked
        in a file by setting the property (\verb+RICEDG_PROVIDE_STATISTICS+) TRUE;   
        (YY, 08/92)

  \item the dependence graph is prettyprintable with a option ``dg''
        (ex. Display module dg); (YY, 08/92)

  \item ricedg was (slightly) modified to use the \verb+context+ field 
        of effects decorating the dependence graph; this is possible
        with the \verb+rice+ option but not with \verb+rfulldg+ or
        \verb+rsdg+ as the additional information they would brought
        is already contained in the context; to use this new option,
        you have to select the  \verb+region_chains+ option to compute
        the use-def chains; FI, 6 August 1992;

\end{enumerate}

\section{SDFI}

Summary Data Flow Information.

\begin{enumerate}

  \item \verb+same_sdfi_p()+ was bugged; this was shown on 
        \verb+ONERA/aile.f+; fixed, FI, 7 August 1992;

\end{enumerate}

\section{Semantics}

\begin{enumerate}

  \item Karr like fix-point operator for induction variables; not implemented;

  \item some integer scalar variables are ignored; JH in EXTR;
        IX, IY, INCX, INCY in SDOT; see transformers and effects

  \item use declarations to derive constraints like N.ge.1; use these
        extra-constraints to get MUST effects information for loop bodies?
        see section {\em Regions};

  \item exploit intrinsics like MOD, MAX, MIN and /; keep it simple;
        do not try to use temporary variables for sub-expressions;

  \item check that common variables are properly handled; they should
        be translated like formal parameters; done, FI, July 1992;

  \item EQUIVALENCE equations are lost

  \item change transformer\_convex\_hull() to take into account distinct
        basis; a 0-dimensional space should be a neutral element (always?); how
        partial subspaces could be dealt with? Is transformer\_identity\_p()
        satisfying?

  \item display summary\_transformer before module declarations when
        prettyprinting transformers;

  \item value\_mappings should be reset in module\_name\_to\_transformers,
        module\_name\_to\_preconditions, prettyprint\_transformers,
        prettyprint\_preconditions

  \item does the name {\em t} of the Fortran program have the magic
        effects? we can't get the effects of the Fortran program that
        has the name {\em t}. when we change the name, everything works
        fine. Strange indeed ?! ( added by lz 05/06/91 )

  \item improve handling of test like \verb+i.eq.1+ to cope with Wegman/Zadeck
        example (see \verb+zadeck.f+); 
        useful for switches too: if-elseif-elseif....

  \item propagate call site preconditions for functions (only call to
        subroutines are handled);

  \item integer variables stored in COMMON are not handled interprocedurally;
        the translation routines wrongly assumed that they were named as
        \verb+common_name:variable_name+; in fact, they are named like
        any other variable as \verb+module_name:variable_name+; fixed,
        FI, July 1992;

  \item a new bug cropped up for equivalenced variables; it appeared
        the first time for ICS'91; it seems that \verb+entity_conflict_p+
        was modified: equivalenced variables do not conflict anymore...
        see \verb+equi.f+;

  \item integer user functions like {\tt m = ifoo()} are not exploited because 
        the expression is not linear; FI: I think it is very unusual to
        know the value returned by a function; LZ: I have a case in
        Hermann Mierendorf code; December 1991;

  \item preconditions are not properly computed after a privatization
        in demo/choles.f:

  \item in func.f : information about function values is lost when
        transformers are computed: see interprocedural preconditions of func.

  \item Pips doesn't work for the empty THEN or ELSE block, needs a
        CONTINUE to be passed by pips. LZ 24/02/93

\begin{verbatim}
   baron@isatis Tests/demo : Init -d -f choles.f ch
   baron@isatis Tests/demo : Perform -m choles priv
   baron@isatis Tests/demo : Select rsdg
   baron@isatis Tests/demo : Display -v
   Workspace ch (default)
   Module CHOLES (default)
   pips -m CHOLES -s PRINT_PARALLELIZED_CODE -b PARALLELPRINTED_FILE ch
   Workspace ch opened
   Selecting rule: PRINT_PARALLELIZED_CODE
   Request: build resource PARALLELPRINTED_FILE for module CHOLES.
     PROPER_EFFECTS       building   PROPER_EFFECTS(CHOLES)
     CUMULATED_EFFECTS    building   CUMULATED_EFFECTS(CHOLES)
     CHAINS               building   CHAINS(CHOLES)
     SUMMARY_EFFECTS      building   SUMMARY_EFFECTS(CHOLES)
     TRANSFORMERS_INTRA_F building   TRANSFORMERS(CHOLES)
     PRECONDITIONS_INTRA  building   PRECONDITIONS(CHOLES)user warning in call_site
   _count: not implemented yet, always returns 1
   pips error in entity_to_new_value: unbounded entity K
\end{verbatim}

        Fixed by FI, 1 August 1992; two changes were performed; (1) conditions
        on loop indices are added only if the loop index has a value;
        (2) privatized scalar integer variables are declared as having
        a value although they do not have any effect; see \verb+mappings.c+

  \item \verb+Bugs/Semantics/lmult.f+: assertion violated in sommet.c;
        this routine computes values close to MAXINT using parentheses
        to avoid overflows; YY/FI: 18 February 1992;

  \item \verb+add_or_kill_equivalenced_variable()+ warns about return
        storage of \verb+isamax+ in \verb+isamax+; this warning does
        seem to be useful, but is it really useful? FI, 23 July 1992

  \item summary preconditions are not always correct; there is no way
        to reset them and they are always increasing when convex hulls
        are performed; so if a summary precondition has been set equal
        to the whole space at some time, it can never be made more
        precise afterwards; see \verb+eval.f+; FI, BB, 30 July 1992;

  \item prettyprint modified (as well as \verb+pipsmake-rc.f.tex+) 
        to display the summary precondition or transformer
        at the beginning of the listing;
        FI, 4 August 1992;

  \item exponential time and space for interprocedural preconditions
        of \verb+onera/aile.f+; heuristics used to trade accuracy
        vs speed; FI, 13 Jan. 1993;

  \item global value translation: an actual parameter was a COMMON
        variable in the caller, and just a formal in the callee;
        translation from the caller into the callee was attempted;
        see \verb+translate_global_value+, FI, 13 Jan. 93;

  \item seems to loop forever on LMULT in LGS.f (QCD Perfect Benchmark);
        likely origin: subtle additions of 2**31... FI, 5 May 1993

  \item value entity \verb+PROFYL:IS1ST#init+ is found with fields type,
        storage and value undefined (\verb+undefined_xxx+) when
        computing preconditions for \verb+TRACK+; the core dump occured
        when the summary precondition of \verb+THINIT+ was computed;
        the bug was not reproducable; it seems due to a long use of
        a workspace; I applied \verb+stf-workspace+ after a first run;
        no idea; FI, 18 June 1993;

  \item call to \verb+sc_minmax_of_variable+ added in
        \verb+add_good_loop_conditions+ to handle symbolic steps
        when their signs are known; suggested by Kathryn McKinley;
        FI, 28 June 1993;

  \item \verb+translate_global_value+ and
        \verb+precondition_intra_to_inter+ are modified to let invisible
        information pass thru intermediate procedures; e.g. A calls B
        and B calls C; A and C share common variables, which are not
        seen from B; however, B summary transformer and precondition
        must take them into account; FI, 29 June 1993;

  \item code added for MIN0 and MAX0 intrinsics; FI, 18 August 1993;

  \item code added for STOP intrinsic; FI, 3 February 1993; this may be
        devastating in a main; STOP is abstracted as the non-feasible
        transformer;

  \item prettyprint of transformers and preconditions with the parsed
        code instead of the code to avoid display troubles introduced by
        the controlizer; FI, 3 February 1994

      \item in {\tt statement\_to\_transformer}, the transformer is not
        copied before being stored in the statement mapping and is then
        returned to the caller. {\tt loop\_to\_transformer} modifies it by
        side effect. The transformer and the preconditions computed with it
        where wrong in some cases (see {\tt
        Validation/Semantics/li92\_1.f}. BC, oct. 94.
\end{enumerate}

\subsection{Wishes:}

\begin{enumerate}

  \item the precondition list should not be reversed from each statement
        to the next;

  \item equalities with negative constants are not nicely printed:
        \verb+I==-1+ is shown as \verb/I+1==0/;

  \item it would be useful to compute the buffer size necessary to
        read a precondition p (i.e. a Pvecteur) each time p is
        written;

  \item it would be nice to accept a user file as an initializer for
        summary preconditions, at least for the main; if the C3/Linear
        format is used, a parser is already available; \verb+user_request+
        could be used to enter a file name; FI, 30 July 1992

  \item LOGICAL variables should be handled; see bdna.f (perfect Club)
        as an example; this is also requested by Stein and Fox in CPE,
        Dec. 93; FI, 10 February 1994

\end{enumerate}

\section{Syntax}

Lexical and syntactic analyzer. 

% Should be called \verb+parser+. BTW, to
% see debug messages, set \verb+PARSER_DEBUG_LEVEL+ instead of 
% \verb+SYNTAX_DEBUG_LEVEL+!
%
% No more: I changed PARSER_DEBUG_LEVEL into SYNTAX_DEBUG_LEVEL; FI, 15
% September 1993

\subsection{Bugs}

\begin{enumerate}

  \item \label{name-spaces}
        unique space for COMMON names and variable names; modify lexical
        analyzer and parser to keep slashes around common names so as to
        distinguish them textually from variables? See MakeCommon and the
        prettyprinter. Or use a general scheme to distinguish external
        variables from internal variables.

        bug in Bugs/Syntax/decl2.f: the common is declared before the variable;
        a nice {\em user} error message is printed; fixed by FI;

        bug in CEA benchmark \verb+matrie.f+
        (also stored in \verb+Bugs/Syntax/matrie.f+): 
        double use of \verb+ZZZZ+ as
        a logical variable name and as a common name; PIPS used to core dump
        because no doucle declaration check was performed when the
        variable was declared before the common; fixed (if I may say so!)
        by FI, 16 January 1992: a proper {\em user} error message 
        is printed before PIPS stops...

        The PIPS grammar \verb+gram.y+ and \verb+MakeIntrinsic+ were
        modified to directly prefix global names
        with \verb+TOP-LEVEL+ instead of \verb+CurrentModule+.
        \verb+MakeAtom+ should also be modified to update entity names
        when it finds out that it is an external entity; \verb+MakeAtom+
        modified mainly through \verb+MakeExternalFunction+; FI, 9 March 1992

  \item unique space for INTRINSIC names and variable names; 

        intrinsics 
        detection could (should) be delayed to link time, but their names
        should not appear as callees (done); else the request for source file
        will abort; check that the linker does not use callees to build
        its link hash table; how is the INTRINSIC statement handled? The
        EXTERNAL one?

        more or less fixed by FI, 21 September 1991; there is still a unique
        name space for intrinsics and variables but a variable can have
        as name the name of an intrinsic;

        further fixed by FI, 16 January 1992; see previous 
        bug \ref{name-spaces};

  \item Pips' parser bug : type problem.
        It is observed when running ``pips'' on BLAS3 subroutines for the 
        control-flow graph.
        The problem comes from the following declaration:
        \begin{verbatim} 
                CHARACTER*( * )    NAME, OPTS
        \end{verbatim}
        pips' parser fails to give a good type to \verb+NAME+ and
        \verb1OPTS1.
        dbx detects that 
        MakeFortranType(t = 9, v = 0x1ee1b8),line 755 in syntax/declaration.c 
        where 9 is too large for a Fortran type. ( Supposed to be 1 to 7 ).
        The sample program is kept in ~pips/Pips/Tests/Bugs/Syntax/ilaenc.f .
        Reported by LZ, Oct. 9, 92.

        Fixed; FI, 15 September 1993

  \item missing declarations observed by Lei Zhou when building the call
        graph; might be due to name captures?

  \item see if the call to FindOrCreateEntity() should/could not be delayed
        in gram.y; when an entity is recognized, only its name should
        be passed in \$\$ instead of a pointer to an entity whose full name
        is still unknown; entities should be effectively created when the
        context is better known, COMMON statement or not, declaration
        or instruction part of the module.

        In fact, this is dealt with in bug \ref{name-spaces}.

  \item accept BLOCKDATA, even if they are not exploited, to avoid core
        dumps on real programs; try to do that when writing a pips fsplit
        in awk (FI); but awk cannot open more than 7 files and is stuck
        when more than 6 subroutines have to be split;

        BLOCKDATA are parsed like other modules; FI, 15 September 1993;

  \item run PIPS on SPEC77 (PerfectClub); look at \verb+del4+, \verb+gloop+ and \verb+gwater+;
        these files are now properly parsed; parallelization bug from
        Yi-qing prevents interprocedural parallelization in \verb+del4+; effects
        bug in \verb+gfidi+ prevents interprocedural parallelization in \verb+gloop+;
        effects bug in \verb+lrgscl+ prevents interprocedural parallelization in
        gwater;

  \item ParserError() should accept varargs; at least errors should be better
        documented using user\_error() before;

  \item {\em conflict for symbol ...}: no line number specified;

  \item move as many modules as possible into ri-util; they are going to
        be used to generate code automatically in PIPS-CRAY and PUMA;
        Corinne has already taken a look at this; Bruno too;

  \item EXTERNAL declaration in onde24.f causes a core dump (not a syntax
        error); fixed by FI, 21 September 1991;

  \item functional formal parameter should be detected by the parser and
        processing should be stopped with a proper warning;

        Theoretically fixed by FI, 18 January 1992, but not checked.
        Checked with formel.f and installed in Validation/Syntax,
        FI, 12 February 1992.

  \item some comments seem to be lost in onde24.f, although they are
        not linked to control statement (i.e. \verb+GO_TO+);

  \item constant \verb+MAXBLOCK+ controls the nesting level; since
        \verb+ELSEIF+ are desugared in nested \verb+IF+, its value of
        10 was too small; it was increased to 20 (FI, 2 December 1991)

  \item wrong user line numbering (detected by Bruno Baron, 23 October 1991);
        observed again by BB, November 12; use {\em validate} to get
        the bug; \verb+stat_num+ was statically initialized; statement
        number depended on the module ordering for parsing (FI, 3
        January 1991); this does not explain the problem with validate
        since everything should be redone in the same order unless
        databases are not destroyed;

  \item unique space for {\em intrinsics} and user names; keywords like
        \verb+RETURN+ are handled as intrinsics and are misinterpreted
        as calls to a user function, it is defined:

\begin{verbatim}
      subroutine return
      return
      end
\end{verbatim}

        PIPS core dumps because this is encoded as recursive calls to
        user function \verb+return+. There is no difference in the RI
        for \verb+RETURN+ and \verb+CALL RETURN+.

        This may have now changed because external names are directly prefixed
        by \verb+TOP-LEVEL:+. Such user function names should
        be refused as redefinition of an intrinsic. 
        See the correction of the first Syntax bug.
        To be checked (FI, 22 January 1992).

  \item \label{complex-constant}
        complex constants are not parsed: \verb+syntax error near ,+;
        discovered because Dongarra parallelizer benchmark was not
        accepted; see Bugs/Syntax/complex.f; FI, 08/01/92; 

        partially fixed at the lexical level: pairs of floating point
        numbers are recongized but not pairs of integer numbers or
        integer/floating point combinations (see paragraph 4.6.1 in
        the standard); FI, 9 January 1992; this does not work: pairs
        of floating point numbers can appear in a function call like
        \verb+FOO(1.,2.)+;back
        to the previous version;

        Fixed by modifying \verb+gram.y+. See comments there.
        FI, 18 January 1992;

        This introduced a bug related to IO list (see below
        \ref{implicit-do}). Complex
        constants are not recognized in PIPS Fortran. They have to
        be replaced by a call to \verb+CMPLX+.

  \item COMMON length are likely to be varying from a module to another.
        They are statically allocated as \verb+TOP-LEVEL:FOO+ instead
        of \verb+CURRENT:FOO+ and common variables are systematically
        added at the current end of the common.

        This bugs was fixed a first time by generating {\em local}
        common names, to be later linked (more or less?). This caused
        problems when a variable and a common have the same name. This
        seems to be more the rule than the exception: programmers
        name their common like the most important variable of the
        common.

        It should be fixed by adding a \verb+common_size_map+ in the
        parser to remember the current size of the common wrt. the
        current module. The consistence between the different
        declarations of a given common could be checked at the
        end of each module syntactic analysis.

        A new bug file should be added to \verb+Bug/Syntax+.

        Fixed by FI, February 1992.

  \item character constants are not handled in a uniform and consistent
        way in statements and in FORMATs. See \verb+Bug/Syntax/string.f+

  \item \label{implicit-do}
        implicit DO in IO lists seem to be refused by the parser; see
        \verb+Bugs/Syntax/edition.f+ from \verb+aile.f+ and
        \verb+Bugs/Syntax/grandg.f+ from \verb+cfc2.f+ (SIDOLO);
        Yi-Qing bumped into the problem many other times in Sidolo
        and ONERA benchmarks; this used to work; 29 January 1992;
        this might be linked to the new rule introduced to parse
        complex constants (see above \ref{complex-constant})

        No simple solution seems to exist; complex constants have
        to be not recognized for the time being; FI, 4 February 1992.

  \item blank commons are not handled properly; see \verb+Lib/syntax/common3.f+
        FI, 6 February 1992;

  \item long lists of values in DATA are not accepted: yacc stack overflow;
        see \verb+Bugs/Syntax/hit.f+; Yi-Qing, 5 February 1992;
        fixed, FI, 6 February 1992;

  \item long list of variables and values found inconsistent by PIPS in
        \verb+AnalyzeData+:
        \begin{quote}
        Too few initializers
        \end{quote}
        but not by f77; see \verb+Bugs/Syntax/calavs.f+, Yi-Qing,
        6 February 1992

        Same problem seems to occur in \verb+Bugs/Syntax/rffti1.f+ which
        is extracted from \verb+APS.f+; the context is easier;
        FI, 14 February 1992;

        MakeDataVar() used to assume that a whole array was initialized
        each time an array element was initialized; AnalyzeData() seemed
        to have a wrong test for the error message \verb+too few initializer+;
        fixed (?), FI, 18 February 1992;

  \item long lists of arguments in CALL are not accepted: yacc stack overflow;
        see \verb+Bugs/Syntax/pset.f+; Lei Zhou, 6 February 1992;
        fixed, FI, 6 February 1992;

  \item \verb+fsplit+ does not indentify subroutines whose names appear
        only in a continuation line; FI, 7 February 1992

  \item empty programs having just one END statement are declared illegal;
        FI, 7 February 1992 (almost a feature; this is due to the grammar;
        PJ tried to update it but failed);

  \item the substring operator is recognized but not accepted; it is not
        possible to generate a regular call to the substring function
        because it may appear as a lhs; however, in SIDOLO, the program
        where we bumped into te problem, all substring operations seem
        to appear in expressions (i.e. rhs)
        FI, June 1991 and 11 February 92.

  \item long lists of formal parameters are not accepted: yacc stack overflow;
        see \verb+run.f+ in Perfect Club benchmark \verb+APS.f+;
        Francois Irigoin, 12 February 1992; 

  \item no overflow check for the comment buffer; array \verb+keywidx+ was thus
        overwritten when parsing \verb+RUN+ in 
        Perfect Club benchmark \verb+APS.f+; Francois Irigoin, 12
        February 1992; overflow check added (and tested) and then
        buffer size doubled; FI, 12 February 1992.

  \item \verb+implicit_type_p()+ seems to be to strict; 
        see \verb+Bugs/Syntax/imprif.f+; because the implicit statement
        occurs (necessarily) after the formal parameter declaration, they
        do not have their current implicit type when they are redeclared;
        see variable \verb+CN+ in \verb+imprif.f+; LZ, FI: 12 February 1992;
        the only short term fix I can think of is to move formal parameter
        declarations {\em before} the \verb+IMPLICIT+ statement;

  \item one more (unidentified) bug with \verb+RUN+ in 
        Perfect Club benchmark \verb+APS.f+; 
        \begin{quote}
        user error in MakeExternalFunction: parser error between lines 303
        and 304 \\
        Formal functions are not supported
        \end{quote}
        Franc,ois Irigoin, 12 February 1992

        In fact, it was a bug in \verb+APS.f+; 
        YY had commented out declarations in \verb+run.f+ thus making regular
        array (formal parameter) look like formal functions; the error
        message should be more informative; fixed;

  \item TABs are not properly expanded; see \verb+Bugs/Syntax/tabs.f+;
        they seem to be systematically replaced by 7 spaces 
        in \verb+GetChar()+; FI, 21 February 1992; fixed, FI, 26
        February 1992 (Column was replaced by col in the test)

  \item pips core dumps with too many ENDIFs instead of issuing a user
        error; see \verb+Bugs/Syntax/endif.f+; missing check in
        \verb+MakeEndif()+; FI, 21 February 1992

  \item SPACEs in empty lines are taken into account to compute the
        truncation at column 73 in the next instruction; FI, 27 February
        1992; fixed by reinitializing col one more time, FI, 27 February 1992;

  \item on \verb+user_error+ the database is not saved; \verb+sserror+
        (i.e. \verb+yyerror+ in gram.y) ends with a regular exit;
        this exit was replaced by a call to \verb+pips_error+; FI, BB,
        27 February 1992;

  \item \verb+LOG+ and \verb+LOG10+ are standard generic intrinsics;
        they were forgotten in bootstrap and effects; 
        see \verb+Validation/Syntax/testlog.f+
        fixed, FI, 27 February 1992;

  \item all symbols declared in an EXTERNAL statement are considered
        FUNCTIONS and their results are given implicit types;
        this clashes when it is discovered later that is is a SUBROUTINE;
        the type consistency test was soften to allow implicit types
        to be redefined; why is the \verb+unknown_type+ not used? See
        above the problems with types of formal parameters. FI, 9 March 1992.

  \item comment lines starting with an asterisk are not recognized
        although this is part of the standard (section 3.2.1);
        see \verb+Bugs/Syntax/asterisk.f+, FI, 10 March 1992;
        fixed, FI, 12 March 1992;

  \item initial lines with a zero in continuation column were mistaken
        as continuation lines; see \verb+Validation/Syntax/cont0.f+;
        FI, 12 March 1992;

  \item PIPS core dumps on the following string declaration:
        \verb+CHARACTER*(*) MESAGE+; see \verb+/Bugs/Syntax/errors.f+; YQ,
        6 Apr. 1992; Fixed, FI, 15 September 1993;

  \item type reconstruction error in \verb+MakeExternalFunction+; see
        \verb+Validation/Syntax/linear.f+ and \verb+formel.f+; the bug
        was introduced by FI when trying to perform link at parse time;
        fixed; FI, 12 July 1992;

  \item TABs were not properly expanded; see \verb+Validation/Syntax/sdot.f+;
        fixed; FI, 14 July 1992;

  \item global \verb+SAVE+ statement without argument list is not recognized;
        FI, 28 July 1992; fixed in gram.y but no action implemented, FI,
        28 July 1992;

  \item calls to function without arguments are not taken into account
        in \verb+CALLEES+; see \verb+Bugs/syntax/funcall.f+;FI, 30 July 1992;

  \item comments carried by \verb+CONTINUE+ statements seem to disappear;
        this may not be due to the parser but to the controlizer;
        FI, 5 August 1992;

  \item comments carried by instruction formated on multiple lines seem
        to disappear; see \verb+Bugs/syntax/contcom.f+; FI, 30 December 1992;

  \item parameterless function calls are analyzed as reference to
        variables; see \verb+Bugs/Syntax/time.f+; FI, 11 March 1993;
        fix on March 11 showed a bug in ri-util on PAUSE;

  \item buffer size increased for very long declarations in ocean.f,
        PerfectClub; FI, 21 June 1993;

  \item {\tt free\_entity()} added in procedure.c when a assumed variable turns
        out to be a function and when that variable has been explictly 
        declared before as in:
\begin{quote}
INTEGER*4 CLOCK

I = CLOCK()
\end{quote}
        This does not solve all problems if LOG = LOG(3) is allowed
        by Fortran standard. See comments in procedure.c.
        FI, 21 June 1993

  \item bug in oa118.f because of the previous modification; the freed
        entity is not removed from the module declaration list; YY, FI,
        24 June 1993; new function \verb+remove_variable_entity()+
        added in \verb+ricedg+ to clean decalrations better; FI, 25 June
        1993

  \item bug in validate: formal functional parameter are no longer
        detected; previous modification is refined according to the
        storage in MakeExternalFunction()+; the bug was not apparent
        because the validation shell scripts were bugged; fixed; FI, 29
        June 1993

  \item bug in ??? (renault.f) in a statement containing an expression looking like
\begin{quote}
ALOG(0.5*ALOG(X))
\end{quote}
        \verb+remove_variable_entity()+ is called on the fly and one of
        the two pointers to \verb+MODULE:ALOG+ is left dangling; a list
        of variables to remove should be built and non-existent
        variables should be removed at the end of the parsing phase;
        CA, FI: 8 July 1993; fixed the same day;

  \item intrinsics \verb+BACKSPACE+ was only partially recognized in the
        parser, and forgotten in bootstrap/bootstrap.c and
        effects/intrinsics.c; CA, FI: 8 July 1993
        
  \item character constants are unduly capitalized (see standard section
        4.8: {\em any character capable of representation in the processor});
        FC: July 1993

  \item computed GOTO are desugared; FI, 15 September1993;

  \item field offset is fixed at the end of the declaration section; it
        was wrong when the type was declared after the first occurence
        of the variable; this bug introduced pseudo-conflict between
        variables in the same common (see TRACK for instance); FI, 1 December 1993

  \item commons already in the symbol table are now properly declared as
        visible in the current module declaration field; FI, 1 December 1993

  \item bug on offset computation for variables of different modules
        fixed; FI, 2 December 1993

  \item CleanLocalVariables() improved to clean up the layout field of
        commons; possible explanantion for a problem with free added;
        FI, 2 December 1993

  \item bug on area layout when types of variables are redeclared: the
        type redeclaration modifies the length used by the variable in
        the area but not its offset; observed on track-fi.f; 1 February
        1994; problem avoided by ignoring type redefinition for
        variables whose store is already defined; FI, 1 February 1994;

  \item bug on area layout (again); previous modification was a mistake;
        bug in {\tt update\_common\_layout()} fixed; validation with retype.f;
        FI, 7 February 1994;

\end{enumerate}

\subsection{Wishes:}

\begin{enumerate}

  \item recognize and handle Fortran macros

  \item recognize and handle ENTRY

\end{enumerate}

\section{Text and Text-Util}

Low-level data structure used by prettyprint oriented functions.

\subsection{Bugs}

\begin{enumerate}

  \item likely to be responsible for many memory leaks (FI); intermediate
        storage is not freed, just in case it might be shared, which happens
        sometimes! Please no more code of that kind in PIPS\verb+:-)+.

  \item the layout is Fortran dependent; information from ri-util,
        namely \verb+STATEMENT_NUMBER_UNDEFINED+, should be used but is
        not available according to the PIPS library hierarchy; this constant
        is assumed {\em negative}; the bug was: multiple line statements
        unconditionally had -001 as statement number for all lines but
        the last one, that had the proper number; FI, 16 November 1991;

  \item strings to long to fit on one line were not broken into parts
        and continuation lines but just printed; code produced by PIPS
        was not compilable; this showed mainly in FORMAT statements;
        FI, 15 March 1993.

\end{enumerate}

\section{Tiling}

No library yet. See {\em wp65}.

\subsection{Wishes:}

\begin{enumerate}

  \item handle not perfectly nested loops by adding guards and optimize
        guards in generated code by moving as many of them as possible
        in loop bounds

  \item handle adjacent loops, like the ones in image processing, by
        using a \verb+switch+-like statement

\end{enumerate}

\section{Top-Level}

\subsection{Bugs:}

\begin{enumerate}

  \item can we assume there exist a split shell script? Is it easier
        to modify than C code? Not compatible with file regeneration
        when pips is stopped? Adapt \verb+top-level/source_file.c+
        and test \verb+Utilities/split.awk+

\end{enumerate}

\section{Transformer}

Low level data structure for semantic analysis (see {\em Semantics}).

\subsection{Bugs:}

\begin{enumerate}

  \item \verb+transformer_filter+ had to be extended to cope with
        empty relations (non-feasable systems)

  \item \verb+transformer_combine+ was extended to cope with dead code
        by generating a non-satisfiable transformer; FI, 27 May 1993;

  \item \verb+print_transformer+ was restricted to well-defined
        transformers; i.e. a pips error is generated; FI, 27 May 1993;
        
  \item \verb+transformer_convex_hulls+ was extended to cope with dead code
        by generating a non-satisfiable transformer; FI, 27 May 1993;

  \item \verb+arguments_union+ was not unioning its second argument;
        FI, 27 May 1993

  \item \verb+value_entity_p()+ added; FI, 29 June 1993

\end{enumerate}

\section{Utilities}

\subsection{Bugs}

\begin{enumerate}

  \item \verb+validate+ does not complain directly 
        when it is run on a non-existing directory (it sends an e-mail message)

  \item \verb+Perform+ does not check the existence of a rule before
        starting Pips and pips; this leads to a core dump for a typo!
        FI, 10 August 1992

  \item \verb+pips-makemake+ modified to use sccs; RK, FI, June 1993

\end{enumerate}

\section{WP65}

Code written for the PUMA project: emulated shared memory scheme.

\begin{enumerate}

  \item allocate scalar variables cyclically over the banks; how are we
        going to remember on which bank is what? One more hash-table?

  \item also allocate array starting cyclically over the banks; the initial
        bank has to be remembered; this is going to make the address equation
        more complicated;

  \item se'parer clairement le calcul du nombre de copies
        a` effectuer du calcul de leur dimension (?!?);

  \item estimate the space complexity;

  \item improve (!) the tiling decision making procedure;

  \item improve the target machine description (see also Complexity and 
        Bruno's work);

  \item \marginpar{CA}
        the message lengths (\verb+LS+) 
        for \verb+WP65_SEND+ and \verb+WP65_RECEIVE+ should be
        expressed in bytes instead of number of elements; 
        the underlying library has no way to guess the data type used;
        (fixed by CA in \verb+build_sc_machine.c+ \& 
        \verb+tiling.c+, January 1992)

  \item in the target machine description, the bank width (i.e. the
        line size \verb+LS+) should be expressed in bytes;
        (have to be fixed in each model.rc)

  \item \verb+model.rc+ should be stored as \verb+model-rc.tex+ in
        \verb+Documentation+ and derived automatically as for other
        \verb+xxx.rc+ files

  \item initial loop indices should be replaced by expressions and not
        by local loop indices when they do not appear in subscript
        expressions; example: \verb+MAT(I,J)= I + J+; general
        expression:
        \begin{quote}
        \verb/I = P I_0 + O + L_I/
        \end{quote}

  \item parallelism over the channel requests cannot be expressed in
        Fortran~3L; it's OK with sockets because IO's are buffered;

  \item make WP65 code validation easier; add a call to generic (?) init
        and validate procedures?

  \item \verb+load code+ should be based on local declarations:

    \begin{enumerate}

        \item multiple references are almost always reads;

        \item local declarations cannot contains elements that
        do not exist in the emulated shared memory; to be proved,
        using the fact that arrays are rectangular;

        \item see with Corinne if the global problem cannot be broken
        into pieces;

        \item use \verb+sc_proj_test+ and other test routines to see
        if thinks would be easier when the bank number or the processor
        number are numerically known;

    \end{enumerate}

  \item \marginpar{LZ} 
        \verb+REAL+ variables are generated with type \verb+REAL*8+
        instead of \verb+REAL*4+

  \item \marginpar{CA} temporary local variables like loop indices are declared
        more than once; (fixed by CA, january 92)

\end{enumerate}

\section{Wpips}

\begin{enumerate}

  \item no problem with X-events sent to a stopped wpips at Rice nor at
        ICASE; we should compile with XView3 available on font (waits
        until /usr/lib/libc.so.1.6 is available: required for XView3);

  \item do not create any database before the user have selected the
        fortran files. This way, should no more core dump after: create
        workspace in a directory without .f file, change directory,
        create workspace again.

  \item {\tt wpips\_user\_error} debugged (RK, January 1993)

  \item numerous interface bugs corrected (RK, FI, January 1993)

\end{enumerate}

\subsection{Wishes:}

\begin{enumerate}

  \item simpler module selection: direct input from the main window;
        selection in any window (e.g. from a displayed call graph
        or from some displayed code); first solution: the module
        selection window can stay as long as you want;

  \item try dynamic linking to reduce startup time

  \item give a new exit choice to avoid storing lots of stuff in the
        database; done;

  \item new entry point like \verb+user_error+ to display information or query
        the user; suppose we write a \verb+check_call_sites+ (as we should have
        according to PIPS contract): how are we going to display binding
        errors (parameter number, type, in/out mismatches)? How are we
        going to display the problem location?

  \item Parascope has a special {\tt snapshot} button to include screen
        views in papers; also available here (xwd -frame | xpr -device
        ps | lpr -Pnec); sa. xselection

  \item delay calls to pipsmake() when a user selects a rule; he may change
        his mind and resources may be unnecessarily destroyed; done by BB;

  \item do not print {\em closing module} when nothing is done; thanks!

  \item add a copyright notification (for instance an extra-window displayed
        while wpips is dynamically linked...)

  \item no documentation available yet; needed for ICS'91 in June 91;

  \item \verb+char * top_level_query_user(char * prompt)+ needed for 
        interactive phases (prettyprint dependences, unrolling, etc...);

\end{enumerate}

\subsection{Modifications}

\begin{enumerate}

  \item new buttons have been added in the menu for icfg, dependence
        graph and various transformations (FI, January 1993)
        
\end{enumerate}

\end{document}
\end
